
\documentclass[12pt,a4paper]{book}
%\makeatletter
%\makeatother
\usepackage{fancyhdr}
\usepackage[headings]{fullpage}

\usepackage[pdftex]{graphicx} % Pour l'insertion d'images
\DeclareGraphicsExtensions{.jpg,.mps,.pdf,.png} % Formats d'images

\usepackage[pdftex]{thumbpdf}      % Vignettes
\usepackage{xcolor} % required for colors by hyperef
\usepackage[pdftex,                %
    bookmarks         = true,%     % Signets
    bookmarksnumbered = true,%     % Signets num�rot�s
    pdfpagemode       = true,%     % Signets/vignettes ferm�s � l'ouverture
%   pdfpagemode       = Fullscreen,
    pdfstartview      = FitV,%     % La page prend toute la hauteur
    pdfpagelayout     = SinglePage,% Vue par page
    colorlinks        = true,%     % Liens en couleur
    linkbordercolor   = white, % Couleur de la bo�te sur les liens normal 
    citebordercolor   = white, % Couleur de la bo�te sur les citations 
    filebordercolor   = white, % Couleur sur la bo�te sur les fichiers 
    urlbordercolor    = white, % Couleur sur la bo�te sur les URL
    linkcolor         = cyan, % Liens internes
    urlcolor          = blue, %  % Couleur des liens externes
    pdfborder         = {0 0 0}%   % Style de bordure : ici, pas de bordure
    ]{hyperref}%                   % Utilisation de HyperTeX


\hypersetup{ % Modifiez la valeur des champs suivants
    pdfauthor   = {Eddy Caron, Yves Caniou, Benjamin Depardon, Ha�kel Gu�mar}
    pdftitle    = {DIET Contributor's Guide},%
    pdfsubject  = {Manual},%
    pdfkeywords = {DIET, Grid-RPC},%
    pdfcreator  = {PDFLaTeX},%
    pdfproducer = {PDFLaTeX}}


%\usepackage[french]{babel}
%\usepackage[latin1]{inputenc}
%\usepackage{multicol}
\usepackage{verbatim}
\usepackage{url}
\usepackage{subfigure}
\usepackage{listings}
\usepackage{xspace}
\usepackage{calc}
\usepackage{marvosym} % More symbols
\graphicspath{{./fig}}

\usepackage{pifont} % for \ding{52} dagda

\newsavebox{\logobox}
\sbox{\logobox}{\includegraphics[scale=0.4]{fig/logo_DIET}}
\newcommand{\logo}{\usebox{\logobox}}

%%%%

\newcounter{rmq}[section]
\setcounter{rmq}{0}
\newenvironment{remarque}{\addtocounter{rmq}{1}\textbf{N.B. \thermq:}}{}

%%%%

%%%%
\renewcommand{\title}{DIET Contributor's Guide}
%%%%

\pagestyle{fancyplain}
\fancyhead[L]{\title}
% \lhead[\fancyplain{\title}{\title}]
%       {\fancyplain{\title}{\title}}
\chead{}
\rhead[\fancyplain{\logo}{\logo}]{\fancyplain{\logo}{\logo}}

\lfoot[\fancyplain{\scriptsize{\copyright} ~INRIA, ENS-Lyon, UCBL, CNRS, SysFera}{\scriptsize{\copyright} ~INRIA, ENS-Lyon, UCBL, CNRS, SysFera}]{\fancyplain{\scriptsize{\copyright} ~INRIA, ENS-Lyon, UCBL, CNRS, SysFera}{\scriptsize{\copyright} ~INRIA, ENS-Lyon, UCBL, CNRS, SysFera}}
\cfoot[\fancyplain{}{}]{\fancyplain{}{}}
\rfoot[\fancyplain{Page~\thepage}{Page~\thepage}]
      {\fancyplain{Page~\thepage}{Page~\thepage}}

\usepackage{todonotes}

\newcommand{\todoall}[2][]{\todoinline[color=blue!20,#1]{\textbf{ALL:} #2}}
\newcommand{\todohg}[2][]{\todoinline[color=purple!50,#1]{\textbf{Ha�kel:} #2\
}}
\newcommand{\todoyc}[2][]{\todoinline[color=pink!70,#1]{\textbf{Yves:} \
#2}}
\newcommand{\todobd}[2][]{\todoinline[color=orange!25,#1]{\textbf{Ben:} \
#2}}
\newcommand{\todofd}[2][]{\todoinline[color=red!50,#1]{\textbf{Fred:} #2}}
\newcommand{\todoam}[2][]{\todoinline[color=orange!50,#1]{\textbf{Adrian:\
} #2}}
\newcommand{\todoec}[2][]{\todoinline[color=yellow!50,#1]{\textbf{Eddy:} #2}}

\newcommand{\todoinline}[2][]{\todo[inline,#1]{TODO: #2}}
\definecolor{blgreen}{rgb}{0.,0.85,0.7}
\newcommand{\todopages}[2][]{\todo[inline,color=blgreen,#1]{#2}}

\newcommand{\diet}{\textsc{Diet}\xspace}
\newcommand{\madag}{MA$_{DAG}$\xspace}
\newcommand{\grpc}{GridRPC\xspace}


\newcommand{\dietmaint}{\diet Maintainer\xspace}
\newcommand{\dietbranch}{\diet Branch Master\xspace}
\newcommand{\dietcontrib}{\diet Contributor\xspace}
\newcommand{\dietguest}{\diet Guest\xspace}
\newcommand{\dietpackage}{\diet Packager\xspace}
\newcommand{\dietversion}{2.8}

\begin{document}

\thispagestyle{empty}
\vspace*{3cm}
\vspace*{3cm}

\begin{center}
\includegraphics[scale=.5]{fig/logo_DIET_big}\\[2ex]
\textbf{\Huge USER'S MANUAL\\[2ex]}
\end{center}

\vfill

\noindent
\small{
\begin{tabular}{ll}
  \textbf{VERSION}  & \dietversion\\
  \textbf{DATE}     & February 2012\\
  \textbf{PROJECT MANAGER}  & Fr\'ed\'eric \textsc{Desprez}.\\

  \textbf{EDITORIAL STAFF}  & Yves \textsc{Caniou}, Eddy
  \textsc{Caron} and David ~\textsc{Loureiro}.\\

  \textbf{AUTHORS STAFF}    & 
\begin{minipage}[t]{12cm}
  Abdelkader \textsc{Amar}, Rapha\"el \textsc{Bolze}, \'Eric
  \textsc{Boix}, Yves \textsc{Caniou}, Eddy \textsc{Caron}, Pushpinder
  Kaur \textsc{Chouhan}, Philippe ~\textsc{Combes}, Sylvain
  \textsc{Dahan}, Holly \textsc{Dail}, Bruno \textsc{Delfabro}, Benjamin
  \textsc{Depardon}, Peter
  \textsc{Frauenkron}, Georg \textsc{Hoesch}, Benjamin \textsc{Isnard},
  Mathieu \textsc{Jan}, Jean-Yves \textsc{L'Excellent}, Ga�l \textsc{Le
    Mahec}, Christophe \textsc{Pera}, Cyrille \textsc{Pontvieux}, Alan
  \textsc{Su}, C\'edric \textsc{Tedeschi}, and Antoine
  \textsc{Vernois}.
\end{minipage} \\
  \textbf{Copyright}& INRIA, ENS-Lyon, UCBL, SysFera
\end{tabular}\\
}
\newpage
\thispagestyle{empty}
\ 
\newpage
\tableofcontents


\section{Introduction}

\section{Contributing to \diet}

\todoall{Honteusement pomp� sur
  http://wiki.openbravo.com/wiki/Projects/POS/Contributor\%27s\_Guide
  mais c'est pas mal d'avoir ce genre de motivation en entr�e de document}.

We truly believe that anyone can be personally benefit by contributing
to \diet project. Whether you're an independent consultant, student,
open source hacker, or even part of a company, contributing could be
great for you.
\begin{enumerate}
\item Contributions to \diet are maintained by the \dietmaint: If you
contribute your enhancements to \diet, they will be part of the product
and will be maintained, documented and tested as part of the \diet
software and all subsequent versions. This means that, after the
initial development investment, you will not need to invest in
upgrading these features anymore and that your upgrades to future
versions will be smoother. 

\item Free up Resources through Teamwork: Contributing to \diet and having
developments supported by us leaves you more time for your own
developments and installations. You develop, and let us take care of
maintaining the code during versions. For example, think of the big
changes occurring in each one of the versions. Whatever you put in our
\diet core is monitored by us for life.

\item Work with People with Common Goals: Whether by leveraging on
other's work or collaborating on a project, push development in your
own direction by leading it. The product will support what you want,
and when you want it. This also helps ensure that your priorities,
issues and features becomes the communities goals.

\item Gain recognition and exposure: Make yourself known in a community
of individuals and companies deploying professional open source
solutions.

\item Network: Work with and meet others just like yourself, while
potentially getting job or professional opportunities worldwide.

\item Learn and Apply Skills: Enhance your knowledge by working on open
technologies like C++, CORBA, Java, ... These are today's industry
standard tools for building professional software solutions.

\end{enumerate}

\section{Participate in the mailing list and help other user}

Three different mailing lists are available on the
website~\url{http://listes.ens-lyon.fr} :
\begin{description}
\item[diet-usr] Dedicated to the users of \diet. \url{https://listes.ens-lyon.fr/sympa/info/diet-usr}
\item[diet-dev]  Dedicated to the developers and contributor of
  \diet. \url{https://listes.ens-lyon.fr/sympa/info/diet-dev}
\item[diet-commits] Dedicated to the commits sent by Git.  Contributor
  s don't forget to add an explicit message when you commit
  something. \url{https://listes.ens-lyon.fr/sympa/info/diet-commits}
\end{description}

\section{Bugs reports}

Have you found a bug? Are you sure? Are you really sure? Then, please, report
it to us using the bugzilla system for DIET available at this URL:
\url{http://graal.ens-lyon.fr/bugzilla}. Even better, you can also
propose a bug fix for the issue.

\section{Contributing ideas and features requests}


Do you have an idea for DIET? 
\begin{itemize}
\item You can fill up a feature request in the bugzilla system as a
  feature (\url{http://graal.ens-lyon.fr/bugzilla}). Better way. 
\item You can also send it to the diet-dev mailing list. 
\end{itemize}

This process just communicates an idea to DIET that will
implement the idea at its discretion. Since DIET it is an open
source project, you can implement an idea into the product. See the
next sections.

\section{Create or improve documentation}

If you found a bug in the documentation or if you have a request or an
idea to submit about the documentation. Please fill a bug report in
the bugzilla system (http://graal.ens-lyon.fr/bugzilla) section
documentation. 

If you have created a significant piece of documentation regarding the
implementation, development or use of \diet. You can contact at least
one member of the \dietmaint. Also you can help improve the current
documentation after registering in our GIT . 

\todoall{Est-ce qu'on fait une branche documentation ?}

\section{Translate documentation}

\dietcontrib is authorized to modify and enhance the documentation,
but also every one can translate the documentation into their
language. Can you speak English and other language? Enjoy !

\section{Organization}

There are five status: 

\begin{description} 
\item \textbf{\dietmaint}. This group is composed of four persons: Eddy Caron
  (Eddy.Caron@ens-lyon.fr), Yves Caniou (Yves.Caniou@ens-lyon.fr),
  Benjamin Depardon (Benjamin.Depardon@sysfera.com) and Ha�kel Gu�mar
  (Haikel.Guemar@sysfera.com). The role of this group is to:
\begin{itemize}
  \item validate the creation of a new branch. 
  \item nominate the \dietbranch for each official branch.
  \item decide with the \dietbranch if a branch  can be merge to the
    master. 
    \item decide to publish a new release. 
\end{itemize}

\item \textbf{\dietbranch}. Each official DIET branch is lead by a
  \dietbranch. His role is to:
  \begin{itemize}
    \item control the quality of the code
    \item check the global  coherency of the \dietcontrib. 
      \item define the roadmap for his branch according to the
        \dietmaint.
        \item informs the \dietmaint group when a branch is ready to merge. 
\end{itemize}
\item \textbf{\dietcontrib}. All contributors a person who have an authorized access to
  write into a branch or into the master. His role is to:
  \begin{itemize}
    \item write a nice and amazing code. 
    \item work in collaboration with his \dietbranch. 
\end{itemize}
 
\item \textbf{\dietguest}. It's a contributor without a write GIT
  access. He can 
\begin{itemize}
  \item  send a patch to a \dietbranch or a member of the \dietmaint. 
    \item become an DIET developer as soon as possible
\end{itemize}

\item \textbf{\dietpackage} He does the merge to the master with the
  help of the \dietbranch. And he builds the package (tarball, Debian,
  Fedora, ...) when a new release is announced by the \dietmaint. 
\end{description}

\section{Becoming an DIET developer}

A DIET developer is an amazing person that is actively involved in
DIET development. For example, if you have a significant contribution
or write regularly enhancements for the product or planning to do it
soon.

\section{Obtaining Write Access to GIT branch}

DIET source code lives in GIT. The main features under development
exist under a GIT branch.  Everyone can read the source code but only
developers can write on it. You need written access permission to DIET
GIT to be able to push your code to the GIT.

\section{Regression tests}

\todobd{Des infos sur Jenkins ? Et comment ajouter de nouveau tests ou
o� trouver la doc pour le faire.}


\section{Coding Standard}

\todohg{Faire un lien sur le document de DIET Coding Standards fait
  par Ha�kel. Est-ce qu'on garde les 15 pages ? Je veux dire par l�
  que je vois bien une version pas trop rebuttante en version Lite
  pour un coder beginner puis rapidement si �a mort on lui passe le
  document en version FULL.}

\section{Packages for distributions and installers}

Can you help to prepare the necessary infrastructure to create
packages for different distributions and installers for different
platforms? We already build packages for Debian and Fedora. 

\section{Legal Aspects}

\todoec{Check avec Nicolas Jourdan en cours}

\bibliographystyle{plain} 
\end{document}
