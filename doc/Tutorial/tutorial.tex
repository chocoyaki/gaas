%****************************************************************************%
%* $Id$ *%
%* DIET tutorial                                                            *%
%*                                                                          *%
%*  Author(s):                                                              *%
%*    - Ludovic BERTSCH (Ludovic.Bertsch@ens-lyon.fr)                       *%
%*    - Philippe COMBES (Philippe.Combes@ens-lyon.fr)                       *%
%*                                                                          *%
%* $LICENSE$                                                                *%
%****************************************************************************%
%* $Log$
%* Revision 1.2  2003/01/31 13:08:07  pcombes
%* Apply Coding Standards
%*
%* Revision 1.1.1.1  2003/01/24 16:28:28  pcombes
%* DIET tutorial based on users's day examples
%****************************************************************************%


\documentclass[11pt,a4paper]{article}


\makeatletter
\def\input@path{{../utils}{../../../../publis/utils/styles/}}
\makeatother
\usepackage{fancyheadings}
\usepackage[french]{babel}
\usepackage[latin1]{inputenc}
\usepackage{palatino}
\usepackage{multicol}
\usepackage{verbatim}
%\usepackage{fullpage}
\usepackage[headings]{fullpage}
\usepackage{url}

\usepackage{graphicx}

\newsavebox{\logobox}
\sbox{\logobox}{\includegraphics[scale=0.3]{../UM/fig/logo_DIET.ps}}
\newcommand{\logo}{\usebox{\logobox}}

\renewcommand{\title}{Travaux Pratiques DIET}

\pagestyle{fancyplain}
\lhead[\fancyplain{\title}{\title}]
      {\fancyplain{\title}{\title}}
\chead{}
\rhead[\fancyplain{\logo}{\logo}]{\fancyplain{\logo}{\logo}}

\lfoot[\fancyplain{INRIA}{INRIA}]{\fancyplain{INRIA}{INRIA}}
\cfoot[\fancyplain{}{}]{\fancyplain{}{}}
\rfoot[\fancyplain{Page~\thepage}{Page~\thepage}]
      {\fancyplain{Page~\thepage}{Page~\thepage}}


\begin{document}

\begin{center}
{\Huge DIET Tutorial \par}
\end{center}



\section{Introduction}

This is a tutorial for DIET (Distributive Interactive Engineering Toolbox). It
comes with three directories: \texttt{exercise2} and \texttt{exercise3} contain
the skeletons for the programs you will have to write, \texttt{solutions}
contains the programs you should write.

To compile the programs, you will need to have omniORB 3 or 4 installed. Some
tips about omniORB installation are given in the DIET User's Manual, which will
be often refered to in this tutorial. However, if you still have problems to
install omniORB 4, please contact your system administrator.
In this document, we will refer to the directory were omniORB has been installed
with the environment variable \texttt{\$\{OMNIORB\_HOME\}}.



\section{Exercise 1: Installing and compiling DIET}

The installation process is described in the User's Manual. Install DIET in the
directory of your choice, ofr instance \texttt{\$\{HOME\}/DIET}. For this
tutorial, FAST will not be used, and as regards omniORB, please add the
\texttt{\$\{OMNIORB\_HOME\}/bin} directory in your \texttt{PATH}, or give the
option \texttt{--with-omniORB=\$\{OMNIORB\_HOME\}} to the \texttt{configure}
script.

Using the command line help, compile {\bf and} install DIET. Refer to the User's
Manual for more information.



\section{Exercise 2: An example of matrix calculation}

The goal of this example is the programming of a simple client/server
application that uses DIET. We will use the context of matrix calculations to 
make this program look more real. We will implement a basic scalar by matrix
product. Then, we will test this program in different schemes of execution.
\par

\subsection{Files skeletons}

The exercise2 directory, located in your home directory contains all the skeleton
files needed for a quick start. Useful pieces of software are also included
in them. This directory contains the following files:

\begin{description}
\item[Makefile]{management of dependencies between source and compiled files}
\item[server.c]{program implementing the service (scalar by matrix product)}
\item[client\_smprod.c]{program using the service defined in server.c: the
    matrix is stored in memory}
\item[client\_smprod\_file.c]{same program than client\_smprod\.c, except that
    the matrix is passed as a file to the server}
\end{description}

\subsection{Server-side implementation}

Using the skeleton of program \texttt{server.c}, write a service of scalar by
matrix product. This service will have the following parameters:

\begin{itemize}
\item{a scalar of type double}
\item{a matrix to be multiplied: all its values will be of type double}
\item{the duration needed for the product to compute: of type float}
\end{itemize}

The resulting values of the product will replace the initial values contained
in the matrix. The matrix will be stored in memory. \par

To start, try to define a detailed interface for the service, i.e. a precise
definition of the \emph{profile} of the service. To do so, look for {\bf in},
{\bf inout} and {\bf out} parameters. \par

Next, program the solve function \texttt{solve\_smprod}, and also the
initialization of the service in the \texttt{main} function. \par

\footnotesize
\begin{verbatim}
int solve_smprod(...) {
}
\end{verbatim}
\normalsize
\noindent
The following function is given to ease your work:
\footnotesize
\begin{verbatim}
int scal_mat_prod(double alpha, double *M, int nb_rows, int nb_cols, float *time)
\end{verbatim}
\normalsize
\noindent
It mutliplies the scalar \texttt{alpha} by the matrix \texttt{M}
(\texttt{nb\_rows}, \texttt{nb\_cols}), giving the duration of this operation in 
seconds.


\subsection{Client-side implementation}

Using the \texttt{client\_smprod.c} skeleton file, write a client for the service
defined above. You will need to initialize a matrix and a scalar with known
values. That way, you will be able to figure out if the answer is correct or
not. \par

You will have to remember that the profile used in the client must match 
exactly the one defined in the server.

\subsection{Setting up and testing the client/server}

The file \texttt{env\_vars.bash} contains all the environment variables needed
for the execution of the programs. Please pay attention to the fact that this is
\textbf{bash}: adapt it to your default shell. Verify the values of those
variables, then load this file using the following method:

\footnotesize
\begin{verbatim}
$ . env_vars.bash               # This is: [dot][SPACE]env_vars.bash
$
\end{verbatim}
\normalsize

When done with this operation, you need to start the name server of omniORB:
omniNames. To do that, you must give a port number with the \texttt{-start}
option, on which the service will be opened (and on which the server
``listens''):
\footnotesize
\begin{verbatim}
$ omniNames -start 2810

Tue Jan 13 14:05:28 2003:

Starting omniNames for the first time.
Wrote initial log file.
Read log file successfully.
Root context is
IOR:010000002b00000049444c3a6f6d672e6f72672f436f734e616d696e672f4e616d696e674
36f6e746578744578743a312e300000010000000000000060000000010102000d000000313430
2e37372e31332e36310000554f0b0000004e616d6553657276696365000200000000000000080
000000100000000545441010000001c0000000100000001000100010000000100010509010100
0100000009010100
Checkpointing Phase 1: Prepare.
Checkpointing Phase 2: Commit.
Checkpointing completed.
$
\end{verbatim}
\normalsize

Then, you have to copy this port number in the omniORB configuration file: the
name and location of this file is given by the environment variable 
\texttt{OMNIORB\_CONFIG} which is defined in the \texttt{env\_vars.bash} file.
\par

Using DIET User's Manual, prepare configuration files (suggestion: place
them in a \texttt{cfgs} directory). You will want to create a hierarchy of agents,
to make it interesting. This hierarchy will contain at least one MA and one LA.
\par

Compile server and client with the Makefile, then, launch the server and the
client. \par

Finally, launch several servers in different windows (that way you will see which
one is activated). Experiment with different hierarchies.


\subsection{Another version of the service}

In this part, you will modify the server to make it support a slightly different
version of the scalar by matrix product. The matrix will be transmitted as a
file, and not anymore in memory. \par

DIET doesn't impose anything about the data format of files, but it would be a
good idea to facilitate your work to use the data format used in the skeleton
files. This format is just simple text~: the file contains a serie of numbers,
separated by 'white' characters. The meaning of the numbers is as follows~:
\begin{itemize}
\item{matrix dimensions (number of rows, number of columns)}
\item{matrix values}
\end{itemize}

\par
Create a file containing a matrix, then implement a new service ``smprod\_file''
with the following parameters:
\begin{itemize}
\item{a scalar of type double}
\item{a file containing the matrix to be multiplied}
\item{the duration of the operation : of type float}
\end{itemize}

The result of the operation will be stored in the same file as the initial values.


\section{The BLAS dgemm}

To compile the programs of this exercise, the library BLAS (Basic Linear Algebric
Subroutines) is required.

The \texttt{dgemm\_} function is part of the BLAS. It performs the following matrix
computation ($:=$ symbolizes the affectation):

$$ C := \alpha A B + \beta C $$

$\alpha$ and $\beta$ are double precision reals, and $A$, $B$ and $C$ are
matrices of double precision reals.
The matrix dimensions are given through three signed integers $m$, $n$ et $k$.
\begin{itemize}
\item{$A$ is ($m$,$k$)}
\item{$B$ is ($k$,$n$)}
\item{$C$ is ($m$,$n$)}
\end{itemize}

This exercise aims at adding a new service in a DIET platform, that performs the
\texttt{dgemm\_} computation. The idea is to interface the existing
\texttt{dgemm\_} function to a DIET SeD. Here is its prototype:
\begin{verbatim}
void dgemm_(char   *tA,
            char   *tB,
            int    *m,
            int    *n,
            int    *k,
            double *alpha,
            double *A,
            int    *lda,
            double *B,
            int    *ldb,
            double *beta,
            double *C,
            int    *ldc);
\end{verbatim}

All parameters are given by address. Parameters \texttt{alpha}, \texttt{beta},
\texttt{m}, \texttt{n}, \texttt{k}, \texttt{A}, \texttt{B} and \texttt{C}
correspond exactly to their respective roles in the computation. \texttt{lda},
\texttt{ldb} and \texttt{ldc} are the \emph{leading dimensions} of the
corresponding matrices. Since matrices are stored in a classical one-dimension
array, it is important to specify if they are stored by rows or by columns.
\texttt{*tA} and \texttt{*tB} are characters which have the following semantics:
\begin{center}
\begin{tabular}{|c|c|l|}\hline
tA  &  \multicolumn{2}{c|}{Storage order of A ($m$,$k$)}\\\hline
'T' &  row-major    & [row 1, row 2, ... , row $m$]\\
'N' &  column-major & [col 1, col 2, ... , col $k$] \\\hline
\end{tabular}
\end{center}

For this exercise, there is no need to explore all possibilities offered by the
storage order or the leading dimension. Just set \texttt{*tA} and \texttt{*tB}
to 'N', and \texttt{lda}, \texttt{ldb} and \texttt{ldc} to the number of rows of
the corresponding matrix.

Once you have specified the \emph{profile} of the service, program a server
that implements this service, and a test client, using the file skeletons in the
\textbf{exercise3} directory. Matrices will be stored in memory.
Eventually, test the client/server architecture, through DIET, in different
contexts of execution.

\end{document}
