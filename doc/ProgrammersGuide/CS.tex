%****************************************************************************%
%* DIET Coding Standards - chapter three of the Programmer's Guide          *%
%*                                                                          *%
%*  Author(s):                                                              *%
%*    - Philippe COMBES (Philippe.Combes@ens-lyon.fr)                       *%
%*                                                                          *%
%* $LICENSE$                                                                *%
%****************************************************************************%
%* $Id$
%* $Log$
%* Revision 1.6  2008/04/07 22:26:26  ecaron
%* Updated files to pdflatex compilation
%*
%* Revision 1.5  2007/11/22 21:07:02  mimbert
%* use latex package fancyhdr instead of fancyheadings which seems deprecated in some recent latex distributions
%*
%* Revision 1.4  2006/12/06 10:00:24  eboix
%*  Coding style comments added for CMakeLists.txt. This is just a reminder
%*  of existing rules, but it will authorize even further punishment for the
%*  ones breaking them. Caveat emptor... --- Injay2461 getting anal on this.
%*
%* Revision 1.3  2003/12/02 10:29:02  sdahan
%* update the doxygen part of the Programmer Guide
%*
%* Revision 1.2  2003/09/17 14:41:28  pcombes
%* Split the .tex according to its chapters.
%*
%* Revision 1.1  2003/09/09 12:42:44  pcombes
%* Reorganization of doc: include CS in a programmer's guide.
%*
%* Revision 1.9  2003/06/03 18:22:05  pcombes
%* License field is kept.
%*
%* Revision 1.8  2003/04/10 11:12:43  pcombes
%* Put the CVS Id field right above the CVS Log field.
%*
%* Revision 1.7  2003/02/07 15:32:49  pcombes
%* Update - noting new in the background
%*
%* Revision 1.3  2003/01/30 15:13:42  pcombes
%* New Coding Standards
%*
%* Revision 1.2  2003/01/13 12:09:00  pcombes
%* UM: client part complete for users's day ...
%*
%* Revision 1.1.1.1  2003/01/07 18:06:18  pcombes
%* Add Coding Standards for DIET.
%****************************************************************************%

\begin{comment}
\documentclass[11pt,a4paper]{article}
\makeatletter
%\makeatother
\usepackage{fancyhdr}
%\usepackage[french]{babel}
%\usepackage[latin1]{inputenc}
\usepackage{multicol}
\usepackage{verbatim}
\usepackage[headings]{fullpage}
\usepackage{url}

\usepackage{graphicx}
\graphicspath{{../UM/fig}}

\newsavebox{\logobox}
\sbox{\logobox}{\includegraphics[scale=0.3]{fig/logo_DIET}}
\newcommand{\logo}{\usebox{\logobox}}

%%%%
\renewcommand{\title}{DIET Coding Standards}
%%%%

\pagestyle{fancyplain}
\lhead[\fancyplain{\title}{\title}]
      {\fancyplain{\title}{\title}}
\chead{}
\rhead[\fancyplain{\logo}{\logo}]{\fancyplain{\logo}{\logo}}

\lfoot[\fancyplain{INRIA}{INRIA}]{\fancyplain{INRIA}{INRIA}}
\cfoot[\fancyplain{}{}]{\fancyplain{}{}}
\rfoot[\fancyplain{Page~\thepage}{Page~\thepage}]
      {\fancyplain{Page~\thepage}{Page~\thepage}}


\begin{document}
%\vspace*{1cm}
\thispagestyle{empty}
\begin {center}
\noindent\textbf{\LARGE{\title}}
\end{center}
\vspace*{1cm}
\end{comment}

In a continuous attempt to unify the structure and the format of the DIET source
code, we have collected a few guidelines here to which we ask you to adhere as
you add new source code or modify the existing code. We ask you to make every
effort to follow our guidelines before you submit your contributions to the
project. We also ask for your help in improving existing incompliant code as you
make modifications to it.


\section{File structure}
\label{fstruct}

\subsection{Headers}

Each file begins with the DIET header, where are precised:\\
\begin{tabular}{l}
- a short title (in one line)  that explains the role of this file (l. 3),\\
- the authors of the file,\\
- three fields that are automatically filled in.\\
\end{tabular}

As an example, here is the header of the \LaTeX\ source of this document, before
\texttt{\$Id\$} and \texttt{\$Log\$} fields have been filled in by CVS
operations:
\footnotesize
\begin{verbatim}
%****************************************************************************%
%* DIET Coding Standards                                                    *%
%*                                                                          *%
%*  Author(s):                                                              *%
%*    - Philippe COMBES (Philippe.Combes@ens-lyon.fr)                       *%
%*                                                                          *%
%* $LICENSE$                                                                *%
%****************************************************************************%
%* $@Id$
%* $@Log$
%****************************************************************************%
\end{verbatim}
\normalsize
% NB: in the verbatim above, the $@ is replaced by $ before the latex
%     compilation to produce the correct output. If we did not use such an
%     artefact, the cvs commits would fill in fields also ...

The CMakeLists.txt are not submitted to this header marking (since it was
considered that there is no technical value attached to them).


The fields \texttt{\$Id\$} and \texttt{\$Log\$} are automatically set at every
CVS commit on the file. The CVS log message is added to the \texttt{\$Log\$}
region, prefixed by the characters that preceeds \texttt{\$Log\$} on its line.
It thus keeps the history of all modifications performed on the file. But it may
increase drastically the size of the file with some useless information. So
please help us purging sometimes this Log part of all minor modifications (such
as bug fixes).

These CVS fields are removed by a shell script when a distribution is built. The
\texttt{\$LICENSE\$} line will also be processed automatically, o be replaced by
a short version of the DIET license, so please be very careful to put one and
only one space between the \texttt{\%*} and the \texttt{\$}, between
\texttt{\$Id\$} and the final \texttt{*\%}, between \texttt{\$LICENSE\$} and the
final \texttt{*\%}, and NO space after the last
significant character of all these lines.

\subsection{Defining a new class}

In a class declaration, the public part comes before the private one. Inner each
part, declare the field first, and all methods then.

The file in which the class is declared should be named
\texttt{$<$ClassName$>$.hh} and the one in which it is implemented should be named 
\texttt{$<$ClassName$>$.cc}.


\section{Naming}

Please follow the chapter 9 of the JAVA coding conventions.

\subsection{Classes}

Class names should be nouns, in mixed case with the first letter of each
internal word capitalized. Try to keep your class names simple and descriptive.
Use whole words-avoid acronyms and abbreviations (unless the abbreviation is
much more widely used than the long form, such as URL or HTML).
\footnotesize
\begin{verbatim}
class AgentImpl
{
  ...
\end{verbatim}
\normalsize


\subsection{Methods}

Methods should be verbs, in mixed case with the first letter lowercase, with the
first letter of each internal word capitalized.
\footnotesize
\begin{verbatim}
run();
addServices();
\end{verbatim}
\normalsize


\subsection{Member Variables}

Member variable names should be short yet meaningful. The choice of a variable
name should be mnemonic - that is, designed to indicate to the casual observer
the intent of its use.
\footnotesize
\begin{verbatim}
float myWidth;
\end{verbatim}
\normalsize


\subsection{Variables}

All instance, class, and class constants are in mixed case with a lowercase
first letter. Internal words start with capital letters.\\

Variable names should not start with underscore \texttt{\_} or dollar sign
\texttt{\$} characters, even though both are allowed. They can follow the
members naming conventions or be lowercase with words separated by underscores.
One-character variable names should be avoided except for temporary "throwaway"
variables. Common names for temporary variables are i, j, k, m, and n for
integers; c, d, and e for characters.
\footnotesize
\begin{verbatim}
int   i;
char* c;
\end{verbatim}
\normalsize


\subsection{Constants and Macros}

The names of variables declared class constants and of ANSI constants, the names
of macros and enum constants should be all uppercase with words separated by
underscores. (ANSI constants should be avoided, for ease of debugging.)
\footnotesize
\begin{verbatim}
#define TRACE_STRUCTURES 10
static int MIN_WIDTH = 4;
typedef enum {
  DIET_CHAR,
  ...
}
\end{verbatim}
\normalsize


\subsection{Types}

Type names must be suffixed with \texttt{\_t}: \texttt{diet\_type\_t}.



\section{Comments and Documentation}

Source code documentation is essential in an increasing group of developers
producing software collabortatively and efficiently.

\subsection{Source code}

Comments within the source code are one way to create a comprehensive
documentation. This insures that the related information is in the right place
and makes it easier for the software engineer to maintain it.

Inside DIET we use line comments to describe certain attributes, block comments
to explain bigger code sections and Doxygen comments to declare modules,
functions and other important parts of the software.

Line comments look like this:
\footnotesize
\begin{verbatim}
  int reqId;      /* Request ID */
or
  int reqId;      // Request ID
...
  } else { /* At least one offered service matches the problem */
or
  } else { // At least one offered service matches the problem
\end{verbatim}
\footnotesize

They are at the end of a line after the code or in the line before it just
like block comments as shown below:
\footnotesize
\begin{verbatim}
    /* Add the new log to the list */
    requestsLog->logMutex.lock();
    requestsLog->append(newLog);
    requestsLog->logMutex.unlock();
\end{verbatim}
\normalsize
\noindent \textbf{NB:} All comments are written \texttt{in english} !!!\\

It can be introduced a hierarchy in comments by the use of various symbols. For
instance, some comments can be used to simply separate the file into main
sections that group functions or methods together:
\footnotesize
\begin{verbatim}
  /****************************************************************************/
  /* Private methods                                                          */
  /****************************************************************************/
or
  /****************************************************************************/
  /* Data structure marshalling                                               */
  /****************************************************************************/
\end{verbatim}
\normalsize

Each method or function must be commented with Doxygen comments (see below)
Inside functions or methods, you can use '//' to mark that the comment is less
important than '/* ... */' comments.
\\


Eventually, please comment even your \verb+#if+ an \verb+#ifdef+ conditions: add
the conditional expression as a comment on the corresponding \verb+#else+ and
\verb+#endif+ lines:
\footnotesize
\begin{verbatim}
#if HAVE_FAST
  ...
#else  // HAVE_FAST
  ...
#endif // HAVE_FAST
\end{verbatim}
\normalsize

\subsubsection{Doxygen}

\textit{Doxygen} is a documentation system very similar to javadoc. It analyses
the header files of your code and build a pretty and easy to use documentation
from it. You can/must add some comments into your header file to describe the
interface of your classes. If the comment start by \texttt{/**} instead of
\texttt{/*}, \textit{Doxygen} takes it as the description of your class or
method. The documentation of Doxygen can be found at
\url{http://www.stack.nl/~dimitri/doxygen/docblocks.html}.

The most important things that must be put in the documentation is the
hypothesis made by our methods. If your class can accept several modifications
in the same time, you put ``this class is thread safe'' in the
documentation. If a methods accept a \texttt{char*} as a parameter but not the
\texttt{NULL} value, the documentation must say it too. It is useless to put
tons and tons of documentation, but nothing must stay undefined and an example
of use can be useful. This is really important when you describes your methods,
every hypothesis about your parameters and the state of the object must be
defined.

The easiest way to learn how to use \textit{Doxygen} is to look at a header
with \textit{Doxygen} comments like the \texttt{src/utils/Counter.hh} file. To
build the documentation run \texttt{doxygen doc/Doxyfile} in the source
directory of diet. The API directory will be created with all the
documentations inside. There is one rule that you must remember : always put
your comment just above your prototype or it will be ignored by
\textit{Doxygen}.

As there are C files in the DIET sources, the flag \texttt{@author} cannot be
used in all DIET source files. Thus, it is important to identify the author
of the file \textbf{in the header}, as explained section \ref{fstruct}.  Using
the flag \texttt{@author} would be redundant.

One last thing, do not put some \texttt{//FIXME blablabla} but use \texttt{/**
@warning blabla */}, \texttt{/** @todo blabla */} and \texttt{/** @bug blabla
*/} directly in the C code. A report will be generated with all the todo,
warning and bugs. It will be easier to know what must be done.

\subsection{Manuals}

As soon as modifications visible from the user's point of view are performed,
they have to be commented in the \textit{User's Manual}. These include changes
in the installation process, in the API or still in the general behaviour of the
platform, the new features that a programmer could add, etc. The \textit{Users's
  Manual} should always be up-to-date.
\\

It is also very important for the developers to maintain this
\textit{Programmer's Guide}, essentially the first two chapters of course.
Indeed, these chapters expose the architecture of the software, its various
parts and their links: this could evoluate a lot with the future developments.
The \textit{Programmer's Guide} is not redundant with the comments put into the
source code, since these comments cannot give a general view of the software,
and some parts of the code require so many comments that it is better to put
them in this \textit{Programmer's Guide}, with a simple reference in the
source code. Moreover, it is a \textit{guide}, which means it should help a
newbie programmer finding quickly what changes to do and where, before having
read the least source code line.


\section{Tabs, Line Length and Indentation}

Please do not use tabs inside of the code and wrap lines after the 80th
character in a line. We do not like tabs because different editors and editor
settings always lead to different displays. However, it is permitted to insert
tabs if and if only they correspond to 8 charcters in the indentation (as for
GNU-Emacs standards).

Lines have to be indented by two spaces for each code block level.  See more
details below, especially in section \ref{s:statements}.

Due to cmake's syntax design choices \texttt{CMakeLists.txt} tend to
quickly feed the 80 characters width of a line. For \texttt{CMakeLists.txt}
files it was thus chosen to use a 2 or 3 characters indentation width.
If we strictly conform to the above limited-tabs recommendation,
\texttt{CMakeLists.txt} should thus remain tab free.

\section{Declarations}


\subsection{Variables}

Initialize variables at declaration. It is accepted not to initialize
"throwaway" variables, and to declare several of them on one single line.

The asterisk of a pointer declaration and the \texttt{\&} of a reference
declaration are parts of the \textbf{type} name. The space has to be put in
between the \texttt{*} or the \texttt{\&} and the type name.
\footnotesize
\begin{verbatim}

\end{verbatim}
\normalsize

\subsection{Functions}

The format of a function prototype should be equivalent with the function
definition. Return type and keywords such as extern, static, export, explicit,
inline, etc, go on the line above the function name. They should start with the
qualifier and return type. There is no space between the function name and the
open parentheses. If the arguments do not fit into one line split the line and
continue the arguments under the first one of the previous line.

\footnotesize
\begin{verbatim}
CORBA::Long
AgentImpl::serverSubscribe(SeD_ptr me, const char* hostname,
                           const SeqCorbaProfileDesc_t& services)
{
�� ...
or
int
unmrsh_in_args_to_profile(diet_profile_t* dest, corba_profile_t* src,
                          const diet_convertor_t* cvt)
{
�� ...
\end{verbatim}
\normalsize

\noindent In the function definition, the brace has to be in column zero again.
\footnotesize
\begin{verbatim}
int
mrsh_profile_to_out_args(corba_profile_t* dest, const diet_profile_t* src,
                         const diet_convertor_t* cvt)
{
  int i, res;
  int arg_idx(0);
  int* args_filled = new int[(dest->last_out + 1)];
  diet_data_t dd;
� ...
\end{verbatim}
\normalsize


\section{Statements}
\label{s:statements}

Statements inside the function body should be indented with 2 spaces like
statements inside a compound statement.  In contrast to functions, the opening
left brace of a compound statement should be at the end of the line beginning
the compound statement and the closing right brace should be alone on a line.
Some examples:\\

\footnotesize
\begin{center}
 \begin{tabular}{l|l}
  \begin{minipage}[c]{.5\linewidth}
   \begin{verbatim}
  for (int i = 0; i < max_nb_s; i++) {
    ...
  }

  do {
��    ...
  } while (!found);



  if (i == 1) {
��    ...
  } else if (i == -5) {
��    ...
  } else {
��    one_function_call();
  }

   \end{verbatim}
  \end{minipage}
&
  \begin{minipage}[c]{.5\linewidth}
   \begin{verbatim}
  while (1) {
��    ...
  }
  
  switch (i) {
  case 1,2:
��    ...
��    break;
  case 3:
��    ...
��    break;
  default:
��    ...
  }

  typedef struct name_s {
��    char *first;
  } name_t
   \end{verbatim}
  \end{minipage}\\
 \end{tabular}
\end{center}

\textrm
\normalsize

There is one space between the keyword and the open parentheses. Braces are in
the same line like the keywords. They should be used even if it is not
necessary. Function calls do not contain spaces between the function name and
the open parentheses.

\section{Clean use of C++ constructs}

Past experience leads us to some best-practice coding advices:
\begin{itemize}
\item Huge parameter lists or function or methods bodies are an indicator that
  it is time to split a function or method into some smaller ones which are
  easier to read and understand.
\item Try to avoid nested expressions where multiple assignments or the
  $?$-operators are envolved.
\item Use \verb+#if 0+ to disable code sections and add a short comment why this
  code was disabled.
%\item Avoid casting a CORBA type variable with a C/C++ type.
\end{itemize}

%\end{document}

