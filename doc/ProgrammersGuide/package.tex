%****************************************************************************%
%* DIET Programmer' guide, Package things                                   *%
%*                                                                          *%
%*  Author(s):                                                              *%
%*    - Eddy Caron (Eddy.Caron@ens-lyon.fr)                                 *%
%*    - Philippe Combes (Philippe.Combes@ens-lyon.fr)                       *%
%*                                                                          *%
%* $LICENSE$                                                                *%
%****************************************************************************%
%* $Id$
%* $Log$
%* Revision 1.1  2009/08/28 15:20:13  ecaron
%* Add information to generate the DIET tarball
%*
%****************************************************************************%

\section{DIET tarball generated}

The \url{bin/scripts} directory contains utilities for the DIET packaging. Here are some notes for the packager:

\begin{itemize} 
\item obtain the last cvs version (called here as DIET\_CVS)
\item Build the documentation with cmake under Bin:
\begin{verbatim}
        cmake ...
        make
\end{verbatim}
\item copy the resulting documentations (and included PostScript files)
\begin{verbatim}
   tar cvf DietDocs.tar ./doc/UsersManual/UsersManual.pdf 
   ./doc/ProgrammersGuide/ProgrammersGuide.pdf 
   ./doc/UsersManual/fig/*.ps      
   ./doc/UsersManual/fig/*.eps     
   ./doc/ProgrammersGuide/fig/*.ps 
   ./doc/ProgrammersGuide/fig/*.eps
   cp DietDocs.tar $DIET_CVS
   cd $DIET_CVS
   tar xvf DietDocs.tar
\end{verbatim}

\item Invoke the packaging script from DIET\_CVS:
\begin{verbatim}
     ./bin/scripts/make_dist.pl
\end{verbatim}
\item The result is placed under the cwd with name:
\begin{verbatim}
   diet-<major>.<minor>.tar.gz (e.g. diet-2.2.tar.gz)
\end{verbatim}
\end{itemize} 

\section{Files selected}

In the Distribution\_files.lst file, there are 4 types of section:

\begin{description}
\item{[Templated]:} Files for all kinds of distribution, with a template header which is to be processed by distrib\_file.sh
\item{[Untemplated]:} File with no template header but to be added in all kinds of distribution.                         
\item{[Devel\_Templated]:} Files with template header for maintainer distributions only.                                            
\item{[Devel\_Untemplated]:} Files with no template header for maintainer distributions only.       
\end{description}
