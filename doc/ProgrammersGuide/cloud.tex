%****************************************************************************%
%* diet Programmer's guide, Cloud submission                                *%
%*                                                                          *%
%*  Author(s):                                                              *%
%*    - Adrian Muresan (adrian.muresan@ens-lyon.fr)                         *%
%*                                                                          *%
%* $LICENSE$                                                                *%
%****************************************************************************%
%* $Id$
%* $Log$
%* Revision 1.1  2010/07/07 15:10:51  amuresan
%* Added Cloud entry for the UsersGuide and ProgrammersGuide.
%*
%****************************************************************************%

The current chapter details the conceptual and implementation details of \diet's Cloud
component. It contains details about the design of the component, the cloud interface
that was used and the API exposed to the \diet programmer.

\section{Objectives}

The goal of the Cloud component is to allow \diet services to use an Amazon EC2 compatible
Cloud platform for on-demand resource provisioning.

\section{Implementation}

Given the goals, the easiest way to use a Cloud platform in \diet is to consider it
a new type of batch system. \diet is easy to extend in this field and all that is needed
is an interface to the Cloud provider and a new implementation for the \textbf{BatchSystem}
abstract class.

\subsection{Eucalyptus SOAP interface}

\textsc{Eucalyptus} has been used as the cloud provider during the development process. It has
been chosen because of its open-source nature and its compatibility with the Amazon EC2 interface.
Managing Virtual Machines in \textsc{Eucalyptus}\footnote{\url{http://open.eucalyptus.com/}}
is done via its web service interface. During
the development, the implemented version of the EC2 interface is 2009-08-15. This corresponds to
version 1.6 of \textsc{Eucalyptus}.

In order to generate a C stub for the web service interface, the gSOAP\footnote{\url{http://www.cs.fsu.edu/~engelen/soap.html}}
package has been used. This automatically generates the interface. The resulting files
are placed in the \verb!<diet_src>/src/util/EucaLib! directory. Please note that the WSSE
plugin for gSOAP should also be installed. This enables Web service security.

Generating the SOAP stub is done in two steps:
\begin{enumerate}
\item{Generate the intermediary header file} - this is necessary for gSOAP:

\verb!wsdl2h -Nec2 -c -o euca.h -t WS-typemap.dat ec2.2008-12-01.wsdl!

In the above command, \textbf{ec2.2008-12-01.wsdl} is a WSDL file describing the web service
interface of the Cloud platform and S-typemap.dat contains type definitions that wsdl2h uses
to parse the wsdl and are required to enable ws-security. The \textbf{-Nec2} option creates
a friendly name (\textbf{ec2}) for the generated structures and functions.

\textbf{Note:} it is necessary to make sure that the generated .h file contains an '#import "wsse.h"'
directive somewhere at the beginning of its content. The generated .h files from ec2 wsdl files do not
contain this directive by default and this causes errors later on. If the generated .h does not contain
the directive, then it should be manually added: \verb!#import "wsse.h"!. One must pay attention as this
statement is an \textbf{import} which is internally used by gSOAP in the second phase and not a C/C++
\textbf{include} statement.
\item{Generate the stub} with a pure C output and client-side only (the server side stub is not needed):

\verb!soapcpp2 -I import -c -C euca.h!

In the above statement, \textbf{-I import} must specify the directory that contains the Web Service Security plugin, \textbf{wsse.h}, which
is used internally by gSOAP.
\end{enumerate}

\textbf{Note:} when including the generated files in a compilation, linking should also be done agains
\textbf{libssl} and \textbf{libcrypto}.

\subsection{Eucalyptus Batch System}



\section{Installation}

Please refer to the users manual.


