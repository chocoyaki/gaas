%**
%*  @file  cloud.tex
%*  @brief  DIET Programmer' guide, Cloud submission
%*  @author  Adrian Muresan (adrian.muresan@ens-lyon.fr) 
%*  @section Licence 
%*    |LICENSE|

The current chapter details the conceptual and implementation details of \diet's Cloud
component. It contains details about the design of the component, the cloud interface
that was used and the API exposed to the \diet programmer.

\section{Objectives}

The goal of the Cloud component is to allow \diet services to use an Amazon EC2 compatible
Cloud platform for on-demand resource provisioning.

\section{Implementation}

Given the goals, the easiest way to use a Cloud platform in \diet is to consider it
a new type of batch system. \diet is easy to extend in this field and all that is needed
is an interface to the Cloud provider and a new implementation for the \textbf{BatchSystem}
abstract class.

\subsection{Eucalyptus SOAP interface}

\textsc{Eucalyptus} has been used as the cloud provider during the development process. It has
been chosen because of its open-source nature and its compatibility with the Amazon EC2 interface.
Managing Virtual Machines in \textsc{Eucalyptus}\footnote{\url{http://open.eucalyptus.com/}}
is done via its web service interface. During
the development, the implemented version of the EC2 interface is 2009-08-15. This corresponds to
version 1.6 of \textsc{Eucalyptus}.

In order to generate a C stub for the web service interface, the gSOAP\footnote{\url{http://www.cs.fsu.edu/~engelen/soap.html}}
package has been used. This automatically generates the interface. The resulting files
are placed in the \verb!<diet_src>/src/util/EucaLib! directory. Please note that the WSSE
plugin for gSOAP should also be installed. This enables Web service security.

Please note that \diet Cloud has been tested with gSOAP version 2.7.16.

Generating the SOAP stub is done in two steps:
\begin{enumerate}
\item{Generate the intermediary header file} - this is necessary for gSOAP:

\verb!wsdl2h -Nec2 -c -o euca.h -t WS-typemap.dat ec2.2008-12-01.wsdl!

In the above command, \textbf{ec2.2008-12-01.wsdl} is a WSDL file describing the web service
interface of the Cloud platform and S-typemap.dat contains type definitions that wsdl2h uses
to parse the wsdl and are required to enable ws-security. The \textbf{-Nec2} option creates
a friendly name (\textbf{ec2}) for the generated structures and functions.

\textbf{Note:} it is necessary to make sure that the generated .h file contains an '\#import "wsse.h"'
directive somewhere at the beginning of its content. The generated .h files from ec2 wsdl files do not
contain this directive by default and this causes errors later on. If the generated .h does not contain
the directive, then it should be manually added: \verb!#import "wsse.h"!. One must pay attention as this
statement is an \textbf{import} which is internally used by gSOAP in the second phase and not a C/C++
\textbf{include} statement.
\item{Generate the stub} with a pure C output and client-side only (the server side stub is not needed):

\verb!soapcpp2 -I import -c -C euca.h!

In the above statement, \textbf{-I import} must specify the directory that contains the Web Service Security plugin, \textbf{wsse.h}, which
is used internally by gSOAP. The resulting source file will contain structure definitions for the types
used by the SOAP interface and methods used for calling the desired web method.
\end{enumerate}

\textbf{Note:} when including the generated files in a compilation, linking should also be done agains
\textbf{libssl} and \textbf{libcrypto}.

Calling a method from the SOAP interface is done by going through the following steps:
\begin{enumerate}
\item Generating the SOAP message with the three security headers required by the EC2 interface
\footnote{\url{http://docs.amazonwebservices.com/AWSEC2/latest/DeveloperGuide/index.html?using-soap-api.html}}:
\begin{enumerate}
\item \textbf{Binary security token} contains the X.509 certificate encoded in base64
\item \textbf{Signature} contains an XML digital signature using a signature algorithm and digest method
\item \textbf{Timestamp} requests to Amazon EC2 are only valid for 5 minutes to prevent replay attacks
\end{enumerate}
\item Instantiating and filling in the structures corresponding to the request and reply of the methods that is to be invoked.
\item Performing the method invokation by calling its corresponding generated C method from the stub.
\item Using the information from the response structure passed to the invoked method.
\end{enumerate}

\subsection{Eucalyptus Batch System}

The Cloud component has been implemented as a \textbf{BatchSystem}. This has been done by subclassing
\textbf{BatchSystem} and implementing its virtual methods.

Running a service call is done in 3 steps:
\begin{enumerate}
\item \textbf{Obtaining the Virtual Machines} through a SOAP call to the corresponding method of the EC2
interface. Note that the VMs are not obtained instantly. Booting a VM takes time. The method returns
instantly and polling is performed until all the VMs have been booted and have an associated IP address.
To prevent infinite waiting, a maximum number of tries is performed.
\item \textbf{Running the service} script on the SeD machine. It has access to the instantiated VMs
inside the script via their IP addresses by using the \verb!DIET_CLOUD_VMS! meta-variable.
\item \textbf{Terminating the VMs} by running another SOAP request to the Cloud front-end coresponding
to the method responsible for termination.
\end{enumerate}

The configuration for a SeD Cloud is done normally through the configuration file. Details about
the configuration file and the valid options can be found in the user's manual.

\section{Installation}

Please refer to the user's manual.


