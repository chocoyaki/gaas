%****************************************************************************%
%* DIET User's FAQ main file                                                *%
%*                                                                          *%
%*  Author(s):                                                              *%
%*    - Philippe COMBES (Philippe.Combes@ens-lyon.fr)                       *%
%*                                                                          *%
%* $LICENSE$                                                                *%
%****************************************************************************%
%* $Id$
%* $Log$
%* Revision 1.4  2010/12/01 12:49:53  ecaron
%* bug fix for the DIET URL
%*
%* Revision 1.3  2010/10/29 01:47:19  bisnard
%* Added FAQ for error message
%*
%* Revision 1.2  2010/09/21 14:49:07  kcoulomb
%* Correctifs sur la FAQ de DIET
%*
%* Revision 1.1  2010/09/14 17:07:28  bdepardo
%* Users FAQ
%*
%****************************************************************************%

\documentclass[12pt,a4paper]{book}
%\makeatletter
%\makeatother
\usepackage{fancyhdr}
\usepackage[headings]{fullpage}

\usepackage[pdftex]{graphicx} % Pour l'insertion d'images
\DeclareGraphicsExtensions{.jpg,.mps,.pdf,.png} % Formats d'images

\usepackage[pdftex]{thumbpdf}      % Vignettes
\usepackage{xcolor} % required for colors by hyperef
\usepackage[pdftex,                %
    bookmarks         = true,%     % Signets
    bookmarksnumbered = true,%     % Signets numérotés
    pdfpagemode       = true,%     % Signets/vignettes fermés à l'ouverture
%   pdfpagemode       = Fullscreen,
    pdfstartview      = FitV,%     % La page prend toute la hauteur
    pdfpagelayout     = SinglePage,% Vue par page
    colorlinks        = true,%     % Liens en couleur
    linkbordercolor   = white, % Couleur de la boîte sur les liens normal 
    citebordercolor   = white, % Couleur de la boîte sur les citations 
    filebordercolor   = white, % Couleur sur la boîte sur les fichiers 
    urlbordercolor    = white, % Couleur sur la boîte sur les URL
    linkcolor         = cyan, % Liens internes
    urlcolor          = blue, %  % Couleur des liens externes
    pdfborder         = {0 0 0}%   % Style de bordure : ici, pas de bordure
    ]{hyperref}%                   % Utilisation de HyperTeX


\hypersetup{ % Modifiez la valeur des champs suivants
    pdfauthor   = {DIET Team},%
    pdftitle    = {DIET User's FAQ},%
    pdfsubject  = {FAQ},%
    pdfkeywords = {DIET, Grid-RPC},%
    pdfcreator  = {PDFLaTeX},%
    pdfproducer = {PDFLaTeX}}


%\usepackage[french]{babel}
%\usepackage[latin1]{inputenc}
%\usepackage{multicol}
\usepackage{verbatim}
\usepackage{url}
\usepackage{subfigure}
\usepackage{listings}
\usepackage{xspace}
\usepackage{calc}
\usepackage{marvosym} % More symbols
\graphicspath{{./fig}}

\usepackage{pifont} % for \ding{52} dagda

\newsavebox{\logobox}
\sbox{\logobox}{\includegraphics[scale=0.4]{fig/logo_DIET}}
\newcommand{\logo}{\usebox{\logobox}}

%%%%

\newcounter{rmq}[section]
\setcounter{rmq}{0}
\newenvironment{remarque}{\addtocounter{rmq}{1}\textbf{N.B. \thermq:}}{}

%%%%

%%%%
\renewcommand{\title}{DIET User's FAQ}
%%%%

\pagestyle{fancyplain}
\fancyhead[L]{\title}
% \lhead[\fancyplain{\title}{\title}]
%       {\fancyplain{\title}{\title}}
\chead{}
\rhead[\fancyplain{\logo}{\logo}]{\fancyplain{\logo}{\logo}}

\lfoot[\fancyplain{\scriptsize{\copyright} ~INRIA, ENS-Lyon, UCBL}{\scriptsize{\copyright} ~INRIA, ENS-Lyon, UCBL}]{\fancyplain{\scriptsize{\copyright} ~INRIA, ENS-Lyon, UCBL}{\scriptsize{\copyright} ~INRIA, ENS-Lyon, UCBL}}
\cfoot[\fancyplain{}{}]{\fancyplain{}{}}
\rfoot[\fancyplain{Page~\thepage}{Page~\thepage}]
      {\fancyplain{Page~\thepage}{Page~\thepage}}


\newcommand{\ie}{\emph{i.e.,} }
\newcommand{\eg}{\emph{e.g.,} }

\newcommand{\dietj}{{\sc Diet}$_J$\xspace}
\newcommand{\sedj}{{\textit SeD}$_J$\xspace}
\newcommand{\seddiet}{{\textit SeD}$_{\sc Diet}$\xspace}
\newcommand{\sedsdiet}{{\textit SeDs}$_{\sc Diet}$\xspace}
\newcommand{\MAj}{{\textit MA}$_J$\xspace}
\newcommand{\MAsj}{{\textit MAs}$_J$\xspace}
\newcommand{\MAdiet}{{\textit MA}$_{\sc Diet}$\xspace}
\newcommand{\MAsdiet}{{\textit MAs}$_{\sc Diet}$\xspace}
\newcommand{\LAj}{{\textit LA}$_J$\xspace}
\newcommand{\LAsj}{{\textit LAs}$_J$\xspace}
\newcommand{\LAdiet}{{\textit LA}$_{\sc Diet}$\xspace}
\newcommand{\LAsdiet}{{\textit LAs}$_{\sc Diet}$\xspace}
\newcommand{\clientj}{{\textit client}$_J$\xspace}
\newcommand{\clientsj}{{\textit clients}$_J$\xspace}
\newcommand{\clientdiet}{{\textit client}$_{\sc Diet}$\xspace}
\newcommand{\dagda}{{\sc Dagda}\xspace}
\newcommand{\diet}{{\sc Diet}\xspace}
\newcommand{\dietforwarder}{{\sc Diet}Forwarder\xspace}
\newcommand{\godiet}{{\sc GoDiet}\xspace}
\newcommand{\madag}{{\sc M$A_{DAG}$}\xspace}
\newcommand{\pse}{PSE\xspace}
\newcommand{\nes}{NES\xspace}
\newcommand{\ptop}{\textit{Peer-to-Peer}}
\newcommand{\red}{\textit{Red}}
\newcommand{\sci}{Scilab}
\newcommand{\scip}{Scilab$_{//}$}
\newcommand{\scalapack}{ScaLAPACK}
\newcommand{\sed}{\textit{SeD}\xspace}
\newcommand{\seds}{\textit{SeD}s\xspace}
\newcommand{\thread}{\textit{thread}}
\newcommand{\threads}{\textit{threads}}
\newcommand{\nsl}{NetSolve}
\newcommand{\fixme}[1]{\fbox{\textsl{{\bf #1}}}}
\newcommand{\dietversion}{2.5}

%%%%
% Document beginning
%%%%

\begin{document}

%%%%
% First sheet
%%%%

\thispagestyle{empty}
\vspace*{3cm}
\vspace*{3cm}

\begin{center}
\includegraphics[scale=.5]{fig/logo_DIET_big}\\[2ex]
\textbf{\Huge USER'S FAQ\\[2ex]}
\end{center}

\vfill

\noindent
\small{
\begin{tabular}{ll}
  \textbf{VERSION}  & \dietversion\\
  \textbf{DATE}     & September 2010\\
  \textbf{PROJECT MANAGER}  & Fr\'ed\'eric \textsc{Desprez}.\\
  \textbf{EDITORIAL STAFF}  & Yves \textsc{Caniou}, Eddy \textsc{Caron} and David ~\textsc{Loureiro}.\\
  \textbf{AUTHORS STAFF}    & 
\begin{minipage}[t]{12cm}
  Eug\`ene \textsc{Pamba Capo-Chichi}.
\end{minipage} \\
  \textbf{Copyright}& INRIA, ENS-Lyon, UCBL
\end{tabular}\\
}

\newpage
\thispagestyle{empty}
\ 

%%%%
% End of first sheet
%%%%


\newpage
\tableofcontents

\setlength{\columnseprule}{1pt}


\noindent The purpose of this document is to give answers to several simple questions in order to facilitate the familiarization with DIET for new users (newbies).\\

\noindent \textcolor{red}{DIET: What does it mean?}\\
DIET is an acronym of Distributed Interactive Engineering Toolbox.\\

\noindent \textcolor{red}{What is the purpose of DIET?}\\
The aim of the DIET project is to develop a set of tools to build computational servers. 
The Distributed Interactive Engineering Toolbox (DIET) project is focused on the development of scalable middleware with initial efforts focused on distributing the scheduling 
problem across multiple agents. It consists of a set of elements that can be used together to build applications using the Grid-RPC paradigm. 
This middleware is able to find an appropriate server according to the information given in the client's request.\\

\noindent \textcolor{red}{How to get/download DIET?}\\
DIET is available at: \url{http://graal.ens-lyon.fr/DIET}.\\
  
\noindent However, the graal members can directly download it from the Graal server using a CVS command. Before connecting to the Graal server, the new graal member has to generate a couple
of private/public keys by executing the ssh-keygen command.\\
The generated public key (and localized on the .ssh directory) has to be sent to the graal administrator: eddy.caron@ens-lyon.fr.\\
 
\noindent \textcolor{red}{How to install DIET?}\\
Before installing DIET, the user needs to install other essential packages which are: 

\begin{itemize}
 \item OMNIORB4 available on \url{http://www.yolinux.com/TUTORIALS/CORBA.html} and,
 \item ccmake on \url{http://www.cmake.org/cmake/help/install.html}.\\
\end{itemize}

\noindent After that DIET can be installed. For more information about installation, the user manual is available on the directory: diet-\dietversion/doc/UsersManual (Chapter II). The installation is also well described on the README in diet-\dietversion/Cmake.\\ 
  
\noindent \textcolor{red}{How to configure OMNIORB4?}\\
Set environment variables:
\begin{verbatim}
export OMNIORB4_DIR="OMNIORB4_install_directory"
export OMNIORB4_INCLUDE_DIR=$OMNIORB4_DIR/include

export OMNINAMES_LOGDIR="/var/omninames" for root 
export OMNINAMES_LOGDIR="/tmp" for non root user 
(For OMNINAMES_LOGDIR, the previous directories can be changed according to the user permission).
 
export PATH=$OMNIORB4_DIR/bin:$PATH
export OMNIORB_CONFIG=$OMNIORB4_DIR/etc/omniORB4.cfg 
\end{verbatim}

\noindent The OMNIORB4\_install\_directory is the path of the directory where OMNIORB4 is installed. It is also essential to export the omniORB4.cfg file:\\


\begin{verbatim}

#****************************************************************************#
# (*) Reference to the CORBA Name server, to which all agents register and
#     connect to get references on other agent
#****************************************************************************#
InitRef = NameService=corbaname::127.0.0.1:2809
supportBootstrapAgent = 1
\end{verbatim}

\noindent \textcolor{red}{What is a simple way to better understand how DIET works?}\\
The scalars example (on diet-\dietversion/dietbin/src/examples/scalars) is a very simple and interesting client/server example to understand how DIET works.\\

\noindent \textcolor{red}{How is the scalars example run?}\\
The first step is to run the CORBA naming service:\\
If you are launching CORBA for the first time, type:\\
$>$ omniNames -start\\
Otherwise you can directly launch\\
$>$ omniNames\\

\noindent This error can occur if use use the start option whereas you have already used the naming service:
\begin{verbatim}
root@capo-chichi:/home/paco/Desktop/DIET/FAQ# omniNames -start
Sat Sep 11 20:27:23 2010:
Error: log file '/var/omninames/omninames-capo-chichi.log' exists.  Can't use -start option.
\end{verbatim}

\noindent There exist two solutions to solve it:
\begin{enumerate}
 \item by executing : $>$ omniNames,

\begin{verbatim}
root@capo-chichi:/home/paco/Desktop/DIET/FAQ# omniNames
Sat Sep 11 20:27:26 2010:
Read log file successfully.
Root context is IOR:010000002b00000049444c3a6f6d672e6f72672f436f734e616d696e672f4e616d696e67436f6e746578744578743a312e300000010000000000000070000000010102000e0000003139322e3136382e312e31303400f90a0b0000004e616d6553657276696365000300000000000000080000000100000000545441010000001c000000010000000100010001000000010001050901010001000000090101000354544108000000ccf6844c010037fa
Checkpointing Phase 1: Prepare.
Checkpointing Phase 2: Commit.
Checkpointing completed.
\end{verbatim}

 \item by erasing the log files generated on OMNINAMES log directory and restarting the command omniNames.
\end{enumerate}

\noindent The second step is to run an agent by using the dietAgent command (diet-\dietversion/dietbin/src/examples/scalars):
$>$dietAgent MA.cfg\\
\noindent The last step is the execution of server and client programs:\\
$>$ ./server\footnote{Depending on the compilation, the programm may be called scalars\_server} Sed.cfg\\
$>$./client\footnote{Depending on the compilation, the programm may be called scalars\_client} client.cfg CADD where CADD is one of the service offer by the Sed.
\begin{verbatim}
/* This server offers all additions of two scalars:                  */
/*   - CADD = sum of two chars                                       */
/*   - BADD = sum of two bytes                                       */
/*   - IADD = sum of two ints                                        */
/*   - LADD = sum of two longs                                       */
/*   - FADD = sum of two floats                                      */
/*   - DADD = sum of two doubles                                     */
\end{verbatim}

\noindent This is one example of the 3 configurations files MA.cfg, SeD.cfg and client.cfg:

\begin{verbatim}
MA.cfg

/***MA configuration file****/
traceLevel = 1
agentType = DIET_MASTER_AGENT
name = MA0

SeD.cfg

/***SeD configuration file****/
traceLevel = 10
parentName = MAO

client.cfg

/***Client configuration file****/
traceLevel = 10
MAName = MA0
\end{verbatim}

\noindent For a better understanding of this example, the code sources of these programs (client.c and server.c) are available on diet-\dietversion/src/examples/scalars.\\

\noindent \textcolor{red}{What is GoDiet?}\\
GoDiet is a Java-based tool for automatic Diet deployment that manages configuration file creation, staging of files, launch of elements, monitoring and reporting on launch
success, and process cleanup when the Diet deployment is no longer needed.\\


\noindent \textcolor{red}{How to get/download GoDiet?} \\
\noindent Graal members have to simply do the following:
\begin{itemize}
\item export CVSROOT=":ext:login@graal.ens-lyon.fr:/home/CVS/graal"
\item cvs co -d GoDiet\_CVS GRAAL/devel/diet/diet-contrib/GoDIET\\
\end{itemize}

\noindent \textcolor{red}{How to install GoDiet?}

\noindent The installation of GoDiet needs to check other essential pakages such as: ant, libgcj and opensdk. GoDiet uses ssh localhost connection for running 
that is why it is important to check it:
\begin{verbatim}
paco@capo-chichi:~$ ssh localhost 
ssh: connect to host localhost port 22: Connection refused
\end{verbatim}

\noindent To solve it, the user has to do:
\begin{verbatim}
paco@capo-chichi:~$ sudo apt-get install ssh
[sudo] password for paco: 
Lecture des listes de paquets... Fait
Construction de l'arbre des dependances       
Lecture des informations d'état... Fait
Les paquets supplémentaires suivants seront installes : 
  openssh-server
Paquets suggeres :
  rssh molly-guard openssh-blacklist openssh-blacklist-extra
Les NOUVEAUX paquets suivants seront installes :
  openssh-server ssh
\end{verbatim}

\noindent And checking:
\begin{verbatim}
paco@capo-chichi:~$ ssh localhost
The authenticity of host 'localhost (127.0.0.1)' can't be established.
RSA key fingerprint is 4f:0f:e3:b3:55:d8:91:e4:81:1c:fc:2c:bf:34:da:72.
Are you sure you want to continue connecting (yes/no)? yes
Warning: Permanently added 'localhost' (RSA) to the list of known hosts.
paco@localhost's password: 
Linux capo-chichi 2.6.32-24-generic #42-Ubuntu SMP Fri Aug 20 14:24:04 UTC 2010 i686 GNU/Linux
Ubuntu 10.04.1 LTS
\end{verbatim}

\noindent After checking all things described previously, GoDIET-2.3.0.jar is generated by the command ant:

\begin{verbatim}
paco@capo-chichi:~/Desktop/DIET/Godiet/GoDiet_CVS$ ant 
Buildfile: build.xml
init:
idlj:
     [echo]  Generating stubs, helper classes and ties from IDL
     [echo]  Generating for tasks ...

package:

idlj_diet:
     [echo]  Generaxting stubs, helper classes and ties from
     [echo]     IDL
                 . 
                 . 
                 .
\end{verbatim}

\noindent \textcolor{red}{How is GoDiet run?}\\
\noindent The command for running GoDiet is: java -jar GoDIET-2.3.0.jar your\_xml\_file
\begin{verbatim}
paco@capo-chichi:~/DIET/Godiet/GoDiet_CVS$ java -jar GoDIET-2.3.0.jar examples/example1.xml
Parsing xml file: examples/example1.xml
\end{verbatim}

\noindent On the example1.xml file, you have to ckeck local paths (with the hdail login), the PATH and LD\_LIBRARY\_PATH:

\begin{verbatim}
<env>
		    <var name="PATH" value="/home/paco/DIET/diet-\dietversion/dietbin/src/examples/scalars:$PATH"/>
		    <var name="LD_LIBRARY_PATH" value="$LD_LIBRARY_PATH"/>
</env>
\end{verbatim}

\noindent After that, use the \textit {launch} command to start the example. For more, you have to follow instructions on the DIET UsersManual (Chapter 9, p. 91).

\noindent \textcolor{red}{What does mean the error message: "DIET ERROR: caught a CORBA exception (TRANSIENT) while submitting problem."?}\\
\noindent The reason is probably that you are using a different version of DIET on the client side and on the server (Master Agent or SeD) side. For example this happens if you use a DIET v2.5 client with a DIET v2.4 MA.

\end{document}

% http://www.gps-sdr.com/
