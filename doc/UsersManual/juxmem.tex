%****************************************************************************%
%* DIET User's Manual JuxMem use chapter file                               *%
%*                                                                          *%
%*  Author(s):                                                              *%
%*    - Mathieu Jan (Mathieu.Jan@irisa.fr)                                  *%
%*                                                                          *%
%* $LICENSE$                                                                *%
%****************************************************************************%

\chapter{JuxMem extension}
\label{ch:juxmem}

\section{Introduction}

With the release of version 2.0 of the DIET toolkit, we introduce the
ability to use JuxMem for managing persistent data blocks. This
section shortly describes how to use JuxMem inside DIET, as it is an
on going work.

\section{Overview of JuxMem}

JuxMem, stands for Juxtaposed Memory, implements the concept of data
sharing service for grid, based on a compromise between DSM systems
and P2P systems. JuxMem decouples data management from grid
computation, by providing location transparency as well as data
persistence in a dynamic environnement. JuxMem is based on the P2P
platform called JXTA, which stands for Juxtaposed. For more
information about JuxMem, please check the available documentation on
the web site of JuxMem~\cite{JuxMem}.

\section{How to configure DIET to use JuxMem?}

DIET currently needs JuxMem version 0.2 to work. This version can be
downloaded on the web site of JuxMem~\cite{JuxMem}. For configuring
and building JuxMem, please check the README file included in this
0.2 release of JuxMem. When the \texttt{--enable-JuxMem} option is
activated, you need to have JuxMem-C build, so please read the
documentation for building JuxMem-C. Currently, for configuring DIET
in order to use JuxMem you need to specify the build path of JuxMem,
JXTA-C and APR using respectively the \texttt{--with-juxmem},
\texttt{--with-jxta-c} and \texttt{--with-apr} options.
Note that APR (Apache Portable Runtime) is a requirement of both
JuxMem-C and JXTA-C.

When DIET is configured to use JuxMem, SeDs are able to store data
blocks inside JuxMem. Please be carefull as it does not mean that you
have a JuxMem platform deployed and usable!  In a first step, you must
deploy a JuxMem platform as described in the documentations of
JuxMem. This JuxMem platform is currently based on JuxMem-J2SE,
JuxMem-C is only used to play the role of a JuxMem client within a
DIET SeD. Please read the README file of JuxMem to build and deployed
a JuxMem platform.

\section{Example}

A simple example of the JuxMem usage inside DIET can be found in the
dmat\_manips sample. The name of the client is
\texttt{clientJuxMem}. This example stores DIET matrices inside JuxMem, 
and allows next computations to retrieve these matrices directly from
JuxMem. Clients therefore avoid unnecessary tranfers of matrices as
they only need to transfer the ID of the data returned by JuxMem. More
documentation and examples will be available in the future.

\section{Troubleshooting}

If you encounter any problem, you can try get help from the
JuxMem-discuss mailing list
\url{<juxmem-discuss@lists.gforge.inria.fr>}. Do not forget to include
in your e-mails the exact error message, your hardware description,
your OS name and version, and the JuxMem version number.  However,
please do understand that this is an on going work and therefore no
full support is provided.
