%****************************************************************************%
%* DIET User's Manual plugin scheduler chapter file                         *%
%*                                                                          *%
%*  Author(s):                                                              *%
%*    - Alan SU (Alan.SU@ens-lyon.fr)                                       *%
%*                                                                          *%
%* $LICENSE$                                                                *%
%****************************************************************************%

%%%%%%%%%%%%%%%%%%%%%%%%%%%%%%%%%%%%%%%%
%% \documentclass{article}

\newenvironment{code}
{\begin{list}{}{\setlength{\leftmargin}{1em}}\item\bfseries\tt}
{\end{list}}

\newenvironment{tinycode}
{\begin{list}{}{\setlength{\leftmargin}{1em}}\item\tiny\bfseries\tt}
{\end{list}}

%% \begin{document}
%%%%%%%%%%%%%%%%%%%%%%%%%%%%%%%%%%%%%%%%

\chapter{Plugin Scheduler Extension}
\label{ch:plugin}

\section{Introduction}

With the release of version 2.0 of the DIET toolkit, we introduce a
\emph{plugin scheduling} facility, designed to allow DIET service
developers to define application-specific performance measures and
to implement corresponding scheduling strategies.  This section
describes the interface to this facility.

\section{Default Scheduling}

The default scheduling strategy of the DIET toolkit is changed subtly
with the release of the plugin scheduler.  Aside from various bugfixes
in the low-level
performance measurement and comparison algorithms, the default
behavior of the scheduler has adapted according to the newly
redesigned framework.  In particular, the logical process used to
order SeDs by default has been changed to account for the
modifications in
the scheduling mechanisms.  The SeD selection process is effected by a
series of pairwise comparisons between
\emph{performance estimations} -- collections of data pertaining to
the capabilities of a particular server in the context of a particular
client request -- throughout the DIET agent hierarchy.  At the level
of a particular agent, we term this process
\emph{server response aggregation}.  The end result of the aggregation
phase is a list of server responses sorted according to the
aggregation method in effect.
By default, the aggregation phase
implements the following ordered sequence of tests:

\begin{enumerate}
\item \textbf{FAST/NWS data}: SeDs compiled with FAST~\cite{Qui02} and
  NWS~\cite{WSH99}
  configuration options enabled are capable of making dynamic
  performance estimates.  If such data
  were generated, these are the metrics on which DIET prefers to
  select servers.
\item \textbf{Round-robin}: In the absence of application- and
  platform-specific performance
  data, the DIET scheduler attempts to probabilistically achieve load
  balance by attempting to assign client requests on a round-robin
  basis.  Essentially each server records a timestamp indicating the
  last time at which it was assigned a job for execution.  Each time a
  request is recieved, the SeD computes the time elapsed since its
  last execution, and among the responses it recieves, DIET agents
  favor SeDs with a longer elapsed time.
\item \textbf{Random}: If for any reason, the SeD is unable to store
  timestamps, the DIET scheduler will chose randomly when
  comparing two otherwise equivalent SeD performance estimations.
\end{enumerate}

In the following sections, we describe the plugin scheduler, providing
the application developer the means to extend the notion of a
performance estimation to include metrics that are
application-specific, and to instruct DIET how to treat those data in
the aggregation phase.


\section{Plugin Scheduler Interface}

Distributed applications are varied and often exhibit performance
behavior specific to the domain from which they arise.  Consequently,
application-specific scheduling approaches are often necessary to
achieve high-performance execution.  We propose an extensible
framework to build
\emph{plugin schedulers}, enabling application developers to tune the
metrics and heuristics by which the DIET system deploys tasks.

The DIET scheduling subsystem is based on the notion that, for the
sake of system efficacy and scalability, the work of
determining the appropriate schedule for a parallel workload should be
distributed across the computational platform.  As each task in
such a parallel workload is submitted to the system for processing,
each Server Daemon (SeD) provides a
\emph{performance estimate} for that task.  These estimates are
propagated up the hierarchy and sorted in a manner that optimizes
certain performance criteria.  The structure of the
performance estimate is static and their interpretation and comparison
is fixed; the focus of the basic plugin scheduler is to enable
application developers to specify performance estimation metrics that
are tailored to their individual needs.

%% This section introduces the principal components of the basic plugin
%% scheduler framework.

\subsection{Estimation Metric Vector}\label{sect:estvector}

The new type \texttt{estVector\_t} is a structure that can manage a
dynamic collection of performance estimation values.  Logically, we
refer to such an object as an \emph{estimation vector}, containing
values that represent the performance profile provided by a
SeD in response to a DIET service request.  This collection of values
may contain either standard performance measures that are used in the
DIET system, or developer-defined values that are meaningful only in
the context of the application being developed.

\subsubsection{Standard Performance Metrics}

To allow users to use the existing default performance estimation
routines (as described in Section~\ref{sect:est_fn}), the following
functions are available to facilitate the construction of custom
performance estimation functions:
\begin{itemize}
\item\texttt{int diet\_estimate\_fast(estVector\_t ev, const diet\_profile\_t* const profilePtr)}:
  store both the FAST- and NWS-based performance estimation metrics
  (contingent on availability of these packages at compilation time)
  into the estimation metric vector \texttt{ev}
\item\texttt{int diet\_estimate\_lastexec(estVector\_t ev,
                           const diet\_profile\_t* const profilePtr)}:
  store the time elapsed since the last execution timestamp (to enable
  the round-robin scheduler) in the estimation metric vector \texttt{ev}
\end{itemize}

In the future, we plan to expand the suite of default estimation
metrics to include other performance estimation systems
(e.g.,~ganglia) and dynamic internal DIET system state information
(e.g.,~queue lengths).

\subsubsection{Developer-defined Performance Metrics}

Application developers may also define performance values to be
included in a SeD response to a client request.  For example, a DIET
SeD that provides a service to examine particular databases may need
to include information about which databases are currently resident in
its disk cache, in order that an appropriate server be identified for
each client request.  To store such values, the SeD developer should
utilize the following interface:
\begin{code}
int diet\_set\_user\_estimate(estVector\_t ev, int userTag, double value);
\end{code}
Such a call instructs DIET to store in the estimation vector
\texttt{ev} the value \texttt{value} associated with the identifier
\texttt{userTag}.


%% The interface for inspecting and
%% manipulating such structures is as follows:
%% \begin{itemize}
%% \item\texttt{estVector\_t new\_estVector()}:
%%   create a new estimation metric vector
%% \item\texttt{int estVector\_addEstimation(estVector\_t ev, diet\_est\_tag\_t tag, double val)}:
%%   add a new estimation (with metric identifier \texttt{tag} and value
%%   \texttt{val}) to the estimation metric vector \texttt{ev}
%% \item\texttt{int estVector\_setEstimation(estVector\_t ev, diet\_est\_tag\_t tag, double val)}:
%%   in the estimation metric vector \texttt{ev}, set the value
%%   associated with metric identifier \texttt{tag} to
%%   value \texttt{val} (to be created if no such tag yet exists, and to
%%   refer to the first such value if multiple instances of the
%%   identifier exist)
%% \item\texttt{int estVector\_numEstimationsByType(estVector\_t ev, diet\_est\_tag\_t tag)}:
%%   return a count of the number of values associated with identifier
%%   \texttt{tag} in the estimation metric vector \texttt{ev}
%% \item\texttt{int estVector\_numEstimations(estVector\_t ev)}:
%%   return the total number of values (regardless of identifier) in the
%%   estimation metric vector \texttt{ev}
%% \item\texttt{diet\_est\_tag\_t estVector\_getEstimationTagByIdx(estVector\_t ev, int idx)}:
%%   return the identifier of the \texttt{idx}$^{th}$ metric in the
%%   estimation metric vector \texttt{ev}
%% \item\texttt{double estVector\_getEstimationValueByIdx(estVector\_t ev,
%%                                          int idx,
%%                                          double errVal)}:
%%   return the \texttt{idx}$^{th}$ value in the estimation metric vector
%%   \texttt{ev} (returning \texttt{errVal} if no such value exists)
%% \item\texttt{double estVector\_getEstimationValue(estVector\_t ev,
%%                                     diet\_est\_tag\_t tag,
%%                                     double errVal)}:
%%   return the first value in the estimation metric vector
%%   \texttt{ev} associated with the identifier \texttt{tag} (returning
%%   \texttt{errVal} if no such value exists)
%% \item\texttt{double estVector\_getEstimationValueNum(estVector\_t ev,
%%                                        diet\_est\_tag\_t tag,
%%                                        double errVal,
%%                                        int idx)}:
%%   return the \texttt{idx}$^{th}$ value in the estimation metric vector
%%   \texttt{ev} associated with the identifier \texttt{tag}
%%   (returning \texttt{errVal} if no such value exists)
%% \item\texttt{void free\_estVector(estVector\_t ev)}:
%%   deallocate an estimation vector and all its stored metrics
%% \end{itemize}

\subsection{Estimation Function}\label{sect:est_fn}

The default behavior of a SeD when a task service request arrives from
its managing agent is to store the following information in the
request profile:
\begin{enumerate}
\item \textbf{FAST-based execution time predictions}: DIET SeDs
  attempt to call FAST
  routines to obtain execution time predictions based on the type of
  service requested, if FAST was available at compilation time.  If
  available, such predictions are stored in the
  performance estimate.
\item \textbf{NWS-based dynamic resource information}: If NWS library
  functions are available, performance estimates may include dynamic
  resource performance information about CPU availability, free
  memory, and network bandwidth.
\item \textbf{Last execution timestamp}: To implement the default
  round-robin behavior in absence of FAST and NWS facilities, each SeD
  stores a timestamp of the last time at which it started a solve
  request.
\end{enumerate}
This is accomplished by using the \texttt{diet\_estimate\_fast} and
\texttt{diet\_estimate\_lastexec} functions described in
Section~\ref{sect:estvector}.

To implement a plugin scheduler, we define an
interface that admits customizable performance estimation routines:
\begin{code}
  typedef estVector\_t (* diet\_perfmetric\_t)(diet\_profile\_t*);\\
  \\
  diet\_perfmetric\_t\\
  diet\_service\_use\_perfmetric(diet\_perfmetric\_t perfmetric\_fn);\\
\end{code}
Thus, the type \texttt{diet\_perfmetric\_t} is a function pointer
that returns a performance estimation (represented by the
\texttt{est\_vector\_t} object), given a DIET service request
profile.  Once such a function has been defined, the application
developer can instruct DIET to use it for all subsequent service
registrations with the
\texttt{diet\_service\_use\_perfmetric} interface.  This function
returns the previously registered performance metric function, if one
was defined (and
\texttt{NULL} otherwise).  This call remains in effect
(i.e.,~continues to modify the performance estimations for newly
registered services) until the
\texttt{diet\_service\_use\_perfmetric} interface is called with a new
performance metric function.

%% \subsection{Using Standard Metrics}

%% To allow users to use the existing default performance estimation
%% routines (as described in Section~\ref{sect:est_fn}), the following
%% functions are available to facilitate the construction of custom
%% performance estimation functions:
%% \begin{itemize}
%% \item\texttt{int diet\_estimate\_fast(estVector\_t ev, const diet\_profile\_t* const profilePtr)}:
%%   store both the FAST- and NWS-based performance estimation metrics
%%   (contingent on availability of these packages at compilation time)
%%   into the estimation metric vector \texttt{ev}
%% \item\texttt{int diet\_estimate\_lastexec(estVector\_t ev,
%%                            const diet\_profile\_t* const profilePtr)}:
%%   store the last execution timestamp (to enable the round-robin
%%   scheduler) in the estimation metric vector \texttt{ev}
%% \end{itemize}

%% In the future, we plan to expand the suite of default estimation
%% metrics to include other performance estimation systems
%% (e.g.,~ganglia) and dynamic internal DIET system state information
%% (e.g.,~queue lengths).

\subsection{Aggregation Methods}

At the time a DIET service is defined, an \emph{aggregation method}~--
the logical mechanism by which SeD responses are sorted~-- is
associated with the service.
By default, DIET agents will favor SeDs that supply
FAST- and/or NWS-based performance prediction data during the
aggregation phase.  If such data are
not available, DIET attempts to achieve load balance by sorting SeDs
using a round-robin or random approach.
If the estimation function is not overridden by the
application developer (as described in Section~\ref{sect:est_fn}),
each SeD will attempt to supply these data in response to a client
request.  We refer to this as the \emph{standard DIET scheduler}.

If application-specific data \emph{are} supplied (i.e.,~the
estimation function has been redefined), an alternative method for
aggregation is needed.  Currently, a basic
\emph{priority scheduler} has been implemented, enabling an
application developer to specify a series of performance values that
are to be optimized in succession.  A developer may implement a
priority scheduler using the following interface:
\begin{code}
diet\_aggregator\_desc\_t* \\
diet\_profile\_desc\_aggregator(diet\_profile\_desc\_t* profile); \\
\\
int \\
diet\_aggregator\_set\_type(diet\_aggregator\_desc\_t* agg, diet\_aggregator\_type\_t atype); \\
\\
int \\
diet\_aggregator\_priority\_max(diet\_aggregator\_desc\_t* agg, diet\_est\_tag\_t tag); \\
\\
int \\
diet\_aggregator\_priority\_min(diet\_aggregator\_desc\_t* agg, diet\_est\_tag\_t tag); \\
\\
int \\
diet\_aggregator\_priority\_maxuser(diet\_aggregator\_desc\_t* agg, int val); \\
\\
int \\
diet\_aggregator\_priority\_minuser(diet\_aggregator\_desc\_t* agg, int val); \\
\end{code}
The \texttt{diet\_profile\_desc\_aggregator} and
\texttt{diet\_aggregator\_set\_type} functions fetch and configure the
aggregator corresponding to a DIET service profile, respectively.
In particular, a priority scheduler is declared by using the
\texttt{DIET\_AGG\_PRIORITY} aggregator type to the latter function.
An agent's aggregation phase (effectively, the sorting of all server
responses from its children) will then use the specified tags to
perform the pairwise server comparisons needed to construct the final
result: a sorted list of server responses.

The four
\texttt{diet\_aggregator\_priority\_*} functions serve to identify the
estimation values to be optimzed during the aggregation phase.  The
\texttt{\_min} and \texttt{\_max} forms indicate that a standard
performance metric (e.g.,~time elapsed since last execution, supplied
by the \texttt{diet\_estimate\_lastexec} function) to be either
minimized or maximized, respectively.  Similarly, the
\texttt{\_minuser} and \texttt{\_maxuser} forms indicate the analogous
operations on user-supplied estimation values.  Calls to these
functions indicate the order of \textbf{precedence} of the tags.  Each
time two server responses need to be compared, the values associated
with the tags specified in the priority aggregator are retrieved.  In
the specified order, pairs or corresponding values are successively
compared, passing to the next tag only if the values for the current
tag are identical.  This process continues until one response is
declared superior to the other, or all tags in the priority aggregator
are exhaused (and the responses are judged equivalent).



%% \subsection{Generic Minimization/Maximization Metrics}
%% \label{sect:genl_min_max}

%% The basic plugin scheduler is based on the notion that each SeD can
%% estimate (in an application-specific manner) its
%% \emph{relative suitability}
%% in absence of information about other SeDs in the system.  Moreover,
%% such an estimate is assumed to be expressible on a numeric scale,
%% where the basic ``greater than'' and ``less than'' comparators
%% indicate the relative suitability for the given task.  Assuming this
%% is the case, DIET's basic plugin scheduling framework provides a means
%% of effecting a greedy task-by-task scheduling based on this metric
%% by simply providing a SeD-level performance estimation function.  This
%% is done with the following procedure:
%% \begin{enumerate}
%% \item Create a performance estimation function (with prototype
%%   conforming to the \texttt{diet\_perfmetric\_t} type).
%% \item Within this function,
%%   \begin{itemize}
%%   \item create an \texttt{estVector\_t} structure using the provided
%%     interface;
%%   \item calculate the relative suitability metric;
%%   \item store this value in the newly created estimation metric vector
%%     tagged with either the 
%%     \texttt{EST\_MINMETRIC} or
%%     \texttt{EST\_MAXMETRIC} identifier in the call to
%%     \texttt{estVector\_setEstimation()}; this tag indicates whether
%%     the suitability metric is a minimization metric (i.e.,~''lesser is
%%     better'') or a maximization metric (i.e.,~''greater is better''),
%%     respectively; and
%%   \item return the created estimation metric vector.
%%   \end{itemize}
%% \item Register this function as the desired performance estimation
%%   function using the \begin{code}diet\_service\_use\_perfmetric\end{code}
%%   interface, \emph{before} registering the DIET service.
%% \end{enumerate}
%% In this manner, SeDs will be chosen according to the optimization of
%% the metric represented by the value generated in the custom
%% performance estimation function.  In the next section, we discuss
%% methods to implement performance estimations that cannot be expressed
%% simply as a numerical value to be optimized.

%% \section{Enhanced Plugin Scheduler}

%% \subsection{User-defined Metrics}

%% \subsection{Application-specific Aggregation}

%% \subsection{Instantiating New Schedulers}

\section{Example}

A new example has been added to the DIET distribution, in the
directory
\begin{code}
src/examples/plugin\_example/
\end{code}
Provided are a DIET server and client, corresponding to a simulation
of a database research application.  In this context, one may use this
example and the corresponding code as a model for the usage of the
plugin scheduling facility.  If the construction of examples was
enabled during DIET configuration, two binaries \texttt{server} and
\texttt{client} will be built in this directory.  Having deployed a
DIET agent hierarchy, the server may be simply instantiated at various
sites in the platform:
\begin{code}
  \$ server <SeD\_config> <DB> [ <DB> ... ]
\end{code}
where \texttt{<DB>} are simply string that represent the existence of
a particular database at the SeD's site.  A client would pose a query
against a set of databases:
\begin{code}
  \$ client <client\_config> <DB> [ <DB> ... ]
\end{code}
The application uses the plugin scheduling facility to prioritize the
existence of databases in selecting a server, and thus, the expected
result is that one of the SeDs with the fewest number of database
mismatches will be selected.

In the \texttt{main} function of the \texttt{server.c} file, the
following block of code (i)~specifies the use of the priority
aggregator for this service, (ii)~declares a performance estimation
function to supply the necessary data at request-time, and
(iii)~defines the order of precedence of the performance values
(i.e.,~minimizing the number of database mismatches, and then
maximizing the elapsed execution time).
\begin{code}
  \{
    /* new section of the profile: aggregator */ \\
    diet\_aggregator\_desc\_t *agg; \\
    agg = diet\_profile\_desc\_aggregator(profile); \\
\\
    /* point the DIET server code to our custom performance function */ \\
    diet\_service\_use\_perfmetric(performanceFn); \\
\\
    /* for this service, use a priority scheduler */ \\
    diet\_aggregator\_set\_type(agg, DIET\_AGG\_PRIORITY); \\
    diet\_aggregator\_priority\_minuser(agg, 0); \\
    diet\_aggregator\_priority\_max(agg, EST\_TIMESINCELASTSOLVE); \\
  \}
\end{code}
The performance function \texttt{performanceFn} is thus defined:
\begin{code}
static estVector\_t \\
performanceFn(diet\_profile\_t* pb) \\
\{ \\
  const char *targetString; \\
  int numMismatch; \\
  estVector\_t perfValues = diet\_estimate\_new\_vector(); \\
\\
  /* string value must be fetched from description; value is NULL */ \\
  targetString = (diet\_paramstring\_get\_desc(diet\_parameter(pb, 0)))->param; \\
  numMismatch = computeMismatches(targetString); \\
\\
  /* store the mismatch value in the user estimate space */ \\
  diet\_set\_user\_estimate(perfValues, 0, numMismatch); \\
\\
  /* also store the timestamp since last execution */ \\
  diet\_estimate\_lastexec(perfValues, pb); \\
\\
  return (perfValues); \\
\}
\end{code}
The function \texttt{computeMismatches} (defined earlier in
\texttt{server.c}), calculates the number of requested databases that
are not present on the SeD making the evaluation.
Together, the \texttt{performanceFn} function and the block of code
added to \texttt{main} serve to customize the generation of
performance information and the treatment of these data in the context
of the simulated database search.
Finally, it should be noted that the client code is not affected by
the introduction of a plugin scheduler.

\section{Future Work}

We have two primary efforts planned for extensions to the plugin
scheduler.
\begin{itemize}
\item \textbf{Additional information services}: We plan to add
  functionality to enable to application developer to access and use
  data concerning the internal state of the DIET server (e.g.,~the
  current length of request queues).  As other performance measurement
  and evaluation tools are developed both within and external to the
  DIET project, we may also create access functions to enable such
  information to be incorporated in the context of the plugin
  scheduler.
\item \textbf{Enhanced aggregation methods}: The plugin scheduler
  implemented in the current release enables the DIET system to
  account for user-defined factors in the server selection process.
  However, the priority aggregation method is fairly rudimentary and
  lacks the power to express many imaginable comparison mechanisms.
  We plan to investigate methods to embed code into DIET agents
  (e.g.,~a simple expression interpreter) in a manner that is secure
  and that preserves performance.
\end{itemize}

%%%%%%%%%%%%%%%%%%%%%%%%%%%%%%%%%%%%%%%%
%% \end{document}
%%%%%%%%%%%%%%%%%%%%%%%%%%%%%%%%%%%%%%%%
