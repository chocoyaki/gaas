%**
%*  @file  cloud_ec2.tex
%*  @brief   diet Programmer's guide, Cloud submission 
%*  @author  - Adrian Muresan (adrian.muresan@ens-lyon.fr)
%*  @section Licence 
%*    |LICENSE|


\chapter{Cloud submissions}\label{chapter:cloudSubmission}
\section{Introduction}

The user must understand the relationship between \diet Cloud and the Cloud system it is using.
A Cloud system is seen as a provider of on-demand resources and as such, the \diet Cloud SeD
is the front-end to the Cloud system from the \diet user's point of view.

The \diet platform remains completely outside of the Cloud system and virtual resources are
instantiated based on how they are needed. These virtual resources contain the task that solves
the service required by the \diet client. It is the responsibility of the \diet service
creator to provide Virtual Machine images that contain the service implementation. 

The \diet Cloud system is part of the \diet Batch system and it is recommended that the user
reads the corresponding chapter before the current.

\section{Compatibility and requirements}

The current version of \diet Cloud has been built on top of the Amazon EC2 SOAP interface,
version 2009-08-15 and is compatible with any cloud system that offers a front-end with
this interface.

We have used our own installation of
\textsc{Eucalyptus}\footnote{\url{http://open.eucalyptus.com/}}  as
well as
\textsc{AmazonEC2}\footnote{\url{http://aws.amazon.com/fr/ec2/}} for
testing and development. The installation process of the cloud system
will not be detailed as it is outside the scope of the current
topic. For details on the installation and usage of the cloud system
please refer to your Cloud provider's documentation. The rest of this
chapter assumes the existing installation and correct functioning of a
Cloud system that the user can access.

To be able to reserve virtual resources on an Amazon EC2 compatible cloud system, the Cloud user
mush have knowledge of the following resources:
\begin{itemize}
\item \verb!The URL of the Cloud system!
\item \verb!Virtual machine images' names!: the user should know what virtual resources she wants
to instantiate. To be more precise, the user must know the names associated to the virtual machine
images, machine kernel images and machine ramdisk images that she wants to use.
\item \verb!Virtual machine instance type!: if the user requires a specific type of virtual machine
that is specialized with larger quantities of virtual CPUs, memory space or disk space offered.
An Amazon EC2 compatible Cloud system offers 5 different types of virtual machines that vary these
three quantities. The exact values are not standard. Eucalyptus provides the possibility of
per-installation configuration of these types.
\item \verb!The number of virtual resources to instantiate!
\item \verb!The X509 certificate that is associated to the Cloud installation!: this is provided by
the Cloud system after its installation and the user's registration in the system.
\item \verb!The user's key!: after registering in the Cloud system, each user receives a key
that is used for signing all the requests that she makes to the Cloud system.
\item \verb!A keypair to use for the reservation!: the user must create a keypair that is to be used
when performing reservations.
\end{itemize}


\section{Configuration for compilation}

In order to enable the compilation and installation of the \diet Cloud system, its corresponding make flag
must be enabled when configuring. This flag is named \verb!DIET_USE_CLOUD_EC2!. This is a part of
the alternate batch system so it also needs to be installed (the \verb!DIET_USE_ALT_BATCH! flag). The \diet
batch system is dependent on CORI so it also needs to be installed (the \verb!DIET_USE_CORI! flag).

The installation of the Cloud system is dependent on an installation of gSOAP\footnote{\url{http://www.cs.fsu.edu/~engelen/soap.html}}
that contains the WSSE plugin.
This is the Web Service Security plugin. It is the responsibility of the \diet user to ensure a propper
installation of gSOAP and its WSSE plugin. The will try to automatically detect an installation of gSOAP
with its components. If this fails, then the \diet user will have to manually specify the installation
path of the gSOAP package and its WSSE plugin.

Please note that \diet Clour has been tested with gSOAP version 2.7.16.

Examples of the use of the \diet Cloud system will also be compiled if the \verb!DIET_BUILD_EXAMPLES! flag
is triggered.

For more information on compiling and installing \diet please see the section dedicated specifically to that purpose.


\section{Server configuration file}

At the time this documentation has been written, Eucalyptus and Amazon
EC2 supported 5 different types of Virtual Machines to
instantiate. These differ from one-another by CPU, memory and disk
characteristics. Eucalyptus allows the modification of these
characteristics. The 5 VM types and their associated default
characteristics inside \textbf{Eucalyptus} are:
\begin{enumerate}
\item{m1.small}: 1 VCPU, 128 MB RAM, 2 GB Disk
\item{c1.medium}: 1 VCPU, 256 MB RAM, 5 GB Disk 
\item{m1.large}: 2 VCPU, 512 MB RAM, 10 GB Disk 
\item{m1.xlarge}: 2 VCPU, 1024 MB RAM, 20 GB Disk 
\item{c1.xlarge}: 4 VCPU, 2048 MB RAM, 20 GB Disk 
\end{enumerate}

\noindent
Configuring the server part is done by using the following options:
\begin{itemize}
\item \textsc{batchName}: the name of the batch system to be used. This is recognized internally by the
\diet Batch System and should always have the value \textbf{eucalyptus}.
\item \textsc{pathToNFS}: path to a temporary directory
\item \textsc{cloudURL}: the URL of the Cloud System's management service
\item \textsc{emiName}: the name of the Virtual Machine Image containing the service
\item \textsc{eriName}: the name of the Ramdisk image to be used when instantiating the service
\item \textsc{ekiName}: the name of the Kernel image to be used when instantiating the service
\item \textsc{vmType}: the type of the Virtual Machine to be instantiated
\item \textsc{keyName}: the name of the keypair created for use in the Cloud System
\item \textsc{vmMinCount}: minimum number of VM instances that is acceptable for the current service call to succede
\item \textsc{vmMaxCount}: maximum number of VM instances for the current service call
\item \textsc{pathToCert}: path to the X.509 certificate given by the Cloud System
\item \textsc{pathToPK}: path to the private key that the current Cloud user has associated to the Cloud System
\end{itemize}

An example of a valid configuration is the following:
\begin{lstlisting}[language=bash,basewidth={.5em,.4em},fontadjust]
batchName = eucalyptus
pathToNFS = /tmp
cloudURL = http://140.77.13.186:8773/services/Eucalyptus
emiName = emi-82011336
eriName = eri-0ADF1561
ekiName = eki-CBC01482 
vmType = m1.small
keyName = mykey
vmMinCount = 1
vmMaxCount = 1
pathToCert = /home/adi/.euca/euca2-admin-0b298100-cert.pem
pathToPK = /home/adi/.euca/euca2-admin-0b298100-pk.pem
\end{lstlisting}

\section{Registering the service}

\diet's Cloud module behaves like any other Batch System. A \sed programed this way
is actually a service residing at the front-end of a Cloud. It will be able to access
directly all the Virtual Machines that the Cloud has instantiated.
Defining and registering the Cloud-enabled \diet service is done in the same way as
defining a batch service. \diet's Batch System module will internally invoke the
\textbf{Eucalyptus} Batch System which will instantiate the Virtual Resources
corresponding to the current service.

\section{Service API}

Programming the \sed Cloud is done in the same way as programming normal Batch System.
The programmer builds a shell script that will execute his job. To facilitate access
to the instantiated Virtual Machines, the \verb!DIET_CLOUD_VMS! meta-variable can
be used inside the shell script. This will automatically be replaced upon submission
with a list of the IP addresses of all the instantiated virtual resources.

\section{Example of client/server}

The \diet sources also contain working examples for the \sed Cloud. These can be
found in: \verb!<diet_src>/src/examples/cloud!. The examples are built automatically
if the \verb!DIET_BUILD_EXAMPLES! and \verb!DIET_HAVE_CLOUD_EC2! make flags are selected.
Note that the user {\it must change} the service configuration file to match his
own Cloud installation.

