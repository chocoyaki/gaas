%**
%*  @file  dynamic.tex
%*  @brief   DIET User's Manual: Dynamic Management
%*  @author  -  Benjamin DEPARDON (Benjamin.Depardon@ens-lyon.fr)
%*  @section Licence 
%*    |LICENSE|


\chapter{Dynamic management}
\label{ch:dynamic}

\section{Dynamically modifying the hierarchy}

\subsection{Motivations}

So far we saw that \diet's hierarchy was mainly static: once the shape
of the hierarchy chosen, and the hierarchy deployed, the only thing
you can do is kill part of the hierarchy, or add new subtrees to the
existing hierarchy. But whenever an agent is killed, the whole
underlying hierarchy is lost. This has several drawbacks: some \sed
will become unavailable, and if you want to reuse the machines on
which those \sed (or agents) are, you need to kill the existing \diet
element, and redeploy a new subtree. Another problem due to this
static asignement of the parent/children links is that if you have an
agent that is overloaded, you cannot move part of its children to an
underloaded agent somewhere else in the hierarchy without once again
killing part of the hierarchy, and deploying once again.


\subsection{``And thus it began to evolve''}

Hence, \diet also has a built-in mode in which you can dynamically modify
its shape using CORBA calls.  In this mode, if a \diet element
cannot reach its parent, when initializing, it won't exit, but will
wait for an order to connect itself to a new parent. Hence, you do not
need to deploy \diet starting from the MA down to the \sed, you can
launch all the elements at once, and then, send the orders for each
element to connect to its correct parent (you do not even need to
follow the shape of the tree, you can start from the bottom to the
tree up to the root, or use a random order, the service tables will be
correctly initialized.)

You now have access to the following CORBA methods:
\begin{itemize}
\item \verb|long bindParent(in string parentName)|: sends an order to
  a \sed or agent to bind to a new parent having the name
  ``\verb|parentName|'' if this parent can be contacted, otherwise the
  element keeps its old parent. If the element already has a parent,
  it unsubscribes itself from the parent, so that this latter is able
  to update its service table and list of children. A \texttt{null} value is
  returned if the change occurred, otherwise a value different from 0
  is returned if a problem occurred.

\item \verb|long disconnect()|: sends an order to disconnect an
  element from its parent. This does not kill the element, but merely
  removes the link between the element and its parent. Thus, the
  underlying hierarchy will be unreachable until the element is
  connected to a new parent.

\item \verb|long removeElement()|: sends an order to a \sed to kill
  itself. The \sed first unsubscribe from its parent before ending
  itself properly.

\item \verb|long removeElement(in boolean recursive)|: same as above
  but for agents. The parameter ``\verb|recursive|'' if true also
  destroys the underlying hierarchy, otherwise only the agent is
  killed.
\end{itemize}

Now, what happens if during a request submission an element receives
an order to change its parent? Actually, nothing will change, as
whenever a request is received a reference to the parent from which
the request originates is locally kept. So if the parent changes
before the request is sent back to the parent, as we keep a local
reference on the parent, the request will be sent back to the correct
``parent''. Hence, for a short period of time, an element can have
multiple parents.

\textbf{WARNING: currently no control is done on whether or not you
  are creating loops in the hierarchy when changing a parent.}

\subsection{Example}

Two examples on how to call those CORBA methods are present in\newline
\texttt{src/examples/dynamic\_hierarchy}:
\begin{itemize}
\item \texttt{connect.cc} sends orders to change the parent of an element.\newline
%
    \texttt{Usage: ./connect <SED|LA> <element name> <parent name>}.

\item \texttt{disconnect.cc} sends orders to disconnect an element
  from its parent. It does not kill the element, but only disconnects
  it from the \diet hierarchy (useful when your platform is not
  heavily loaded and you want to use only part of the
  hierarchy)\newline
%
  \texttt{Usage: ./disconnect <SED|LA> <element name>}.

\item \texttt{remove.cc} sends orders to remove an element.\newline
%
    \texttt{Usage: ./remove <SED|AGENT> <element name> [recursive: 0|1]}
\end{itemize}


\section{Changing offered services}

\subsection{Presentation}
A \sed does not necessarily need to declare all its services
initially, \ie as presented in Chapter~\ref{ch:server} before
launching the \sed via \verb|diet_SeD(...)|. One could want to
initially declare a given set of services, and then, depending on
parameters, or external events, one could want to modify this set of
services. An example of such usage is to spawn a service that is in
charge of cleaning temporary files when they won't be needed nor by
this \sed, nor by any other \sed or clients, and when this service is
called, it cleans whatever needs to be cleaned, and then this service
is removed from the service table.

Adding a service has already been introduced in
Chapter~\ref{ch:server}: using \verb|diet_service_table_add(...)| you
can easily add a new service (be it before running the \sed or within
a service). Well, removing a service is as easy, you only need to call
one of these methods: {\footnotesize
\begin{verbatim}
int diet_service_table_remove(const diet_profile_t* const profile);
int diet_service_table_remove_desc(const diet_profile_desc_t* const profile);
\end{verbatim}
}

So basically, when you want to remove the service that is called, you
only need to pass the \verb|diet_profile_t| you receive in the solve
function to \verb|diet_service_table_remove|. If you want to remove
another service, you need to build its profile description (just as if
you wanted to create a new service), and pass it to
\verb|diet_service_table_remove_desc|.


\subsection{Example}

The following example (present in \texttt{src/examples/dyn\_add\_rem})
initially declares one service. This service receives an integer $n$
as parameter. It creates $n$ services, and removes the service that
has just been called. Hence a service can only be called once, but it
spawns $n$ new services.

{\footnotesize
\begin{verbatim}
#include <iostream>
#include <sstream>
#include <cstring>

#include "DIET_server.h"
#include "DIET_Dagda.h"

/* begin function prototypes*/
int service(diet_profile_t *pb);
int add_service(const char* service_name);
/* end function prototypes*/

static unsigned int NB = 1;

template <typename T>
std::string toString( T t ) {
    std::ostringstream oss;
    oss << t;
    return oss.str();
}

/* Solve Function */
int
service(diet_profile_t* pb) {
  int *nb;

  if (pb->pb_name)
    std::cout << "## Executing " << pb->pb_name << std::endl;
  else {
    std::cout << "## ERROR: No name for the service" << std::endl;
    return -1;
  }

  diet_scalar_get(diet_parameter(pb,0), &nb, NULL);
  std::cout << "## Will create " << *nb << " services." << std::endl;

  for (int i = 0; i < *nb; i++) {
    add_service(std::string("dyn_add_rem_" + toString(NB++)).c_str());
  }

  std::cout << "## Services added" << std::endl;
  diet_print_service_table();

  /* Removing */
  std::cout << "## Removing service " << pb->pb_name << std::endl;
#ifdef HAVE_ALT_BATCH
  pb->parallel_flag = 1;
#endif
  diet_service_table_remove(pb);
  std::cout << "## Service removed" << std::endl;

  /* Print service table */
  diet_print_service_table();

  return 0;
}

/* usage function */
int
usage(char* cmd) {
  std::cerr << "Usage: " << cmd << " <SeD.cfg>" << std::endl;
  return -1;
}

/* add_service function: declares SeD's service */
int
add_service(const char* service_name) {
  diet_profile_desc_t* profile = NULL;
  unsigned int pos = 0;

  /* Set profile parameters: */
  profile = diet_profile_desc_alloc(strdup(service_name),0,0,0);

  diet_generic_desc_set(diet_param_desc(profile,pos++),DIET_SCALAR, DIET_INT);

  /* Add service to the service table */
  if (diet_service_table_add(profile, NULL, service )) return 1;

  /* Free the profile, since it was deep copied */
  diet_profile_desc_free(profile);

  std::cout << "Service '" << service_name << "' added!" << std::endl;

  return 0;
}

int checkUsage(int argc, char ** argv) {
  if (argc != 2) {
    usage(argv[0]);
    exit(1);
  }
  return 0;
}

/* MAIN */
int
main( int argc, char* argv[]) {
  int res;
  std::string service_name = "dyn_add_rem_0";

  checkUsage(argc, argv);

  /* Add service */
  diet_service_table_init(1);
  add_service(service_name.c_str());

  /* Print service table and launch daemon */
  diet_print_service_table();
  res = diet_SeD(argv[1],argc,argv);
  return res;
}
\end{verbatim}
}


\subsection{Going further}

Finally, another example is provided in
\texttt{src/examples/dynamicServiceMgr} showing how to dynamically
load and unload libraries containing services. Hence, a client can
send a library to as server, and for as long as the library is
compiled for the right architecture, the server will be able to load
it, and instanciate the service present in the library. The service
can further be called by other clients, and whenever it is not
required anymore, it can be easily removed.

%%% Local Variables:
%%% mode: latex
%%% ispell-local-dictionary: "american"
%%% mode: flyspell
%%% fill-column: 79
%%% End:
