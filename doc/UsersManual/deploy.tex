%****************************************************************************%
%* DIET User's Manual deploying chapter file                                *%
%*                                                                          *%
%*  Author(s):                                                              *%
%*    - Philippe COMBES (Philippe.Combes@ens-lyon.fr)                       *%
%*                                                                          *%
%* $LICENSE$                                                                *%
%****************************************************************************%
%* $Id$
%* $Log$
%* Revision 1.2  2004/01/07 10:27:33  cpera
%* Add asynchronous call example and useAsyncAPI config parameter.
%*
%* Revision 1.1  2003/09/09 12:38:20  pcombes
%* Reorganization of doc: UM becomes UsersManual.
%*
%* Revision 1.10  2003/06/02 13:47:05  pcombes
%* Fix footnotesize.
%*
%* Revision 1.9  2003/05/23 09:23:35  pcombes
%* Add suggestions from Jean-Yves. Thanks !
%*
%* Revision 1.8  2003/05/15 14:17:58  pcombes
%* UM 0.7
%*
%* Revision 1.5  2003/01/22 17:34:53  pcombes
%* User Manual, v. 0.6.4
%*
%* Revision 1.4  2003/01/21 12:17:02  pcombes
%* Update UM to API 0.6.3, and "hide" data structures.
%*
%* Revision 1.3  2003/01/13 12:09:00  pcombes
%* UM: client part complete for users's day ...
%****************************************************************************%

\chapter{Deploying a DIET platform}
\label{ch:deploying}


%====[ Deploying CORBA services ]==============================================
\section{Deploying CORBA services}
\label{sec:CORBA_services}

So far, only one CORBA service is needed: the Naming Service. It is provided by
a name server that must be launched before all DIET entities. Then, these DIET
entities must be passed a reference to the $<$host:port$>$ of this name server.

In this section, we try to give some basic information to start using DIET with
a basic CORBA configuration. Please refer to the documentation of your ORB if
you need more details.

\subsection{omniNames}

\subsubsection{The server}

To launch the omniORB name server, first check that the path of the omniORB
libraries is in your environment variable \texttt{LD\_LIBRARY\_PATH}, then
specify the log directory, through the environment variable
\texttt{OMNINAMES\_LOGDIR} (or, with \textbf{omniORB 4}, at compile time,
through the \texttt{--with-omniNames-logdir} option of the omniORB configure
script). If there is no log file in this directory, \texttt{omniNames} needs to
be given the port through which its client will connect. It can be launched
as follows:
{\footnotesize
\begin{verbatim}
~ > omniNames -start

Fri Dec 13 14:46:02 2002:

Starting omniNames for the first time.
Wrote initial log file.
Read log file successfully.
Root context is IOR:010000002b00000049444c3a6f6d672e6f72672f436f734e616d696e672f4e61
6d696e67436f6e746578744578743a312e300000010000000000000060000000010102000d0000003134
302e37372e31332e34360000fa0a0b0000004e616d655365727669636500020000000000000008000000
0100000000545441010000001c0000000100000001000100010000000100010509010100010000000901
0100
Checkpointing Phase 1: Prepare.
Checkpointing Phase 2: Commit.
Checkpointing completed.
\end{verbatim}
}

This sets an omniORB name server which listens to clients
connections on default port 2809. If omniNames has already been launched once,
\emph{ie} there is already some log files in the log directory, using the
\texttt{-start} option causes an error. The port is actually read from old
log files:
{\footnotesize
\begin{verbatim}
~ > omniNames -start 2810

Fri Dec 13 15:02:36 2002:

Error: log file '/tmp/omninames-toto.log' exists.  Can't use -start option.
~ > omniNames  

Fri Dec 13 15:02:40 2002:

Read log file successfully.
Root context is IOR:010000002b00000049444c3a6f6d672e6f72672f436f734e616d696e672f4e61
6d696e67436f6e746578744578743a312e300000010000000000000060000000010102000d0000003134
302e37372e31332e34360000fa0a0b0000004e616d655365727669636500020000000000000008000000
0100000000545441010000001c0000000100000001000100010000000100010509010100010000000901
0100
Checkpointing Phase 1: Prepare.
Checkpointing Phase 2: Commit.
Checkpointing completed.
\end{verbatim}
}


\subsubsection{The client}

Every DIET entity needs to connect to the CORBA name server: it is the way to
discover each other. The reference to the omniORB name server is written in the
configuration file, whose path is given to omniORB through the environment
variable \texttt{OMNIORB\_CONFIG} (or, with \textbf{omniORB 4}, at compile time,
through the \texttt{--with-omniORB-config} option of the configure script). Some
examples of such a configuration file are given in the directory
\texttt{src/examples/cfgs} of the DIET source files and installed in
\texttt{$<$install\_dir$>$/etc}. The lines concerning the name server, in the
omniORB configuration file, are built as follows:
\begin{description}
 \item{omniORB 3:}
{\footnotesize
\begin{verbatim}
ORBInitialHost <name server hostname>
ORBInitialPort <name server port>
\end{verbatim}
}
 \item{omniORB 4:}
{\footnotesize
\begin{verbatim}
InitRef = NameService=corbaname::<name server hostname>:<name server port>
\end{verbatim}
}
The name server port is the one given in argument to the \texttt{-start} option
of \texttt{omniNames}.
\end{description}

%\subsection*{The TAO name server}
%\subsubsection{The server}
%\subsubsection{The client}


%====[ DIET configuration file ]===============================================
\section{DIET configuration file} 

Launching a DIET entity needs a configuration file. Some fully commented
examples of such configuration files are given in the directory
\texttt{src/examples/cfgs} of the DIET source files and installed in
\texttt{$<$install\_dir$>$/etc}. Please note that:
\begin{itemize}
\item comments start with '\#' and finish at the end of the current
  line,
\item meaningful lines have the format: \texttt{keyword = value}, following the
  format of configuration files for omniORB 4,
\item keywords are case sensitive.
\end{itemize}

\subsection{Tracing API}

\noindent
\texttt{traceLevel} \ \ \emph{default}\texttt{ = 1}\\
This option controls debugging trace output. The following levels are defined:

\begin{center}
 \footnotesize
 \begin{tabular}{p{.1\linewidth}p{.8\linewidth}}
  level $=$ 0  & Print only errors\\
  level $<$ 5  & Print errors and messages for the main steps (such as ``Got a
  request'') - default\\
  level $<$ 10 & Print errors and messages for all steps\\
  level $=$ 10 & Print errors, all steps, and some important structures (such
  as the list of offered services)\\
  level $>$ 10 & Print all DIET messages AND omniORB messages corresponding to
  an omniORB traceLevel of (level~-~10)
 \end{tabular}
\end{center}


\subsection{Client side parameters}

\noindent
\texttt{MAName} \ \ \emph{default}\texttt{ = }\emph{none}\\
This is a \textbf{compulsory} parameter that specifies the name of the Master
Agent to connect to. The MA must have registered with this same name to the
CORBA name server.


\subsection{Agent side parameters}

\noindent
\texttt{agentType} \ \ \emph{default}\texttt{ = }\emph{none}\\
As DIET offers only one executable for both types of agent, it is
\textbf{compulsory} to specify which kind of agent must be launched. Two values
are available: \texttt{DIET\_MASTER\_AGENT} and \texttt{DIET\_LOCAL\_AGENT}.
They have aliases, respectively \texttt{MA} and \texttt{LA}.
\\

\noindent
\texttt{name} \ \ \emph{default}\texttt{ = }\emph{none}\\
This is a \textbf{compulsory} parameter that specifies the name which the agent
will register with to the CORBA name server.


\subsection{LAs and SeD side parameters}

\noindent
\texttt{parentName} \ \ \emph{default}\texttt{ = }\emph{none}\\
This is a \textbf{compulsory} parameter for Local Agents only ; it indicates the
name of the parent (an LA or the MA) to register to.


\subsection{FAST options}

\noindent
\texttt{fastUse} \ \ \emph{default}\texttt{ = 1}\\
This option activates the requests to FAST. It is ignored if DIET was
compiled without FAST, and defaults to 1 otherwise. In the latter case, it is
useful to set it to 0 if one wishes to perform some tests
without deploying the whole NWS/LDAP platform.

The following options are ignored if DIET was compiled without FAST or if
\texttt{fastUse} is set to 0.

\subsubsection{LDAP options}

\noindent
\texttt{ldapUse} \ \ \emph{default}\texttt{ = 0} for agents, \texttt{ = 1} for
SeDs\\
This option activates the use of LDAP in FAST requests. It defaults to 0 for
agents (which do not need to connect to the LDAP base) and to 1 for SeDs.
\\

The following two options are ignored if \texttt{ldapUse} is set to 0.
\\

\noindent
\texttt{ldapBase} \ \ \emph{default}\texttt{ = }\emph{none}\\
Specify the \texttt{host:port} address of the LDAP base where FAST gets the
results of its benchmarks.
\\

\noindent
\texttt{ldapMask} \ \ \emph{default}\texttt{ = }\emph{none}\\
Specify the mask used for requests to the LDAP base. It must match the one given
in the \texttt{.ldif} file of the server that was added to the base.


\subsubsection{NWS options}

\noindent
\texttt{nwsUse} \ \ \emph{default}\texttt{ = 1}\\
This option activates the use of NWS in FAST requests. If 0, FAST will use an
internal sensor for the performance of the machine, but will not be able to
evaluate communication times.
\\

The following two options are ignored if \texttt{nwsUse} is set to 0.
\\

\noindent
\texttt{nwsNameserver} \ \ \emph{default}\texttt{ = }\emph{none}\\
Specify the \texttt{host:port} address of the NWS name server.
\\

\noindent
\texttt{nwsForecaster} \ \ \emph{default}\texttt{ = }\emph{none}\\
Specify the \texttt{host:port} address of the NWS forecaster.
\\


\subsection{Miscellanous options}

\texttt{endPoint} \ \ \emph{default}\texttt{ = }\emph{2809}\\
This option specifies the listening port of an agent or an SeD. It is very
useful when they are behind a firewall. It defaults to 2809, but if this port is
already busy, the OS chooses a port number.

\texttt{useAsyncAPI} \ \ \emph{default}\texttt{ = }\emph{1}\\
Specify use of asynchronous call and so create objects managing it
on client side.

%====[ Example ]=========================================
\section{Example}
\label{sec:deploy_ex}

As shown in Section \ref{init}, the hierarchy is built from top to down: the
children register to their direct parent.

Here is an example of a complete platform deployment. Let us assume that:

\begin{itemize}
\item DIET was compiled with FAST on all machines used,
\item the LDAP server is launched on the machine \texttt{ldaphost} and listens
  on the port 9000,
\item the NWS name server is launched on the machine \texttt{nwshost} and
  listens on the port 9001,
\item the NWS forecaster is launched on the machine \texttt{nwshost} and
  listens on the port 9002,
\item the NWS sensors are launched on every machine we use.
\end{itemize}


\subsubsection{Launching the MA}

For such a platform, the MA configuration file could be:
\tt
\begin{center}
 \footnotesize
 \begin{tabular}{lcll}
  \multicolumn{4}{l}{\# file MA\_example.cfg, configuration file for an MA}\\
  agentType    &=&DIET\_MASTER\_AGENT&\\
  name         &=&MA\_example        &\\
  \#traceLevel &=&5                  &\# default\\
  \#endPoint   &=&2809               &\# default\\
  \#fastUse    &=&1                  &\# default\\
  \#ldapUse    &=&0                  &\# default\\
  \#nwsUse     &=&1                  &\# default\\
  nwsNameserver&=&nwshost:9001       &\\
  nwsForecaster&=&nwshost:9002       &\\
 \end{tabular}
\end{center}
\rm

This configuration file is the only argument to the executable \texttt{dietAgent},
that is installed in \texttt{$<$install\_dir$>$/bin}. Provided
\texttt{$<$install\_dir$>$/bin} is in your PATH environment variable, run
{\footnotesize
\begin{verbatim}
~ > dietAgent MA_example.cfg

Master Agent MA_example started.
\end{verbatim}
}


\subsubsection{Launching an LA}

For such a platform, an LA configuration file could be:
\tt
\begin{center}
 \footnotesize
 \begin{tabular}{lcll}
  \multicolumn{4}{l}{\# file LA\_example.cfg, configuration file for an LA}\\
  agentType    &=&DIET\_LOCAL\_AGENT&\\
  name         &=&LA\_example       &\\
  parentName   &=&MA\_example       &\\
  \#traceLevel &=&5                 &\# default\\
  \#endPoint   &=&2809              &\# default\\
  \#fastUse    &=&1                 &\# default\\
  \#ldapUse    &=&0                 &\# default\\
  \#nwsUse     &=&1                 &\# default\\
  nwsNameserver&=&nwshost:9001      &\\
  nwsForecaster&=&nwshost:9002      &\\
 \end{tabular}
\end{center}
\rm

This configuration file is the only argument to the executable \texttt{dietAgent},
that is installed in \texttt{$<$install\_dir$>$/bin}. This LA will register as a
son of MA\_example. Run
{\footnotesize
\begin{verbatim}
~ > dietAgent LA_example.cfg

Local Agent LA_example started.
\end{verbatim}
}

\subsubsection{Launching a server}

For such a platform, an \sed\ configuration file could be:
\tt
\begin{center}
 \footnotesize
 \begin{tabular}{lcll}
  \multicolumn{4}{l}{\# file SeD\_example.cfg, configuration file for an SeD}\\
  parentName   &=&LA\_example        &\\
  \#traceLevel &=&5                 &\# default\\
  \#endPoint   &=&2809              &\# default\\
  \#fastUse    &=&1                 &\# default\\
  \#ldapUse    &=&1                 &\# default\\
  ldapBase     &=&ldaphost:9000     &\\
  ldapMask     &=&dc=LIP,dc=ens-lyon,dc=fr&\\
  \#nwsUse     &=&1                 &\# default\\
  nwsNameserver&=&nwshost:9001      &\\
  nwsForecaster&=&nwshost:9002      &\\
 \end{tabular}
\end{center}
\rm

This \sed\ will register as a son of LA\_example. Run the executable that you
linked with the DIET SeD library, and do not forget that the first argument of
\texttt{diet\_SeD} must be the path of the configuration file above.


\subsubsection{Launching a client}

Our client must connect to the MA\_example:
\tt
\begin{center}
 \footnotesize
 \begin{tabular}{lcll}
  \multicolumn{4}{l}{\# file SeD\_example.cfg, configuration file for an SeD}\\
  MAName       &=&LA\_example        &\\
  \#traceLevel &=&5                 &\# default\\
 \end{tabular}
\end{center}
\rm

Run the executable that you linked with the DIET client library, and do not
forget that the first argument of \texttt{diet\_initialize} must be the path of
the configuration file above.

