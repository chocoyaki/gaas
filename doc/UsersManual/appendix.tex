%****************************************************************************%
%* DIET User's Manual appendix file                                         *%
%*                                                                          *%
%*  Author(s):                                                              *%
%*    - Philippe COMBES (Benjamin.Depardon@ens-lyon.fr)                     *%
%*                                                                          *%
%* $LICENSE$                                                                *%
%****************************************************************************%
%* $Id$
%* $Log$
%* Revision 1.1  2009/09/08 13:46:34  bdepardo
%* Added appendix.
%* Currently the page layout is broken in the appendix.
%*
%****************************************************************************%

\section{Configuration files}

\begin{table}[h]
  \centering
  \begin{tabular}[h]{|p{1cm}|l|l|p{5cm}|l|}
    \hline
    Element & Option & Type & Description & Mode\\
    \hline

    All & traceLevel & Integer & 
    traceLevel for the DIET agent:
    0  DIET prints only warnings and errors on the standard error output.
    1  [default] DIET prints information on the main steps of a call.
    5  DIET prints information on all internal steps too.
    10  DIET prints all the communication structures too.
    >10  (traceLevel - 10) is given to the ORB to print CORBA messages too.
    & All \\
    \hline

    Client & MAName & String & MA name & All \\
    \hline

    Agent & agentType & Agent type &
    Master Agent or Local Agent ? As there is only one executable
    for both agent types, it is COMPULSORY to specify the type of this
    agent:
    DIET\_MASTER\_AGENT (or MA) or DIET\_LOCAL\_AGENT (or LA) & All \\
    \hline

    All & dietPort & Integer & the listening port of the agent. If not
    specified, let the ORB get a port from the system (if the default
    2809 was busy). & All \\
    \hline

    All & dietHostname & String &
    the listening interface of the agent. If not specified,
    let the ORB get the hostname from the system (the first one if several 
    one are available). & All \\
    \hline

    Agent and \sed & name & String &
    The name of the element. The ORB configuration files of the clients
    and the children of this MA (LAs and SeDs) must point at the same CORBA
    Naming Service as the one pointed at by the ORB configuration file of
    this agent. & All \\
    \hline

    LA and \sed & parentName & String &
    the name of the agent to which the element will register. This
    agent must have registered at the same CORBA Naming Service that is 
    pointed to by your ORB configuration. & All \\
    \hline

    Agent and \sed & fastUse & Boolean & 
    If set to 0, all LDAP and NWS parameters are ignored, and all
    requests to FAST are disabled (when DIET is compiled with FAST). This is
    useful for testing a DIET platform without deploying an LDAP base nor an
    NWS platform. & \\
    \hline

    Agent and \sed & ldapUse & Boolean & 
    0 tells FAST not to look for the services in an LDAP base. & \\
    \hline

    Agent and \sed & ldapBase & Adress &
    <host:port> of the LDAP base that stores FAST-known services. & \\
    \hline

    Agent and \sed & ldapMask & String &
    the mask which is registered in the LDAP base. & \\
    \hline

    Agent and \sed & nwsUse & Boolean & 
    0 tells FAST not to use NWS for its comm times forecasts. & \\
    \hline

    Agent and \sed & nwsNameserver & Adress &
    <host:port> of the NWS nameserver. & \\
    \hline

    Agent and \sed & nwsForecaster & Adress & 
    & \\
    \hline

    Agent and \sed & useAsyncAPI & Boolean &
    & \\
    \hline

    Agent and \sed & useLogService & Boolean &
    1 to use the LogService for monitoring. & \\
    \hline

    Agent and \sed & lsOutbuffersize & Integer &
    the size of the buffer for outgoing messages. & \\
    \hline

    Agent and \sed & lsFlushinterval & Integer &
    the flush interval for the outgoing message buffer. & \\
    \hline

    MA & neighbours & String &
    A list of Master Agent that must be contacted to build a
    federation. The format is a list of host:port. & \\
    \hline

    MA & minimumNeighbours & Integer &
    Minimum number of connected neighbours. If the agent
    has less that this number of connected neighbours, is going to find some
    new connections. & \\
    \hline

    MA & maximumNeighbours & Integer &
    maximum number of connected neighbours. The agent does
    not accept a greater number of connection to build the federation than
    maximumNeighbours. & \\
    \hline

    MA & updateLinkPeriod & Integer &
    The agent check at a reagular time basis that all it's
    neighbours are still alive and try to connect to a new one if the number
    of connections is less than minimumNeighbours. updateLinkPeriod indicate
    the period in second between two checks. & \\
    \hline

    MA & bindServicePort & Integer &
    port used by the Master Agent to share its IOR. & \\ 
    \hline

    \sed & useConcJobLimit & Boolean & 
    & All \\
    \hline

    \sed & maxConcJobs & Integer &
    & All \\
    \hline

    & locationID & String &
    & \\
    \hline

    Client & MADAGNAME & String &
    the name of the MA DAG agent to wich the client will connect. &
    Workflow \\
    \hline

    & USEWFLOGSERVICE & String &
    & Workflow \\
    \hline

    & schedulerModule & String &
    The path to the scheduler library file. & User scheduling \\
    \hline

    & moduleConfigFile & String &
    Optional configuration file for the module. & User scheduling \\
    \hline

    \sed & batchName & String &
    The reservation batch system's name. & Batch \\
    \hline

    \sed & batchQueue & String & 
    The name of the queue where the job will be submitted. & Batch \\
    \hline

    \sed & pathToNFS & String &
    Path to an NFS directory where you have read/write rights. & Batch \\
    \hline

    \sed & pathToTmp & String &
    Path to a temporary directory where you have read/write rights. &
    Batch \\
    \hline

    \sed & internOARbatchQueueName & String &
    should only be useful when using CORI features
    with OAR 1.6x & Batch \\
    \hline

    & initRequestID & Integer &
    & \\
    \hline

    Agent and \sed & ackFile & String &
    Path to a file that will be created when the element is ready to
    execute. & Acknowledge file \\
    \hline

    & maxMsgSize & Integer &
    & Dagda \\
    \hline

    & maxDiskSpace & Integer &
    & Dagda \\
    \hline

    & maxMemSpace & Integer &
    & Dagda \\
    \hline

    & cacheAlgorithm & String &
    & Dagda \\
    \hline

    & shareFiles & Boolean &
    & Dagda \\
    \hline

    & dataBackupFile & String &
    & Dagda \\
    \hline

    & restoreOnStart & Boolean &
    & Dagda \\
    \hline

    & storageDirectory & String &
    & Dagda or Batch \\
    \hline


    Client & USE\_SPECIFIC\_SCHEDULING & String &
    This option specifies the scheduler the client will use whenever it submits
    a request:
    - BURST\_REQUEST: round robin on the available SeDs
    - BURST\_LIMIT: only allow a certain number of request per SeD in parallel
    the limit can be set with "void setAllowedReqPerSeD(unsigned ix)"
    & Custom Client Scheduling (CCS) \\
    \hline

    Client & clientMaxNbSeD & Integer &
    The maximum number of SeD the client should receive. & All \\
    \hline

  \end{tabular}
  \caption{Configuration files options}
  \label{tab:cfg_files_opt}
\end{table}