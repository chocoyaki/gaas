%****************************************************************************%
%* DIET User's Manual appendix file                                         *%
%*                                                                          *%
%*  Author(s):                                                              *%
%*    - Philippe COMBES (Benjamin.Depardon@ens-lyon.fr)                     *%
%*                                                                          *%
%* $LICENSE$                                                                *%
%****************************************************************************%
%* $Id$
%* $Log$
%* Revision 1.7  2010/02/25 07:15:57  ycaniou
%* DAGDA -> macro + sc
%* Add info logs � dagda.tex
%* Formatage document + ispell
%*
%* Revision 1.6  2010/02/25 06:45:38  ycaniou
%* Add local variables
%*
%* Revision 1.5  2010/02/02 19:41:43  bdepardo
%* A few corrections
%*
%* Revision 1.4  2010/02/01 06:55:44  ycaniou
%* Typo
%*
%* Revision 1.3  2010/01/21 14:05:24  bdepardo
%* Available options present for the Diet elements.
%* Please fill in the blanks.
%*
%* Revision 1.2  2009/10/26 07:23:07  bdepardo
%* Added chapter on dynamic hierarchy management.
%*
%* Revision 1.1  2009/09/08 13:46:34  bdepardo
%* Added appendix.
%* Currently the page layout is broken in the appendix.
%*
%****************************************************************************%

\chapter{Appendix}
\section{Configuration files}

\begin{description}
\item{\bf{traceLevel}}
  \begin{itemize}
  \item Component: All
    \item Mode: All
  \item Type: Integer
  \item Description: traceLevel for the \diet agent:
    \begin{itemize}
    \item  0 \diet prints only warnings and errors on the standard error
      output,
    \item 1 [default] \diet prints information on the main steps
      of a call,
    \item 5 \diet prints information on all internal steps too,
    \item 10 \diet prints all the communication structures too,
    \item $>10$ (traceLevel - 10) is given to the ORB to print CORBA messages
      too.
    \end{itemize}
  \end{itemize}

\item{\bf{MAName}}
  \begin{itemize}
  \item Component: Client
  \item Mode: All
  \item Type: String
  \item Description: Master Agent name.
  \end{itemize}

\item{\bf{agentType}}
  \begin{itemize}
  \item Component: Agent (MA and LA)
  \item Mode: All
  \item Type: Agent type
  \item Description: Master Agent or Local Agent? As there is only one
    executable for both agent types, it is COMPULSORY to specify the type
    of this agent: DIET\_MASTER\_AGENT (or MA) or DIET\_LOCAL\_AGENT (or
    LA).
  \end{itemize}

\item{\bf{dietPort}}
  \begin{itemize}
  \item Component: All
  \item Mode: All
  \item Type: Integer
  \item Description: the listening port of the agent. If not
    specified, let the ORB get a port from the system (if the default
    2809 was busy).
  \end{itemize}

\item{\bf{dietHostName}}
  \begin{itemize}
  \item Component: All
  \item Mode: All
  \item Type: String
  \item Description: the listening interface of the agent. If not specified,
    let the ORB get the hostname from the system (the first one if several 
    one are available).
  \end{itemize}

\item{\bf{name}}
  \begin{itemize}
  \item Component: Agent and \sed
  \item Mode: All
  \item Type: String
  \item Description: The name of the element. The ORB configuration files of the clients
    and the children of this MA (LAs and SeDs) must point at the same CORBA
    Naming Service as the one pointed at by the ORB configuration file of
    this agent.
  \end{itemize}

\item{\bf{parentName}}
  \begin{itemize}
  \item Component: LA and \sed
  \item Mode: All
  \item Type: String
  \item Description: the name of the agent to which the element will
    register. This agent must have registered at the same CORBA Naming
    Service that is  pointed to by your ORB configuration.
  \end{itemize}


\item{\bf{fastUse}}
  \begin{itemize}
  \item Component: Agent and \sed
  \item Mode: FAST
  \item Type: Boolean
  \item Description: If set to 0, all LDAP and NWS parameters are
    ignored, and all requests to FAST are disabled (when \diet is compiled
    with FAST). This is useful for testing a \diet platform without
    deploying an LDAP base nor an NWS platform.
  \end{itemize}

\item{\bf{ldapUse}}
  \begin{itemize}
  \item Component: Agent and \sed
  \item Mode: FAST
  \item Type: Boolean
  \item Description: 0 tells FAST not to look for the services in an
    LDAP base.
  \end{itemize}

\item{\bf{ldapBase}}
  \begin{itemize}
  \item Component: Agent and \sed
  \item Mode: FAST
  \item Type: Address 
  \item Description: <host:port> of the LDAP base that stores
    FAST-known services.
  \end{itemize}

\item{\bf{ldapMask}}
  \begin{itemize}
  \item Component: Agent and \sed
  \item Mode: FAST
  \item Type: String
  \item Description: the mask which is registered in the LDAP base.
  \end{itemize}

\item{\bf{nwsUse}}
  \begin{itemize}
  \item Component: Agent and \sed
  \item Mode: FAST
  \item Type: Boolean
  \item Description: 0 tells FAST not to use NWS for its comm times
    forecasts.
  \end{itemize}

\item{\bf{nwsNameserver}}
  \begin{itemize}
  \item Component: Agent and \sed
  \item Mode: FAST
  \item Type: Address
  \item Description: <host:port> of the NWS nameserver.
  \end{itemize}

\item{\bf{nwsForecaster}}
  \begin{itemize}
  \item Component: Agent and \sed
  \item Mode: FAST
  \item Type: Address
  \item Description: FAST
  \end{itemize}

\item{\bf{useAsyncAPI}}
  \begin{itemize}
  \item Component: Agent and \sed
  \item Mode: ?
  \item Type: Boolean
  \item Description: ?
  \end{itemize}

\item{\bf{useLogService}}
  \begin{itemize}
  \item Component: Agent and \sed
  \item Mode: All
  \item Type: Boolean
  \item Description: 1 to use the LogService for monitoring.
  \end{itemize}

\item{\bf{lsOutbuffersize}}
  \begin{itemize}
  \item Component: Agent and \sed
  \item Mode: All
  \item Type: Integer
  \item Description: the size of the buffer for outgoing messages.
  \end{itemize}

\item{\bf{lsFlushinterval}}
  \begin{itemize}
  \item Component: Agent and \sed
  \item Mode: All
  \item Type: Integer
  \item Description: the flush interval for the outgoing message buffer.
  \end{itemize}

\item{\bf{neighbours}}
  \begin{itemize}
  \item Component: MA
  \item Mode: Multi MA
  \item Type: String
  \item Description: A list of Master Agent that must be contacted to
    build a federation. The format is a list of host:port.
  \end{itemize}

\item{\bf{minimumNeighbours}}
  \begin{itemize}
  \item Component: MA
  \item Mode: Multi MA
  \item Type: Integer
  \item Description: Minimum number of connected neighbours. If the
    agent has less that this number of connected neighbours, is going to
    find some new connections. 
  \end{itemize}

\item{\bf{maximumNeighbours}}
  \begin{itemize}
  \item Component: MA
  \item Mode: Integer 
  \item Type: Multi MA
  \item Description: maximum number of connected neighbours. The agent
    does not accept a greater number of connection to build the federation
    than maximumNeighbours.
  \end{itemize}

\item{\bf{updateLinkPeriod}}
  \begin{itemize}
  \item Component: MA
  \item Mode: Multi MA
  \item Type: Integer
  \item Description: The agent check at a regular time basis that all
    it's neighbours are still alive and try to connect to a new one if the
    number of connections is less than
    \emph{minimumNeighbours}. \emph{updateLinkPeriod} indicate the period
    in second between two checks.
  \end{itemize}

\item{\bf{bindServicePort}}
  \begin{itemize}
  \item Component: MA
  \item Mode: All
  \item Type: Integer
  \item Description: port used by the Master Agent to share its IOR.
  \end{itemize}

\item{\bf{useConcJobLimit}}
  \begin{itemize}
  \item Component: \sed
  \item Mode: All
  \item Type: Boolean
  \item Description: ?
  \end{itemize}

\item{\bf{maxConcJobs}}
  \begin{itemize}
  \item Component: \sed
  \item Mode: All
  \item Type: Integer
  \item Description: ?
  \end{itemize}

\item{\bf{locationID}}
  \begin{itemize}
  \item Component: ?
  \item Mode: ?
  \item Type: String
  \item Description: ?
  \end{itemize}

\item{\bf{MADAGNAME}}
  \begin{itemize}
  \item Component: Client
  \item Mode: Workflow
  \item Type: String
  \item Description: the name of the \madag agent to wich the client
    will connect.
  \end{itemize}

\item{\bf{USEWFLOGSERVICE}}
  \begin{itemize}
  \item Component: ?
  \item Mode: Workflow
  \item Type: String
  \item Description: ?
  \end{itemize}

\item{\bf{schedulerModule}}
  \begin{itemize}
  \item Component: ?
  \item Mode: User scheduling
  \item Type: String
  \item Description: The path to the scheduler library file.
  \end{itemize}

\item{\bf{moduleConfigFile}}
  \begin{itemize}
  \item Component: ?
  \item Mode: User scheduling
  \item Type: String
  \item Description: Optional configuration file for the module.
  \end{itemize}

\item{\bf{batchName}}
  \begin{itemize}
  \item Component: \sed
  \item Mode: Batch
  \item Type: String
  \item Description: The reservation batch system's name.
  \end{itemize}

\item{\bf{batchQueue}}
  \begin{itemize}
  \item Component: \sed
  \item Mode: Batch
  \item Type: String
  \item Description: The name of the queue where the job will be submitted.
  \end{itemize}

\item{\bf{pathToNFS}}
  \begin{itemize}
  \item Component: \sed
  \item Mode: Batch
  \item Type: String
  \item Description: Path to an NFS directory where you have read/write rights.
  \end{itemize}

\item{\bf{pathToTmp}}
  \begin{itemize}
  \item Component: \sed
  \item Mode: Batch
  \item Type: String
  \item Description: Path to a temporary directory where you have
    read/write rights.
  \end{itemize}

\item{\bf{internOARbatchQueueName}}
  \begin{itemize}
  \item Component: \sed
  \item Mode: Batch
  \item Type: String
  \item Description: only useful when using CORI batch features with
    OAR 1.6
  \end{itemize}

\item{\bf{initRequestID}}
  \begin{itemize}
  \item Component: ?
  \item Mode: ?
  \item Type: Integer
  \item Description: ?
  \end{itemize}

\item{\bf{ackFile}}
  \begin{itemize}
  \item Component: Agent and \sed
  \item Mode: Acknowledge file
  \item Type: String
  \item Description: Path to a file that will be created when the
    element is ready to execute.
  \end{itemize}

\item{\bf{maxMsgSize}}
  \begin{itemize}
  \item Component: ?
  \item Mode: \dagda
  \item Type: Integer
  \item Description: ? 
  \end{itemize}

\item{\bf{maxDiskSpace}}
  \begin{itemize}
  \item Component: ?
  \item Mode: \dagda
  \item Type: Integer
  \item Description: ?
  \end{itemize}

\item{\bf{maxMemSpace}}
  \begin{itemize}
  \item Component: ?
  \item Mode: \dagda
  \item Type: Integer
  \item Description: ?
  \end{itemize}

\item{\bf{cacheAlgorithm}}
  \begin{itemize}
  \item Component: ?
  \item Mode: \dagda
  \item Type: String
  \item Description: ? 
  \end{itemize}

\item{\bf{shareFiles}}
  \begin{itemize}
  \item Component: ?
  \item Mode: \dagda
  \item Type: Boolean
  \item Description: ?
  \end{itemize}

\item{\bf{dataBackupFile}}
  \begin{itemize}
  \item Component: ?
  \item Mode: \dagda
  \item Type: String
  \item Description: ? 
  \end{itemize}

\item{\bf{restoreOnStart}}
  \begin{itemize}
  \item Component: ?
  \item Mode: \dagda
  \item Type: Boolean
  \item Description: ?
  \end{itemize}

\item{\bf{storageDirectory}}
  \begin{itemize}
  \item Component: ?
  \item Mode: \dagda or Batch
  \item Type: String
  \item Description: ? 
  \end{itemize}

\item{\bf{USE\_SPECIFIC\_SCHEDULING}}
  \begin{itemize}
  \item Component: Client
  \item Mode: Custom Client Scheduling (CCS)
  \item Type: String
  \item Description: 
    This option specifies the scheduler the client will use whenever it submits
    a request:
    \begin{itemize}
    \item BURST\_REQUEST: round robin on the available \sed
    \item BURST\_LIMIT: only allow a certain number of request per \sed in
      parallel the limit can be set with "void
      setAllowedReqPerSeD(unsigned ix)"
    \end{itemize}
  \end{itemize}

\item{\bf{clientMaxNbSeD}}
  \begin{itemize}
  \item Component: Client
  \item Mode: All
  \item Type: Integer
  \item Description: The maximum number of \sed the client should receive.
  \end{itemize}

\end{description}

%%% Local Variables:
%%% mode: latex
%%% ispell-local-dictionary: "american"
%%% mode: flyspell
%%% fill-column: 79
%%% End:
