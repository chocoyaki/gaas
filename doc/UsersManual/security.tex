%**
%*  @file  security.tex
%*  @brief   DIET User's Manual Security chapter file 
%*  @author  - Guillaume Verger (guillaume.verger@inria.fr) 
%*  @section Licence 
%*    |LICENSE|

\newcommand{\auth}{\emph{Auth}\xspace}

\chapter{Security extension}
\label{ch:Securityextension}

In this chapter, we introduce the security extension of \diet. It is 
possible to use encrypted communications between agents, \seds and clients. 
To do so, we need and authority able to deliver signed certificates. Each 
entity will then have a certificate, which will ensure its authentication.

\section{Preparing the files}

Communications in \diet are handled by Corba. In order to have secured 
communications in \diet, we have to specify Corba we use ssl, and then 
provides keys and certificates to use it.

\subsection{Defining an authority}

When a component will register to the \diet hierarchy, it will need to be 
authenticated for its parent to accept the connection. This can be achieved 
by exhibiting a signed certificate from a known authority. We will now 
define the different steps to create your own certificate as an authority 
\auth.

\subsection{Getting a signed certificate}

In order to prove a \diet entity as being authenticated, it needs to have a 
certificate signed by \auth. Suppose we have \verb+privatekey.pem+ the file 
containing the private key for the entity, and \verb+signedcert.pem+ the 
certificate which has been signed by \auth for the entity. The entity has 
to concatenate the two files into a single one, the key followed by the 
certificate.

\section{Setting the configuration files}

The instructions about how to compile \diet with the Security extension are 
available in Section~\ref{sec:securityinstall} and the configuration 
instructions are available in Section~\ref{sec:securityconfig}.



%%% Local Variables:
%%% mode: latex
%%% ispell-local-dictionary: "american"
%%% mode: flyspell
%%% fill-column: 79
%%% End:
