%**
%*  @file  security.tex
%*  @brief   DIET User's Manual Security chapter file 
%*  @author  - Guillaume Verger (guillaume.verger@inria.fr) 
%*  @section Licence 
%*    |LICENSE|

\newcommand{\auth}{\emph{Auth}\xspace}

\chapter{Security extension}
\label{ch:Securityextension}

In this chapter, we introduce the security extension of \diet. It is 
possible to use encrypted communications between agents, \seds and clients. 
To do so, we need and authority able to deliver signed certificates. Each 
entity will then have a certificate, which will ensure its authentication.

\section{Preparing the files}

Communications in \diet are handled by Corba. In order to have secured 
communications in \diet, we have to specify Corba we use ssl, and then 
provides keys and certificates to use it.

\subsection{Defining an authority}

For a component to register to the \diet hierarchy, it will need to be 
authenticated for its parent to accept the connection. This can be achieved 
by exhibiting a signed certificate from a known authority. This means each
entity has the certificate of the authority (which is public), a private key and
a certificate generated thanks to this private key and signed by the authority.

\section{Setting the configuration files}

The instructions about how to compile \diet with the Security extension are 
available in Section~\ref{sec:securityinstall} and the configuration 
instructions are available in Section~\ref{sec:securityconfig}.
An example of scripts generating the certificates and the configuration
files is available in \verb+src/sexamples/security/+.



%%% Local Variables:
%%% mode: latex
%%% ispell-local-dictionary: "american"
%%% mode: flyspell
%%% fill-column: 79
%%% End:
