%**
%*  @file  client_scheduler.tex
%*  @brief   DIET User's Manual: client schedulers 
%*  @author  - Gael Le Mahec (gael.le.mahec@u-picardie.fr)
%*  @section Licence 
%*    |LICENCE|

\section{Scheduler at client level}

In this section we introduce a new way to define a scheduling policy
in \diet: at the client level. When a client submits a requests, it
receives an ordered list of available servers. However, the order of
the list might not suit the requirements of the client. Hence, we
introduce the ``Custom Client Scheduling'' mode. You need to compile
\diet with the option \texttt{DIET\_USE\_CCS}. Then you need to specify
in the client configuration file the scheduler you want to use with
the option \texttt{USE\_SPECIFIC\_SCHEDULING}. By
default, two schedulers are available: \texttt{BURST\_REQUEST} which
does a round-robin on the \sed (all the encountered \sed so far), and
\texttt{BURST\_LIMIT} which only allow a certain number of request per
\sed in parallel the limit can be set with
\verb|void setAllowedReqPerSeD(unsigned ix)|.
Hence for example you can add the following line in your client
configuration file:
\begin{verbatim}
USE_SPECIFIC_SCHEDULING = BURST_REQUEST
\end{verbatim}



\subsection{Create a new scheduler: \diet side}


A new scheduling mode can be added by modifying the
\texttt{SpecificClientScheduler} class present in
\texttt{src/client/SpecificClientScheduler.\{cc,hh\}}.
The virtual function \texttt{schedule} is here to test the name of the
chosen scheduler, and to execute the corresponding method.
Here is an example to add a mode named \texttt{DUMMY\_MODE}:
\begin{verbatim}
void
SpecificClientScheduler::schedule(const char * scheduling_name,
                                  SeD_var& chosenServer,
                                  corba_response_t * response) {
…..
  if (! strcmp(scheduling_name, "DUMMY_MODE") ) {
    dummyScheduling(chosenServer, response);
  }
…...
} // end schedule

void
SpecificClientScheduler::dummyScheduling(const char * scheduling_name,
                                         SeD_var& chosenServer,
                                         corba_response_t * response) {
  // ADD PROCESSING IN ORDER TO GIVE A VALUE TO chosenServer
  ….
} // end dummyScheduling
\end{verbatim}


\subsection{Create a new scheduler: user side}

We can add a new scheduling mode without having to modify \diet
sources. For this you need to create a class that inherits from
\texttt{SpecificClientScheduler} and which implements the
\texttt{schedule} method. Here is an example of how you can add a
scheduling mode which always chooses the same server (this is of
course usually not the kind of schedule you would like to have, but
this serves as an illustration).

\subsubsection{Create the ``SpecificClientScheduler'' class}

\begin{verbatim}
class MyScheduler : public SpecificClientScheduler {
public:
  MyScheduler();
  virtual ~MyScheduler();
  virtual void schedule(const char * scheduling_name, SeD_var& chosenServer,
           corba_response_t * response);
private:
  void dummyScheduling(const char * scheduling_name, SeD_var& chosenServer,
                  corba_response_t * response);
};
\end{verbatim}

\subsubsection{Redefine the ``schedule'' method}

\begin{verbatim}
void
MyScheduler::schedule(const char * scheduling_name,
                      SeD_var& chosenServer,
                      corba_response_t * response) {
  SpecificClientScheduler::schedule(scheduling_name, chosenServer, response);
  if (! strcmp(scheduling_name, "DUMMY_SCHEDULING")) {
    this->dummyScheduling(scheduling_name, chosenServer, response);
  }
} // end schedule
\end{verbatim}


\subsubsection{Write scheduling method}

\begin{verbatim}
void
MyScheduler::dummyScheduling(const char * scheduling_name,
                             SeD_var& chosenServer,
                             corba_response_t * response) {
  static vector<SeD_var> allSeDs;

  // Add new SeDs
  for (unsigned int ix=0; ix < response->servers.length(); ix++) {
    unsigned int jx;
    bool found = false;
    while (!found && (jx < allSeDs.size())) {
      if (allSeDs[jx] == response->servers[ix].loc.ior) 
        found = true;
      else
        jx++;
    }
    if (!found) {
      allSeDs.push_back(response->servers[ix].loc.ior);
    }
  } // end for

  cout << "*** Using my own scheduling method to change the SeD !!!!" << endl;
  chosenServer = allSeDs[0];
}
\end{verbatim}


\subsubsection{Modify client code}

\begin{verbatim}
MyScheduler * scheduler = new MyScheduler();
MyScheduler::setScheduler(scheduler);
// DIET code execution
…
delete scheduler;
\end{verbatim}