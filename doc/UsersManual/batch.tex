%****************************************************************************%
%* DIET Programmer' guide, batch/parallel submissions                       *%
%*                                                                          *%
%*  Author(s):                                                              *%
%*    - Yves Caniou (yves.caniou@ens-lyon.fr)                               *%
%*                                                                          *%
%* $LICENSE$                                                                *%
%****************************************************************************%
%* $Id$
%* $Log$
%* Revision 1.3  2008/04/08 09:59:09  ecaron
%* Correct batch.tex evrsion
%*
%* Revision 1.2  2008/04/07 23:55:37  ecaron
%* Fix overfull box
%*
%* Revision 1.10  2008/04/07 22:26:26  ecaron
%* Updated files to pdflatex compilation
%*
%* Revision 1.9  2006/11/30 14:14:16  ycaniou
%* diet -> textsc{Diet} (Tkx Abdelkader)
%*
%* Revision 1.8  2006/11/30 13:58:07  aamar
%* Remove an extra \
%*
%* Revision 1.7  2006/11/28 20:40:30  ycaniou
%* Only headers
%*
%* Revision 1.6  2006/11/27 08:13:58  ycaniou
%* Added missing fields Id and Log in headers
%****************************************************************************%

\chapter{Batch and parallel submissions}
\label{chapter:parallelSubmission}
\section{Introduction}

Most of resources in a grid are a parallel one. Computational grids
are even considered as hierachical sets of parallel resources, as we
can see in ongoing project like the french research grid project,
Grid'5000\cite{grid5000} (for the moment, 9 sites are involved), or
like the EGEE\footnote{\url{http://public.eu-egee.org/}} project ({\it
Enabling Grids for E-science in Europe}), composed of more than a
hundred centers in 31 countries. Then, in order to provide transparent
access to resources, grid middleware must supply efficient mechanisms
to provide parallel services.

Because parallel resources are managed differently on each site, it is
neither the purpose of \diet to deal with the deployment of parallel
tasks inside the site, nor manage copies of data which can possibly be
on NFS. \diet implements mechanisms for a \sed programmer to easily
provide a service that can be portable on different sites; for clients
to request services which can be explicitly sequential, parallel or
solved in the real transparent and efficient metacomputing way: only
the name of the service is given and \diet chooses the best resource
where to solve the problem.

%****************************************************************************%
\section{Terminology}
%****************************************************************************%

Because a good understanding comes with correct terms, we provide here
the definition of the terms that we will use thereafter.

Servers provide {\it services}, e.g instanciation of {problems} that a
server can solve: for example, two services can provide the resolution
of the same problem, one being sequential the other parallel. A \diet~
{\it task}, also called a {\it job}, is created by the {\it request}
of a client: it refers to the resolution of a service on a given
server.

A service can be sequential or (exclusive) parallel, in which case its
resolution requires numerous processors of a parallel resource (a
parallel machine or a cluster of workstations). If parallel, the task
can be modeled with the MPI standard, or composed of multiple
sequential tasks (deployed for example with \verb!ssh!) resolving a
single service: it is often the case with data parallelism problems.

Note that when dealing with batch reservation systems, we will likely
speak about {\it jobs} rather than about {\it tasks}.

%****************************************************************************%
\section{Configuration for compilation}
%****************************************************************************%

Compilation must necessary be done with \verb!cmake! (you can find
some information about cmake within \diet sources in
\verb!<diet_src>/Cmake/README!) because it is not maintained
with the autotools anymore.

Batch submission mechanisms were first built using two libraries:
elagi\footnote{\url{http://grail.sdsc.edu/projects/elagi/}} and
appleseeds\footnote{\url{http://grail.sdsc.edu/projects/appleseeds/}},
developped in the University of California of San-Diego. It now only
relies on the appleseeds library, that can be downloaded at the
download section on the site. Note that if you want to use \diet on
MAC OS 64 bits, you have to patch appleseeds by doing:
\begin{itemize}
\item Remove declaration of variable "environ" in \verb!envseed.h!
\item In propertyseed.c, replace the following line
\begin{lstlisting}[language=c,basewidth={.5em,.4em},fontadjust]
extern char **environ;
\end{lstlisting}
by
\begin{lstlisting}[language=c,basewidth={.5em,.4em},fontadjust]
#include <crt_externs.h>
#define environ (*_NSGetEnviron())
\end{lstlisting}
\item Then, you can compile and install it with traditional \verb!make! and \verb!make install!
\end{itemize}

You must enable the batch flag and provide the path of the appleseeds
installation in cmake arguments. Typically, if you build \diet from
the command line, you can use the following:

\begin{lstlisting}[language=bash,label=dietConfig.sh,basewidth={.5em,.4em},fontadjust]

   ccmake $diet_src_path                          \
          -DDIET_USE_BATCH:BOOL=ON                \
          -DAPPLESEEDS_DIR:PATH=$appleseeds_path

\end{lstlisting}

You also have to provide a directory where the \sed can read and write
data on the parallel resource. Its path is defined within the
\verb!cmake! variable \verb!DIET_BATCH_WRITING_PATH!. Note that the
\verb!$HOME! directory when the compilation is done is provided as
default.

Please note that this directory is used by \diet to store the new
built script that is submitted to the batch scheduler. In consequence,
because certain batch schedulers (like OAR) need the script to be
available on all resources, {\it this directory might be on NFS}
(remember that \diet cannot replicate the script before submission
because of access rights).

%****************************************************************************%
\section{Parallel systems}
%****************************************************************************%

Single parallel systems are surely the less deployed in actual
computing grids. They are usually composed of a frontale node where
clients log in, and from which they can log on numerous nodes and
execute their parallel jobs, without any kind of reservation. Some
problems occur with such a use of parallel resources: multiple
parallel tasks can share a single processor, hence delaying the
execution of all applications using it; during the deployment, the
application must at least check the connectivity of the resources; if
performance is wanted, some monitoring has to be performed by the
application.

%****************************************************************************%
\section{Batch system}
%****************************************************************************%

Generally, a parallel resource is managed by a batch system, and jobs
are submitted to a site queue. The batch system is responsible for
managing parallel jobs: it schedules each job and, it determines and
allocates the resources needed for its execution. 

There are many batch system, among which
Torque\footnote{\url{http://old.clusterresources.com/products/torque/}}
(a fork of
PSB\footnote{\url{http://www.clusterresources.com/pages/products/torque-resource-manager.php}}),
Loadleveler\footnote{\url{http://www-03.ibm.com/servers/eserver/clusters/software/loadleveler.html}}
(developped by IBM), SunGrid
Engine\footnote{\url{http://www.sun.com/software/gridware/}} (SGE,
developped by Sun), OAR\footnote{\url{http://oar.imag.fr}} (developped
at the IMAG lab). Each one implements its own language syntax (with
its own mnemonics), as well as its own scheduler. Jobs can generally
access the identity of the reserved nodes through a file during their
execution, and are assured to exclusively possess them.

%Diagram batch

%****************************************************************************%
\section{Client extended API}
%****************************************************************************%

Even if older client codes must be recompiled (because internal
structures have evolved), they don't necessarily need modifications.

\diet now provides to the client means to request exclusively
sequential services, parallel services, or let \diet choose the best
implementation of a problem for efficiency purposes (according to the
scheduling metric). The API now incorporates the calling functions to
perform the former, the latter being the default and accessed through
the traditional \verb!diet_call()! function:

\begin{lstlisting}[language=c,basewidth={.5em,.4em},fontadjust]
diet_error_t diet_parallel_call(diet_profile_t* profile) ;
diet_error_t diet_sequential_call(diet_profile_t* profile) ;
\end{lstlisting}
%****************************************************************************%
\section{Parallel and batch server extended API and configuration
file}
%****************************************************************************%

There are too many diverse scenarii about the communication and
execution of parallel applications: the code can be a MPI code or
composed of different interacting programs possibly launched via ssh
on every nodes; input and output files can use NFS if this file system
is present, or they can be splitted and uploaded to each node
participating to the calculus.

Then, we will see: what supplementary information has to be provided
in the server configuration file; how to write a submission script
from within a \sed; and how to record the parallel/batch service.

\subsection{Registering the service}

A server is mostly built like described in section~\ref{ch:server}. In
order to let the \sed know that the service defined within the profile
is a parallel one, the \sed programmer must use the function:

\begin{lstlisting}[language=c,basewidth={.5em,.4em},fontadjust]
void diet_profile_desc_set_parallel(diet_profile_desc_t* profile)
\end{lstlisting}

By default, a service is registered as sequential. Nevertheless, for
code readability reasons, we also give the pendant function to
explicitly register a sequantial service:

\begin{lstlisting}[language=c,basewidth={.5em,.4em},fontadjust]
void diet_profile_desc_set_sequential(diet_profile_desc_t* profile)
\end{lstlisting}

\underline{\bf Note:} a \sed only launches parallel services or none
at all, in which case it solves exclusively sequential problems. If
both kinds of jobs are tried to be recorded in a unique \sed, \diet
reports an error.

\subsection{Server configuration file}

\subsubsection{\bf Batch server} 

The programmer of a batch service available in a \sed has not to worry
about which batch system to submit, because \diet provides all the
mechanisms to transparently submit the job to them.

%FIXME: � quel point parle t-on d'elagi ?
\diet is able to submit batch scripts to numerous batch schedulers,
among which: "condor", "dqs", "loadleveler", "lsf", "pbs", "sge" or
"oar".
% FIXME: do we give shell, condor which are not batch? Do we give all
% even if only oar is tested?
The name of the batch scheduler managing the parallel resource where
the \sed is running has to be incorporated with the keyword
\verb!batchName! in the server configuration file. Only this makes the
\sed know how to submit a job correctly.

Furthermore, if there is no default queue, the \diet deployer must
also provide the queue on which jobs have to be submitted, with the
keyword \verb!batchQueue!.

For example, the server configuration file can contain the following
lines:

\begin{lstlisting}[language=bash,label=dietConfig.sh,basewidth={.5em,.4em},fontadjust]
batchName = oar
batchQueue = queue_9_13
\end{lstlisting}

\subsubsection{\bf Parallel server}

With the aim to not extend the \diet API too much, we consider that a
parallel server corresponds to an ordinary shell that submits the
generated script. In consequence, the server configuration file should
contain the following line

\begin{lstlisting}[language=bash,label=dietConfig.sh,basewidth={.5em,.4em},fontadjust]
batchName = shellscript
\end{lstlisting}

\subsection{Server API for writing services}

The writing of a service corresponding to a parallel or batch job is
very simple. The \sed programmer builds a shell script that he would
have normally used to execute the job, i.e., a script that must take
care of data replication and executable invocation.

In order for the service to be system independent, the \sed API
provides some meta-variables which can be used in the script. Note
only the first and second ones given below are correct {\it for
parallel jobs}, even if no error will occur if the \sed programmer use
the other ones:

\begin{itemize}
\item \verb!$DIET_NAME_FRONTALE!: frontale name
\item \verb!$DIET_USER_NBPROCS!: number of processors
\item \verb!$DIET_BATCH_NODESLIST!: list of reserved nodes
\item \verb!$DIET_BATCH_NBNODES!: number of reserved nodes
\item \verb!$DIET_BATCH_NODESFILE!: name of the file containing the
identity of the reserved nodes
\item \verb!$DIET_BATCH_JOBID!: batch job ID
\end{itemize}

Once the script written in a string, it is given as an argument to the
following function:
\begin{lstlisting}[language=bash,label=dietConfig.sh,basewidth={.5em,.4em},fontadjust]
int diet_submit_parallel(diet_profile_t* pb, char* script)
\end{lstlisting}
\subsection{Example of the client/server 'concatenation' problem}

There is a fully commented client/server example in
\verb!<diet_src>/src/examples/Batch! directory. It contains 3 servers:
one sequential, one parallel and one batch, and a single client. The
latter is configurable to simply ask for only sequential, or
explicitly parallel services, or to let \diet choose the best (by
default, 2 processors are used and the scheduling algorithm is
Round-Robin). We consequently give the MPI code which is called from
parallel (and batch) \sed (not reproduced here). Note that the user
{\it must change} some paths in the \sed codes, according to the site
where he deploys \diet. We reproduce it here.

%\parbox[b]{.5\textwidth}{
{%\twocolumn
  \tiny
  \lstinputlisting[title={\bf Synchronous client code},language=c,label=client.c,basewidth={.5em,.4em},fontadjust]{src/examples/Batch/client.c}
}

\newpage

{%\twocolumn
  \tiny
  \lstinputlisting[title={\bf Batch server code},language=c,basewidth={.5em,.4em},fontadjust]{src/examples/Batch/batch_server.c}
}

{%\twocolumn
  \tiny
  \lstinputlisting[title={\bf Parallel server code},language=c,basewidth={.5em,.4em},fontadjust]{src/examples/Batch/parallel_server.c}
}

{%\twocolumn
  \tiny
  \lstinputlisting[title={\bf Sequential server code},language=c,basewidth={.5em,.4em},fontadjust]{src/examples/Batch/sequential_server.c}
}

%\onecolumn

