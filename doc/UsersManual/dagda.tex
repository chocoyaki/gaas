%**
%*  @file  dagda.tex
%*  @brief   DIET User's Manual: DAGDA 
%*  @author  - Ga�l Le Mahec (Gael.le.mahec@ens-lyon.fr)
%*  @section Licence 
%*    |LICENSE|


\chapter{\dagda: Data Manager}
\label{ch:dagda}

\dagda (\textbf{D}ata \textbf{A}rrangement for \textbf{G}rid and
\textbf{D}istributed \textbf{A}pplications) is a new data manager for
\diet. \dagda offers to the \diet application developers a simple
and efficient way to manage the data. It was not designed to replace
the JuxMem extension but to be possibly coupled with it. In a future
work, \dagda will be divided in two parts: The \dagda data manager and
the \dagda data interface. The data interface will make interactions
between \dagda, JuxMem, FTP etc. and other data transfer/management
protocols. In this chapter, we will present the current version of
\dagda which is an alternative data manager for \diet with several
advanced data management features.

\section{Overview}

\dagda allows data explicit or implicit replications and advanced data
management on the grid. It was designed to be backward compatible with
previously developed applications for \diet which benefit
transparently of the data replications. Moreover, \dagda limits the
data size loaded in memory to a user-fixed value and avoids CORBA
errors when transmitting too large data regarding to the ORB
configuration. 

\dagda offers a new way to manage the data on \diet. The API allows
the application developer to replicate, move, add or delete a data to
be reused later or by another application. Each component of \diet can
interact with \dagda and the data manipulation can be done from a
client application, a server or an agent through a plug-in scheduler.

A \dagda component is associated to each element in a \diet platform (client,
Master Agent, Local Agent, SeD). These components are connected following the
\diet deployment topology. Figure~\ref{fig:DAGDAarch} shows how the \dagda and
\diet classical components are connected. Contrary to a \diet architecture,
each \dagda component has the same role. It can store, transfer or move a
data. The client's \dagda component is isolated of the architecture and
communicates only with the chosen SeDs \dagda components when necessary. When
searching for a data, \dagda uses its hierarchical topology to contact the data
managers. Among the data managers having one replicate of the data, \dagda
chooses the \textit{``best''} source to transfer it. To make this choice \dagda
uses some statistics collected from previous data transfers between the
nodes. By not using dynamic information, it is unsure that \dagda really chose
the ``best'' nodes for the transfers. In a future version, we will introduce some
facilities to estimate the time needed to transfer a data and to improve the
choice of a data stored on the grid. To do the data transfers, \dagda uses the
pull model: it is the destination node that ask for the data transfer.
\begin{figure}[h]
\centerline{\includegraphics[width=0.7\linewidth]{fig/dagdaArch}}
\caption{\dagda architecture in \diet.\label{fig:DAGDAarch}}
\end{figure}

Figure~\ref{fig:DAGDAarch} presents how \dagda manages the data when a
client submit a job. In this example, the client wants to use some
data stored on the grid and some personal data. He wants to obtain
some results and to store some others on the grid. Some of these
output data are already stored on the platform and they should be
updated after the job execution.
\begin{enumerate}
  \item The client sends a request to the Master Agent.
  \item The Master agent returns one or more \sed references.
  \item The client sends its request to the chosen node. The
    parameters data are identified by a unique ID and the problem
    profile contains a reference to the client's data manager.
  \item Receiving the request the \sed asks the client to transfer the
    data of the user and it asks to the \dagda architecture to obtain
    the persistent data already stored on the platform.
  \item The \sed executes the job. After the execution, the \sed stores
    the output data and it informs the client that the data are ready
    to be downloaded. It also asks the architecture to update the
    modified output data.
  \item The client upload its results and the data are updated on the
    nodes.
\end{enumerate}

%Using \dagda, each data stored on the platform can be replicated as many
%time as there is free space on the platform. To avoid \diet to take up all
%the storage ressources (disk or memory) of the nodes, \dagda allows to fix
%a limit to the data size that can be stored for each \dagda component. When
%a job needs a data, it asks \dagda for it using its unique ID and gets it
%from the ``best source''. \dagda uses some statistic informations to determine
%which node should be used for the transfers, but could be easily extended to
%use some dynamic metrology parameters. \dagda also offers a way to manage
%automatically the data by choosing a \textit{cache replacement algorithm}
%for each node. The user can choose a persistence mode for each replica:
%\begin{itemize}
%  \item[-] Persistent: The data is kept as long as possible on the node
%    but can be deleted when the node needs some storage resources.
%  \item[-] Sticky: The data will stay on the node until a user chooses
%    to delete it explicitly.
%\end{itemize}
%Then, the three cache replacement algorithms implemented in \dagda work as
%follows:
%\begin{itemize}
%  \item \emph{Least Recently Used} (LRU): The least recently used persistent
%    data of sufficient size is deleted.
%  \item \emph{Least Frequently Used} (LFU): The least frequently used
%    persistent data of sufficient size is deleted.
%  \item \emph{First In First Out} (FIFO): Among the persistent data of
%    sufficient size, the \textit{oldest} is deleted.
%\end{itemize}
%None of the above algorithms controls if another replica of the data
%exists on the platform before to delete it. The only way to ensure
%at least one replica of a data leaves on the platform is to declare
%one of them as \emph{sticky} data.

%Extending the \diet API, \dagda allows to explicitly manage the data in the
%platform by communicating directly with the \dagda hierarchy. However, \dagda
%does not deal with the replicas consistency and the data availability. The
%user is responsible of what he does with the data. Even if \dagda offers
%a way to update a persistent data, if a job or a user modifies the data
%while the updating process is executing, some side effects can appear.

\newcommand{\tabCell}[2]{%
  \begin{minipage}{#1}
    \vspace*{1mm}
    \scriptsize #2
    \vspace*{1mm}
  \end{minipage}
}
\section{The \dagda configuration options}
\dagda introduces new configuration options that can be defined for
all the \dagda components. None of these options are mandatory to use
\dagda. Figure \ref{fig:DAGDAoptions} presents all the \dagda
available options, their meaning and default values.
\begin{figure}[h]
\begin{tabular}{|l|l|l|c|c|c|}
\hline
%& & & & & \\
\tabCell{2.9cm}{\vspace*{0.5cm}\centering\textbf{Option}} &
\tabCell{5cm}{\vspace*{0.5cm}\centering\textbf{Description}} &
\tabCell{4cm}{\vspace*{0.5cm}\centering\textbf{Default value}} &
\rotatebox{270}{\centering\bf Client} & \rotatebox{270}{\centering\bf Agent } &
\rotatebox{270}{\centering\bf SeD} \\
%& & & & &\\
\hline
%& & & & &\\
storageDirectory &
\tabCell{5cm}{The directory on which \dagda will store the data files} &
\tabCell{4cm}{The \textit{/tmp} directory.} &
\ding{52} & \ding{52} & \ding{52} \\
%& & & & &\\
\hline
maxMsgSize &
\tabCell{5cm}{The maximum size of a CORBA message sent by \dagda.} &
\tabCell{4cm}{The omniORB \textit{giopMaxMsgSize} size.} &
\ding{52} & \ding{52} & \ding{52} \\
%& & & & &\\
\hline
maxDiskSpace &
\tabCell{5cm}{The maximum disk space used by \dagda to store the data. If set
to 0, \dagda will not take care of the disk usage.} &
\tabCell{4cm}{The available disk space on the disk partition chosen by the
  \textit{storageDirectory} option.} &
\ding{52} & \ding{52} & \ding{52} \\
\hline
maxMemSpace &
\tabCell{5cm}{The maximum memory space used by \dagda to store the data. If set
to 0, \dagda will not take care of the memory usage.} &
\tabCell{4cm}{No maximum memory usage is set. Same effect than to choose 0.} &
\ding{52} & \ding{52} & \ding{52} \\
\hline
cacheAlgorithm &
\tabCell{5cm}{The cache replacement algorithm used when \dagda needs more space
to store a data. Possible values are: \textit{LRU, LFU, FIFO}} &
\tabCell{4cm}{No cache replacement algorithm. \dagda never replace a data by
another one.} &
\ding{52} & \ding{52} & \ding{52} \\
\hline
shareFiles &
\tabCell{5cm}{The \dagda component shares its file data with all its children
(when the path is accessible by them, for example, if the storage directory is
on a NFS partition). Value can be 0 or 1.} &
\tabCell{4cm}{No file sharing - 0} &
\ding{56} & \ding{52} & \ding{56} \\
\hline
dataBackupFile &
\tabCell{5cm}{The path to the file that will be used when \dagda save all its
stored data/data path when asked by the user (Checkpointing).} &
\tabCell{4cm}{No checkpointing is possible.} &
\ding{56} & \ding{52} & \ding{52} \\
\hline
restoreOnStart &
\tabCell{5cm}{\dagda will load the \textit{dataBackupFile} file at start and
restore all the data recorded at the last checkpointing event. Possible values
are 0 or 1.} &
\tabCell{4cm}{No file loading on start - 0} &
\ding{56} & \ding{52} & \ding{52} \\
\hline
\end{tabular}
\caption{\dagda configuration options}
\label{fig:DAGDAoptions}
\end{figure}

\section{Cache replacement algorithm}
When a data is replicated on a site, it is possible that not enough
disk/memory space is available. In that case, \dagda allows to choose
a strategy to delete a persistent data. Only a simple persistent data
can be deleted, the sticky ones are never deleted by the chosen
algorithm. \dagda offers three algorithms to manage the cache
replacement:
\begin{itemize}
  \item LRU: The least recently used persistent data of sufficient
    size is deleted.
  \item LFU: The least frequently used persistent data of sufficient
    size is deleted.
  \item FIFO:  Among the persistent data of sufficient size, the
    \textit{oldest} is deleted.
\end{itemize}

\section{The \dagda API}
By compiling \diet with the \dagda extension activated, the
\textit{DIET\_Dagda.h} file is installed on the \diet include
directory.  This file contains some data management functions and
macros.
\subsection{Note on the memory management}
On the \sed side, \dagda and the \sed share the same data pointers, that
means that if the pointer is a local variable reference, when \dagda
will use the data, it will read an unallocated variable. The users
should allways allocate the data with a \textit{``malloc''/``calloc''} or
\textit{``new''} call on the \sed and agent sides. Because \dagda takes
the control of the data pointer, there is no risk of memory leak even
if the service allocate a new pointer at each call. The data lifetime
is managed by \dagda and the data will be freed according to its
persistence mode.\\[4mm]
\begin{minipage}{2cm}
  \centering
  \textbf{{\Huge \Biohazard}}
\end{minipage}
\begin{minipage}{\textwidth - 2cm}
\textbf{On the \sed and agent sides, \dagda takes the control of the data
pointers. To free a data may cause major bugs which could be very hard to
find. The users could only free a \diet data on the client side after the end of
a transfer.}
\end{minipage}

\subsection{Synchronous data transfers}
All of the following functions return at the end of the transfer or
if an error occured. They all return an integer value: 0 if the
operation succeeds, another value if it fails.

\subsubsection{\dagda \textit{put} data macros/functions}
\label{sec:syncPutFunctions}
The following functions put a data on the \dagda hierarchy to be used
later.  The last parameter is always a pointer to a C-string which
will be initialized with a pointer to the ID string of the data. This
string is allocated by \dagda and can be freed when the user does not
need it anymore.  The first parameter is always a pointer to the
data: for a scalar value a pointer on the data, for a vector, matrix
or string, a pointer on the first element of the data. The
\textit{``value''} argument for a file is a C-string containing the path
of this file. The persistence mode for a data managed by \dagda should
allways be DIET\_PERSISTENT or DIET\_STICKY.  The VOLATILE and
*\_RETURN modes do not make sense in this data management context.

% \begin{itemize}
%   \item Synchronous: The called function returns when the transfer is ended.
%   \item Asynchronous with control: The called function returns immediately.
%     The user can wait the end of the transfer by calling to a \dagda wait
%     function.
%   \item Asynchronous without control: The called function returns immediately.
%     The user cannot wait the end of the transfer. These functions should only
%     be used on the SeD or Agent sides.
% \end{itemize}
% \subsubsection{Synchronous data transfers.}
\begin{itemize}
  \item[-] \verb#dagda_put_scalar(void* value, diet_base_type_t base_type,#\\
           \verb#                 diet_persistence_mode_t mode, char** ID)#:\\
           This macro adds to the platform the scalar data of type
           \textit{``base\_type''} pointed by \textit{``value''} with the
           persistence mode \textit{``mode''} (DIET\_PERSISTENT or DIET\_STICKY)
           and initializes \textit{``*ID''} with the ID of the data.
  \item[-] \verb#dagda_put_vector(void* value, diet_base_type_t base_type,#\\
           \verb#                 diet_persistent_mode_t mode, size_t size, char** ID)#:\\
           This macro adds to the platform the vector of \textit{``size''}
           \textit{``base\_type''} elements pointed by \textit{``value''} with the
           persistence mode \textit{``mode''} and stores the data ID in
           \textit{``ID''}.
  \item[-] \verb#dagda_put_matrix(void* value, diet_base_type_t base_type,#\\
         \verb#                 diet_persistence_mode_t mode, size_t nb_rows,#\\
         \verb#                 size_t nb_cols, diet_matrix_order_t order, char** ID)#:\\
           This macro adds to the platform the \textit{``base\_type''} matrix of
           dimension \textit{``nb\_rows''} $\times$ \textit{``nb\_cols''} stored in
           \textit{``order''} order. The data ID is stored on \textit{``ID''}.
  \item[-] \verb#dagda_put_string(char* value, diet_persistence_mode_t mode, char** ID)#:\\
           This macro adds to the platform the string pointed by
           \textit{``value''} with the persistence mode \textit{``mode''} and
           stores the data ID into \textit{``ID''}.
  \item[-] \verb#dagda_put_file(char* path, diet_persistence_mode_t mode, char**ID)#:\\
           This macro adds the file of path \textit{``path''} with the persistence
           mode \textit{``mode''} to the platform and stores the data ID into
           \textit{``ID''}
 \end{itemize}

\subsubsection{\dagda \textit{get} data macros/functions}
\label{sec:syncGetFunctions}
The following API functions are defined to obtain a data from \dagda
using its ID:
\begin{itemize}
  \item[-] \verb#dagda_get_scalar(char* ID, void** value,#\\
           \verb#                 diet_base_type_t* base_type)#:\\
    The scalar value using the ID \textit{``ID''} is obtained from
    \dagda and the \textit{``value''} argument is initialized with a
    pointer to the data.  The \textit{``base\_type''} pointer content is
    set to the data base type. This last parameter is optional and
    can be set to NULL if the user does not want to get the
    \textit{``base\_type''} value.

  \item[-] \verb#dagda_get_vector(char* ID, void** value,#\\
           \verb#                 diet_base_type_t* base_type, size_t* size)#:\\
    The vector using the ID \textit{``ID''} is obtained from \dagda. The
    \textit{``value''} argument is initialized with a pointer to the
    first vector element. The \textit{``base\_type''} content are
    initialized with the base type and size of the vector. These two
    parameters can be set to NULL if the user does not take care about
    it.
  \item[-] \verb#dagda_get_matrix(char* ID, void** value,#\\
           \verb#                 diet_base_type_t* base_type, size_t* nb_r,#\\
           \verb#                 size_t* nb_c, diet_matrix_order_t* order)#:\\
    The matrix using the ID \textit{``ID''} is obtained from \dagda. The
    \textit{``value''} argument is initialized with a pointer to the
    first matrix element. The \textit{``base\_type''}, \textit{``nb\_r''},
    \textit{``nb\_c''} and \textit{``order''} arguments contents are
    repectively set to the base type of the matrix, the number of
    rows, the number of columns and the matrix order. All of these
    parameters can be set to NULL if the user does not take care about
    it.
  \item[-] \verb#dagda_get_string(char* ID, char** value)#:\\
    The string of ID \textit{``ID''} is obtained from \dagda and the
    \textit{value} content is set to a pointer on the first string character.
  \item[-] \verb#dagda_get_file(char* ID, char** path)#:\\
    The file of ID \textit{``ID''} is obtained from \dagda and the
    \textit{``path''} content is set to a pointer on the first path string
    character.
\end{itemize}

\subsection{Asynchronous data transfers}
With \dagda, there are two ways to manage the asynchronous data
transfers, depending of the data usage:
\begin{itemize}
  \item With end-of-transfer control: \dagda maintains a reference to
    the transfer thread. It only releases this reference after a call
    to the corresponding waiting function. The client developer should
    always use these functions, that's why a data ID is only returned
    by the \textit{``dagda\_wait\_*''} and
    \textit{``dagda\_wait\_data\_ID''} functions.
  \item Without end-of-transfer control: The data is loaded from/to
    the \dagda hierarchy without the possibility to wait for the end
    of the transfer.  These functions should only be called from an
    agent plugin scheduler, a \sed plugin scheduler or a \sed if the
    data transfer without usage of the data is one of the objectives
    of the called service. The data adding functions without control
    should be used very carefully because there is no way to be sure
    the data transfer is achieved or even started.    
\end{itemize}
With asynchronous transfers, the user should take care of the data
lifetime because \dagda does not duplicate the data pointed by the
passed pointer.  For example, if the program uses a local variable
reference to add a data to the \dagda hierarchy and goes out of the
variable scope, a crash could occured because the data pointer could
be freed by the system before \dagda has finished to read it.
\subsubsection{\dagda asynchronous \textit{put} macros/functions}
The arguments to these functions are the same than for the synchronous
ones.  See Section \ref{sec:syncPutFunctions} for more details. All of
these functions return a reference to the data transfer which is an
unsigned int. This value will be passed to the
\textit{``dagda\_wait\_data\_ID''} function.
\begin{itemize}
\item[-] \verb#dagda_put_scalar_async(void* value, diet_base_type_t base_type,#\\
         \verb#                       diet_persistence_mode_t mode)#
\item[-] \verb#dagda_put_vector_async(void* value, diet_base_type_t base_type,#\\
         \verb#                       diet_persistence_mode_t mode, size_t size)#
\item[-] \verb#dagda_put_matrix_async(void* value, diet_base_type_t base_type,#\\
         \verb#                       diet_persistence_mode_t mode, size_t nb_rows,#\\
         \verb#                       size_t nb_cols, diet_matrix_order_t order)#
\item[-] \verb#dagda_put_string_async(char* value, diet_persistence_mode_t mode)#
\item[-] \verb#dagda_put_file_async(char* path, diet_persistence_mode_t mode)#
\end{itemize}
After a call to one of these functions, the user can obtain the data ID by
calling the \textit{``dagda\_wait\_data\_ID''} function with a transfer
reference.
\begin{itemize}
  \item[-] \verb#dagda_wait_data_ID(unsigned int transferRef, char** ID)#:\\
    The \textit{``transferRef''} argument is the value returned by a
    \textit{``dagda\_put\_*\_async''} function. The \textit{``ID''} content will
    be initialized to a pointer on the data ID.
\end{itemize}

\subsubsection{\dagda asynchronous \textit{get} macros/functions}
The only argument needed for one of these functions is the data ID.
All of these functions return a reference to the data transfer which
is an unsigned int. This value will be passed to the corresponding
\textit{``dagda\_wait\_*''} functions described later.
\begin{itemize}
\item[-] \verb#dagda_get_scalar_async(char* ID)#
\item[-] \verb#dagda_get_vector_async(char* ID)#
\item[-] \verb#dagda_get_matrix_async(char* ID)#
\item[-] \verb#dagda_get_string_async(char* ID)#
\item[-] \verb#dagda_get_file_async(char* ID)#
\end{itemize}

After asking for an asynchronous transfer, the user has to wait the 
end by calling the corresponding \textit{``dagda\_wait\_*''} function.
 The arguments of these functions are the same than for the synchronous
 \textit{``dagda\_get\_*''} functions. 
 See Section \ref{sec:syncGetFunctions} for more details.

\begin{itemize}
\item[-] \verb#dagda_wait_scalar(unsigned int transferRef, void** value,#\\
         \verb#                  diet_base_type_t* base_type)#
\item[-] \verb#dagda_wait_vector(unsigned int transferRef, void** value,#\\
         \verb#                  diet_base_type_t* base_type, size_t* size)#
\item[-] \verb#dagda_wait_matrix(unsigned int transferRef, void** value,#\\
         \verb#                  diet_base_type_t* base_type, size_t* nb_r,#\\
         \verb#                  size_t* nb_c, diet_matrix_order_t* order)#
\item[-] \verb#dagda_wait_string(unsigned int transferRef, char** value)#
\item[-] \verb#dagda_wait_file(unsigned int transferRef, char** path)#
\end{itemize}

A plugin scheduler developer often wants to make an
asynchronous data transfer to the local \diet node. Problems can arise if you
want to wait the completion of the tranfer before returning. 
But with the previously defined functions, \dagda maintains a reference to
the transfer thread which will be released after a call to the waiting
function. To avoid \dagda to keep infinitely these references, the
user should call the \textit{``dagda\_load\_*''} functions instead of
the \textit{``dagda\_get\_*\_async''} ones.

\begin{itemize}
\item[-] \verb#dagda_load_scalar(char* ID)#
\item[-] \verb#dagda_load_vector(char* ID)#
\item[-] \verb#dagda_load_matrix(char* ID)#
\item[-] \verb#dagda_load_string(char* ID)#
\item[-] \verb#dagda_load_file(char* ID)#
\end{itemize}

\subsection{Data checkpointing with \dagda}
\dagda allows the \sed administrator to choose a file where \dagda will
store all the data it's managing. When a \sed has a configured and valid
path name to a backup file (\textit{``dataBackupFile''} option in the
configuration file), a client can ask to the agents or \seds \dagda
components to save the data.\\

The \verb#dagda_save_platform()# function, which can only be called
from a client, records all the data managed by the agents' or \seds'
\dagda components that allow it.\\ Then, the \textit{``restoreOnStart''}
configuration file option asks to the \dagda component to restore the
data stored on the \textit{``dataBackupFile''} file when the component
starts. This mechanism allows to stop the \diet platform for a while
and restart it conserving the same data distribution.

\subsection{Create data ID aliases}
For many applications using large sets of data shared by several
users, to use an automatically generated ID to retrieve a data is
difficult or even impossible. \dagda allows the user to define data
aliases, using human readable and expressive strings to retrieve a
data ID. Two functions are defined to do it:
\begin{itemize}
\item[-]
  \verb#dagda_data_alias(const char* id, const char* alias)#:\\ Tries
  to associate \textit{``alias''} to \textit{``id''}. If the alias is
  already defined, returns a non zero value. A data can have several
  aliases but an alias is always associated to only one data.
\item[-]
  \verb#dagda_id_from_alias(const char* alias, char** id)#:\\ This
  function tries to retrieve the data id associated to the alias.
\end{itemize}

\subsection{Data replication}
After a data has been added to the \dagda hierarchy, the users can
choose to replicate it explicitely on one or several \diet nodes. With
the current \dagda version, we allow to choose the nodes where the
data will be replicated by hostname or \dagda component ID. In future
developments, it will be possible to select the nodes differently. To
maintain backward compatibility, the replication function uses a
C-string to define the replication rule.
\begin{itemize}
\item[-] \verb#dagda_replicate_data(const char* id, const char* rule)#
\end{itemize}
The replication rule is defined as follows:\\
``$<$Pattern target$>$:$<$identification pattern$>$:$<$Capacity overflow
behavior$>$''\\
\begin{itemize}
\item The \textit{pattern target} can be ``ID'' or ``host''.
\item The \textit{identification pattern} can contain some
  \textit{wildcards} characters. (for example
  \textit{``*.lyon.grid5000.fr''} is a valid pattern.
\item The \textit{capacity overflow behavior} can be ``replace'' or
  ``noreplace''.  ``replace'' means the cache replacement algorithm will
  be used if available on the target node (a data could be deleted
  from the node to leave space to store the new one). ``noreplace''
  means that the data will be replicated on the node if and only if
  there is enough storage capacity on it.
\end{itemize}

For example, \textit{``host:capricorne-*.lyon.*:replace''} is a valid
replication rule.

\section{On the correct usage of \dagda}
Some things to keep in mind when using \dagda as data manager for
\diet:
\begin{itemize}
\item All the data managed by \dagda are entirely managed by \dagda:
  The user don't have to free them. \dagda avoids memory leaks, so the user
  does not have to worry about the memory management for the data
  managed by \dagda.
\item When using more than one \dagda component on a node, the user
  should define a different storage directory for each component. For
  example, the Master Agent and one \sed are launched on the same
  computer: the user can define the storage directory of the Master
  Agent as ``/tmp/MA'' and the one for the \sed as ``/tmp/SeD1''. Do
  not forget to create the directories before to use \dagda. This tip
  avoids many bugs which are really hard to find.
\item The \dagda API can be used to transfer the parameters of a
  service, but it should not be used as this. If an application needs
  a data which is only on the client, the user should transmit it
  through the profile.  The \dagda API should be used to share,
  replicate or retrieve an existing data. Using the API allows the
  user to optimize their applications, not to proceed to a diet\_call
  even if it works fine. Indeed, the \dagda client component is not
  linked to the \diet hierarchy, so using the API to add a data and
  then to use it as a profile parameter makes \dagda to do additional
  and useless transfers.
\item \dagda can be used without any configuration, but it is always
  a good idea to define all the \dagda parameters in the configuration
  files.
\end{itemize}

For any comment or bug report on \dagda, please contact G. Le Mahec at
the following e-mail address: \url{gael.le.mahec@u-picardie.fr}.

\section{Future works}
The next version of \dagda will allow the users to develop their own
cache replacement algorithms and network capacity measurements
methods. \dagda will be separated in two parts: A data management
interface and the \dagda data manager itself. \dagda will implement
the GridRPC data management API extension.

%%% Local Variables:
%%% mode: latex
%%% ispell-local-dictionary: "american"
%%% mode: flyspell
%%% fill-column: 80
%%% End:
