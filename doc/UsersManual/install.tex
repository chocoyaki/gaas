%**
%*  @file  install.tex
%*  @brief   DIET User's Manual installing chapter file 
%*  @author  - Eddy CARON (Eddy.Caron@ens-lyon.fr)
%*           - Pushpinder Kaur Chouhan (Pushpinder.Kaur.Chouhan@ens-lyon.fr)
%*           - Philippe COMBES (Philippe.Combes@ens-lyon.fr) 
%*  @section Licence 
%*    |LICENSE|


\chapter{\diet installation}
\label{ch:installing}

%====[ Dependencies ]==========================================================
\section{Dependencies}
\label{sec:dependencies}

\subsection{General remarks on \diet platform dependencies}

\diet is itself written in C/C++ and for limited parts in java. \diet is based
on CORBA and thus depends on the chosen CORBA implementation.  Additionally,
some of \diet extensions make a strong use of libraries themselves written in
C/C++ and java. Thus, we could expect \diet to be effective on any platform
offering decent version of such compilers.

\diet undergoes daily regression tests (see
\url{http://cdash.inria.fr/CDash/index.php?project=DIET}) on various hardwares,
a couple of Un*x based operating systems (under different distributions), MacOSX and AIX,
and mainly with GCC. But thanks to users reports (punctual deployments and
special tests conduced before every release), \diet is known to be effective on
a wide range of platforms.

Nevertheless, if you encounter installation difficulties don't hesitate to post
on \diet's users mailing list: \verb+diet-usr@listes.ens-lyon.fr+ (for the
archives refer to \url{http://graal.ens-lyon.fr/DIET/mail-lists.html}).  If you
find a bug in \diet, please don't hesitate to submit a bug report on
\url{http://graal.ens-lyon.fr/bugzilla}. If you have multiple bugs to report,
please make multiple submissions, rather than submitting multiple bugs in a
single report.

\subsection{Hardware dependencies}
\diet is fully tested on Linux/i386 and Linux/i686 platforms. \diet is known to
be effective on Linux/Sparc, Linux/i64, Linux/amd64, Linux/Alpha,
Linux/PowerPC, AIX/PowerPC, MacOS/PowerPC and Windows XP(Cygwin)/i386 platforms. 
At some point in \diet history, \diet used to be tested on the Solaris/Sparc platform...

\subsection{Supported compilers}
\diet is supported on GCC with versions ranging from 3.2.X to 4.3.4.  Note that
due to omniORB 4 (see \ref{sec:software_dependencies}) requirements towards
thread-safe management of exception handling, compiling \diet with GCC
%\verb+gcc+
requires at least the version 2.96.
%\verb+gcc-2.96+. 
\diet is also supported on XL compiler (IBM) and Intel compiler.

\subsection{Operating system dependencies}
\diet is fully tested on Linux [with varying distributions like Debian, Red Hat
  Enterprise Linux (REL-ES-3), Fedora Core (5), gentoo], on AIX (5.3) on MacOSX
(Darwin 8) and on Windows (Cygwin 1.5.25 and Cygwin 1.7.1).

\subsection{Software dependencies}
\label{sec:software_dependencies}

As explained in Section~\ref{sec:CORBA}, CORBA is used for all communications
inside the platform.  The implementations of CORBA currently supported in \diet
is \textbf{omniORB 4} which itself depends on \textbf{Python}.

\noindent 
\textbf{NB:} We have noticed that some problems occur with \textbf{Python 2.3}:
the C++ code generated by idl could not be compiled. It has been patched in
\diet, but some warnings may still appear.

\textbf{omniORB 4} itself also depends on \textbf{OpenSSL} in case of you wish to
secure your \diet platform. In order to deploy CORBA services with omniORB, a configuration file and a log
directory are required: see Section \ref{sec:CORBA_services} for a complete
description of the services.  Their paths can be given to omniORB either at
runtime (through the well-known environment variables
\texttt{\$OMNIORB\_CONFIG} and \texttt{\$OMNINAMES\_LOGDIR}), and/or at omniORB
compile time (with the \linebreak\texttt{--with-omniORB-config} and
\texttt{--with-omniNames-logdir} options.)  Some examples provided in the \diet
sources depend on the BLAS and \scalapack\ libraries. However the compilation
of those BLAS and \scalapack\ dependent examples are optional.

%====[ Compilation ]===========================================================
\section{Compiling the platform}
\label{sec:compil_platform}

\diet compilation process moved away from the traditional \verb+autotools+ way
of things to a tool named \verb+cmake+ (mainly to benefit from \verb+cmake+'s
built-in regression tests mechanism).

Before compiling \diet itself, first install the above mentioned (cf
Section~\ref{sec:software_dependencies}) dependencies.  Then untar the \diet
archive and change current directory to its root directory.

%==================
\subsection{Obtaining and installing cmake per se}
\diet requires using \verb+cmake+ at least version \verb+2.4.3+. For
many popular distributions \verb+cmake+ is incorporated by default or
at least \verb+apt-get+ (or whatever your distro package installer
might be) is \verb+cmake+ aware. Still, in case you need to install
an up-to-date version \verb+cmake+'s official site distributes many
binary versions (alas packaged as tarballs) which are made available
at \url{http://www.cmake.org/HTML/Download.html}. Optionally, you
can download the sources and recompile them: this simple process
(\verb+./bootstrap; make; make install+) is described at
\url{http://www.cmake.org/HTML/Install.html}.

%==================
\subsection{Configuring \diet's compilation: cmake quick introduction}
If you are already experienced with \verb+cmake+ then using it to
compile \diet should provide no surprise. \diet respects
\verb+cmake+'s best practices \eg by clearly separating the source
tree from the binary tree (or compile tree), by exposing the main
configuration optional flag variables prefixed with \verb+DIET_+ (and
by hiding away the technical variables) and by not postponing
configuration difficulties (in particular the handling of external
dependencies like libraries) to compile stage.

\verb+cmake+ classically provides two ways for setting configuration
parameters in order to generate the makefiles in the form of two
commands \verb+ccmake+ and \verb+cmake+ (the first one has an extra
"c" character):
\begin{description}
\item{\verb+ccmake [options] <path-to-source>+}\\ in order to specify
  the parameters interactively through a GUI interface
\item{\verb+cmake [options] <path-to-source> [-D<var>:<type>=<value>]+}\\ in
  order to define the parameters with the \verb+-D+ flag directly from
  the command line.
\end{description}
In the above syntax description of both commands, {\verb+<path-to-source>+}
specifies a path to the top level of the source tree (\ie the directory where
the top level CMakeLists.txt file is to be encountered). Also the current
working directory will be used as the root of the build tree for the project
(out of source building is generally encouraged especially when working on a
CVS tree).

\noindent
Here is a short list of \verb+cmake+ internal parameters that are worth
mentioning:
\begin{itemize}
\item
  \verb+CMAKE_BUILD_TYPE+ controls the type of build mode among which
  \verb+Debug+ will produce binaries and libraries with the debugging
  information
\item
   \verb+CMAKE_VERBOSE_MAKEFILE+ is a Boolean parameter which when set to ON
   will generate makefiles without the .SILENT directive. This is useful for
   watching the invoked commands and their arguments in case things go wrong.
\item
   \verb+CMAKE_C[XX]_FLAGS*+ is a family of parameters used for the
   setting and the customization of various C/C++ compiler options.
\item
   \verb+CMAKE_INSTALL_PREFIX+ variable defines the location of the install
   directory (defaulted to \verb+/usr/local+ on Un*x).  This is cmake's
   portable equivalent of the autotools configure's \texttt{--prefix=} option.
\end{itemize}
%
Eventually, here is a short list of \verb+ccmake+ interface tips:
\begin{itemize}
\item
  when lost, look at the bottom lines of the interface which always summarizes
  \verb+ccmake+'s most pertinent options (corresponding keyboard shortcuts)
  depending on your current context
\item
  hitting the "h" key will direct you \verb+ccmake+ embedded tutorial and a
  list of keyboard shortcuts (as mentioned in the bottom lines, hit "e" to
  exit)
\item
  up/down navigation among parameter items can be achieved with the up/down
  arrows
\item
  when on a parameter item, the line in inverted colors (close above the bottom
  of the screen) contains a short description of the selected parameter as well
  as the set of possible/recommended values
\item
  toggling of boolean parameters is made with enter
\item
  press \verb+enter+ to edit path variables
\item
  when editing a \verb+PATH+ typed parameter the \verb+TAB+ keyboard shortcut
  provides an emacs-like (or bash-like) automatic path completion.
\item
  toggling of advanced mode (press "t") reveals hidden parameters
\end{itemize}
 
%==================
\subsection{A ccmake walk-through for the impatients}

Assume that \verb+CVS_DIET_HOME+ represents a path to the top level directory
of \diet sources.  This \diet sources directories tree can be obtained by \diet
users by expanding the \diet current source level distribution tarball. But for
the \diet developers this directories tree simply corresponds to the directory
GRAAL/devel/diet/diet of a cvs checkout of the \diet sources hierarchy.
Additionally, assume we created a build tree directory and \verb+cd+ to it (in
the example below we chose \verb+CVS_DIET_HOME/Bin+ as build tree, but feel
free to follow your conventions):
\begin{itemize}
\item
  \verb+cd CVS_DIET_HOME/Bin+
\item
  \verb+ccmake ..+ to enter the GUI
  \begin{itemize}
  \item press \verb+c+ (equivalent of bootstrap.sh of the autotools)
  \item toggle the desired options \eg \verb+DIET_BUILD_EXAMPLES+. 
  \item specify the \verb+CMAKE_INSTALL_PREFIX+ parameter (if you wish to
    install in a directory different from \verb+/usr/local+)
  \item press \verb+c+ again, for checking required dependencies
  \item check all the parameters preceded with the * (star) character
    whose value was automatically retrieved by \verb+cmake+.
  \item provide the required information \ie fill in the proper values for all
    parameters whose value is terminated by NOT-FOUND
  \item iterate the above process of parameter checking, toggle/specification
    and configuration until all configuration information is satisfied
  \item press \verb+g+ to generate the makefile
  \item press \verb+q+ to exit ccmake
  \end{itemize}
\item
  \verb+make+ in order to classically launch the compilation process
\item
  \verb+make install+ when installation is required
\end{itemize}

%==================
\subsection{\diet's main configuration flags}

Here are the main configuration flags:
\begin{itemize}
\item
  \verb+OMNIORB4_DIR+ is the path to the omniORB4 installation
  directory (only relevant when omniORB4 was not installed in
  /usr/local).\\ Example:
  \verb+cmake .. -DOMNIORB4_DIR:PATH=$HOME/local/omniORB-4.1.5+

\item
  \verb+DIET_BUILD_EXAMPLES+ activates the compilation of a set of
  general client/server examples. Note that some specific examples
  (\eg \verb+DIET_BUILD_BLAS_EXAMPLES+) require some additional flag
  to be activated too.

\item
  \verb+DIET_BUILD_LIBRARIES+ which is enabled by default, activates the
  compilation of the \diet libraries. Disabling this option is only useful if
  you wish to restrict the compilation to the construction of the
  documentation.
\end{itemize}

%==================
\subsection{\diet's extensions configuration flags}

\diet has many extensions (some of them are still) experimental. These
extensions most often rely on external packages that need to be
pre-installed. One should notice that some of those extensions offer concurrent
functionalities. This explains the usage of configuration flags in order to
obtain the compilation of the desired extensions.

\begin{itemize}
\item
  \verb+DIET_BUILD_BLAS_EXAMPLES+ option activates the compilation of
  the BLAS based \diet examples, as a sub-module of examples.  The
  BLAS~\footnote{\url{http://www.netlib.org/blas/}} (Basic Linear
  Algebra Subprograms) are high quality ``building block'' routines
  for performing basic vector and matrix operations.  Level 1 BLAS do
  vector-vector operations, Level 2 BLAS do matrix-vector operations,
  and Level 3 BLAS do matrix-matrix operations.  Because the BLAS are
  efficient, portable, and widely available, they're commonly used in
  the development of high quality linear algebra software. \diet uses
  BLAS to build demonstration examples of client/server.  Note that
  the option \verb+DIET_BUILD_BLAS_EXAMPLES+ can only be effective
  when \verb+DIET_BUILD_EXAMPLES+ is enabled.
  \verb+DIET_BUILD_BLAS_EXAMPLES+ is disabled by default.

\item
  \verb+DIET_USE_ALT_BATCH+ enables the transparent submission to
  batch servers. See Chapter~\ref{chapter:parallelSubmission} for more
  details.

\item
  \verb+DIET_USE_WORKFLOW+ enables the support of workflow. For the support
  of workflows inside \diet, Xerces and Xqilla libraries are mandatory 
  (see \url{http://xerces.apache.org/xerces-c/} and
  \url{http://xqilla.sourceforge.net/HomePage}). For more details about the
  workflow support in \diet see Chapter \ref{ch:workflows}.

\item
  \label{sec:multimainstall}
  \verb+DIET_WITH_MULTI_MA+ activates the so called MULTI Master Agent
  support which allows the user to connect several MA for them to act
  as bounded.
  When this option is activated, such a bounded MA is allowed to search
  for a \sed into the MA hierarchies it is connected to.

\item
  \verb+DIET_WITH_STATISTICS+ enables the generation of statistics logs.
  The logs can be obtained for any element in the DIET hierarchy. To do
  so, you have to define the \texttt{DIET\_STAT\_FILE\-\_NAME} environment variable.
  For instance : \\
\centerline{\texttt{export DIET\_STAT\_FILE\_NAME=/tmp/client}}
then calling
  a client will generate the statistics for the client in the /tmp/client
  file.

\item
  \verb+DIET_USE_USERSCHED+ allows using custom plugin schedulers at the \diet
  agents level. See chapter~\ref{ch:plugin}.

\item
  \label{sec:securityinstall}
  \verb+DIET_USE_SECURITY+ enables ssl communications between DIET entities
  (agents, SeDs and clients). This extension uses the SSL part of OmniORB,
  so you need the library omnisslTP to be installed. In order to do that
  OmniORB should be compiled on an environment where OpenSSL is available.
  Please refer to the OmniORB doc if necessary\footnote{Chapter 8.9 of
  \url{http://omniorb.sourceforge.net/omni41/omniORB/omniORB008.html}}.
  To run \diet with security on, keys and certificates will be specified in the
  configuration files of each entities (see Appendix~\ref{ch:appendix} page~\pageref{appendix_ssl})

\end{itemize}

%==================
\subsection{\diet's advanced configuration flags}
\noindent

Eventually, some configuration flags control the general result of the
compilation or some developers extensions:
\begin{itemize}
\item
  \verb+BUILD_TESTING+ is a conventional variable (which is not a cmake
  internal variable) which specifies that the regression tests should also be
  compiled.

\item
  \verb+BUILD_SHARED_LIBS+ is a cmake internal variable which
  specifies whether the libraries should be dynamics as opposed to
  static (on Mac system this option is automatically set to \verb!ON!,
  as static compilation of binaries seems to be forbidden on these
  systems)

% \fixme{S�rement � remplacer par l'�quivalent pour CDash ou � enlever non ?}
\item
  \verb+Maintainer+ By default cmake offers four different build modes
  that one toggles by positioning \verb+CMAKE_BUILD_TYPE+ built-in
  variable (to \verb+Debug+, \verb+Release+, \verb+RelWithDebInfo+ and
  \verb+MinSizeRel+). \verb+Maintainer+ is an additional mode which
  fulfills two basic needs of the task of the maintainer of \diet. The
  first preventive task is to provide code free from any compilation
  and link warnings. The second corresponds to the snafu stage which
  is to debug the code. For reaching those goals the \verb+Maintainer+
  build type sets the compilers flags, respectively the linker flags,
  with all the possible warning flags activated, resp. with the
  additional debug flags.

\item
  Several options (not accessible through the ccmake GUI) can control the
  installation paths for libraries, binaries, includes and modules. By default
  the value of these install paths are respectively
  \verb+${CMAKE_INSTALL_PREFIX}/lib${LIB_SUFFIX}+,
  \verb+${CMAKE_INSTALL_PREFIX}/bin+, \verb+${CMAKE_INSTALL_PREFIX}/include+,
  \verb+${CMAKE_INSTALL_PREFIX}/share/cmake/Modules+. They can independently be
  set up using the following variables \verb+LIB_INSTALL_DIR+,
  \verb+BIN_INSTALL_DIR+, \verb+INSTALL_INSTALL_DIR+ and
  \verb+CMAKE_MOD_INSTALL_DIR+. You can set those on the command line when
  calling cmake or ccmake, for example \verb+cmake .. -DLIB_INSTALL_DIR=${HOME}/lib+.
\end{itemize}

%==================
\subsection{Compiling and installing}

\subsubsection{Summarizing the configuration choices}
Once the configuration is properly made one can check the choices made by
looking the little summary proposed by cmake.  This summary should look like
(\verb+[...]+ denotes eluded portions): {\footnotesize

%%  - Install prefix: /home/diet/local/diet
%%  - OmniORB found: YES
%%    * OmniORB directory: /home/diet/local/omniORB-4.0.7
%%    * OmniORB includes: /home/diet/local/omniORB-4.0.7/include
%%    * OmniORB libraries: /home/diet/local/omniORB-4.0.7/lib/libomniDynamic4.so;
%%      [...]libomniORB4.so;[...]libomnithread.so;[...]libCOS4.so;[...]
%%  - General options:
%%    * Documentation: ON
%%    * Examples: ON
%%    * BLAS Examples: ON
%%  - Options set:
%%    * Batch: ON
%%      -- Appleseeds directory: /home/diet/local/appleseeds-2.2.1
%%      -- Appleseeds includes:  [...]appleseeds-2.2.1/include/appleseeds
%%      -- Appleseeds library:   [...]appleseeds-2.2.1/lib/libappleseeds.a
%%    * CORI: ON

\begin{verbatim}
~/DIET > ./cmake ..
[...]
- XXXXXXXXXXXXXXXXXXXXXXXXXXXXXXXXXXXXXXXXXXXXXXXXXXXXXXXXXXXXXXXXXXXXXXXXXX
-- XXXXXXXXXXXXXXXXXXXXXX  DIET configuration summary  XXXXXXXXXXXXXXXXXXXXXX
-- XXXXXXXXXXXXXXXXXXXXXXXXXXX 2010/03/31-07:47:15 XXXXXXXXXXXXXXXXXXXXXXXXXX
-- XXX System name Linux
-- XXX - Install prefix: /home/diet/local/diet
-- XXX - C compiler   : /usr/bin/gcc
-- XXX    * version   : 4.3.4
-- XXX    * options   :  -Dinline="static __inline__" -Dconst="" -std=gnu99
-- XXX - CXX compiler : /usr/bin/c++
-- XXX    *  version  : 4.3.4
-- XXX    *  options  : -lpthread -g -D__linux__
-- XXX - OmniORB found: YES
-- XXX   * OmniORB version: 4.1.2
-- XXX   * OmniORB directory:
-- XXX   * OmniORB includes: /usr/include
-- XXX   * OmniORB libraries: [...]libomniDynamic4.so;[...]libomniORB4.so;[...]libomnithread.so
-- XXX - General options:
-- XXX   * Examples: ON
-- XXX   * BLAS Examples: ON
-- XXX - Options set:
-- XXX   * Batch: ON
-- XXX   * Statistics: ON
-- XXXXXXXXXXXXXXXXXXXXXXXXXXXXXXXXXXXXXXXXXXXXXXXXXXXXXXXXXXXXXXXXXXXXXXXXXX
[...]
\end{verbatim}
} A more complete, yet technical, way of making sure is to check the
content of the file named \verb+CMakeCache.txt+ (generated by cmake in
the directory from which cmake was invocated). When exchanging with
the developers list it is a recommendable practice to join the content
of this file which summarizes your options and also the automatic
package/library detections made by cmake.

\subsubsection{Compiling stage}
You are now done with the configuration stage (equivalent of both the
\verb+bootstrap.sh+ and \verb+./configure+ stage of the
\verb+autotools+). You are now back to your platform level
development tools, \ie \verb+make+ when working on Unices. Hence you
can now proceed with the compiling process by launching \verb+make+.

\subsubsection{Testing}
If you configured \diet with the \verb+BUILD_TESTING+ you can easily run the
regression tests by invoking the \verb+make test+. This is equivalent to
invoking \verb+ctest+ command (ctest is part of cmake
package). \verb+ctest --help+ provides a summary of the advanced options of
\verb+ctest+ among which we recommend the \verb+--verbose+ option.

\subsubsection{Installation stage}
After compiling (linking, and testing) you can optionally proceed with the
installation stage with the \verb+make install+ command.

%==============================================================================
\section{Diet client/server examples}
\label{section:diet-examples}

A set of various examples of \diet server/client are provided within the \diet
archive, here are some of the provided examples:
\begin{itemize}
\item{\texttt{Batch}}: A simple basic example on how to use the batch API is
  given here: no IN or INOUT args, the client receives as a result the number
  of processors on which the service has been executed. The service only writes
  to a file, with batch-independent mnemonics, some information on the batch
  system.

\item{\texttt{BLAS}}: the server offers the \texttt{dgemm} BLAS
  functionality. We plan to offer all BLAS (Basic Linear Algebraic Subroutines)
  in the future. Since this function computes $C = \alpha AB + \beta C$, it can
  also compute a matrix-matrix product, a sum of square matrices, etc. All
  these services are offered by the BLAS server. Two clients are designed to
  use these services: one (\texttt{dgemm\_client.c}) is designed to use the
  \texttt{dgemm\_} function only, and the other one (\texttt{client.c}) to use
  all BLAS functions (but currently only \texttt{dgemm\_}) and sub-services,
  such as \texttt{MatPROD}.

\item{\texttt{dmat\_manips}}: the server offers matrix manipulation routines:
  transposition (\texttt{T}), product (\texttt{MatPROD}) and sum
  (\texttt{MatSUM}, \texttt{SqMatSUM} for square matrices, and
  \texttt{SqMatSUM\_opt} for square matrices but re-using the memory space of
  the second operand for the result). Any subset of these operations can be
  specified on the command line. The last two of them are given for
  compatibility with a BLAS server as explained below.

\item{\texttt{file\_transfer}}: the server computes the sizes of two input
  files and returns them. A third output parameter may be returned; the server
  decides randomly whether to send back the first file. This is to show how to
  manage a variable number of arguments: the profile declares all arguments
  that may be filled, even if they might not be all filled at each
  request/computation.

\item{\texttt{\scalapack}}: the server is designed to offer all
  \scalapack\  (parallel version of the LAPACK library) functions but only
  manages the \texttt{pdgemm\_} function so far. The \texttt{pdgemm\_} routine
  is the parallel version of the \texttt{dgemm\_} function, so that the server
  also offers all the same sub-services. Two clients are designed to use these
  services: one (\texttt{pdgemm\_client.c}) is designed to use the
  \texttt{pdgemm\_} function only, and the other one (\texttt{client.c}) to use
  all \scalapack\ functions and sub-services, such as \texttt{MatPROD}.

\item{\texttt{workflow}}: The programs in this directory are examples that
  demonstrate how to use the workflow feature of diet.  The files representing
  the workflows that can be tested are stored in xml sub-directory. For each
  workflow, you can find the required services in the corresponding xml file
  (check the path attribute of each node element).  For the scalar manipulation
  example, you can use \texttt{scalar\_server} that gathers four different
  elementary services.
\end{itemize}

% \fixme{Rajouter les exemples dynamics, etc ou alors supprimer workflow et on
%   laisse que les exemples de base? -> Ou on dit qu'on fournit notamment les
%   exemples list�s l� ?}

\subsection{Compiling the examples}
\label{subsection:compiling-examples}

\verb+cmake+ will set the examples to be compiled when setting the
\verb+DIET_BUILD_EXAMPLES+ to \verb+ON+ which can be achieved by
toggling the corresponding entry of \verb+ccmake+ GUI's or by adding
\verb+-DDIET_BUILD_EXAMPLES:BOOL=ON+ to the command line arguments of
\verb+[c]cmake+ invocation. Note that this option is disabled by
default.

The compilation of the examples, respectively the installation, is executed on
the above described invocation of \verb+make+, resp. \verb+make install+
stages. The binary of the examples are placed in the
\texttt{$<$install\_dir$>$/bin/examples} sub-directory of the installation
directory. Likewise, the samples of configuration files located in
\texttt{src/examples/cfgs} are processed by \texttt{make install} to create
ready-to-use configuration files in \texttt{src/examples/cfgs} and then copied
into \texttt{$<$install\_dir$>$/etc/cfgs}.

%%% Local Variables:
%%% mode: latex
%%% ispell-local-dictionary: "american"
%%% mode: flyspell
%%% fill-column: 79
%%% End:
