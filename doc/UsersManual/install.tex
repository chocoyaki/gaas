%****************************************************************************%
%* DIET User's Manual installing chapter file                               *%
%*                                                                          *%
%*  Author(s):                                                              *%
%*    - Philippe COMBES (Philippe.Combes@ens-lyon.fr)                       *%
%*                                                                          *%
%* $LICENSE$                                                                *%
%****************************************************************************%
%* $Id$
%* $Log$
%* Revision 1.1  2003/09/09 12:38:20  pcombes
%* Reorganization of doc: UM becomes UsersManual.
%*
%* Revision 1.12  2003/06/23 13:14:09  pcombes
%* Update example to new configuration summary.
%*
%* Revision 1.11  2003/06/16 17:39:55  pcombes
%* One word about gcc-2.96.
%*
%* Revision 1.10  2003/06/02 13:47:05  pcombes
%* Fix footnotesize.
%*
%* Revision 1.9  2003/05/23 09:23:35  pcombes
%* Add suggestions from Jean-Yves. Thanks !
%*
%* Revision 1.8  2003/05/15 14:17:58  pcombes
%* UM 0.7
%*
%* Revision 1.6  2003/01/24 16:58:54  pcombes
%* UM 0.6.4
%*
%* Revision 1.5  2003/01/22 17:34:53  pcombes
%* User Manual, v. 0.6.4
%****************************************************************************%


\chapter{Installing}
\label{ch:installing}

%====[ Dependencies ]==========================================================
\section{Dependencies}
\label{sec:dependencies}

\subsection{Hardware dependencies}

DIET has only been tested on Linux i386 and i686 platforms.\\
Solaris - sparc, Linux - sparc and Linux - alpha will soon be supported.

\subsection{Software dependencies}

The three parts of DIET (the client, the agent, and the \sed ) can be used
separately, and their dependencies are different from each other.

As explained in Section \ref{sec:CORBA}, CORBA is used for all
communications inside the platform. So all of the three parts depend on it. The
implementations of CORBA currently supported in DIET are:
\begin{itemize}
 \item{\textbf{omniORB 3}} which depends on \textbf{Python 1.5}
 \item{\textbf{omniORB 4}} which depends on \textbf{Python 2.1} or later,
                           and on \textbf{OpenSSL} if you would like your DIET
                           platform to be secure.
% \item{soon \textbf{TAO 1.3}} which depends on \textbf{ACE} (but TAO is always
%                           provided with ACE)
\end{itemize}
We strongly recommend omniORB 4, as it is easier to install, and provides an SSL
support. In order to deploy CORBA services with omniORB, a configuration file
and a log directory are required: see Section \ref{sec:CORBA_services} for
complete description of the services. Their path can be given to omniORB through
environments variables, and/or, for \textbf{omniORB 4} only, at compile time,
with the \texttt{--with-omniORB-config} and \texttt{--with-omniNames-logdir}
options.

\noindent 
\textbf{NB:} We have noticed that some problems occur with \textbf{Python 2.3}:
the generated C++ code could not be compiled. It has been patched in DIET, but
some warnings still appear.
\\

Since omniORB needs a thread-safe management handling, compiling DIET with
\verb+gcc+ requires at least \verb+gcc-2.96+.
\\

The agent and server parts of DIET can depend on FAST (see �\ref{sec:FAST}).
Although not absolutely compulsory, it is strongly recommended: indeed, all
communication and computation times are set to infinity if FAST is not
installed.  FAST depends itself on:
\begin{itemize}
 \item{\textbf{DB 2}} the Berkeley Database routines
 \item{\textbf{GSL}} the GNU Scientific Library
 \item{\textbf{OpenLDAP}} an implementation of the Lightweight Directory Access
                          Protocol
 \item{\textbf{NWS}} the Network Weather Service
\end{itemize}
For more details about FAST dependencies, please refer to the FAST User's Manual
soon available. It is important to understand basically how FAST works, and the
role of its dependencies, to deactivate the ones not needed by the user.\\


Finally, some examples provided in the DIET sources depend on the BLAS and
\scalapack\ libraries (cf. �\ref{sec:compil_platform}). However all examples do
not have to be compiled.


%====[ Compilation ]===========================================================
\section{Compiling the platform}
\label{sec:compil_platform}

Once all the dependencies you need are installed, untar the DIET archive and
change to its root directory. A configure script will prepare DIET for
compiling: its main options are described below, but please, run
\texttt{configure --help} to get an up-to-date and complete usage description.

{\footnotesize
\begin{verbatim}
~ > tar xzf DIET.tgz
~ > cd DIET
~/DIET > ./configure --help=short
Configuration of DIET:
...
\end{verbatim}
}

\subsection{Optional features for configuration}

{\footnotesize
\begin{verbatim}
  --enable-doc            enable the module doc, documentation about DIET
\end{verbatim}
}
\noindent This option activates the compilation and installation of the DIET
documents, which is disabled by default, because it is very sensitive to the
version of your \LaTeX\ compiler. The output postscript files are provided in
the archive.

{\footnotesize
\begin{verbatim}
  --disable-examples      disable the module examples, basic DIET examples
\end{verbatim}
}
\noindent This option deactivates the compilation of the DIET examples, which is
enabled by default.

{\footnotesize
\begin{verbatim}
  --enable-BLAS           enable the module BLAS, an example for calling BLAS
                          functions through DIET
\end{verbatim}
}
\noindent This option activates the compilation of the DIET BLAS examples, as a
sub-module of examples (which means that this option has no effect if examples
are disabled) - disabled by default.

{\footnotesize
\begin{verbatim}
 --enable-ScaLAPACK      enable the module ScaLAPACK, an example for calling
                          ScaLAPACK functions through DIET
\end{verbatim}
}
\noindent This option activates the compilation of the DIET \scalapack\ examples,
as a sub-module of examples (which means that this option has no effect if
examples are disabled) - disabled by default.

%
% FIXME: Sylvain should document this option if it is a permanent option.
%
\begin{comment}
{\footnotesize
\begin{verbatim}
  --enable-multi-MA       enable multi-MA architecture
\end{verbatim}
}
\end{comment}


\subsection{Optional packages for configuration}

\subsubsection{The \texttt{--with-PKG-extra} option}

For all packages that DIET depends on, the \texttt{configure} script provides an
option that let the user define the arguments needed to compile with the
libraries of the package PKG: \texttt{--with-PKG-extra}. This is useful when
the \texttt{configure} script does not succeed on its own to get necessary
compilation option.

Let us take the example of a user who wishes to compile the BLAS examples (see
the BLAS subsection below). On the platform he/she uses, it is necessary to
specify the following arguments to compile with the BLAS library:

{\footnotesize \texttt{-lm -L/home/theuser/lib -lblas -lg2c -lm} }

\noindent
Let us assume that specifying the option
\texttt{--with-BLAS-libraries=/home/theuser/lib} is not enough for the
\texttt{configure} script to compile DIET examples. Then, the user will have to
add the following arguments to his/her \texttt{configure} command line:

{\footnotesize \texttt{--with-BLAS-extra="-lm -L/home/theuser/lib -lblas -lg2c
  -lm"} }

\subsubsection{omniORB}
{\footnotesize
\begin{verbatim}
  --with-omniORB=DIR      specify the root installation directory of omniORB
  --with-omniORB-includes=DIR
                          specify exact headers directory for omniORB
  --with-omniORB-libraries=DIR
                          specify exact libraries directory for omniORB
  --with-omniORB-extra=ARG|"ARG1 ARG2 ..."
                          specify extra args for the linker to find the
                          omniORB libraries (use "" in case of several args)
\end{verbatim}
}
\noindent This group of options lets the user define all necessary paths to
compile with omniORB. Generally, \texttt{--with-omniORB=} would be enough, and
the other options are provided for ugly installations of omniORB.\\
\textbf{NB:} having the executable \texttt{omniidl} in the PATH environment
variable should be enough in most cases.

\subsubsection{FAST}
{\footnotesize
\begin{verbatim}
  --with-FAST=DIR         installation root directory for FAST (optional)
  --with-FAST-bin=DIR     installation directory for fast-config (optional)
\end{verbatim}
}
\noindent This group of options lets the user define all necessary paths to
compile with FAST. Generally, \texttt{--with-FAST=} would be enough, and the
other options are provided for ugly installations of FAST. There is no need to
specify includes, libraries nor extra arguments, since FAST provides a tool
\texttt{fast-config} that does the job for us.\\
\textbf{NB1:} having the executable \texttt{fast-config} in the PATH environment
variable should be enough in most cases.\\
\textbf{NB2:} it is possible to specify \texttt{--without-FAST}, which overrides
\texttt{fast-config} detection.

\subsubsection{BLAS}
{\footnotesize
\begin{verbatim}
  --with-BLAS=DIR         specify the root installation directory of BLAS
  --with-BLAS-includes=DIR
                          specify exact headers directory for BLAS
  --with-BLAS-libraries=DIR
                          specify exact libraries directory for BLAS
  --with-BLAS-extra=ARG|"ARG1 ARG2 ..."
                          specify extra args for the linker to find the BLAS
                          libraries (use "" in case of several args)
\end{verbatim}
}
\noindent This group of options lets the user define all necessary paths to
compile with the BLAS libraries. Generally, \texttt{--with-BLAS=} would be
enough, and the other options are provided for ugly installation of the BLAS.\\
\textbf{NB:} these options have no effect if the module example and/or its
sub-module BLAS are disabled.

\subsubsection{\scalapack}
{\footnotesize
\begin{verbatim}
  --with-ScaLAPACK=DIR    specify the root installation directory of ScaLAPACK
  --with-ScaLAPACK-includes=DIR
                          specify exact headers directory for ScaLAPACK
  --with-ScaLAPACK-libraries=DIR
                          specify exact libraries directory for ScaLAPACK
  --with-ScaLAPACK-extra=ARG|"ARG1 ARG2 ..."
                          specify extra args for the linker to find the
                          ScaLAPACK libraries (use "" in case of several args)
\end{verbatim}
}
\noindent This group of options lets the user define all necessary paths to
compile with the \scalapack\ libraries. Normally, \texttt{--with-ScaLAPACK=}
should be enough, but the other options are provided because the installation of
the \scalapack\ libraries is often ugly, and thus difficult to detect
automatically: for instance, the \texttt{--with-ScaLAPACK-extra} option is
useful to integrate BLACS and MPI libraries, which are useful for ScaLAPACK.
\\
\textbf{NB:} these options has no effect if the module example and/or its
sub-module \scalapack\ are disabled.


\subsection{From the configuration to the compilation}

A default option for configure scripts is \texttt{--prefix=}, which specifies
where binary files, documents and configuration files will be installed. It is
important to set this option: it defaults to \texttt{DIET/install}.

The configuration will return with an error if no ORB was found. So please help
the configure script to find the ORB with the \texttt{--with-omniORB*} options.\\


If everything went OK, the configuration ends with a summary of the options that
were selected and what it was possible to get. The output will look like:
{\footnotesize
\begin{verbatim}
~/DIET > ./configure --enable-doc --enable-BLAS
DIET successfully configured as follows:
 - documents:       yes
 - examples:        dmat_manips, file_transfer, BLAS
 - FAST:            found /home/pcombes/DIET/FAST/bin/fast-config
 - ORB:             omniORB 4 in /usr
 - prefix:          /home/pcombes/DIET/install
 - multi-MA:        no
 - target platform: i686-pc-linux-gnu
 
Please run make help to get compilation instructions.
~/DIET > make help
====================================================================
 Usage : make help     : shows this help screen
         make agent    : builds DIET agent executable
         make SeD      : builds DIET SeD library
         make client   : builds DIET client library
         make examples : builds basic examples
         make install  : copy files into /home/pcombes/DIET/install
====================================================================
\end{verbatim}
}

This lets the user choose the DIET parts he/she wants to compile and install. It
is recommended for beginners to compile the client, the SeD and the agent in one
single step with \texttt{make all}. But please, pay attention to the fact that
\texttt{make all} does not install DIET in the prefix provided at configuration
time. To do this, run \texttt{make install}.

\texttt{make install} will run the compilation for all DIET entities before the
installation itself. Thus, if the user wants to compile only the agent, for
instance, and install it, he must run:
{\footnotesize
\begin{verbatim}
~/DIET > cd src/agent
~DIET/src/agent > make install
\end{verbatim}
}


\section{Compiling the examples}

Four series of examples are provided in the DIET archive:
\begin{itemize}
\item{\texttt{file\_transfer}}: the server computes the sizes of two input files
  and returns them (with or without, randomly, the first file) as output
  arguments to the client.
  
\item{\texttt{dmat\_manips}}: the server offers matrix manipulation routines:
  transposition (\texttt{T}), product (\texttt{MatPROD}) and sum
  (\texttt{MatSUM}, \texttt{SqMatSUM} for square matrices, and
  \texttt{SqMatSUM\_opt} for square matrices but re-using the memory space of
  the second operand for the result) Any subset of these operations can be
  specified on the command line. The last two of them are given for
  compatibility with a BLAS server - see below
  
\item{\texttt{BLAS}}: the server is designed to offer all BLAS (Basic Linear
  Algebric Subroutines) functions but only manages the \texttt{dgemm\_} function
  so far. Since this function computes $C = \alpha AB + \beta C$, it can also
  compute a matrix-matrix product, a sum of square matrices, etc. All these
  services are offered by the BLAS server. Two clients are designed to use these
  services: one (\texttt{dgemm\_client.c}) is designed to use the
  \texttt{dgemm\_} function only, and the other one (\texttt{client.c}) to use
  all BLAS functions (but currently only \texttt{dgemm\_}) and sub-services,
  such as \texttt{MatPROD}.
  
\item{\texttt{\scalapack}}: the server is designed to offer all \scalapack\ 
  (parallel version of the LAPACK library) functions but manages only the
  \texttt{pdgemm\_} function so far. The \texttt{pdgemm\_} is actually the
  parallel version of the \texttt{dgemm\_} function, so that the server also
  offers all the same sub-services. Two clients are designed to use these
  services: one (\texttt{pdgemm\_client.c}) is designed to use the
  \texttt{pdgemm\_} function only, and the other one (\texttt{client.c}) to use
  all \scalapack\ functions and sub-services, such as \texttt{MatPROD}.
\end{itemize}

Running \texttt{make install} in the \texttt{examples} directory (or in the root
directory, when DIET is configured with \texttt{--enable-examples}) does not
copy binary files into \texttt{$<$install\_dir$>$/bin}, but in
\texttt{examples/bin}: this odd behaviour is due to some limitations of the
\textsf{automake} tool.

Likewise, the samples of configuration files located in \texttt{examples/cfgs}
are processed by \texttt{make install} to create ready-to-use configuration
files in \texttt{examples/etc}, and then copied into
\texttt{$<$install\_dir$>$/etc}. Successive calls to make install will not erase
the configuration files created and copied at first time, except if \texttt{make
  uninstall} is called inbetween ; thus the user will not lose his/her changes
to these files.

