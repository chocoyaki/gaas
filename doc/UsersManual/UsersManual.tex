%****************************************************************************%
%* DIET User's Manual main file                                             *%
%*                                                                          *%
%*  Author(s):                                                              *%
%*    - Philippe COMBES (Philippe.Combes@ens-lyon.fr)                       *%
%*                                                                          *%
%* $LICENSE$                                                                *%
%****************************************************************************%
%* $Id$
%* $Log$
%* Revision 1.8  2004/01/29 17:08:47  ecaron
%* Add suggestions from Frederic Desprez. Thanks !
%*
%* Revision 1.7  2004/01/21 23:23:03  ecaron
%* Add suggestions from Jean-Yves. Thanks !
%*
%* Revision 1.6  2004/01/15 23:59:31  ecaron
%* Add corrections from Holly Dail
%*
%* Revision 1.5  2003/12/15 00:13:53  ecaron
%* New structure for the first sheet
%*
%* Revision 1.4  2003/12/12 14:42:44  pkchouha
%*  define \diet_version in UserManual.tex to be 1.0
%*
%* Revision 1.3  2003/12/12 12:36:17  ecaron
%* Change call to diet_initialize()
%* Correct some bug
%*
%* Revision 1.2  2003/09/17 14:41:28  pcombes
%* Split the .tex according to its chapters.
%*
%* Revision 1.1  2003/09/09 12:38:20  pcombes
%* Reorganization of doc: UM becomes UsersManual.
%*
%* Revision 1.9  2003/06/16 17:40:33  pcombes
%* Date
%*
%* Revision 1.8  2003/05/23 09:23:35  pcombes
%* Add suggestions from Jean-Yves. Thanks !
%*
%* Revision 1.7  2003/05/15 14:17:58  pcombes
%* UM 0.7
%*
%* Revision 1.4  2003/01/22 17:34:53  pcombes
%* User Manual, v. 0.6.4
%*
%* Revision 1.3  2003/01/21 12:17:02  pcombes
%* Update UM to API 0.6.3, and "hide" data structures.
%*
%* Revision 1.2  2003/01/13 12:09:00  pcombes
%* UM: client part complete for users's day ...
%*
%* Revision 1.1.1.1  2002/12/13 17:06:34  pcombes
%* User's Manual - architecture
%****************************************************************************%

\documentclass[12pt,a4paper]{book}
\makeatletter
\makeatother
\usepackage{fancyheadings}
\usepackage[headings]{fullpage}
%\usepackage[french]{babel}
%\usepackage[latin1]{inputenc}
%\usepackage{multicol}
\usepackage{verbatim}
\usepackage{url}
\usepackage{subfigure}

\usepackage{graphicx}
\graphicspath{{./fig}}

\newsavebox{\logobox}
\sbox{\logobox}{\includegraphics[scale=0.3]{fig/logo_DIET.ps}}
\newcommand{\logo}{\usebox{\logobox}}



%%%%
\renewcommand{\title}{DIET User's Manual}
%%%%

\pagestyle{fancyplain}
\lhead[\fancyplain{\title}{\title}]
      {\fancyplain{\title}{\title}}
\chead{}
\rhead[\fancyplain{\logo}{\logo}]{\fancyplain{\logo}{\logo}}

\lfoot[\fancyplain{INRIA}{INRIA}]{\fancyplain{INRIA}{INRIA}}
\cfoot[\fancyplain{}{}]{\fancyplain{}{}}
\rfoot[\fancyplain{Page~\thepage}{Page~\thepage}]
      {\fancyplain{Page~\thepage}{Page~\thepage}}


\newcommand{\ptop}{\textit{Peer-to-Peer}}
\newcommand{\red}{\textit{Red}}
\newcommand{\sci}{Scilab}
\newcommand{\scip}{Scilab$_{//}$}
\newcommand{\scalapack}{ScaLAPACK}
\newcommand{\sed}{\textit{SeD}}
\newcommand{\thread}{\textit{thread}}
\newcommand{\threads}{\textit{threads}}
\newcommand{\nsl}{NetSolve}
\newcommand{\fixme}[1]{\fbox{\textsl{{\bf #1}}}}
\newcommand{\dietversion}{1.0}

%%%%
% Document beginning
%%%%

\begin{document}



%%%%
% First sheet
%%%%

\thispagestyle{empty}
\vspace*{3cm}
\vspace*{3cm}

\begin{center}
\includegraphics[scale=.5]{fig/LOGO_DIET.ps}\\[2ex]
\textbf{\Huge USER'S MANUAL\\[2ex]}
\end{center}

\vfill


\noindent
\small{
\begin{tabular}{ll}
  \textbf{VERSION}  & 1.0\\
  \textbf{DATE}     & December 2003\\
  \textbf{PROJECT MANAGER}  & Fr\'ed\'eric \textsc{Desprez}.\\
  \textbf{EDITORIAL STAFF}  & Eddy \textsc{Caron} and Philippe ~\textsc{Combes}.\\
  \textbf{AUTHORS STAFF}    & 
\begin{minipage}[t]{12cm}
  Eddy \textsc{Caron}, Pushpinder Kaur \textsc{Chouhan}, Philippe ~\textsc{Combes},
  Sylvain \textsc{Dahan}, Bruno \textsc{Delfabro}, Georg \textsc{Hoesch}, Christophe \textsc{Pera}, Cyrille \textsc{Pontvieux}, Jean-Yves \textsc{L'Excellent} and Antoine \textsc{Vernois}.
\end{minipage} \\
  \textbf{Copyright}& INRIA
\end{tabular}\\
}

\newpage
\thispagestyle{empty}
\ 

%%%%
% End of first sheet
%%%%


\newpage
\tableofcontents


%
% Introduction
%
\newpage
\addcontentsline{toc}{chapter}{Introduction}
\chapter*{Introduction}


DIET stands for Distributed Interactive Engineering Toolbox. It is a
toolbox for easily developing Application Service Provider systems,
based on the Client/Agent/Server scheme. Agents are the schedulers of
this toolbox. In DIET, user's requests are served via RPC.

The RPC paradigm~\cite{CM00,MNS+00} is a good candidate for building
to build Problem Solving Environments (PSE) for numerical applications
on the Grid~\cite{HR00}. Several tools following this approach exist,
such as \nsl~\cite{nug}, NINF~\cite{NSS99}, NEOS~\cite{FMM00}, or
RCS~\cite{AGM97}. They are most commonly referred to as Network
Enabled Server (NES) environments~\cite{MNS+00}. Such environments
usually have five different components: \emph{Clients} that submit
problems they have to solve to \emph{Servers}, a \emph{Database} that
contains information about software and hardware resources, a
\emph{Scheduler} that chooses an appropriate server
depending on the problem sent and the information contained in the
database, and finally \emph{Monitors} that acquire information about
the status of the computational resources.

DIET is based on the five component types described above, but a
hierarchy of schedulers (agents) are used to address the scalability
problems. Clients always submit their to one of a particular agent
(their Master Agent or MA). Below the MA is a hierarchy of agents
(Local Agents or LA) which share the workload for scheduling jobs.

%
% A DIET platform
%
\newpage
%****************************************************************************%
%* DIET User's Manual description chapter file                              *%
%*                                                                          *%
%*  Author(s):                                                              *%
%*    - Philippe COMBES (Philippe.Combes@ens-lyon.fr)                       *%
%*                                                                          *%
%* $LICENSE$                                                                *%
%****************************************************************************%
%* $Id$
%* $Log$
%* Revision 1.3  2004/01/21 23:23:03  ecaron
%* Add suggestions from Jean-Yves. Thanks !
%*
%* Revision 1.2  2004/01/15 23:59:31  ecaron
%* Add corrections from Holly Dail
%*
%* Revision 1.1  2003/09/09 12:38:20  pcombes
%* Reorganization of doc: UM becomes UsersManual.
%*
%* Revision 1.9  2003/06/02 13:47:05  pcombes
%* Fix footnotesize.
%*
%* Revision 1.8  2003/05/15 14:17:58  pcombes
%* UM 0.7
%*
%* Revision 1.5  2003/01/22 17:34:53  pcombes
%* User Manual, v. 0.6.4
%*
%* Revision 1.4  2003/01/21 12:17:02  pcombes
%* Update UM to API 0.6.3, and "hide" data structures.
%*
%* Revision 1.3  2003/01/13 12:09:00  pcombes
%* UM: client part complete for users's day ...
%****************************************************************************%

\chapter{A DIET platform}
\label{ch:description}

DIET is built upon the \emph{Server Daemons}. The scheduler is
scattered across a hierarchy of \emph{Local Agents} and \emph{Master
  Agents}. NWS~\cite{WSH99} sensors are placed on every node of the
hierarchy to collect resource availabilities, which are used by an
application-centric performance prediction tool called FAST
\cite{Qui02}.  Figure~\ref{fig:platform} shows the hierarchical
organization of DIET.

\begin{figure}[htb]
 \begin{center}
  \resizebox{.7\linewidth}{!}{\includegraphics{fig/global_platform.eps}}
  \label{fig:platform}
  \caption{A hierarchy of DIET agents}
 \end{center}
\end{figure}

%====[ DIET components ]=======================================================
\section{DIET components}
\label{sec:components}

The different components of our software architecture are the following:      

\begin{description}
%....[ Client ]................................................................
\item \textbf{Client}\\
  A client is an application which uses DIET to solve problems.  Many types of clients are able to connect to DIET from a web page, a PSE such as
  Matlab or \sci, or from a compiled program.
%....[ Master Agent (MA) ].....................................................
\item \textbf{Master Agent (MA)}\\
  An MA receives computation requests from clients. These requests refer to some
  DIET problems listed on a reference web page. Then the MA collects computation
  abilities from the servers and chooses the best one. The reference of the
  chosen server is returned to the client. A client can be connected to an MA by
  a specific name server or a web page which stores the various MA locations.

%....[ Local Agent (LA) ]......................................................
\item \textbf{Local Agent (LA)}\\
  An LA aims at transmitting requests and information between MAs and
  servers.  The information stored on an LA is the list of the
  requests being processed and performance of all servers in its
  subtree that can solve a given problem. Depending on the underlying
  network topology, a hierarchy of LAs may be deployed between an MA
  and the servers. Of course, the function of an LA is to do a partial
  scheduling on its subtree, which reduces the job of the MA.

%....[ Server Daemon (SeD) ]...................................................
\item \textbf{Server Daemon (SeD)}\\
  A SeD encapsulates a computational server. For instance it can be
  located on the entry point of a parallel computer. The information
  stored on a SeD is a list of the data available on its server (with
  their distribution and the way to access them), the list of problems
  that can be solved on it, and all information concerning its load
  (memory available, number of resources available, \ldots). A SeD
  declares the problems it can solve to its parent LA.  An eD can give
  performance prediction for a given problem thanks to the module
  FAST, described in the next section.

\end{description}

%\begin{figure}[htb]
% \begin{center}
%  \resizebox{.6\linewidth}{!}{\includegraphics{fig/global_platform.eps}}
%  \label{fig:platform}
%  \caption{A hierarchy of DIET agents}
% \end{center}
%\end{figure}


%====[ FAST: FAST AGENT'S SYSTEM TIMER ]=======================================
\section{FAST: Fast Agent's System Timer}
\label{sec:FAST}

FAST \cite{Qui02} is a tool for dynamic performance forecasting in a
Grid environment. As shown in Figure~\ref{fig:fast-overview}, FAST is
composed of several layers and relies on low-level software. First it
uses a network and CPU monitoring software to handle dynamically
changing resources, like workload or bandwidth.  FAST uses the Network
Weather Service (NWS)~\cite{WSH99} a distributed system that
periodically monitors and dynamically forecasts the performance of
various network and computational resources. The resource
availabilities acquisition module of FAST uses and enhances NWS.
Indeed, if there is no direct NWS monitoring between two machines,
FAST automatically searches for the shortest path between them in the
graph of monitored links. It estimates the bandwidth as the minimum of
those in the path and the latency as the sum of those measured. This
allows for the availability of more predictions when DIET is deployed
over a hierarchical network.

\begin{figure}[htb]
  \begin{center}
    \resizebox{.75\linewidth}{!}{\includegraphics{fig/FAST.eps}}
    \caption{FAST overview}
    \label{fig:fast-overview}
  \end{center}
\end{figure}

In addition to the system availabilities, FAST can also forecast the time and
space needs of computational routines, depending on both the parameter set and
the machine where the computation would take place.  For this, FAST 
benchmarks the routines at installation time on each machine for a
representative set of parameters. After polynomial data fitting, the results
are stored in an LDAP tree.  The user API of FAST is composed of a small set
of functions that combine resource availabilities and routine needs from
low-level software to produce ready-to-use values. These results can be
combined into analytical models by the parallel extension~\cite{CS02} to
forecast execution times of parallel routines.

Thus DIET components, as any FAST client, can access information like the time
needed to move a given amount of data between two SeDs, the time to solve a
problem with a given set of computational resources managed by a SeD, or the
combination of these two quantities.\\

For more details about FAST, please read the FAST Reference Manual~\footnote{\url{http://graal.ens-lyon.fr/FAST/docs/index.html}}.


%====[ CORBA ]=================================================================
\section{Communications inner to the platform}
\label{sec:CORBA}

NES environments can be implemented using a classic socket communication layer.
NINF and \nsl\ are implemented that way. Several problems to this approach
have been pointed out such as the lack of portability or the limitation of
opened sockets. Our aim is to implement and deploy a distributed NES
environment that works at a wider scale. Distributed object environments, such
as \emph{Java}, \emph{DCOM} or CORBA have proven to be a good base for
building applications that manage access to distributed services. They not only
provide transparent communications in heterogeneous networks, but they also
offer a framework for the large scale deployment of distributed
applications. Being open and language independent, CORBA was chosen as the
communication layer in DIET.

As recent implementations of CORBA provide communication time close to
that of sockets, CORBA is well suited to support distributed resources
and applications in a large scale Grid environment. New specialized
services can be easily published and existing services can also be
used.  Therefore, CORBA is a good choice for the
development of Grid specific services. DIET is based upon
\emph{OmniORB 3}~\cite{OMNIORB} or later, a free CORBA implementation
which provides good communication performance.


%====[ DIET INITIALIZATION ]===================================================
\section{DIET initialization}
\label{init}

Figure~\ref{fig:init} shows each step of the initialization of a simple Grid
system. The architecture is built in hierarchical order, each component
connecting to its parent. The MA is the first entity to be started~(1). It waits
for connections from LAs or requests from clients.

\begin{figure}[hbt]
  \begin{center}
    \resizebox{9cm}{!}{\includegraphics{fig/init.eps}}
    \caption{Initialization of a DIET system.}
    \label{fig:init}
  \end{center}
\end{figure}

Then when an LA is launched, it subscribes to the MA~(2). At this step of the
system initialization, two kinds of components can connect to the LA: an
\sed ~(3), which manages some computational resource, or another LA~(4), to add a
hierarchical level in this branch. When the \sed\ registers to an LA, it
publishes a list of the services it offers, which is forwarded to the parent
agent until the MA.
Finally, any client can access the registered resource through the platform: it
can contact an MA~(5) to get a reference to the best server available and then
directly connect to it to launch the computation.

The architecture of the hierarchy is described in configuration files and each
component transmits the local configuration to its parent. Thus, the system
administration can also be hierarchical. For instance, an MA can manage a domain
like a university, providing prioritary access to users of this domain. Then
each laboratory can run an LA, while each team of the laboratory can run some
other LAs to administrate its own servers. This hierarchical administration of
the system allows local changes in the configuration without interfering with
the whole platform.



%====[ Solving a problem ]=====================================================
\section{Solving a problem}
\label{sec:solvepb}

Assuming that the architecture described in Section \ref{sec:components}
includes several servers able to solve the same problem, and that each operand
needed for the computation is available on one single server.

%%%%%%%%%%%%%%%
%% FIXME for DIET v1.1
%%%%%%%%%%%%%%%
% , the example
% presented in Figure~\ref{fig:submit_pb} considers the submission of the
% problem \texttt{F()} involving data \texttt{A} and \texttt{B}.
%\begin{figure}[hbt]
%  \begin{center}
%    \resizebox{8cm}{!}{\includegraphics{fig/submit_pb.eps}}
%    \caption{Problem submission example}
%    \label{fig:submit_pb}
%  \end{center}
%\end{figure}
%%%%%%%%%%%%%%%

The algorithm presented below lets an MA choose among its servers the one which
will perform the computation. This decision is made in four steps:
\begin{itemize}
\item the MA propagates the client request through its subtrees down to the
  capable servers~; actually, each agent only knows which among its children manage
  the service, and it forwards the client request to them only~;
\item each server that can satisfy the request calls FAST to estimate the
  computation time necessary to process the request, and sends this
  estimation back to its ``parent'' (the LA)~;
\item each LA of the tree that receives one or more positive answers from its
  children sorts the servers and forwards their answers to the MA through the
  hierarchy~;
\item once the MA has collected all the answers from its direct children, it
  chooses a pool of fast servers and sends their references to the client.
\end{itemize}

%%%%%%%%%%%%%%%
%% FIXME:
%%  Memory aspects should be treated here.
%%%%%%%%%%%%%%%

%%%%%%%%%%%%%%%
%% FIXME for DIET v1.1
%%%%%%%%%%%%%%%
% In order to solve the problem itself, the client connects to one of
% the servers chosen: it sends its local data and specifies if the
% results should be kept in-place for further computation or if they
% should be brought back. The transfer of persistent operands is
% performed at this stage.
%%%%%%%%%%%%%%%


%
% Installing
%
\newpage
%****************************************************************************%
%* DIET User's Manual installing chapter file                               *%
%*                                                                          *%
%*  Author(s):                                                              *%
%*    - Eddy CARON (Eddy.Caron@ens-lyon.fr)                                 *%
%*    - Pushpinder Kaur Chouhan (Pushpinder.Kaur.Chouhan@ens-lyon.fr)       *%
%*    - Philippe COMBES (Philippe.Combes@ens-lyon.fr)                       *%
%*                                                                          *%
%* $LICENSE$                                                                *%
%****************************************************************************%
%* $Id$
%* $Log$
%* Revision 1.27  2006/10/12 09:10:28  eboix
%*   More cmake's docs. --- Injay2461
%*
%*
%* Revision 1.25  2006/05/12 12:12:32  sdahan
%* Add some documentation about multi-MA
%*
%* Bug fix:
%*  - segfault when the neighbours configuration line was empty
%*  - deadlock when a MA create a link on itself
%*
%* Revision 1.24  2006/01/25 16:52:55  pfrauenk
%* CoRI : renaming of the chapter performance prediction with fast
%* 	to performance prediction, add of the CoRI Usersmanual,
%* 	changes in the plugin scheduler
%*
%* Revision 1.23  2005/08/26 08:32:42  rbolze
%* remove --enable-logservice feature.
%*
%* Revision 1.22  2005/07/12 08:54:10  ecaron
%* fix BLAS and ScaLAPACK (no fixme in fact)
%*
%* Revision 1.21  2005/06/27 19:26:52  hdail
%* - Moved introduction to FAST to description section with intro to multi-MA and
%*   gave both chapter references.
%* - Changed version number to 2.0.
%* - Moved info on compiling FAST itself to fast section from install section.
%*   install section still explains how to configure DIET with FAST.
%*
%* Revision 1.20  2005/06/24 15:07:29  hdail
%* Continuing english corrections and updating.
%*
%* Revision 1.19  2005/06/14 08:06:20  ecaron
%* Update ./configure output
%*
%* Revision 1.18  2005/05/29 13:51:22  ycaniou
%* Moved the section concerning FAST from description to a new chapter about FAST
%* and performances prediction.
%* Moved the section about convertors in the FAST chapter.
%* Modified the small introduction in chapter 1.
%* The rest of the changes are purely in the format of .tex files.
%*
%* Revision 1.17  2005/05/20 19:06:01  mjan
%* Short description of how to configure DIET for JuxMem
%*
%* Revision 1.16  2004/10/25 08:59:56  sdahan
%* add the multi-MA documentation
%*
%* Revision 1.15  2004/07/12 08:33:58  rbolze
%* explain how to copy cfgs file in install_dir/etc directory and correct my english
%****************************************************************************%

\chapter{DIET installation}
\label{ch:installing}

%====[ Dependencies ]==========================================================
\section{Dependencies}
\label{sec:dependencies}

\subsection{Platform dependencies}

DIET is fully tested on Linux/i386 and Linux/i686 platforms.
But DIET also supports other architectures (thanks to users reports or
to punctual deployments). In particular DIET is known to be effective
on Linux/Sparc, Linux/i64, Linux/amd64 and Linux/Alpha platforms.
At some point DIET used to be tested on the Solaris/Sparc platform (although
it's been a long time).

If you find a bug in DIET, please don't hesitate to submit a bug report at
\url{http://graal.ens-lyon.fr/bugzilla}. If you have multiple bugs
to report, please make multiple submissions, rather than submitting
multiple bugs in a single report.

As for the supported C++ compilers, DIET undergoes daily regression tests
(see \url{http://graal.ens-lyon.fr/DietDashboard}) with GCC's g++ (ranging
from version 3.2.X to 4.1.x).

\subsection{Software dependencies}
\label{sec:software_dependencies}

As explained in Section~\ref{sec:CORBA}, CORBA is used for all
communications inside the platform. The implementations of CORBA
currently supported in DIET is \textbf{omniORB 4} which depends on
\textbf{Python 2.1} or later, and on \textbf{OpenSSL} if you wish
to secure your DIET platform.~\footnote{If you want to deploy a
secure DIET platform, SSL support is not yet implemented  in DIET, but
an easy way to do so is to deploy DIET on a VPN.}
% \item{soon \textbf{TAO 1.3}} which depends on \textbf{ACE} (but TAO is always
%                           provided with ACE)

In order to deploy CORBA services with omniORB, a configuration file
and a log directory are required: see Section
\ref{sec:CORBA_services} for a complete description of the services.
Their paths can be given to omniORB through environment variables
(\texttt{\$OMNIORB\_CONFIG} and \texttt{\$OMNINAMES\_LOGDIR}), and/or,
for \textbf{omniORB 4} only, at compile time, with the
\texttt{--with-omniORB-config} and \texttt{--with-omniNames-logdir}
options.

\noindent 
\textbf{NB:} We have noticed that some problems occur with
\textbf{Python 2.3}: the C++ code generated could not be compiled. It
has been patched in DIET, but some warnings still appear.  \\

Since omniORB needs thread-safe management of exception handling,
compiling DIET with \verb+gcc+ requires at least \verb+gcc-2.96+.  \\

Some examples provided in the DIET sources depend on the BLAS
and \scalapack\ libraries. However the compilation of those BLAS dependent
examples are optional.


%====[ Compilation ]===========================================================
\section{Compiling the platform}
\label{sec:compil_platform}

DIET compilation process is moving from the traditional \verb+autotools+
way of things to a tool named \verb+cmake+ (mainly to benefit from
\verb+cmake+'s built-in regression tests mechanism).
During a short transitional period both compiling tools will be available. 
Nevertheless, whenever if possible, \verb+cmake+ should be seen as the
privileged (and supported) tool.
The \verb+autotools+ should be definitively deprecated in the next version.
The following of this section explains the usage of both tools.

Before compiling DIET itself, first install the above mentioned
(cf section~\ref{sec:software_dependencies}) dependencies.
Then untar the DIET archive and change current directory to its root directory.

%====[ Compilation with cmake ]================================================
\subsection{Compiling DIET with cmake}

%==================
\subsubsection{Obtaining and installing cmake per se}
DIET requires using \verb+cmake+ at least version \verb+2.4.3+.
For many popular distributions \verb+cmake+ is incorporated by
default or at least \verb+apt-get+ (or whatever your distro package installer
might be) is \verb+cmake+ aware.
Still, in case you need to install an up-to-date version \verb+cmake+'s
official site distributes many binary versions (alas packaged as tarballs)
as made available at 
\url{http://www.cmake.org/HTML/Download.html}.
Optionally, you can download the sources and recompile them: this simple
process (\verb+./bootstrap; make; make install+) is described at
\url{http://www.cmake.org/HTML/Install.html}.

%==================
\subsubsection{Configuring DIET's compilation: cmake quick introduction}
If you are already experienced with \verb+cmake+ then using it to compile
DIET should provide no surprise. 
DIET respects \verb+cmake+'s best practices e.g.~by clearly separating the
source tree from the binary tree (or compile tree), by exposing the main
configuration optional flag variables prefixed with \verb+DIET_+ (and by
hiding away the technical variables) and by not postponing configuration
difficulties (in particular the handling of external dependencies like
libraries) to compile stage.

\verb+Cmake+ classically provides two ways for setting configuration
parameters in order to generate the makefiles in the form of two
commands \verb+ccmake+ and \verb+cmake+ (the first one has an extra "c"
character): use
\begin{description}
\item{\verb+ccmake [options] <path-to-source>+}\\
  in order to specify the parameters interactively through a GUI interface
\item{\verb+cmake [options] <path-to-source> [-D<var>:<type>=<value>]+}\\
  in order to define the parameters with the \verb+-D+ flag directly
  from the command line.
\end{description}
In the above syntax description of both commands, {\verb+<path-to-source>+}
specifies a path to the top level of the source tree (i.e. where the directory
where the top level CMakeLists.txt file is to be encountered). Also
the current working directory will be used as the root of the build tree for
the project (out of source building is generally encouraged specially
when working on a CVS tree).

Here is a short list of \verb+cmake+ internal parameters that are worth
mentioning:
\begin{itemize}
\item
  \verb+CMAKE_BUILD_TYPE+ controls the type of build mode among which 
  \verb+Debug+ will produce binaries and libraries with the debugging
   information
\item
   \verb+CMAKE_VERBOSE_MAKEFILE+ is a Boolean parameter which when set to
   ON will generate makefiles without the .SILENT directive. This is
   useful for watching the invoked commands and their arguments in case
   things go wrong.
\item
   \verb+CMAKE_C[XX]_FLAGS*+ is a family of parameters used for
   the setting and the customization of various C/C++ compiler options.
\end{itemize}
%
Eventually, here is short list of \verb+ccmake+ interface tips:
\begin{itemize}
\item
  when lost, look at the bottom lines of the interface which always
  summarizes \verb+ccmake+'s most pertinent options (corresponding
  keyboard shortcuts) depending on your current context
\item
  hitting the "h" key will direct you \verb+ccmake+ embedded tutorial
  and a list of keyboard shortcuts (as mentioned in the bottom
  lines, hit "e" to exit)
\item
  up/down navigation among parameter items can be achieved with the
  up/down arrows
\item
  when on a parameter item, the line in inverted colors (close above the
  bottom of the screen) contains a short description of the selected
  parameter as well as the set of possible/recommended values
\item
  toggling of boolean parameters is made with enter
\item
  press \verb+enter+ to edit path variables
\item
  when editing a \verb+PATH+ typed parameter the \verb+TAB+ keyboard
  shortcut provides an emacs-like (or bash-like) automatic path completion.
\item
  toggling of advanced mode (press "t") reveals hidden parameters
\end{itemize}
 
%==================
\subsubsection{DIET's compilation configuration: a ccmake walk-through}

Assume that \verb+CVS_DIET_HOME+ represents a path to the top level
directory of a tree of DIET sources.
This directory tree was either obtained by expanding the DIET current
source level distribution tarball, or for DIET developers corresponds to
the directory GRAAL/devel/diet/diet of a cvs checkout of the DIET sources
hierarchy.
Additionally, assume we created a build tree directory and \verb+cd+
to it (in the example below we chose \verb+CVS_DIET_HOME/Bin+ as
build tree, but feel free to follow your conventions):
\begin{itemize}
\item
  \verb+cd CVS_DIET_HOME/Bin+
\item
  \verb+ccmake ..+ to enter the GUI
  \begin{itemize}
  \item press \verb+c+ (equivalent of bootstrap.sh of the autotools)
  \item toggle the desired options e.g. \verb+DIET_BUILD_EXAMPLES+ or
     \verb+DIET_USE_JXTA+. 
  \item specify the \verb+CMAKE_INSTALL_PREFIX+ parameter (if you wish
     to install in a directory different from \verb+/usr/local+,
  \item press \verb+c+ again, for checking required dependencies
  \item check all the parameters preceded with the * (star) character
     whose value was automatically retrieved by \verb+cmake+.
  \item provide the required information i.e. feel in the proper values
     for all parameters whose value is terminated by NOT-FOUND
  \item iterate the above process of parameter checking, toggle/specification
     and configuration until all configuration information is satisfied
  \item press \verb+g+ to generate the makefile
  \item press \verb+q+ to exit ccmake
  \end{itemize}
\item
  \verb+make+ in order to classically launch the compilation process
\item
  \verb+make install+ when installation is required
\end{itemize}

%==================
\subsubsection{DIET's main optional configuration flags}

Here are the main configuration flags:
\begin{itemize}
\item
  \verb+OMNIORB4_DIR+ is the path to the omniORB4 installation directory
  (only relevant when omniORB4 was not installed in /usr/local).
  Example: \verb+cmake .. -DOMNIORB4_DIR:PATH=$HOME/local/omniORB-4.0.7+

\item
  \verb+DIET_BUILD_DOCUMENTATION+ for building LaTeX-based user's guide,
  the developer's guide and the doxygenated documentation.
  This option is disabled by default, since it is very sensitive to the
  version of your \LaTeX\ compiler and it also relies on many subdependencies
  (e.g. \verb+doxygen+ or \verb+fig2dev+)

\item
  \verb+DIET_BUILD_EXAMPLES+ activates the compilation of a set of
  general client/server examples. Note that some specific examples
  (e.g. \verb+DIET_USE_BLAS+) require some additional flag to be activated
  too.

\item
  \verb+DIET_BUILD_LIBRARIES+ which is enabled by default, activates the
  compilation of the DIET libraries. This option is only useful if you 
  wish to restrict the compilation to the construction of the documentation.

\item
  \verb+DIET_WITH_STATISTICS+ enables the generation of statistics logs
\end{itemize}

\noindent
DIET has many extensions (some of them are still) experimental. These
extensions most often rely on external packages than need to be pre-installed
and some of those extensions offer concurrent functionalities. This explains
the usage of configuration flags in order to obtain the compilation of
the desired extensions.

\begin{itemize}
\item
  \verb+DIET_USE_BLAS+ option activates the compilation of
  the DIET BLAS examples, as a sub-module of examples.
  The BLAS~\footnote{\url{http://www.netlib.org/blas/}} (Basic Linear
  Algebra Subprograms) are high quality ``building block'' routines for
  performing basic vector and matrix operations.
  Level 1 BLAS do vector-vector operations, Level 2 BLAS do matrix-vector
  operations, and Level 3 BLAS do matrix-matrix operations.
  Because the BLAS are efficient, portable, and widely available,
  they're commonly used in the development of high quality linear algebra
  software.
  DIET uses BLAS to build demonstration examples of client/server.
  Note that the  \verb+DIET_USE_BLAS+ can only be effective when 
  \verb+DIET_BUILD_EXAMPLES+ is enabled.
  \verb+DIET_USE_BLAS+ is disabled by default.

\item
  \verb+DIET_USE_CORI+ CoRI, which stands for COllector of Resource
  Information, provides a framework for probing hardware and performance
  information about the SeD.
  CoRI also yields a very basic set of probing resources which are
  heavily dependent on the system calls available for the considered platform.
  When this option is activated (disabled by default), the user can either
  define new collectors or use existing collectors (like FAST, see the
  \verb+DIET_USE_FAST+ option) through CoRI's interface.
  CoRI thus provides a possible tactical approach for tuning the performance
  of your favorite plug-in scheduler.
  Chapter~\ref{chapter:performance} describes in more details CoRI and its
  possible usage within DIET.

\item
  \verb+DIET_USE_FAST+ activates DIET support of FAST (refer to
  \url{http://graal.ens-lyon.fr/FAST/} a grid aware dynamic forecasting
  library. FAST in turn relies on many sub-libraries (GSL, BDB, NWS, LDAP).
  Thus, the activation of this option can only be recommended for advanced
  users...

\item
  \verb+DIET_USE_BATCH+ activates DIET support of Appleseeds (refer to
  \url{http://grail.sdsc.edu/projects/appleseeds/}) based batch
  extensions.

\item
  \verb+DIET_USE_JXTA+ activates the so called MULTI-Master-Agent
  support. This support is based on the JXTA layer (refer to
  \url{http://www.jxta.org/}). This is to be opposed with
  \verb+DIET_WITH_MULTI_MA+ which offers similar functionalities
  but based on CORBA.

\item
  \verb+DIET_WITH_MULTI_MA+ activates the so called MULTI Master Agent
  support. This support is based on the CORBA layer (which is to be
  opposed with \verb+DIET_USE_JXTA+ which offers similar functionalities
  but based on JXTA.

\item
  \verb+DIET_USE_FD+ for activating Fault Detector.

\item
  \verb+DIET_USE_WORKFLOW+ enables the support of workflow.
\end{itemize}

\noindent
Eventually, some configuration flags control the general result of the
compilation or some developers extensions:
\begin{itemize}
\item
  \verb+BUILD_SHARED_LIBS+ is a cmake internal variable which specifies
  whether the libraries should be dynamics as opposed to static

\item
  \verb+DIET_USE_DART+ enables DART reporting system (refer to 
  \url{http://public.kitware.com/Dart}) which is used for constructing
  DIET's dashboard (see \url{http://graal.ens-lyon.fr/DietDashboard}).
\end{itemize}

%====[ Compilation with the autotools ]========================================
\subsection{Compiling DIET with the autotools (under deprecation)}
A configure script will prepare DIET for
compiling: its main options are described below, but please, run
\texttt{configure --help} to get an up-to-date and complete usage
description.\\

\noindent
{\footnotesize
\texttt{$\sim$> tar xzf DIET\_}\dietversion\texttt{.tgz} \\
\texttt{$\sim$> cd DIET} \\
\texttt{$\sim$/DIET> ./configure --help=short} \\
\texttt{Configuration of DIET} \dietversion \texttt{:} \\
\texttt{...}
}

\subsubsection{Optional features for configuration}

{\footnotesize
\begin{verbatim}
  --disable-FEATURE       do not include FEATURE (same as --enable-FEATURE=no)
  --enable-FEATURE[=ARG]  include FEATURE [ARG=yes]
  --enable-maintainer-mode enable make rules and dependencies not useful
                          (and sometimes confusing) to the casual installer
\end{verbatim}

\begin{verbatim}
  --enable-doc            enable the module doc, documentation about DIET
\end{verbatim}
}
\noindent This option activates the compilation and installation of
the DIET documents, which is disabled by default, because it is very
sensitive to the version of your \LaTeX\ compiler. The output
postscript files are provided in the archive.

{\footnotesize
\begin{verbatim}
  --disable-examples      disable the module examples, basic DIET examples
\end{verbatim}
}
\noindent This option deactivates the compilation of the DIET
examples, which is enabled by default.

{\footnotesize
\begin{verbatim}
  --enable-BLAS           enable the module BLAS, an example for calling BLAS
                          functions through DIET
\end{verbatim}
}
\noindent This option activates the compilation of the DIET BLAS
examples, as a sub-module of examples (which means that this option
has no effect if examples are disabled) - disabled by default.


{\footnotesize
\begin{verbatim}
 --enable-ScaLAPACK      enable the module ScaLAPACK, an example for calling
                          ScaLAPACK functions through DIET
\end{verbatim}
}
\noindent \textit{Experimental}.  This option activates the
compilation of the DIET \scalapack\ examples, as a sub-module of
examples (which means that this option has no effect if examples are
disabled) - disabled by default.

{\footnotesize
\begin{verbatim}
  --enable-JXTA-mode      enable DIET/JXTA architecture
\end{verbatim}
}
\noindent \textit{Experimental}.  This option allows the user to
deploy DIET$_{JXTA}$ architectures.  When this option is activated
(disabled by default), DIET allows only this type of architecture to
be deployed.

\label{sec:multimainstall}
{\footnotesize
\begin{verbatim}
  --enable-multi-MA       enable multi-MA architecture
\end{verbatim}
}
\noindent \textit{Experimental}.  This option allows the user to
connect several MA to be linked together. When this option is
activated, it allow a MA to looking for a SeD into the other MA
hierarchies.

{\footnotesize
\begin{verbatim}
  --enable-JuxMem       enable the use of JuxMem for managing persistent data
\end{verbatim}
}
\noindent \textit{Experimental}.  This option allows the user to
interact with JuxMem for the management of persistent data. When
this option is activated (disabled by default), a SeD can store data
blocks within JuxMem. Chapter~\ref{ch:juxmem} describes in more
details JuxMem and its use inside DIET.

{\footnotesize
\begin{verbatim}
  --enable-cori       enable the use of CoRI for collecting information 
                      about the SeD
\end{verbatim}
}
\noindent \textit{Experimental}.  This option allows the user to
get hardware and performance information about the SeD. When
this option is activated (disabled by default), the user can call FAST and other 
collectors by the interface CoRI. It can be used to improve the plug-in scheduler.
Chapter~\ref{chapter:performance} describes in more details CoRI and its use inside DIET.

{\footnotesize
\begin{verbatim}
  --enable-stats          enable generation of statistics logs
  --disable-dependency-tracking Speeds up one-time builds
  --enable-dependency-tracking  Do not reject slow dependency extractors
  --enable-shared=PKGS    build shared libraries default=yes
  --enable-static=PKGS    build static libraries default=yes
  --enable-fast-install=PKGS optimize for fast installation default=yes
  --disable-libtool-lock  avoid locking (might break parallel builds)
\end{verbatim}
}

%%{\footnotesize
%%\begin{verbatim}
%%  --enable-logservice     enable monitoring through LogService
%%\end{verbatim}
%%}
%%\noindent \textit{Deprecated}.  Support for the LogService monitoring
%%software is now always compiled in.  Note that support for LogService
%%is built-in and you do not have to install LogService to compile
%%DIET.

\subsubsection{Optional packages for configuration}


{\footnotesize
\begin{verbatim}
  --with-PACKAGE[=ARG]    use PACKAGE [ARG=yes]
  --without-PACKAGE       do not use PACKAGE (same as --with-PACKAGE=no)
  --with-gnu-ld           assume the C compiler uses GNU ld default=no
  --with-pic              try to use only PIC/non-PIC objects default=use both
\end{verbatim}
}

%% \subsubsection{The \texttt{--with-PKG-extra} option}

%% For all packages that DIET depends on, the \texttt{configure} script provides an
%% option that let the user define the arguments needed to compile with the
%% libraries of the package PKG: \texttt{--with-PKG-extra}. This is useful when
%% the \texttt{configure} script does not succeed on its own to get necessary
%% compilation option.

%% Let us take the example of a user who wishes to compile the BLAS examples (see
%% the BLAS subsection below). On the platform he/she uses, it is necessary to
%% specify the following arguments to compile with the BLAS library:

%% {\footnotesize \texttt{-lm -L/home/theuser/lib -lblas -lg2c -lm} }

%% \noindent
%% Let us assume that specifying the option
%% \texttt{--with-BLAS-libraries=/home/theuser/lib} is not enough for the
%% \texttt{configure} script to compile DIET examples. Then, the user will have to
%% add the following arguments to his/her \texttt{configure} command line:

%% {\footnotesize \texttt{--with-BLAS-extra="-lm -L/home/theuser/lib -lblas -lg2c
%%   -lm"} }

\paragraph{omniORB}
{\footnotesize
\begin{verbatim}
  --with-omniORB=DIR      specify the root installation directory of omniORB
  --with-omniORB-includes=DIR
                          specify exact header directory for omniORB
  --with-omniORB-libraries=DIR
                          specify exact library directory for omniORB
  --with-omniORB-extra=ARG|"ARG1 ARG2 ..."
                          specify extra conftest.c -o conftest for the linker
                          to find the omniORB libraries (use "" in case of 
                          several conftest.c -o conftest)
\end{verbatim}
}
\noindent This group of options lets the user define all necessary
paths to compile with omniORB. Generally, \texttt{--with-omniORB=DIR}
should be enough, and the other options are provided for ugly
installations of omniORB.\\ \textbf{NB:} having the executable
\texttt{omniidl} in the PATH environment variable should be enough in
most cases.

\paragraph{FAST}\label{fast_compil}
{\footnotesize
\begin{verbatim}
  --with-FAST=DIR         installation root directory for FAST (optional)
  --with-FAST-bin=DIR     installation directory for fast-config (optional)
\end{verbatim}
}
\noindent This group of options lets the user define all necessary
paths to compile with FAST; Generally, \texttt{--with-FAST=DIR} should
be enough, and the other options are provided for difficult
installations of FAST. There is no need to specify includes, libraries
and extra arguments, since FAST provides a tool \texttt{fast-config}
that does this job automatically.\\ \textbf{NB1:} having the
executable \texttt{fast-config} in the PATH environment variable
should be enough in most cases.\\ \textbf{NB2:} it is possible to
specify \texttt{--without-FAST}, which overrides \texttt{fast-config}
detection.

\paragraph{BLAS}

The BLAS~\footnote{\url{http://www.netlib.org/blas/}} (Basic Linear
Algebra Subprograms) are high quality ``building block'' routines for
performing basic vector and matrix operations.  Level 1 BLAS do
vector-vector operations, Level 2 BLAS do matrix-vector operations,
and Level 3 BLAS do matrix-matrix operations. Because the BLAS are
efficient, portable, and widely available, they're commonly used in
the development of high quality linear algebra software.

{\footnotesize
\begin{verbatim}
  --with-BLAS=DIR  specify the root installation directory of BLAS (Basic
                   Linear Algebric Subroutines)

  --with-BLAS-includes=DIR
                   specify exact header directory for BLAS
  --with-BLAS-libraries=DIR
                   specify exact library directory for BLAS
  --with-BLAS-extra=ARG|"ARG1 ARG2 ..."
                   specify extra conftest.c -o conftest for the linker to find
                   the BLAS libraries (use "" in case of several 
                   conftest.c -o conftest)
\end{verbatim}
}
\noindent This group of options lets the user define all necessary
paths to compile with the BLAS libraries. Generally,
\texttt{--with-BLAS=DIR} should be enough, and the other options are
provided for difficult installation of the BLAS.\\ \textbf{NB:} these
options have no effect if the module example and/or its sub-module
BLAS are disabled.

\paragraph{\scalapack}

The ScaLAPACK\footnote{\url{http://www.netlib.org/scalapack/}} (or
Scalable LAPACK) library includes a subset of LAPACK routines
redesigned for distributed memory MIMD parallel computers. It is
currently written in a Single-Program-Multiple-Data style using
explicit message passing for interprocessor communication. It assumes
matrices are laid out in a two-dimensional block cyclic decomposition.

Like LAPACK, the ScaLAPACK routines are based on block-partitioned
algorithms in order to minimize the frequency of data movements
between different levels of the memory hierarchy. For such machines,
the memory hierarchy includes the off-processor memory of other
processors, in addition to the hierarchy of registers, cache, and
local memory on each processor.  The fundamental building blocks of
the ScaLAPACK library are distributed memory versions (PBLAS) of the
Level 1, 2 and 3 BLAS, and a set of Basic Linear Algebra Communication
Subprograms (BLACS) for communication tasks that arise frequently in
parallel linear algebra computations. In the ScaLAPACK routines, all
interprocessor communication occurs within the PBLAS and the BLACS.
One of the design goals of ScaLAPACK was to have the ScaLAPACK
routines resemble their LAPACK equivalents as much as possible.

For detailed information on ScaLAPACK, please refer to the ScaLAPACK
Users' Guide~\cite{BCC+97}.


{\footnotesize
\begin{verbatim}
  --with-ScaLAPACK=DIR    specify the root installation directory of ScaLAPACK
                          (parallel version of the LAPACK library)

  --with-ScaLAPACK-includes=DIR
                          specify exact header directory for ScaLAPACK
  --with-ScaLAPACK-libraries=DIR
                          specify exact library directory for ScaLAPACK
  --with-ScaLAPACK-extra=ARG|"ARG1 ARG2 ..."
                          specify extra conftest.c -o conftest for the linker to find 
                          the ScaLAPACK libraries (use "" in case of several 
                          conftest.c -o conftest)
\end{verbatim}
}

\noindent This group of options lets the user define all necessary
paths for compilation with the \scalapack\ libraries. Normally,
\texttt{--with-ScaLAPACK=DIR} could be enough, but the other options
are provided because the installation of the \scalapack\ libraries is
often difficult, and thus difficult to detect automatically. For
instance, the \texttt{--with-ScaLAPACK-extra} option is useful to
integrate BLACS and MPI libraries, which are useful for ScaLAPACK.  \\
\textbf{NB:} these options have no effect if the module example and/or
its sub-module \scalapack\ are disabled.

\subsubsection{From configuration to compilation}

An important option for configure scripts is \texttt{--prefix=}, which
specifies where binary files, documents and configuration files will
be installed. It is important to set this option; otherwise it 
defaults to \texttt{DIET/install}.

The configuration will return with an error if no ORB was found. So
please help the configure script to find the ORB with the
\texttt{--with-omniORB*} options.\\


If everything went OK, the configuration ends with a summary of the
options that were selected and what it was possible to get. The output
will look like: {\footnotesize
\begin{verbatim}
~/DIET > ./configure --enable-examples --enable-doc
DIET successfully configured as follows:
 - Batch:           no
 - documents:       yes
 - examples:        dmat_manips, file_transfer, scalars
 - FAST:            disabled
 - CORI:            disabled
 - ORB:             omniORB 4 in /usr
 - prefix:          /home/username/DIET/install
 - multi-MA:        no
 - DIET J:          no
 - Juxmem:          no
 - target platform: i686-pc-linux-gnu
 
Please run make help to get compilation instructions.
~/DIET > make help
====================================================================
 Usage : make help     : shows this help screen
         make agent    : builds DIET agent executable
         make SeD      : builds DIET SeD library
         make client   : builds DIET client library
         make examples : builds basic examples
         make          : builds everything configured
         make install  : copy files into /home/username/DIET/install
====================================================================
\end{verbatim}
}

This helps the user to choose which DIET parts to compile and
install. It is recommended for beginners to compile the client,
the SeD and the agent in one single step with \texttt{make all}. But
please pay attention to the fact that \texttt{make all} does not
install DIET in the prefix provided at configuration time. To do this,
run \texttt{make install}.

\texttt{make install} will run the compilation for all DIET entities
before the installation itself. Thus, if the user wants to compile
only the agent, for instance, and install it, he must run:
{\footnotesize
\begin{verbatim}
~/DIET > cd src/agent
~DIET/src/agent > make install
\end{verbatim}
}


\section{Compiling the examples}

Four series of examples are provided in the DIET archive:
\begin{itemize}
\item{\texttt{file\_transfer}}: the server computes the sizes of two
  input files and returns them. A third output parameter may be
  returned; the server decides randomly whether to send back the
  first file. This is to show how to manage a variable number of
  arguments: the profile declares all arguments that may be filled,
  even if they might not be all filled at each request/computation.

\item{\texttt{dmat\_manips}}: the server offers matrix manipulation
  routines: transposition (\texttt{T}), product (\texttt{MatPROD}) and
  sum (\texttt{MatSUM}, \texttt{SqMatSUM} for square matrices, and
  \texttt{SqMatSUM\_opt} for square matrices but re-using the memory
  space of the second operand for the result). Any subset of these
  operations can be specified on the command line. The last two of
  them are given for compatibility with a BLAS server as explained below.
  
\item{\texttt{BLAS}}: the server offers the \texttt{dgemm} BLAS
  functionality.  We plan to offer all BLAS (Basic Linear Algebric
  Subroutines) in the future. Since this function computes
  $C = \alpha AB + \beta C$, it can also compute a matrix-matrix
  product, a sum of square matrices, etc. All these services are
  offered by the BLAS server. Two clients are designed to use these
  services: one (\texttt{dgemm\_client.c}) is designed to use the
  \texttt{dgemm\_} function only, and the other one
  (\texttt{client.c}) to use all BLAS functions (but currently only
  \texttt{dgemm\_}) and sub-services, such as \texttt{MatPROD}.
  
\item{\texttt{\scalapack}}: the server is designed to offer all
  \scalapack\  (parallel version of the LAPACK library) functions but
  only manages the \texttt{pdgemm\_} function so far. The
  \texttt{pdgemm\_} routine is the parallel version of the
  \texttt{dgemm\_} function, so that the server also offers all the
  same sub-services. Two clients are designed to use these services:
  one (\texttt{pdgemm\_client.c}) is designed to use the
  \texttt{pdgemm\_} function only, and the other one
  (\texttt{client.c}) to use all \scalapack\ functions and
  sub-services, such as \texttt{MatPROD}.
\end{itemize}

\subsection{Compiling the examples with cmake}
\verb+Cmake+ will compile the examples when setting the 
\verb+DIET_BUILD_EXAMPLES+ to \verb+ON+ which can be achieved by
toggling the corresponding entry of \verb+ccmake+ GUI's or by adding
\verb+-DDIET_BUILD_EXAMPLES:BOOL=ON+ to the command line
arguments of \verb+[c]cmake+ invocation.

\subsection{Compiling the examples with the autotools}

Running \texttt{make install} in the \texttt{examples} directory (or
in the root directory, when DIET is configured with
\texttt{--enable-examples}) does not copy binary files into
\texttt{$<$install\_dir$>$/bin}, but in \texttt{examples/bin}: this
odd behaviour is due to some limitations of the \textsf{automake} tool.

Likewise, the samples of configuration files located in
\texttt{examples/cfgs} are processed by \texttt{make install} to
create ready-to-use configuration files in \texttt{examples/etc} and
then copied into \texttt{$<$install\_dir$>$/etc}. Successive calls to
make install will not erase the configuration files created and copied
the first time, except if \texttt{make uninstall} is called inbetween
; thus the user will not lose changes to these files.



%
% DIET data
%
\newpage
%****************************************************************************%
%* DIET User's Manual data chapter file                                     *%
%*                                                                          *%
%*  Author(s):                                                              *%
%*    - Philippe COMBES (Philippe.Combes@ens-lyon.fr)                       *%
%*                                                                          *%
%* $LICENSE$                                                                *%
%****************************************************************************%
%* $Id$
%* $Log$
%* Revision 1.12  2005/06/14 08:16:06  ecaron
%* Typo
%*
%* Revision 1.11  2005/04/22 10:35:53  ycaniou
%* The fourth arg of diet_string_set(), size_t length, does not exist
%*
%* Revision 1.10  2004/08/27 17:04:36  ctedesch
%* - update DIET J chapter :
%* use of the PIF DIET
%* JXTA -> DIET J
%*
%* - syntax corrections in data chapter (_ -> \_)
%*
%* Revision 1.9  2004/07/20 08:34:20  bdelfabr
%* corrections. Thanks Matthias.
%*
%* Revision 1.8  2004/07/01 08:53:10  bdelfabr
%* add data management part in section 3.
%*
%* Revision 1.7  2004/02/10 01:58:08  ecaron
%* Add suggesstions from Matthias Colin. Thanks!
%*
%* Revision 1.6  2004/02/10 00:38:25  ecaron
%* Add suggestions from Jean-Yves L'Excellent. Thanks !
%*
%* Revision 1.5  2004/01/29 17:08:47  ecaron
%* Add suggestions from Frederic Desprez. Thanks !
%*
%* Revision 1.3  2004/01/21 00:25:13  ecaron
%* Add suggestions from Holly Dail. Thanks !
%*
%* Revision 1.2  2003/12/12 10:15:29  avernois
%*
%* precision :les types DIET_SCOMPLEX et DIET_DCOMPLEX 
%*            sont encore en status de TODO
%*
%* Revision 1.1  2003/09/09 12:38:20  pcombes
%* Reorganization of doc: UM becomes UsersManual.
%*
%* Revision 1.7  2003/06/02 13:47:05  pcombes
%* Fix footnotesize.
%*
%* Revision 1.6  2003/05/23 09:23:35  pcombes
%* Add suggestions from Jean-Yves. Thanks !
%*
%* Revision 1.5  2003/05/15 14:17:58  pcombes
%* UM 0.7
%*
%* Revision 1.2  2003/01/24 16:58:54  pcombes
%* UM 0.6.4
%*
%* Revision 1.1  2003/01/22 17:34:53  pcombes
%* User Manual, v. 0.6.4
%****************************************************************************%

\chapter{DIET data}
\label{ch:data}

It is important that DIET can manipulate data to optimize copies and memory
allocation, to estimate data transfer and computation time, etc.  Therefore the
data must be fully described: their types first, then all the attributes
associated to these types.



\section{Data types}
\label{sec:types}

DIET defines a precise set of data types to be used to describe the arguments of
the services (on the server side) and of the problems (on the client side).

The DIET data types are defined in the file
\texttt{$<$install\_dir$>$/include/DIET\_data.h}. The user will also find in
this file various function prototypes to manipulate all DIET data types. Please
refer to this file for a complete and up-to-date API description.

To keep DIET type descriptions generic, two main sets are used: base and
composite types.

\subsection{Base types}
\label{ssec:base}

Base types are defined in an enum type \texttt{diet\_base\_type\_t} and have the
following semantics:
\begin{center}
\footnotesize
\begin{tabular}{|l|l|c|}
\hline
\textbf{Type}&\textbf{Description}&\textbf{Size in octets}\\
\hline
\textsf{DIET\_CHAR}     & Character                &  1\\
\textsf{DIET\_BYTE}     & Octet                    &  1\\
\textsf{DIET\_INT}      & Signed integer           &  4\\
\textsf{DIET\_LONGINT}  & Long signed integer      &  8\\
\textsf{DIET\_FLOAT}    & Simple precision real    &  4\\
\textsf{DIET\_DOUBLE}   & Double precision real    &  8\\
\hline\hline
\textsf{DIET\_SCOMPLEX} & Simple precision complex &  8\\
\textsf{DIET\_DCOMPLEX} & Double precision complex & 16\\
\hline
\end{tabular}
\end{center}

\begin{itemize}
\item[NB:] \textsf{DIET\_SCOMPLEX} and \textsf{DIET\_DCOMPLEX} are not implemented yet.
\end{itemize}


\subsection{Composite types}
\label{ssec:complex}

Composite types are defined in an enum type \texttt{diet\_type\_t}:
\begin{center}
\footnotesize
\begin{tabular}{|l|l|}
\hline
\textbf{Type}&\textbf{Possible base types}\\
\hline
\textsf{DIET\_SCALAR} & all base types\\
\textsf{DIET\_VECTOR} & all base types\\
\textsf{DIET\_MATRIX} & all base types\\
\textsf{DIET\_STRING} & \textsf{DIET\_CHAR}\\
\textsf{DIET\_FILE}   & \textsf{DIET\_CHAR}\\
\hline
\end{tabular}
\end{center}

Each of these types requires specific parameters to completely describe the
data (see Figure \ref{fig:data}).


\subsection{Persistence mode}
\label{ssec:persismode}
Persistence mode is defined in an enum type \texttt{diet\_persistence\_mode\_t}

\begin{center}
\footnotesize
\begin{tabular}{|l|l|}
\hline
\textbf{mode}&\textbf{Description}\\
\hline
\textsf{DIET\_VOLATILE} & not stored\\
\textsf{DIET\_PERSISTENT\_RETURN} & stored on server, movable and copy back to client\\
\textsf{DIET\_PERSISTENT} & stored on server and movable\\
\textsf{DIET\_STICKY} & stored and non movable\\
\textsf{DIET\_STICKY\_RETURN} & stored, non movable and copy back to client\\
\hline
\end{tabular}
\end{center}

\begin{itemize}
\item[NB:] \textsf{DIET\_STICKY\_RETURN} is not implemented yet.
\end{itemize}

\section{Data description}
\label{sec:datadesc}

Each parameter of a client problem is manipulated by DIET using the following
structure:
{\footnotesize
\begin{verbatim}
typedef struct diet_arg_s diet_arg_t;
struct diet_arg_s{
  diet_data_desc_t desc;
  void            *value;
};
typedef diet_arg_t diet_data_t;
\end{verbatim}
}

The second field is a pointer to the memory zone where the parameter data are
stored. The first one consists of a complete DIET data description, which is
better described by a figure than with C code, since it can be set and accessed
through API functions. Figure \ref{fig:data} shows the data classification used
in DIET. Every ``class'' inherits from the root ``class'' \texttt{data}, and
could also be a parent of more detailed classes of data in future versions of
DIET.

\begin{figure}[hpt]
 \begin{center}
  \includegraphics[scale=.5]{fig/data.eps}
  \caption{Argument/Data structure description.}
  \label{fig:data}
 \end{center}
\end{figure}


\section{Data management}
\label{sec:datamgt}

\subsection{Data identifier}
\label{ssec:dataid}
The data identifier is generated by the MA. The data identifier is a
string field that contains the MA name, the number of the session plus
the number of the data in the problem (incremental) plus the string
``id''.  This is the \texttt{id} field of the
\texttt{diet\_data\_desc\_t} structure.

{\footnotesize
\begin{verbatim}
typedef struct {
  char* id;  
  diet_persistence_mode_t  mode;
  ....
} diet_data_desc_t;
\end{verbatim}
}

For example, \textbf{id.MA1.1.1} will identify the first data
in the first session submitted on the Master Agent \textbf{MA1}.


\begin{itemize}
\item[NB:] the field ``id'' of the identifier will be next replaced by a
client identifier. This is not implemented yet.
\end{itemize}

\subsection{Data file}
\label{ssec:datafile}

The name of the file is generated by a Master Agent. It is created
during the \texttt{diet\_initialize()} call. The name of the file is
the aggregation of the string ID\_FILE plus the name of the MA plus
the number of the session.  

A file is created only when there are some persistent data in the
session.  

For example, \textbf{ID\_FILE.MA1.1} means the identifiers
of the persistent data stored are in the file corresponding to the
first session in the Master Agent \textbf{MA1}.

The file is stored in the \texttt{/tmp} directory.

\begin{itemize}
\item[NB:] for the moment, when a data is erased from the platform, the
file isn't updated.
\end{itemize}


\section{Manipulating DIET structures}
\label{sec:manip}

The user will notice that the API to the DIET data structures consists of
modifier and accessor functions only: no allocation function is required, since
\texttt{diet\_profile\_alloc} (see Section \ref{sec:pbdesc}) allocates all
necessary memory for all argument \textbf{descriptions}. This avoids the
temptation for the user to allocate the memory for these data structures twice
(which would lead to DIET errors while reading profile arguments). Please see
the example in Section \ref{sec:pbex} for a typical use.
\\

Moreover, the user should know that arguments of the \texttt{\_set} functions
that are passed by pointers are \textbf{not} copied, in order to save memory.
This is true for the \emph{value} arguments, but also for the \emph{path} in
\texttt{diet\_file\_set}. Thus, the user keeps ownership of the memory zones
pointed at by these pointers, and he/she must be very careful not to alter it
during a call to DIET.

\subsection{Set functions}

%The data persistence is not available in this current
%version~\footnote{the data persistence will be available in DIET
%  v1.1}, thus fix the mode of the diet persistence parameter to 0.

\label{sec:setfun}
{\footnotesize
\begin{verbatim}
/**
 * On the server side, these functions should not be used on arguments, but only
 * on convertors (see section 5.5).
 * If mode                                is DIET_PERSISTENCE_MODE_COUNT, 
 * or if base_type                        is DIET_BASE_TYPE_COUNT,
 * or if order                            is DIET_MATRIX_ORDER_COUNT,
 * or if size, nb_rows, nb_cols or length is 0,
 * or if path                             is NULL,
 * then the corresponding field is not modified.
 */

int
diet_scalar_set(diet_arg_t* arg, void* value, diet_persistence_mode_t mode,
                diet_base_type_t base_type);
int
diet_vector_set(diet_arg_t* arg, void* value, diet_persistence_mode_t mode,
                diet_base_type_t base_type, size_t size);

/* Matrices can be stored by rows or by columns */
typedef enum {
  DIET_COL_MAJOR = 0,
  DIET_ROW_MAJOR,
  DIET_MATRIX_ORDER_COUNT
} diet_matrix_order_t;

int
diet_matrix_set(diet_arg_t* arg, void* value, diet_persistence_mode_t mode,
                diet_base_type_t base_type,
                size_t nb_rows, size_t nb_cols, diet_matrix_order_t order);
int
diet_string_set(diet_arg_t* arg, char* value, diet_persistence_mode_t mode);

/* Computes the file size
   ! Warning ! The path is not duplicated !!! */
int
diet_file_set(diet_arg_t* arg, diet_persistence_mode_t mode, char* path);
\end{verbatim}
}


\subsection{Access functions}
\label{sec:accessfun}
{\footnotesize
\begin{verbatim}
/**
 * A NULL pointer is not an error (except for arg): it is simply IGNORED.
 * For instance,
 *   diet_scalar_get(arg, &value, NULL),
 * will only set the value to the value field of the (*arg) structure.
 * 
 * NB: these are macros that let the user not worry about casting (int **)
 * or (double **) etc. into (void **).
 */

/**
 * Type: int diet_scalar_get((diet_arg_t *), (void *),
 *                           (diet_persistence_mode_t *))
 */
#define diet_scalar_get(arg, value, mode) \
        _scalar_get(arg, (void *)value, mode)
/**
 * Type: int diet_vector_get((diet_arg_t *), (void **),
 *                           (diet_persistence_mode_t *), (size_t *))
 */
#define diet_vector_get(arg, value, mode, size) \
        _vector_get(arg, (void **)value, mode, size)
/**
 * Type: int diet_matrix_get((diet_arg_t *), (void **),
 *                           (diet_persistence_mode_t *),
 *                           (size_t *), (size_t *), (diet_matrix_order_t *))
 */
#define diet_matrix_get(arg, value, mode, nb_rows, nb_cols, order) \
        _matrix_get(arg, (void **)value, mode, nb_rows, nb_cols, order)
/**
 * Type: int diet_string_get((diet_arg_t *), (char **),
 *                           (diet_persistence_mode_t *), (size_t *))
 */
#define diet_string_get(arg, value, mode, length) \
        _string_get(arg, (char **)value, mode, length)
/**
 * Type: int diet_file_get((diet_arg_t *),
 *                         (diet_persistence_mode_t *), (size_t *), (char **))
 */
#define diet_file_get(arg, mode, size, path) \
        _file_get(arg, mode, size, (char **)path)
\end{verbatim}
}


\section{Data Management functions}

\begin{itemize}
\item {The \texttt{store\_id} method} is used to store the identifier of a persistent data.
 It also allows to give a description of the data stored.
 This method has to be called after the \texttt{diet\_call()}
\begin{verbatim}
  store_id(char* argID,char *msg);
\end{verbatim}

\item the \texttt{diet\_use\_data} method allows the client to use a data
that is already inside the platform.
\begin{verbatim}
  diet_use_data(diet_arg_t* arg,char* argID);
\end{verbatim}
This function replaces the set functions (see Section \ref{sec:setfun}).



\begin{itemize}
\item[NB:] a mecanisms of publication of data identifiers isnt
implemented yet. So, exchanges of identifiers between end-users that
want to share data must be explicitly done.
\end{itemize}



\item {The \texttt{diet\_free\_persistent\_data} method} allows the
client to remove a persistent data from the platform.
\begin{verbatim}
  diet_free_persistent_data(char *argID);
\end{verbatim}

\end{itemize}


{\footnotesize
\begin{verbatim}

/*******************************************************************
 *   Add handler argID and text message msg in the identifier file *
 ******************************************************************/

void 
store_id(char* argID, char* msg);


/** sets only identifier : data is present inside the platform */

void
diet_use_data(diet_arg_t* arg, char* argID);


/******************************************************************
 *  Free persistent data identified by argID                     *
 *****************************************************************/
int
diet_free_persistent_data(char* argID);

\end{verbatim}
}


\subsection{Free functions}
\label{sec:freefun}

The amount of data  pointed at by value fields should be freed through a DIET
API function:
{\footnotesize
\begin{verbatim}
/****************************************************************************/
/* Free the amount of data pointed at by the value field of an argument.    */
/* This should be used ONLY for VOLATILE data,                              */
/*    - on the server for IN arguments that will no longer be used          */
/*    - on the client for OUT arguments, after the problem has been solved, */
/*      when they will no longer be used.                                   */
/* NB: for files, this function removes the file and frees the path (since  */
/*     it has been dynamically allocated by DIET in both cases)             */
/****************************************************************************/

int
diet_free_data(diet_arg_t* arg);
\end{verbatim}
}


\section{Problem description}
\label{sec:pbdesc}

For DIET to match the client problem with a service, servers and clients must
``speak the same language'', \emph{ie} they must use the same problem
description. A unified way to describe problems is to use a name and define its
profile with the type \texttt{diet\_profile\_t}:
{\footnotesize
\begin{verbatim}
typedef struct {
  int         last_in, last_inout, last_out;
  diet_arg_t *parameters;
} diet_profile_t;
\end{verbatim}
}

%
% FIXME:
% as soon as persistency is integrated, this should become the table Eddy
% prepared to explain the transfer policy, depending on the modes.
%

The field \emph{parameters} consists of a \texttt{diet\_arg\_t} array of size
$last\_out + 1$. Arguments can be
\begin{description}
\item{IN:}    The data are sent to the server. The memory is allocated
  by the user.
\item{INOUT:} The data are allocated by the user as for the IN
  arguments, then sent to the server and brought back into the same memory zone
  after the computation has completed, without any copy. Thus freeing this
  memory while the computation is performed on the server would result in a
  segmentation fault when the data are brought back onto the client.
\item{OUT:} The data are created on the server and brought back into a
  newly allocated zone on the client. This allocation is performed by
  DIET. After the call has returned, the user can find its result in
  the zone pointed at by the \emph{value} field. Of course, DIET
  cannot guess how long the user needs these data for, so it lets
  him/her free the memory allocated with \texttt{diet\_free\_data}.
\end{description}

This behaviour will be modified soon with the introduction of the persistence
modes that will let the user leave some data on the servers for later
computations.

The fields \emph{last\_in}, \emph{last\_inout} and \emph{last\_out} of the
\texttt{diet\_profile\_t} structure respectively point at the indexes in the
\emph{parameters} array of the last IN, INOUT and OUT arguments.

Functions to create and destroy such profiles are defined with the prototypes
below:
{\footnotesize
\begin{verbatim}
diet_profile_t *diet_profile_alloc(int last_in, int last_inout, int last_out);
int diet_profile_free(diet_profile_t *profile);
\end{verbatim}
}



\section{Examples}
\label{sec:pbex}

\subsection{Example 1 : without persistency}
Let us consider the product of a scalar by a matrix: the matrix must be
multiplied in-place, and the computation time must be returned.  This
problem has one IN argument (the scalar factor), one INOUT argument (the matrix)
and one OUT argument (the computation time), so its profile will be built as
follows:
\begin{center}
\includegraphics[scale=.35]{fig/smprod.eps}
\end{center}

Here are the lines of C code to generate such a profile:
{\footnotesize
\begin{verbatim}
  double  factor;
  double *matrix;
  float  *time;
  // Init matrix at least, factor and time too would be better ...
  // ...
  diet_profile_t profile = diet_profile_alloc(0, 1, 2); // last_in, last_inout, last_out
  diet_scalar_set(diet_parameter(profile,0), &factor, 0, DIET_DOUBLE);
  diet_matrix_set(diet_parameter(profile,1), matrix,  0, DIET_DOUBLE, 5, 6, DIET_ROW_MAJOR);
  diet_scalar_set(diet_parameter(profile,2), NULL,    0, DIET_FLOAT);
\end{verbatim}
}

\begin{itemize}
\item[NB1:] If there is no IN argument, \emph{last\_in} must be set to
  -1, if there is no INOUT argument is needed, \emph{last\_inout} must
  be equal to \emph{last\_in}, and if there is no OUT argument,
  \emph{last\_out} must be equal to \emph{last\_inout}.
\item[NB2:] The \emph{value} argument for \texttt{\_set} functions
  (\ref{sec:setfun}) is ignored for OUT arguments, since DIET
  allocates the necessary memory space when the corresponding data are
  transfered from the server, so set value to NULL.
\end{itemize}

\subsection{Example 2 : using persistency}


Let us consider the following problem : $C=A*B$, with A,B and C persistent matrices.


{\footnotesize
\begin{verbatim}
  double *A, *B, *C; 
  // matrices initialization
  ...
  diet_initialize();
  strcpy(path,"MatPROD");
  profile = diet_profile_alloc(path, 1, 1, 2);
  diet_matrix_set(diet_parameter(profile,0),
                  A, DIET_PERSISTENT, DIET_DOUBLE, mA, nA, oA);
  print_matrix(A, mA, nA, (oA == DIET_ROW_MAJOR));
  diet_matrix_set(diet_parameter(profile,1),
                  B, DIET_PERSISTENT, DIET_DOUBLE, mB, nB, oB);
  print_matrix(B, mB, nB, (oB == DIET_ROW_MAJOR));
  diet_matrix_set(diet_parameter(profile,2),
                  NULL, DIET_PERSISTENT_RETURN, DIET_DOUBLE, mA, nB, oC);
  
  if (!diet_call(profile)) {
    diet_matrix_get(diet_parameter(profile,2),&C, NULL, &mA, &nB, &oC);
    store_id(profile->parameters[2].desc.id,"matrix C of doubles");
    store_id(profile->parameters[1].desc.id,"matrix B of doubles");
    store_id(profile->parameters[0].desc.id,"matrix A of doubles");
    print_matrix(C, mA, nB, (oC == DIET_ROW_MAJOR));
      
  }

  diet_profile_free(profile);
  diet_finalize();
\end{verbatim}
}

Then, a client submits the problem : $D=E+C$ with C already present in the platform.
We consider that the handle of C is ``id.MA1.1.3''.

{\footnotesize
\begin{verbatim}
  double *C, *D, *E; 
  // matrices initialization
  ...
  diet_initialize();

  strcpy(path,"MatSUM");
  profile2 = diet_profile_alloc(path, 1, 1, 2);
  
  printf("second pb\n\n");
  diet_use_data(diet_parameter(profile2,0), "id.MA1.1.3");
  diet_matrix_set(diet_parameter(profile2,1),
                  E, DIET_PERSISTENT, DIET_DOUBLE, mA, nB, oE);
  print_matrix(E, mA, nB, (oE == DIET_ROW_MAJOR));
  diet_matrix_set(diet_parameter(profile2,2),
                  NULL, DIET_PERSISTENT_RETURN, DIET_DOUBLE, mA, nB, oD);
  
  if (!diet_call(profile2)) {
    diet_matrix_get(diet_parameter(profile2,2), &D, NULL, &mA, &nB, &oD);
   print_matrix(D, mA, nB, (oD == DIET_ROW_MAJOR));
   store_id(profile2->parameters[2].desc.id,"matrice D de doubles");
   store_id(profile2->parameters[1].desc.id,"matrice E de doubles");
  
  }
  diet_profile_free(profile2);
  diet_free_persistent_data("id.MA1.1.3");
  diet_finalize();
\end{verbatim}
}  

Note that when in the same session, when a data is persistent and used
twice, we cannot know what will be its identifier. However, the
identifier is stored in the structure of the first profile use for
computations. For example, consider a matrix A built with
\texttt{diet\_matrix\_set()} method as following~: {\footnotesize
\begin{verbatim}
  ...
  diet_profile_t *profile;
  ...
  diet_matrix_set(diet_parameter(profile,0),
                  E, DIET_PERSISTENT, DIET_DOUBLE, mA, nA, oA);
  ...
\end{verbatim}
} After the first \texttt{diet\_call}, the identifier of A is stored in
the profile (in \texttt{profile->parameters[0].desc.id}). So, for the
second call, we will have the following instruction in order to use A~:
{\footnotesize
\begin{verbatim}
  ...
  diet_profile_t *profile2;
  ...
  diet_use_data(diet_parameter(profile2,0),profile->parameters[0].desc.id);
  ...
\end{verbatim}
}

\begin{itemize}
\item[NB:] when using this method, the first profile (here
\texttt{profile}) must not be freed before its use. 
\end{itemize}


%
% Building a client program
%
\newpage
%****************************************************************************%
%* DIET User's Manual client chapter file                                   *%
%*                                                                          *%
%*  Author(s):                                                              *%
%*    - Eddy CARON      (Eddy.Caron@ens-lyon.fr)                            *%
%*    - Philippe COMBES (Philippe.Combes@ens-lyon.fr)                       *%
%*    - Christophe Pera (Christophe.Pera@ens-lyon.fr)                       *%
%*                                                                          *%
%* $LICENSE$                                                                *%
%****************************************************************************%
%* $Id$
%* $Log$
%* Revision 1.17  2010/01/21 14:05:58  bdepardo
%* DIET -> \diet
%* SeD -> \sed
%* GoDIET -> \godiet
%*
%* Revision 1.16  2008/07/11 10:11:53  ecaron
%* - Ajout def API (un user a demand�)
%* - Pas mal de modifications sur le chapitre des workflows
%* 	* Corrections
%* 	* Prises en compte des remarques de Brice
%* - Fixme pour Benjamin corriger la vision 2 archi -> 1 archi
%*
%* Revision 1.15  2006/11/15 22:36:50  eboix
%*   Documentation concerning user compilation updated. --- Injay2461
%*
%* Revision 1.14  2005/07/12 21:44:28  hdail
%* - Correcting small problems throughout
%* - Modified deployment chapter to have a real section for deploying via GoDIET
%* - Adding short xml example without the comments to make a figure in GoDIET
%*   section.
%*
%* Revision 1.13  2005/06/27 08:59:41  hdail
%* Updating interfaces to agree with current versions and adding memory cleanup
%* to end of example programs.
%*
%* Revision 1.12  2005/06/14 08:15:40  ecaron
%* Typo
%*
%* Revision 1.11  2004/07/08 15:59:10  mcolin
%* correct the asynchronous example from the tutorial and user manual
%* FIXME :
%*  - still a bug with the INOUT parameter : the matrix is not modified
%*  - User Manuel [5.1] : the first example service/solve_service doesn't
%*  match with the actual version of DIET (pb with pointer and diet_string_get
%*  which doesn't exist)
%*
%* Revision 1.10  2004/02/10 00:44:24  ecaron
%* Add suggestions from Christophe Pera. Thanks !
%*
%* Revision 1.9  2004/02/02 15:57:15  ecaron
%* Add suggestion from Christophe Pera. Thanks !
%*
%* Revision 1.8  2004/01/29 17:08:47  ecaron
%* Add suggestions from Frederic Desprez. Thanks !
%*
%* Revision 1.7  2004/01/27 00:21:17  ecaron
%* Add suggestions from Jean-Yves (Thanks!)
%*
%* Revision 1.6  2004/01/21 00:25:13  ecaron
%* Add suggestions from Holly Dail. Thanks !
%*
%* Revision 1.5  2004/01/07 10:27:33  cpera
%* Add asynchronous call example and useAsyncAPI config parameter.
%*
%* Revision 1.4  2004/01/05 13:06:53  ecaron
%* Update example to DIET 1.0
%****************************************************************************%

\chapter{Building a client program}
\label{ch:client}

The most difficult part of building a client program is to understand how
to describe the problem interface. Once this step is done, it is
fairly easy to build calls to \diet.

\section{Structure of a client program}
\label{sec:cl_struct}

Since the client side of \diet is a library, a client program has to define a
\texttt{main} function that uses \diet through function calls. The complete
client-side interface is described in the files
\texttt{DIET\_data.h} (see Chapter \ref{ch:data}) and
\texttt{DIET\_client.h} found in \texttt{$<$install\_dir$>$/include}.
Please refer to these two files for a complete and
up-to-date API~\footnote{Application programming interface}
description, and include at least the latter at the beginning of
your source code (\texttt{DIET\_client.h} includes \texttt{DIET\_data.h}):
{\footnotesize
\begin{verbatim}
#include <stdio.h>
#include <stdlib.h>

#include "DIET_client.h"

int main(int argc, char *argv[])
{
  diet_initialize(configuration_file, argc, argv);
  // Successive DIET calls ...
  diet_finalize();
}
\end{verbatim}
}

The client program must open its \diet session with a call to
\texttt{diet\_initialize}, which parses the configuration file to set
all options and get a reference to the \diet Master Agent. The session
is closed with a call to \texttt{diet\_finalize}, which frees all
resources associated with this session on the client. Note that
memory allocated for all INOUT and OUT arguments brought back onto
the client during the session is not freed during
\texttt{diet\_finalize}; this allows the user to continue to use the
data, but also requires that the user explicitly free the memory.
The user must also free the memory he or she allocated for IN
arguments.

\section{Client API}
\label{sec:clAPI}

The client API follows the GridRPC definition \cite{gridRPC:02}: all
\texttt{diet\_} functions are ``duplicated'' with \texttt{grpc\_}
functions.  Both \texttt{diet\_initialize}/\texttt{grpc\_initialize}
and \texttt{diet\_finalize}/\texttt{grpc\_finalize} belong to the
GridRPC API. 
 
    A problem is managed through a \emph{function\_handle}, that
associates a server to a problem name. For compliance with GridRPC
\diet accepts \texttt{diet\_function\_handle\_init}, but the server 
specified in the call will be ignored; \diet is designed to
automatically select the best server. The structure allocation is
performed through the function
\texttt{diet\_function\_handle\_default}.

The \emph{function\_handle} returned is associated to the problem description,
its profile, in the call to \texttt{diet\_call}.

\section{Examples}
\label{sec:cl_ex}

Let us consider the same example as in Section \ref{sec:pbex}, but
for synchronous and asynchronous calls.  Here, the client
configuration file is given as the first argument on the command
line, and we decide to hardcode the matrix, its factor, and the name
of the problem.

\subsection{Synchronous call}
\texttt{smprod}
%~\footnote{Source code available in \texttt{doc/tutorial/solutions/exercise2/client\_smprod.c}} 
for scalar by matrix product.

{\footnotesize
\begin{verbatim}
#include <stdio.h>
#include <stdlib.h>
#include <math.h>
#include "DIET_client.h"

int main(int argc, char **argv)
{
  int i;
  double  factor = M_PI; /* Pi, why not ? */
  double *matrix;        /* The matrix to multiply */
  float  *time   = NULL; /* To check that time is set by the server */

  diet_profile_t         *profile;

  /* Allocate the matrix: 60 lines, 100 columns */
  matrix = malloc(60 * 100 * sizeof(double));
  /* Fill in the matrix with dummy values (who cares ?) */
  for (i = 0; i < (60 * 100); i++) {
    matrix[i] = 1.2 * i;
  }
  
  /* Initialize a DIET session */
  diet_initialize("./client.cfg", argc, argv);

  /* Create the profile as explained in Chapter 3 */
  profile = diet_profile_alloc("smprod",0, 1, 2); // last_in, last_inout, last_out
  
  /* Set profile arguments */
  diet_scalar_set(diet_parameter(profile,0), &factor, 0, DIET_DOUBLE);
  diet_matrix_set(diet_parameter(profile,1), matrix,  0, DIET_DOUBLE, 60, 100, DIET_COL_MAJOR);
  diet_scalar_set(diet_parameter(profile,2), NULL,    0, DIET_FLOAT);
  
  if (!diet_call(profile)) { /* If the call has succeeded ... */
     
    /* Get and print time */
    diet_scalar_get(diet_parameter(profile,2), &time, NULL);
    if (time == NULL) {
      printf("Error: time not set !\n");
    } else {
      printf("time = %f\n", *time);
    }

    /* Check the first non-zero element of the matrix */
    if (fabs(matrix[1] - ((1.2 * 1) * factor)) > 1e-15) {
      printf("Error: matrix not correctly set !\n");
    }
  }

  /* Free profile */
  diet_profile_free(profile);
  diet_finalize();
  free(matrix);
  free(time);
}
\end{verbatim}
}

\subsection{Asynchronous call}
\texttt{smprod}
%~\footnote{Source code available in \texttt{doc/tutorial/solutions/exercise2/client\_smprodAsync.c}} 
for scalar by matrix product.
{\footnotesize
\begin{verbatim}
#include <stdio.h>
#include <stdlib.h>
#include <math.h>
#include "DIET_client.h"

int main(int argc, char **argv)
{
  int i, j;
  double  factor = M_PI; /* Pi, why not ? */
  size_t m, n; /* Matrix size */
  double *matrix[5];        /* The matrix to multiply */
  float  *time   = NULL; /* To check that time is set by the server */

  diet_profile_t         *profile[5];
  diet_reqID_t rst[5] = {0,0,0,0,0};

  m = 60;
  n = 100;
 
  /* Initialize a DIET session */
  diet_initialize("./client.cfg", argc, argv);

  /* Create the profile as explained in Chapter 3 */
  for (i = 0; i < 5; i++){
    /* Allocate the matrix: m lines, n columns */
    matrix[i] = malloc(m * n * sizeof(double));
    /* Fill in the matrix with dummy values (who cares ?) */
    for (j = 0; j < (m * n); j++) {
      matrix[i][j] = 1.2 * j;
    }
    profile[i] = diet_profile_alloc("smprod",0, 1, 2); // last_in, last_inout, last_out
  
    /* Set profile arguments */
    diet_scalar_set(diet_parameter(profile[i],0), &factor, 0, DIET_DOUBLE);
    diet_matrix_set(diet_parameter(profile[i],1), matrix[i],  0, DIET_DOUBLE,
                    m, n, DIET_COL_MAJOR);
    diet_scalar_set(diet_parameter(profile[i],2), NULL,    0, DIET_FLOAT);
  }
  
  /* Call Diet */
  int rst_call = 0;
  
  for (i = 0; i < 5; i++){
     if ((rst_call = diet_call_async(profile[i], &rst[i])) != 0)  
        printf("Error in diet_call_async return -%d-\n", rst_call);
     else {
       printf("request ID value = -%d- \n", rst[i]);
       if (rst[i] < 0) {
         printf("error in request value ID\n");
         return 1;
       }
     }
     rst_call = 0;
  }   

  /* Wait for Diet answers */
  if ((rst_call = diet_wait_and((diet_reqID_t*)&rst, (unsigned int)5)) != 0)
     printf("Error in diet_wait_and\n");
  else {
    printf("Result data for requestID");
    for (i = 0; i < 5; i++) printf(" %d ", rst[i]);
    for (i = 0; i < 5; i++){
      /* Get and print time */
      diet_scalar_get(diet_parameter(profile[i],2), &time, NULL);
      if (time == NULL) {
        printf("Error: time not set !\n");
      } else {
        printf("time = %f\n", *time);
      }

      /* Check the first non-zero element of the matrix */
      if (fabs(matrix[i][1] - ((1.2 * 1) * factor)) > 1e-15) {
        printf("Error: matrix not correctly set !\n");
      }
    }
  }
  /* Free profiles */
  for (i = 0; i < 5; i++){
    diet_cancel(rst[i]);
    diet_profile_free(profile[i]);
    free(matrix[i]);
  }
  free(time);
  diet_finalize();
  return 0;
}
\end{verbatim}
}

\section{Compilation}
\label{sec:cl_comp}

After compiling the client program, the user must link it with the
\diet libraries and the CORBA libraries.

\subsection{Compilation when using Makefiles}

When using Makefiles, the easiest way to compile a program using \diet
with all necessary flags and link it with the proper libraries is to
trust the \texttt{Makefile.inc} available in 
\texttt{$<$include\_dir$>$/include} by including it at the beginning
of the program makefile.

The \texttt{Makefile.inc} defines the variables:
\begin{itemize}
\item \texttt{CCFLAGS\_DIET} which contains pre-processing instructions 
to be used when you compile C code,
\item \texttt{CXXFLAGS\_DIET} which contains pre-processing instructions
to be used when you compile C++ code,
\item \texttt{DIET\_CLIENT\_LIBS} which contains the linker instructions
in otder to link your program against the \diet client library.
\item \texttt{CC} and \texttt{CXX} which contains the name of the C compiler,
respectively C++ compiler, used to compiled \diet itself and which you
might choose to compile your own programs (in order to guarantee the 
compatibility of the compilers).
\end{itemize}
\noindent
The \texttt{doc/ExternalExample} directory contains an example of
such a user Makefile, which goes:
{\footnotesize
\begin{verbatim}
# The following inclusion provides convenient make variables for compiling
# and linking against the DIET library (DIET_HOME is an environment variable
# containing the path to a DIET installation directory):
include ${DIET_HOME}/include/Makefile.inc

all: simple_client simple_server

simple_client: simple_client.c $(DIET_CLIENT_PREREQ)
        $(CXX) -g $(CXXFLAGS_DIET) $< $(DIET_CLIENT_LIBS) -o $@
simple_server: simple_server.c  $(DIET_SERVER_PREREQ)
        $(CC) -g $(CCFLAGS_DIET) $< $(DIET_SERVER_LIBS) -o $@
clean:
        rm -f simple_client simple_server

\end{verbatim}
}

Since the \texttt{doc/ExternalExample} directory also contains the
cited \texttt{simple\_server.c} and \texttt{simple\_client.c}, one
can easily test this Makefile.

\subsection{Compilation when using cmake}

The \texttt{doc/ExternalExample} directory also contains a
\texttt{CMakeFile.txt} file which illustrates the cmake way of compiling
this simple client/server example:
{\footnotesize
\begin{verbatim}
PROJECT( DIETSIMPLEEXAMPLE )

SET( CMAKE_MODULE_PATH ${DIETSIMPLEEXAMPLE_SOURCE_DIR}/Cmake )
FIND_PACKAGE( Diet )

# On success use the information we just recovered:
INCLUDE_DIRECTORIES( ${DIET_INCLUDE_DIR} )
LINK_DIRECTORIES( ${DIET_LIBRARY_DIR} )

### Define a simple server...
ADD_EXECUTABLE( simple_server simple_server.c )
TARGET_LINK_LIBRARIES( simple_server ${DIET_SERVER_LIBRARIES} )
INSTALL_TARGETS( /bin simple_server )

### ... and it's associated simple client.
ADD_EXECUTABLE( simple_client simple_client.c )
TARGET_LINK_LIBRARIES( simple_client ${DIET_CLIENT_LIBRARIES} )
INSTALL_TARGETS( /bin simple_client )
\end{verbatim}
}

In order to test drive the cmake configuration of this example, and
assuming the \texttt{DIET\_HOME} points to a directory containing
an installation of \diet, simply try:

{\footnotesize
\begin{verbatim}
export DIET_HOME=<path_to_a_DIET_instal_directory>
cd doc/ExternalExample
mkdir Bin
cd Bin
cmake -DDIET_DIR:PATH=$DIET_HOME -DCMAKE_INSTALL_PREFIX:PATH=/tmp/DIETSimple ..
make
make install
\end{verbatim}
}



% Building a server application
%
\newpage
%**
%*  @file  server.tex
%*  @brief   DIET User's Manual server chapter file 
%*  @author   - Eddy CARON (Eddy.Caron@ens-lyon.fr)
%*            - Philippe COMBES (Philippe.Combes@ens-lyon.fr)
%*            - Georg Hoesh (Georg.Hoesh@ens-lyon.fr)
%*  @section Licence 
%*    |LICENSE|


\chapter{Building a server application}
\label{ch:server}

A \diet server program is the link between the \diet Server Deamon
(SeD) and the libraries that implement the service to offer.

\section{Structure of the program}
\label{sec:sv_struct}

Much like the client side, the \diet SeD is a library. So the server developer
needs to define the \texttt{main} function. Within the \texttt{main}, the \diet
server will be launched with a call to \texttt{diet\_SeD} which will never
return (except if some errors occur, or if a SIGINT or SIGTERM signal is sent
to the \sed). The complete server side interface is described in the files
\texttt{DIET\_data.h} (see Chapter~\ref{ch:data}) and \texttt{DIET\_server.h}
found in \texttt{$<$install\_dir$>$/include}. Do not forget to include the
\texttt{DIET\_server.h} (\texttt{DIET\_server.h} includes
\texttt{DIET\_data.h}) at the beginning of your server source code.

{\footnotesize
\begin{verbatim}
#include <stdio.h>
#include <stdlib.h>

#include "DIET_server.h"
\end{verbatim}
}

The second step is to define a function whose prototype is ``\diet-normalized''
and which will be able to convert the function into the library function prototype.
Let us consider a library function with the following prototype:
{\footnotesize
\begin{verbatim}
int service(int arg1, char *arg2, double *arg3);
\end{verbatim}
}

This function cannot be called directly by \diet, since such a prototype is hard
to manipulate dynamically. The user must define a ``solve'' function whose
prototype only consists of a \texttt{diet\_profile\_t}.
This function will be called by the \diet \sed through a pointer.
{\footnotesize
\begin{verbatim}
int solve_service(diet_profile_t *pb)
{
   int    *arg1;
   char   *arg2;
   double *arg3;

   diet_scalar_get(diet_parameter(pb,0), &arg1, NULL);
   diet_string_get(diet_parameter(pb,1), &arg2, NULL);
   diet_scalar_get(diet_parameter(pb,2), &arg3, NULL);
   return service(*arg1, arg2, arg3);
}
\end{verbatim}
}

Several API functions help the user to write this ``solve''
function, particularly for getting IN arguments as well as setting
OUT arguments.

\subsubsection*{Getting IN, INOUT and OUT arguments}

The \texttt{diet\_*\_get} functions defined in \texttt{DIET\_data.h} are still
usable here. Do not forget that the necessary memory space for OUT arguments is
allocated by \diet. So the user should call the \texttt{diet\_*\_get} functions
to retrieve the pointer to the zone his/her program should write to.

\subsubsection*{Setting INOUT and OUT arguments}

To set INOUT and OUT arguments, use the \texttt{diet\_*\_desc\_set} defined
in \texttt{DIET\_server.h}, these are helpful for writing ``solve''
functions only. Using these functions, the server developer must keep in
mind the fact that he cannot alter the memory space pointed to by
value fields on the server. Indeed, this would make \diet confused
about how to manage the data{\footnote{And the server developer
should not be confused by the fact that
\texttt{diet\_scalar\_desc\_set} uses a value, since scalar values
are copied into the data descriptor.}}.

{\footnotesize
\begin{verbatim}
/**
 * If value                 is NULL,
 * or if order              is DIET_MATRIX_ORDER_COUNT,
 * or if nb_rows or nb_cols is 0,
 * or if path               is NULL,
 * then the corresponding field is not modified.
 */

int
diet_scalar_desc_set(diet_data_t* data, void* value);

// No use of diet_vector_desc_set: size should not be altered by server

// You can alter nb_r and nb_c, but the total size must remain the same
int
diet_matrix_desc_set(diet_data_t* data,
                     size_t nb_r, size_t nb_c, diet_matrix_order_t order);

// No use of diet_string_desc_set: length should not be altered by server

int
diet_file_desc_set(diet_data_t* data, char* path);
\end{verbatim}
}


\section{Server API}
\label{sec:svAPI}


\subsubsection*{Defining services}

First, declare the service(s) that will be offered{\footnote{It is
possible to declare several services in a single SeD.}}.
Each service is described by a profile description called
\texttt{diet\_profile\_desc\_t} since the service does not specify
the sizes of the data. The \texttt{diet\_profile\_desc\_t} type is
defined in \texttt{DIET\_server.h}, and is very similar to
\texttt{diet\_profile\_t}. The difference is that the prototype is
described with the generic parts of \emph{diet\_data\_desc} only,
whereas the client description uses full \emph{diet\_data}.
{\footnotesize
\begin{verbatim}
file DIET_data.h:
     struct diet_data_generic {
       diet_data_type_t type;
       diet_base_type_t base_type;
     };

file DIET_server.h:
     typedef struct diet_data_generic diet_arg_desc_t;

     typedef struct {
       char*            path;
       int              last_in, last_inout, last_out;
       diet_arg_desc_t* param_desc;
     } diet_profile_desc_t;

diet_profile_desc_t* diet_profile_desc_alloc(const char* path,
                        int last_in, int last_inout, int last_out);
int diet_profile_desc_free(diet_profile_desc_t* desc);

diet_profile_desc_t *diet_profile_desc_alloc(int last_in, int last_inout, int last_out);

int diet_profile_desc_free(diet_profile_desc_t *desc);
\end{verbatim}
}

Each profile can be allocated with \texttt{diet\_profile\_desc\_alloc} with the
same semantics as for \texttt{diet\_profile\_alloc}. Every argument of the
profile will then be set with \texttt{diet\_generic\_desc\_set} defined in
\texttt{DIET\_server.h}.

\subsubsection*{Declaring services}

Every service must be added in the service table before the server is
launched. The complete service table API is defined in \texttt{DIET\_server.h}:
{\footnotesize
\begin{verbatim}
typedef int (* diet_solve_t)(diet_profile_t *);
int diet_service_table_init(int max_size);
int diet_service_table_add(diet_profile_desc_t *profile,
                           diet_convertor_t    *cvt,
                           diet_solve_t         solve_func);
void diet_print_service_table();
\end{verbatim}
}

The parameter \texttt{diet\_solve\_t solve\_func} is the type of the
\texttt{solve\_service} function: a function pointer used by \diet to launch the
computation.

The parameter \texttt{diet\_convertor\_t *cvt} is to be used in combination
with scheduling facilities (if available). It is there to allow profile conversion (for
multiple services, or when differences occur between the \diet and the scheduling facility
profile). Profile conversion is complicated and will be treated
separately in Chapter~\ref{chapter:performance}.

\section{Example}
\label{sec:sv_ex}

Let us consider the same example as in Chapter \ref{ch:client}, where
a function \texttt{scal\_mat\_prod} performs the product of a matrix
and a scalar and returns the time required for the computation: {\footnotesize
\begin{verbatim}
int scal_mat_prod(double alpha, double *M, int nb_rows, int nb_cols, float *time);
\end{verbatim}
}
Our program will first define the solve function that consists of the link
between \diet and this function. Then, the \texttt{main} function defines one service and
adds it in the service table with its associated solve function.
{\footnotesize
\begin{verbatim}
#include <stdio.h>
#include "DIET_server.h"
#include "scal_mat_prod.h"

int solve_smprod(diet_profile_t *pb)
{
  double *alpha;
  double *M;
  float  time;
  size_t m, n;
  int res;

  /* Get arguments */
  diet_scalar_get(diet_parameter(pb,0), &alpha, NULL);
  diet_matrix_get(diet_parameter(pb,1), &M, NULL, &m, &n, NULL);
  /* Launch computation */
  res = scal_mat_prod(*alpha, M, m, n, &time);
  /* Set OUT arguments */
  diet_scalar_desc_set(diet_parameter(pb,2), &time);
  /* Free IN data */
  diet_free_data(diet_parameter(pb,0));

  return res;
}

int main(int argc, char* argv[])
{
  diet_profile_desc_t *profile;
  
  /* Initialize table with maximum 1 service */
  diet_service_table_init(1);
  /* Define smprod profile */
  profile = diet_profile_desc_alloc("smprod",0, 1, 2);
  diet_generic_desc_set(diet_param_desc(profile,0), DIET_SCALAR, DIET_DOUBLE);
  diet_generic_desc_set(diet_param_desc(profile,1), DIET_MATRIX, DIET_DOUBLE);
  diet_generic_desc_set(diet_param_desc(profile,2), DIET_SCALAR, DIET_FLOAT);
  /* Add the service (the profile descriptor is deep copied) */
  diet_service_table_add(profile, NULL, solve_smprod);
  /* Free the profile descriptor, since it was deep copied. */
  diet_profile_desc_free(profile);

  /* Launch the SeD: no return call */
  diet_SeD("./SeD.cfg", argc, argv);

  return 0;
}
\end{verbatim}
}

\section{Compilation}
\label{sec:sv_comp}

After compiling her/his server program, the user must link it with the \diet
and CORBA libraries. This process is very similar to the one described
for the client in section \ref{sec:cl_comp}. Please refer to this section
for details.

%%% Local Variables:
%%% mode: latex
%%% ispell-local-dictionary: "american"
%%% mode: flyspell
%%% fill-column: 79
%%% End:


%
% Deploying a DIET platform
%
\newpage
%****************************************************************************%
%* DIET User's Manual deploying chapter file                                *%
%*                                                                          *%
%*  Author(s):                                                              *%
%*    - Philippe COMBES (Philippe.Combes@ens-lyon.fr)                       *%
%*                                                                          *%
%* $LICENSE$                                                                *%
%****************************************************************************%
%* $Id$
%* $Log$
%* Revision 1.2  2004/01/07 10:27:33  cpera
%* Add asynchronous call example and useAsyncAPI config parameter.
%*
%* Revision 1.1  2003/09/09 12:38:20  pcombes
%* Reorganization of doc: UM becomes UsersManual.
%*
%* Revision 1.10  2003/06/02 13:47:05  pcombes
%* Fix footnotesize.
%*
%* Revision 1.9  2003/05/23 09:23:35  pcombes
%* Add suggestions from Jean-Yves. Thanks !
%*
%* Revision 1.8  2003/05/15 14:17:58  pcombes
%* UM 0.7
%*
%* Revision 1.5  2003/01/22 17:34:53  pcombes
%* User Manual, v. 0.6.4
%*
%* Revision 1.4  2003/01/21 12:17:02  pcombes
%* Update UM to API 0.6.3, and "hide" data structures.
%*
%* Revision 1.3  2003/01/13 12:09:00  pcombes
%* UM: client part complete for users's day ...
%****************************************************************************%

\chapter{Deploying a DIET platform}
\label{ch:deploying}


%====[ Deploying CORBA services ]==============================================
\section{Deploying CORBA services}
\label{sec:CORBA_services}

So far, only one CORBA service is needed: the Naming Service. It is provided by
a name server that must be launched before all DIET entities. Then, these DIET
entities must be passed a reference to the $<$host:port$>$ of this name server.

In this section, we try to give some basic information to start using DIET with
a basic CORBA configuration. Please refer to the documentation of your ORB if
you need more details.

\subsection{omniNames}

\subsubsection{The server}

To launch the omniORB name server, first check that the path of the omniORB
libraries is in your environment variable \texttt{LD\_LIBRARY\_PATH}, then
specify the log directory, through the environment variable
\texttt{OMNINAMES\_LOGDIR} (or, with \textbf{omniORB 4}, at compile time,
through the \texttt{--with-omniNames-logdir} option of the omniORB configure
script). If there is no log file in this directory, \texttt{omniNames} needs to
be given the port through which its client will connect. It can be launched
as follows:
{\footnotesize
\begin{verbatim}
~ > omniNames -start

Fri Dec 13 14:46:02 2002:

Starting omniNames for the first time.
Wrote initial log file.
Read log file successfully.
Root context is IOR:010000002b00000049444c3a6f6d672e6f72672f436f734e616d696e672f4e61
6d696e67436f6e746578744578743a312e300000010000000000000060000000010102000d0000003134
302e37372e31332e34360000fa0a0b0000004e616d655365727669636500020000000000000008000000
0100000000545441010000001c0000000100000001000100010000000100010509010100010000000901
0100
Checkpointing Phase 1: Prepare.
Checkpointing Phase 2: Commit.
Checkpointing completed.
\end{verbatim}
}

This sets an omniORB name server which listens to clients
connections on default port 2809. If omniNames has already been launched once,
\emph{ie} there is already some log files in the log directory, using the
\texttt{-start} option causes an error. The port is actually read from old
log files:
{\footnotesize
\begin{verbatim}
~ > omniNames -start 2810

Fri Dec 13 15:02:36 2002:

Error: log file '/tmp/omninames-toto.log' exists.  Can't use -start option.
~ > omniNames  

Fri Dec 13 15:02:40 2002:

Read log file successfully.
Root context is IOR:010000002b00000049444c3a6f6d672e6f72672f436f734e616d696e672f4e61
6d696e67436f6e746578744578743a312e300000010000000000000060000000010102000d0000003134
302e37372e31332e34360000fa0a0b0000004e616d655365727669636500020000000000000008000000
0100000000545441010000001c0000000100000001000100010000000100010509010100010000000901
0100
Checkpointing Phase 1: Prepare.
Checkpointing Phase 2: Commit.
Checkpointing completed.
\end{verbatim}
}


\subsubsection{The client}

Every DIET entity needs to connect to the CORBA name server: it is the way to
discover each other. The reference to the omniORB name server is written in the
configuration file, whose path is given to omniORB through the environment
variable \texttt{OMNIORB\_CONFIG} (or, with \textbf{omniORB 4}, at compile time,
through the \texttt{--with-omniORB-config} option of the configure script). Some
examples of such a configuration file are given in the directory
\texttt{src/examples/cfgs} of the DIET source files and installed in
\texttt{$<$install\_dir$>$/etc}. The lines concerning the name server, in the
omniORB configuration file, are built as follows:
\begin{description}
 \item{omniORB 3:}
{\footnotesize
\begin{verbatim}
ORBInitialHost <name server hostname>
ORBInitialPort <name server port>
\end{verbatim}
}
 \item{omniORB 4:}
{\footnotesize
\begin{verbatim}
InitRef = NameService=corbaname::<name server hostname>:<name server port>
\end{verbatim}
}
The name server port is the one given in argument to the \texttt{-start} option
of \texttt{omniNames}.
\end{description}

%\subsection*{The TAO name server}
%\subsubsection{The server}
%\subsubsection{The client}


%====[ DIET configuration file ]===============================================
\section{DIET configuration file} 

Launching a DIET entity needs a configuration file. Some fully commented
examples of such configuration files are given in the directory
\texttt{src/examples/cfgs} of the DIET source files and installed in
\texttt{$<$install\_dir$>$/etc}. Please note that:
\begin{itemize}
\item comments start with '\#' and finish at the end of the current
  line,
\item meaningful lines have the format: \texttt{keyword = value}, following the
  format of configuration files for omniORB 4,
\item keywords are case sensitive.
\end{itemize}

\subsection{Tracing API}

\noindent
\texttt{traceLevel} \ \ \emph{default}\texttt{ = 1}\\
This option controls debugging trace output. The following levels are defined:

\begin{center}
 \footnotesize
 \begin{tabular}{p{.1\linewidth}p{.8\linewidth}}
  level $=$ 0  & Print only errors\\
  level $<$ 5  & Print errors and messages for the main steps (such as ``Got a
  request'') - default\\
  level $<$ 10 & Print errors and messages for all steps\\
  level $=$ 10 & Print errors, all steps, and some important structures (such
  as the list of offered services)\\
  level $>$ 10 & Print all DIET messages AND omniORB messages corresponding to
  an omniORB traceLevel of (level~-~10)
 \end{tabular}
\end{center}


\subsection{Client side parameters}

\noindent
\texttt{MAName} \ \ \emph{default}\texttt{ = }\emph{none}\\
This is a \textbf{compulsory} parameter that specifies the name of the Master
Agent to connect to. The MA must have registered with this same name to the
CORBA name server.


\subsection{Agent side parameters}

\noindent
\texttt{agentType} \ \ \emph{default}\texttt{ = }\emph{none}\\
As DIET offers only one executable for both types of agent, it is
\textbf{compulsory} to specify which kind of agent must be launched. Two values
are available: \texttt{DIET\_MASTER\_AGENT} and \texttt{DIET\_LOCAL\_AGENT}.
They have aliases, respectively \texttt{MA} and \texttt{LA}.
\\

\noindent
\texttt{name} \ \ \emph{default}\texttt{ = }\emph{none}\\
This is a \textbf{compulsory} parameter that specifies the name which the agent
will register with to the CORBA name server.


\subsection{LAs and SeD side parameters}

\noindent
\texttt{parentName} \ \ \emph{default}\texttt{ = }\emph{none}\\
This is a \textbf{compulsory} parameter for Local Agents only ; it indicates the
name of the parent (an LA or the MA) to register to.


\subsection{FAST options}

\noindent
\texttt{fastUse} \ \ \emph{default}\texttt{ = 1}\\
This option activates the requests to FAST. It is ignored if DIET was
compiled without FAST, and defaults to 1 otherwise. In the latter case, it is
useful to set it to 0 if one wishes to perform some tests
without deploying the whole NWS/LDAP platform.

The following options are ignored if DIET was compiled without FAST or if
\texttt{fastUse} is set to 0.

\subsubsection{LDAP options}

\noindent
\texttt{ldapUse} \ \ \emph{default}\texttt{ = 0} for agents, \texttt{ = 1} for
SeDs\\
This option activates the use of LDAP in FAST requests. It defaults to 0 for
agents (which do not need to connect to the LDAP base) and to 1 for SeDs.
\\

The following two options are ignored if \texttt{ldapUse} is set to 0.
\\

\noindent
\texttt{ldapBase} \ \ \emph{default}\texttt{ = }\emph{none}\\
Specify the \texttt{host:port} address of the LDAP base where FAST gets the
results of its benchmarks.
\\

\noindent
\texttt{ldapMask} \ \ \emph{default}\texttt{ = }\emph{none}\\
Specify the mask used for requests to the LDAP base. It must match the one given
in the \texttt{.ldif} file of the server that was added to the base.


\subsubsection{NWS options}

\noindent
\texttt{nwsUse} \ \ \emph{default}\texttt{ = 1}\\
This option activates the use of NWS in FAST requests. If 0, FAST will use an
internal sensor for the performance of the machine, but will not be able to
evaluate communication times.
\\

The following two options are ignored if \texttt{nwsUse} is set to 0.
\\

\noindent
\texttt{nwsNameserver} \ \ \emph{default}\texttt{ = }\emph{none}\\
Specify the \texttt{host:port} address of the NWS name server.
\\

\noindent
\texttt{nwsForecaster} \ \ \emph{default}\texttt{ = }\emph{none}\\
Specify the \texttt{host:port} address of the NWS forecaster.
\\


\subsection{Miscellanous options}

\texttt{endPoint} \ \ \emph{default}\texttt{ = }\emph{2809}\\
This option specifies the listening port of an agent or an SeD. It is very
useful when they are behind a firewall. It defaults to 2809, but if this port is
already busy, the OS chooses a port number.

\texttt{useAsyncAPI} \ \ \emph{default}\texttt{ = }\emph{1}\\
Specify use of asynchronous call and so create objects managing it
on client side.

%====[ Example ]=========================================
\section{Example}
\label{sec:deploy_ex}

As shown in Section \ref{init}, the hierarchy is built from top to down: the
children register to their direct parent.

Here is an example of a complete platform deployment. Let us assume that:

\begin{itemize}
\item DIET was compiled with FAST on all machines used,
\item the LDAP server is launched on the machine \texttt{ldaphost} and listens
  on the port 9000,
\item the NWS name server is launched on the machine \texttt{nwshost} and
  listens on the port 9001,
\item the NWS forecaster is launched on the machine \texttt{nwshost} and
  listens on the port 9002,
\item the NWS sensors are launched on every machine we use.
\end{itemize}


\subsubsection{Launching the MA}

For such a platform, the MA configuration file could be:
\tt
\begin{center}
 \footnotesize
 \begin{tabular}{lcll}
  \multicolumn{4}{l}{\# file MA\_example.cfg, configuration file for an MA}\\
  agentType    &=&DIET\_MASTER\_AGENT&\\
  name         &=&MA\_example        &\\
  \#traceLevel &=&5                  &\# default\\
  \#endPoint   &=&2809               &\# default\\
  \#fastUse    &=&1                  &\# default\\
  \#ldapUse    &=&0                  &\# default\\
  \#nwsUse     &=&1                  &\# default\\
  nwsNameserver&=&nwshost:9001       &\\
  nwsForecaster&=&nwshost:9002       &\\
 \end{tabular}
\end{center}
\rm

This configuration file is the only argument to the executable \texttt{dietAgent},
that is installed in \texttt{$<$install\_dir$>$/bin}. Provided
\texttt{$<$install\_dir$>$/bin} is in your PATH environment variable, run
{\footnotesize
\begin{verbatim}
~ > dietAgent MA_example.cfg

Master Agent MA_example started.
\end{verbatim}
}


\subsubsection{Launching an LA}

For such a platform, an LA configuration file could be:
\tt
\begin{center}
 \footnotesize
 \begin{tabular}{lcll}
  \multicolumn{4}{l}{\# file LA\_example.cfg, configuration file for an LA}\\
  agentType    &=&DIET\_LOCAL\_AGENT&\\
  name         &=&LA\_example       &\\
  parentName   &=&MA\_example       &\\
  \#traceLevel &=&5                 &\# default\\
  \#endPoint   &=&2809              &\# default\\
  \#fastUse    &=&1                 &\# default\\
  \#ldapUse    &=&0                 &\# default\\
  \#nwsUse     &=&1                 &\# default\\
  nwsNameserver&=&nwshost:9001      &\\
  nwsForecaster&=&nwshost:9002      &\\
 \end{tabular}
\end{center}
\rm

This configuration file is the only argument to the executable \texttt{dietAgent},
that is installed in \texttt{$<$install\_dir$>$/bin}. This LA will register as a
son of MA\_example. Run
{\footnotesize
\begin{verbatim}
~ > dietAgent LA_example.cfg

Local Agent LA_example started.
\end{verbatim}
}

\subsubsection{Launching a server}

For such a platform, an \sed\ configuration file could be:
\tt
\begin{center}
 \footnotesize
 \begin{tabular}{lcll}
  \multicolumn{4}{l}{\# file SeD\_example.cfg, configuration file for an SeD}\\
  parentName   &=&LA\_example        &\\
  \#traceLevel &=&5                 &\# default\\
  \#endPoint   &=&2809              &\# default\\
  \#fastUse    &=&1                 &\# default\\
  \#ldapUse    &=&1                 &\# default\\
  ldapBase     &=&ldaphost:9000     &\\
  ldapMask     &=&dc=LIP,dc=ens-lyon,dc=fr&\\
  \#nwsUse     &=&1                 &\# default\\
  nwsNameserver&=&nwshost:9001      &\\
  nwsForecaster&=&nwshost:9002      &\\
 \end{tabular}
\end{center}
\rm

This \sed\ will register as a son of LA\_example. Run the executable that you
linked with the DIET SeD library, and do not forget that the first argument of
\texttt{diet\_SeD} must be the path of the configuration file above.


\subsubsection{Launching a client}

Our client must connect to the MA\_example:
\tt
\begin{center}
 \footnotesize
 \begin{tabular}{lcll}
  \multicolumn{4}{l}{\# file SeD\_example.cfg, configuration file for an SeD}\\
  MAName       &=&LA\_example        &\\
  \#traceLevel &=&5                 &\# default\\
 \end{tabular}
\end{center}
\rm

Run the executable that you linked with the DIET client library, and do not
forget that the first argument of \texttt{diet\_initialize} must be the path of
the configuration file above.




%%%%
% BIBLIO
%%%%

\bibliographystyle{plain}
\bibliography{UsersManual}

\end{document}
