%****************************************************************************%
%* DIET User's Manual main file                                             *%
%*                                                                          *%
%*  Author(s):                                                              *%
%*    - Philippe COMBES (Philippe.Combes@ens-lyon.fr)                       *%
%*                                                                          *%
%* $LICENSE$                                                                *%
%****************************************************************************%
%* $Id$
%* $Log$
%* Revision 1.68  2011/05/10 09:20:17  bdepardo
%* Updated DIET version
%*
%* Revision 1.67  2011/05/03 12:44:15  bdepardo
%* Updated copyright
%*
%* Revision 1.66  2011/05/02 10:01:43  bdepardo
%* Corrected a few typos.
%* Renamed "round-robin" scheduler into "least recently used" scheduler as this
%* is closer to the behavior of this scheduler.
%*
%* Revision 1.65  2011/01/12 00:44:05  ecaron
%* Update DIET version
%*
%* Revision 1.64  2010/09/14 17:07:30  bdepardo
%* Users FAQ
%*
%* Revision 1.63  2010/09/10 13:05:29  bdepardo
%* Typos
%*
%* Revision 1.62  2010/09/10 07:36:45  glemahec
%* DIET forwarders documentation
%*
%* Revision 1.61  2010/07/07 15:10:52  amuresan
%* Added Cloud entry for the UsersGuide and ProgrammersGuide.
%*
%* Revision 1.60  2010/03/31 02:40:21  ycaniou
%* Compilation (additional package required)
%*
%* Revision 1.59  2010/03/29 20:58:29  ecaron
%* Update date and revision number
%*
%* Revision 1.58  2010/02/25 08:55:43  ycaniou
%* Add a couple of macro like \seddiet \sedj (differs from \sed$_\diet$ which
%*   adds a space)
%* Changed *.tex accordingly
%*
%* Revision 1.57  2010/02/25 07:15:57  ycaniou
%* DAGDA -> macro + sc
%* Add info logs � dagda.tex
%* Formatage document + ispell
%*
%* Revision 1.56  2010/02/25 06:45:38  ycaniou
%* Add local variables
%*
%* Revision 1.55  2010/02/25 06:38:15  ycaniou
%* Add local variable
%* + Ispell
%* + eg, NES etc. -> \eg..
%* Version of supported GCC updated
%* + formatage .tex
%*
%* Revision 1.54  2010/02/25 06:07:08  ycaniou
%* PSE et NES sans textsc
%*
%* Revision 1.53  2010/01/21 14:05:58  bdepardo
%* DIET -> \diet
%* SeD -> \sed
%* GoDIET -> \godiet
%*
%* Revision 1.52  2009/10/26 07:23:07  bdepardo
%* Added chapter on dynamic hierarchy management.
%*
%* Revision 1.51  2009/09/08 13:46:34  bdepardo
%* Added appendix.
%* Currently the page layout is broken in the appendix.
%*
%* Revision 1.50  2008/07/16 23:56:09  ecaron
%* Date for the UsereManual 2.3
%****************************************************************************%
\documentclass[12pt,a4paper]{book}
%\makeatletter
%\makeatother
\usepackage{fancyhdr}
\usepackage[headings]{fullpage}

\usepackage[pdftex]{graphicx} % Pour l'insertion d'images
\DeclareGraphicsExtensions{.jpg,.mps,.pdf,.png} % Formats d'images

\usepackage[pdftex]{thumbpdf}      % Vignettes
\usepackage{xcolor} % required for colors by hyperef
\usepackage[pdftex,                %
    bookmarks         = true,%     % Signets
    bookmarksnumbered = true,%     % Signets num�rot�s
    pdfpagemode       = true,%     % Signets/vignettes ferm�s � l'ouverture
%   pdfpagemode       = Fullscreen,
    pdfstartview      = FitV,%     % La page prend toute la hauteur
    pdfpagelayout     = SinglePage,% Vue par page
    colorlinks        = true,%     % Liens en couleur
    linkbordercolor   = white, % Couleur de la bo�te sur les liens normal 
    citebordercolor   = white, % Couleur de la bo�te sur les citations 
    filebordercolor   = white, % Couleur sur la bo�te sur les fichiers 
    urlbordercolor    = white, % Couleur sur la bo�te sur les URL
    linkcolor         = cyan, % Liens internes
    urlcolor          = blue, %  % Couleur des liens externes
    pdfborder         = {0 0 0}%   % Style de bordure : ici, pas de bordure
    ]{hyperref}%                   % Utilisation de HyperTeX


\hypersetup{ % Modifiez la valeur des champs suivants
    pdfauthor   = {DIET Team},%
    pdftitle    = {DIET Users Manual},%
    pdfsubject  = {Manual},%
    pdfkeywords = {DIET, Grid-RPC},%
    pdfcreator  = {PDFLaTeX},%
    pdfproducer = {PDFLaTeX}}


%\usepackage[french]{babel}
%\usepackage[latin1]{inputenc}
%\usepackage{multicol}
\usepackage{verbatim}
\usepackage{url}
\usepackage{subfigure}
\usepackage{listings}
\usepackage{xspace}
\usepackage{calc}
\usepackage{marvosym} % More symbols
\graphicspath{{./fig}}

\usepackage{pifont} % for \ding{52} dagda

\newsavebox{\logobox}
\sbox{\logobox}{\includegraphics[scale=0.4]{fig/logo_DIET}}
\newcommand{\logo}{\usebox{\logobox}}

%%%%

\newcounter{rmq}[section]
\setcounter{rmq}{0}
\newenvironment{remarque}{\addtocounter{rmq}{1}\textbf{N.B. \thermq:}}{}

%%%%

%%%%
\renewcommand{\title}{DIET User's Manual}
%%%%

\pagestyle{fancyplain}
\fancyhead[L]{\title}
% \lhead[\fancyplain{\title}{\title}]
%       {\fancyplain{\title}{\title}}
\chead{}
\rhead[\fancyplain{\logo}{\logo}]{\fancyplain{\logo}{\logo}}

\lfoot[\fancyplain{\scriptsize{\copyright} ~INRIA, ENS-Lyon, UCBL, CNRS, SysFera}{\scriptsize{\copyright} ~INRIA, ENS-Lyon, UCBL, CNRS, SysFera}]{\fancyplain{\scriptsize{\copyright} ~INRIA, ENS-Lyon, UCBL, CNRS, SysFera}{\scriptsize{\copyright} ~INRIA, ENS-Lyon, UCBL, CNRS, SysFera}}
\cfoot[\fancyplain{}{}]{\fancyplain{}{}}
\rfoot[\fancyplain{Page~\thepage}{Page~\thepage}]
      {\fancyplain{Page~\thepage}{Page~\thepage}}


\newcommand{\ie}{\emph{i.e.,}\xspace}
\newcommand{\eg}{\emph{e.g.,}\xspace}
\newcommand{\etc}{\emph{etc.}\xspace}

\newcommand{\dietj}{{\sc Diet}$_J$\xspace}
\newcommand{\sedj}{{\textit SeD}$_J$\xspace}
\newcommand{\seddiet}{{\textit SeD}$_{\sc Diet}$\xspace}
\newcommand{\sedsdiet}{{\textit SeDs}$_{\sc Diet}$\xspace}
\newcommand{\MAj}{{\textit MA}$_J$\xspace}
\newcommand{\MAsj}{{\textit MAs}$_J$\xspace}
\newcommand{\MAdiet}{{\textit MA}$_{\sc Diet}$\xspace}
\newcommand{\MAsdiet}{{\textit MAs}$_{\sc Diet}$\xspace}
\newcommand{\LAj}{{\textit LA}$_J$\xspace}
\newcommand{\LAsj}{{\textit LAs}$_J$\xspace}
\newcommand{\LAdiet}{{\textit LA}$_{\sc Diet}$\xspace}
\newcommand{\LAsdiet}{{\textit LAs}$_{\sc Diet}$\xspace}
\newcommand{\clientj}{{\textit client}$_J$\xspace}
\newcommand{\clientsj}{{\textit clients}$_J$\xspace}
\newcommand{\clientdiet}{{\textit client}$_{\sc Diet}$\xspace}
\newcommand{\dagda}{{\sc Dagda}\xspace}
\newcommand{\diet}{{\sc Diet}\xspace}
\newcommand{\dietforwarder}{{\sc Diet}Forwarder\xspace}
\newcommand{\godiet}{{\sc GoDiet}\xspace}
\newcommand{\madag}{{\sc M$A_{DAG}$}\xspace}
\newcommand{\pse}{PSE\xspace}
\newcommand{\nes}{NES\xspace}
\newcommand{\ptop}{\textit{Peer-to-Peer}}
\newcommand{\red}{\textit{Red}}
\newcommand{\sci}{Scilab}
\newcommand{\scip}{Scilab$_{//}$}
\newcommand{\scalapack}{ScaLAPACK}
\newcommand{\sed}{\textit{SeD}\xspace}
\newcommand{\seds}{\textit{SeD}s\xspace}
\newcommand{\thread}{\textit{thread}}
\newcommand{\threads}{\textit{threads}}
\newcommand{\nsl}{NetSolve}
\newcommand{\fixme}[1]{\fbox{\textsl{{\bf #1}}}}
\newcommand{\dietversion}{2.8}

%%%%
% Document beginning
%%%%

\begin{document}

%%%%
% First sheet
%%%%

\thispagestyle{empty}
\vspace*{3cm}
\vspace*{3cm}

\begin{center}
\includegraphics[scale=.5]{fig/logo_DIET_big}\\[2ex]
\textbf{\Huge USER'S MANUAL\\[2ex]}
\end{center}

\vfill

\noindent
\small{
\begin{tabular}{ll}
  \textbf{VERSION}  & \dietversion\\
  \textbf{DATE}     & August 2011\\
  \textbf{PROJECT MANAGER}  & Fr\'ed\'eric \textsc{Desprez}.\\
  \textbf{EDITORIAL STAFF}  & Yves \textsc{Caniou}, Eddy \textsc{Caron} and David ~\textsc{Loureiro}.\\
  \textbf{AUTHORS STAFF}    & 
\begin{minipage}[t]{12cm}
  Abdelkader \textsc{Amar}, Rapha\"el \textsc{Bolze}, \'Eric
  \textsc{Boix}, Yves \textsc{Caniou}, Eddy \textsc{Caron}, Pushpinder
  Kaur \textsc{Chouhan}, Philippe ~\textsc{Combes}, Sylvain
  \textsc{Dahan}, Holly \textsc{Dail}, Bruno \textsc{Delfabro}, Benjamin
  \textsc{Depardon}, Peter
  \textsc{Frauenkron}, Georg \textsc{Hoesch}, Benjamin \textsc{Isnard},
  Mathieu \textsc{Jan}, Jean-Yves \textsc{L'Excellent}, Ga�l \textsc{Le
    Mahec}, Christophe \textsc{Pera}, Cyrille \textsc{Pontvieux}, Alan
  \textsc{Su}, C\'edric \textsc{Tedeschi}, and Antoine
  \textsc{Vernois}.
\end{minipage} \\
  \textbf{Copyright}& INRIA, ENS-Lyon, UCBL
\end{tabular}\\
}

\newpage
\thispagestyle{empty}
\ 

%%%%
% End of first sheet
%%%%


\newpage
\tableofcontents


%
% Introduction
%
\newpage
\addcontentsline{toc}{chapter}{Introduction}
\chapter*{Introduction}

% \href{mailto:mail@exemple.com}{Mon mail} % Lien email
% \href{http://exemple.com}{Mon site web}  % Lien web
% \href{fichier.pdf}{Mon fichier}          % Lien vers un fichier
%\Acrobatmenu{FullScreen}{Plein �cran}    % Plein �cran/fen�tr�

Resource management is one of the key issues for the development of
efficient Grid environments. Several approaches co-exist in today's
middleware platforms. The granularity of computation (or communication) and 
dependencies between computations can have a great influence on the
software choices.

The first approach provides the user with a uniform view of
resources. This is the case of GLOBUS~\cite{Globus} which provides
transparent MPI communications (with MPICH-G2) between distant nodes
but does not manage load balancing issues between these nodes. It's
the user's task to develop a code that will take into account the
heterogeneity of the target architecture. Grid extensions to
classical batch processing provide an alternative approach with
projects like Condor-G~\cite{Condor} or Sun
GridEngine~\cite{SunGridEngine}. Finally, peer-to-peer~\cite{Oram01}
or Global computing~\cite{germain01global} can be used for fine
grain and loosely coupled applications.

A second approach provides a semi-transparent access to computing
servers by submitting jobs to servers offering specific
computational services. This model is known
as the Application Service Provider (ASP) model where providers offer,
not necessarily for free, computing resources (hardware and software)
to clients in the same way as Internet providers offer network
resources to clients. The programming granularity of this model is
rather coarse. One of the advantages of this approach is that end
users do not need to be experts in parallel programming to benefit
from high performance parallel programs and computers. This model is
closely related to the classical Remote Procedure Call (RPC)
paradigm. On a Grid platform, RPC (or
GridRPC~\cite{MNS+00,NMSDLC02}) offers easy access to available
resources from a Web browser, a Problem Solving Environment (\pse), or a
simple client program written in C, Fortran, or Java.  It also
provides more transparency by hiding the selection and allocation of
computing resources. We favor this second approach.

In a Grid context, this approach requires the implementation of
middleware to facilitate client access to remote resources. In the ASP
approach, a common way for clients to ask for resources to solve their
problem is to submit a request to the middleware. The middleware will
find the most appropriate server that will solve the problem on behalf
of the client using a specific software. Several environments, usually
called Network Enabled Servers (\nes), have developed such a paradigm:
NetSolve~\cite{nug}, Ninf~\cite{NSS99}, NEOS~\cite{FMM00},
OmniRPC~\cite{SHTS01}, and more recently \diet developed in the
\textsc{Graal} project. A common feature of these environments is that
they are built on top of five components: clients, servers, databases,
monitors and schedulers. Clients solve computational requests on
servers found by the \nes. The \nes schedules the requests on the
different servers using performance information obtained by monitors
and stored in a database.

\diet stands for Distributed Interactive Engineering Toolbox. It is a
toolbox for easily developing Application Service Provider systems on
Grid platforms, based on the Client/Agent/Server scheme. Agents are
the schedulers of this toolbox. In \diet, user requests are served via
RPC.

\diet follows the GridRPC API defined within the Open Grid
Forum~\cite{GridRPC}.

%
% A DIET platform
%
\newpage
%****************************************************************************%
%* DIET User's Manual description chapter file                              *%
%*                                                                          *%
%*  Author(s):                                                              *%
%*    - Philippe COMBES (Philippe.Combes@ens-lyon.fr)                       *%
%*                                                                          *%
%* $LICENSE$                                                                *%
%****************************************************************************%
%* $Id$
%* $Log$
%* Revision 1.3  2004/01/21 23:23:03  ecaron
%* Add suggestions from Jean-Yves. Thanks !
%*
%* Revision 1.2  2004/01/15 23:59:31  ecaron
%* Add corrections from Holly Dail
%*
%* Revision 1.1  2003/09/09 12:38:20  pcombes
%* Reorganization of doc: UM becomes UsersManual.
%*
%* Revision 1.9  2003/06/02 13:47:05  pcombes
%* Fix footnotesize.
%*
%* Revision 1.8  2003/05/15 14:17:58  pcombes
%* UM 0.7
%*
%* Revision 1.5  2003/01/22 17:34:53  pcombes
%* User Manual, v. 0.6.4
%*
%* Revision 1.4  2003/01/21 12:17:02  pcombes
%* Update UM to API 0.6.3, and "hide" data structures.
%*
%* Revision 1.3  2003/01/13 12:09:00  pcombes
%* UM: client part complete for users's day ...
%****************************************************************************%

\chapter{A DIET platform}
\label{ch:description}

DIET is built upon the \emph{Server Daemons}. The scheduler is
scattered across a hierarchy of \emph{Local Agents} and \emph{Master
  Agents}. NWS~\cite{WSH99} sensors are placed on every node of the
hierarchy to collect resource availabilities, which are used by an
application-centric performance prediction tool called FAST
\cite{Qui02}.  Figure~\ref{fig:platform} shows the hierarchical
organization of DIET.

\begin{figure}[htb]
 \begin{center}
  \resizebox{.7\linewidth}{!}{\includegraphics{fig/global_platform.eps}}
  \label{fig:platform}
  \caption{A hierarchy of DIET agents}
 \end{center}
\end{figure}

%====[ DIET components ]=======================================================
\section{DIET components}
\label{sec:components}

The different components of our software architecture are the following:      

\begin{description}
%....[ Client ]................................................................
\item \textbf{Client}\\
  A client is an application which uses DIET to solve problems.  Many types of clients are able to connect to DIET from a web page, a PSE such as
  Matlab or \sci, or from a compiled program.
%....[ Master Agent (MA) ].....................................................
\item \textbf{Master Agent (MA)}\\
  An MA receives computation requests from clients. These requests refer to some
  DIET problems listed on a reference web page. Then the MA collects computation
  abilities from the servers and chooses the best one. The reference of the
  chosen server is returned to the client. A client can be connected to an MA by
  a specific name server or a web page which stores the various MA locations.

%....[ Local Agent (LA) ]......................................................
\item \textbf{Local Agent (LA)}\\
  An LA aims at transmitting requests and information between MAs and
  servers.  The information stored on an LA is the list of the
  requests being processed and performance of all servers in its
  subtree that can solve a given problem. Depending on the underlying
  network topology, a hierarchy of LAs may be deployed between an MA
  and the servers. Of course, the function of an LA is to do a partial
  scheduling on its subtree, which reduces the job of the MA.

%....[ Server Daemon (SeD) ]...................................................
\item \textbf{Server Daemon (SeD)}\\
  A SeD encapsulates a computational server. For instance it can be
  located on the entry point of a parallel computer. The information
  stored on a SeD is a list of the data available on its server (with
  their distribution and the way to access them), the list of problems
  that can be solved on it, and all information concerning its load
  (memory available, number of resources available, \ldots). A SeD
  declares the problems it can solve to its parent LA.  An eD can give
  performance prediction for a given problem thanks to the module
  FAST, described in the next section.

\end{description}

%\begin{figure}[htb]
% \begin{center}
%  \resizebox{.6\linewidth}{!}{\includegraphics{fig/global_platform.eps}}
%  \label{fig:platform}
%  \caption{A hierarchy of DIET agents}
% \end{center}
%\end{figure}


%====[ FAST: FAST AGENT'S SYSTEM TIMER ]=======================================
\section{FAST: Fast Agent's System Timer}
\label{sec:FAST}

FAST \cite{Qui02} is a tool for dynamic performance forecasting in a
Grid environment. As shown in Figure~\ref{fig:fast-overview}, FAST is
composed of several layers and relies on low-level software. First it
uses a network and CPU monitoring software to handle dynamically
changing resources, like workload or bandwidth.  FAST uses the Network
Weather Service (NWS)~\cite{WSH99} a distributed system that
periodically monitors and dynamically forecasts the performance of
various network and computational resources. The resource
availabilities acquisition module of FAST uses and enhances NWS.
Indeed, if there is no direct NWS monitoring between two machines,
FAST automatically searches for the shortest path between them in the
graph of monitored links. It estimates the bandwidth as the minimum of
those in the path and the latency as the sum of those measured. This
allows for the availability of more predictions when DIET is deployed
over a hierarchical network.

\begin{figure}[htb]
  \begin{center}
    \resizebox{.75\linewidth}{!}{\includegraphics{fig/FAST.eps}}
    \caption{FAST overview}
    \label{fig:fast-overview}
  \end{center}
\end{figure}

In addition to the system availabilities, FAST can also forecast the time and
space needs of computational routines, depending on both the parameter set and
the machine where the computation would take place.  For this, FAST 
benchmarks the routines at installation time on each machine for a
representative set of parameters. After polynomial data fitting, the results
are stored in an LDAP tree.  The user API of FAST is composed of a small set
of functions that combine resource availabilities and routine needs from
low-level software to produce ready-to-use values. These results can be
combined into analytical models by the parallel extension~\cite{CS02} to
forecast execution times of parallel routines.

Thus DIET components, as any FAST client, can access information like the time
needed to move a given amount of data between two SeDs, the time to solve a
problem with a given set of computational resources managed by a SeD, or the
combination of these two quantities.\\

For more details about FAST, please read the FAST Reference Manual~\footnote{\url{http://graal.ens-lyon.fr/FAST/docs/index.html}}.


%====[ CORBA ]=================================================================
\section{Communications inner to the platform}
\label{sec:CORBA}

NES environments can be implemented using a classic socket communication layer.
NINF and \nsl\ are implemented that way. Several problems to this approach
have been pointed out such as the lack of portability or the limitation of
opened sockets. Our aim is to implement and deploy a distributed NES
environment that works at a wider scale. Distributed object environments, such
as \emph{Java}, \emph{DCOM} or CORBA have proven to be a good base for
building applications that manage access to distributed services. They not only
provide transparent communications in heterogeneous networks, but they also
offer a framework for the large scale deployment of distributed
applications. Being open and language independent, CORBA was chosen as the
communication layer in DIET.

As recent implementations of CORBA provide communication time close to
that of sockets, CORBA is well suited to support distributed resources
and applications in a large scale Grid environment. New specialized
services can be easily published and existing services can also be
used.  Therefore, CORBA is a good choice for the
development of Grid specific services. DIET is based upon
\emph{OmniORB 3}~\cite{OMNIORB} or later, a free CORBA implementation
which provides good communication performance.


%====[ DIET INITIALIZATION ]===================================================
\section{DIET initialization}
\label{init}

Figure~\ref{fig:init} shows each step of the initialization of a simple Grid
system. The architecture is built in hierarchical order, each component
connecting to its parent. The MA is the first entity to be started~(1). It waits
for connections from LAs or requests from clients.

\begin{figure}[hbt]
  \begin{center}
    \resizebox{9cm}{!}{\includegraphics{fig/init.eps}}
    \caption{Initialization of a DIET system.}
    \label{fig:init}
  \end{center}
\end{figure}

Then when an LA is launched, it subscribes to the MA~(2). At this step of the
system initialization, two kinds of components can connect to the LA: an
\sed ~(3), which manages some computational resource, or another LA~(4), to add a
hierarchical level in this branch. When the \sed\ registers to an LA, it
publishes a list of the services it offers, which is forwarded to the parent
agent until the MA.
Finally, any client can access the registered resource through the platform: it
can contact an MA~(5) to get a reference to the best server available and then
directly connect to it to launch the computation.

The architecture of the hierarchy is described in configuration files and each
component transmits the local configuration to its parent. Thus, the system
administration can also be hierarchical. For instance, an MA can manage a domain
like a university, providing prioritary access to users of this domain. Then
each laboratory can run an LA, while each team of the laboratory can run some
other LAs to administrate its own servers. This hierarchical administration of
the system allows local changes in the configuration without interfering with
the whole platform.



%====[ Solving a problem ]=====================================================
\section{Solving a problem}
\label{sec:solvepb}

Assuming that the architecture described in Section \ref{sec:components}
includes several servers able to solve the same problem, and that each operand
needed for the computation is available on one single server.

%%%%%%%%%%%%%%%
%% FIXME for DIET v1.1
%%%%%%%%%%%%%%%
% , the example
% presented in Figure~\ref{fig:submit_pb} considers the submission of the
% problem \texttt{F()} involving data \texttt{A} and \texttt{B}.
%\begin{figure}[hbt]
%  \begin{center}
%    \resizebox{8cm}{!}{\includegraphics{fig/submit_pb.eps}}
%    \caption{Problem submission example}
%    \label{fig:submit_pb}
%  \end{center}
%\end{figure}
%%%%%%%%%%%%%%%

The algorithm presented below lets an MA choose among its servers the one which
will perform the computation. This decision is made in four steps:
\begin{itemize}
\item the MA propagates the client request through its subtrees down to the
  capable servers~; actually, each agent only knows which among its children manage
  the service, and it forwards the client request to them only~;
\item each server that can satisfy the request calls FAST to estimate the
  computation time necessary to process the request, and sends this
  estimation back to its ``parent'' (the LA)~;
\item each LA of the tree that receives one or more positive answers from its
  children sorts the servers and forwards their answers to the MA through the
  hierarchy~;
\item once the MA has collected all the answers from its direct children, it
  chooses a pool of fast servers and sends their references to the client.
\end{itemize}

%%%%%%%%%%%%%%%
%% FIXME:
%%  Memory aspects should be treated here.
%%%%%%%%%%%%%%%

%%%%%%%%%%%%%%%
%% FIXME for DIET v1.1
%%%%%%%%%%%%%%%
% In order to solve the problem itself, the client connects to one of
% the servers chosen: it sends its local data and specifies if the
% results should be kept in-place for further computation or if they
% should be brought back. The transfer of persistent operands is
% performed at this stage.
%%%%%%%%%%%%%%%


%
% Installing
%
\newpage
%****************************************************************************%
%* DIET User's Manual installing chapter file                               *%
%*                                                                          *%
%*  Author(s):                                                              *%
%*    - Eddy CARON (Eddy.Caron@ens-lyon.fr)                                 *%
%*    - Pushpinder Kaur Chouhan (Pushpinder.Kaur.Chouhan@ens-lyon.fr)       *%
%*    - Philippe COMBES (Philippe.Combes@ens-lyon.fr)                       *%
%*                                                                          *%
%* $LICENSE$                                                                *%
%****************************************************************************%
%* $Id$
%* $Log$
%* Revision 1.27  2006/10/12 09:10:28  eboix
%*   More cmake's docs. --- Injay2461
%*
%*
%* Revision 1.25  2006/05/12 12:12:32  sdahan
%* Add some documentation about multi-MA
%*
%* Bug fix:
%*  - segfault when the neighbours configuration line was empty
%*  - deadlock when a MA create a link on itself
%*
%* Revision 1.24  2006/01/25 16:52:55  pfrauenk
%* CoRI : renaming of the chapter performance prediction with fast
%* 	to performance prediction, add of the CoRI Usersmanual,
%* 	changes in the plugin scheduler
%*
%* Revision 1.23  2005/08/26 08:32:42  rbolze
%* remove --enable-logservice feature.
%*
%* Revision 1.22  2005/07/12 08:54:10  ecaron
%* fix BLAS and ScaLAPACK (no fixme in fact)
%*
%* Revision 1.21  2005/06/27 19:26:52  hdail
%* - Moved introduction to FAST to description section with intro to multi-MA and
%*   gave both chapter references.
%* - Changed version number to 2.0.
%* - Moved info on compiling FAST itself to fast section from install section.
%*   install section still explains how to configure DIET with FAST.
%*
%* Revision 1.20  2005/06/24 15:07:29  hdail
%* Continuing english corrections and updating.
%*
%* Revision 1.19  2005/06/14 08:06:20  ecaron
%* Update ./configure output
%*
%* Revision 1.18  2005/05/29 13:51:22  ycaniou
%* Moved the section concerning FAST from description to a new chapter about FAST
%* and performances prediction.
%* Moved the section about convertors in the FAST chapter.
%* Modified the small introduction in chapter 1.
%* The rest of the changes are purely in the format of .tex files.
%*
%* Revision 1.17  2005/05/20 19:06:01  mjan
%* Short description of how to configure DIET for JuxMem
%*
%* Revision 1.16  2004/10/25 08:59:56  sdahan
%* add the multi-MA documentation
%*
%* Revision 1.15  2004/07/12 08:33:58  rbolze
%* explain how to copy cfgs file in install_dir/etc directory and correct my english
%****************************************************************************%

\chapter{DIET installation}
\label{ch:installing}

%====[ Dependencies ]==========================================================
\section{Dependencies}
\label{sec:dependencies}

\subsection{Platform dependencies}

DIET is fully tested on Linux/i386 and Linux/i686 platforms.
But DIET also supports other architectures (thanks to users reports or
to punctual deployments). In particular DIET is known to be effective
on Linux/Sparc, Linux/i64, Linux/amd64 and Linux/Alpha platforms.
At some point DIET used to be tested on the Solaris/Sparc platform (although
it's been a long time).

If you find a bug in DIET, please don't hesitate to submit a bug report at
\url{http://graal.ens-lyon.fr/bugzilla}. If you have multiple bugs
to report, please make multiple submissions, rather than submitting
multiple bugs in a single report.

As for the supported C++ compilers, DIET undergoes daily regression tests
(see \url{http://graal.ens-lyon.fr/DietDashboard}) with GCC's g++ (ranging
from version 3.2.X to 4.1.x).

\subsection{Software dependencies}
\label{sec:software_dependencies}

As explained in Section~\ref{sec:CORBA}, CORBA is used for all
communications inside the platform. The implementations of CORBA
currently supported in DIET is \textbf{omniORB 4} which depends on
\textbf{Python 2.1} or later, and on \textbf{OpenSSL} if you wish
to secure your DIET platform.~\footnote{If you want to deploy a
secure DIET platform, SSL support is not yet implemented  in DIET, but
an easy way to do so is to deploy DIET on a VPN.}
% \item{soon \textbf{TAO 1.3}} which depends on \textbf{ACE} (but TAO is always
%                           provided with ACE)

In order to deploy CORBA services with omniORB, a configuration file
and a log directory are required: see Section
\ref{sec:CORBA_services} for a complete description of the services.
Their paths can be given to omniORB through environment variables
(\texttt{\$OMNIORB\_CONFIG} and \texttt{\$OMNINAMES\_LOGDIR}), and/or,
for \textbf{omniORB 4} only, at compile time, with the
\texttt{--with-omniORB-config} and \texttt{--with-omniNames-logdir}
options.

\noindent 
\textbf{NB:} We have noticed that some problems occur with
\textbf{Python 2.3}: the C++ code generated could not be compiled. It
has been patched in DIET, but some warnings still appear.  \\

Since omniORB needs thread-safe management of exception handling,
compiling DIET with \verb+gcc+ requires at least \verb+gcc-2.96+.  \\

Some examples provided in the DIET sources depend on the BLAS
and \scalapack\ libraries. However the compilation of those BLAS dependent
examples are optional.


%====[ Compilation ]===========================================================
\section{Compiling the platform}
\label{sec:compil_platform}

DIET compilation process is moving from the traditional \verb+autotools+
way of things to a tool named \verb+cmake+ (mainly to benefit from
\verb+cmake+'s built-in regression tests mechanism).
During a short transitional period both compiling tools will be available. 
Nevertheless, whenever if possible, \verb+cmake+ should be seen as the
privileged (and supported) tool.
The \verb+autotools+ should be definitively deprecated in the next version.
The following of this section explains the usage of both tools.

Before compiling DIET itself, first install the above mentioned
(cf section~\ref{sec:software_dependencies}) dependencies.
Then untar the DIET archive and change current directory to its root directory.

%====[ Compilation with cmake ]================================================
\subsection{Compiling DIET with cmake}

%==================
\subsubsection{Obtaining and installing cmake per se}
DIET requires using \verb+cmake+ at least version \verb+2.4.3+.
For many popular distributions \verb+cmake+ is incorporated by
default or at least \verb+apt-get+ (or whatever your distro package installer
might be) is \verb+cmake+ aware.
Still, in case you need to install an up-to-date version \verb+cmake+'s
official site distributes many binary versions (alas packaged as tarballs)
as made available at 
\url{http://www.cmake.org/HTML/Download.html}.
Optionally, you can download the sources and recompile them: this simple
process (\verb+./bootstrap; make; make install+) is described at
\url{http://www.cmake.org/HTML/Install.html}.

%==================
\subsubsection{Configuring DIET's compilation: cmake quick introduction}
If you are already experienced with \verb+cmake+ then using it to compile
DIET should provide no surprise. 
DIET respects \verb+cmake+'s best practices e.g.~by clearly separating the
source tree from the binary tree (or compile tree), by exposing the main
configuration optional flag variables prefixed with \verb+DIET_+ (and by
hiding away the technical variables) and by not postponing configuration
difficulties (in particular the handling of external dependencies like
libraries) to compile stage.

\verb+Cmake+ classically provides two ways for setting configuration
parameters in order to generate the makefiles in the form of two
commands \verb+ccmake+ and \verb+cmake+ (the first one has an extra "c"
character): use
\begin{description}
\item{\verb+ccmake [options] <path-to-source>+}\\
  in order to specify the parameters interactively through a GUI interface
\item{\verb+cmake [options] <path-to-source> [-D<var>:<type>=<value>]+}\\
  in order to define the parameters with the \verb+-D+ flag directly
  from the command line.
\end{description}
In the above syntax description of both commands, {\verb+<path-to-source>+}
specifies a path to the top level of the source tree (i.e. where the directory
where the top level CMakeLists.txt file is to be encountered). Also
the current working directory will be used as the root of the build tree for
the project (out of source building is generally encouraged specially
when working on a CVS tree).

Here is a short list of \verb+cmake+ internal parameters that are worth
mentioning:
\begin{itemize}
\item
  \verb+CMAKE_BUILD_TYPE+ controls the type of build mode among which 
  \verb+Debug+ will produce binaries and libraries with the debugging
   information
\item
   \verb+CMAKE_VERBOSE_MAKEFILE+ is a Boolean parameter which when set to
   ON will generate makefiles without the .SILENT directive. This is
   useful for watching the invoked commands and their arguments in case
   things go wrong.
\item
   \verb+CMAKE_C[XX]_FLAGS*+ is a family of parameters used for
   the setting and the customization of various C/C++ compiler options.
\end{itemize}
%
Eventually, here is short list of \verb+ccmake+ interface tips:
\begin{itemize}
\item
  when lost, look at the bottom lines of the interface which always
  summarizes \verb+ccmake+'s most pertinent options (corresponding
  keyboard shortcuts) depending on your current context
\item
  hitting the "h" key will direct you \verb+ccmake+ embedded tutorial
  and a list of keyboard shortcuts (as mentioned in the bottom
  lines, hit "e" to exit)
\item
  up/down navigation among parameter items can be achieved with the
  up/down arrows
\item
  when on a parameter item, the line in inverted colors (close above the
  bottom of the screen) contains a short description of the selected
  parameter as well as the set of possible/recommended values
\item
  toggling of boolean parameters is made with enter
\item
  press \verb+enter+ to edit path variables
\item
  when editing a \verb+PATH+ typed parameter the \verb+TAB+ keyboard
  shortcut provides an emacs-like (or bash-like) automatic path completion.
\item
  toggling of advanced mode (press "t") reveals hidden parameters
\end{itemize}
 
%==================
\subsubsection{DIET's compilation configuration: a ccmake walk-through}

Assume that \verb+CVS_DIET_HOME+ represents a path to the top level
directory of a tree of DIET sources.
This directory tree was either obtained by expanding the DIET current
source level distribution tarball, or for DIET developers corresponds to
the directory GRAAL/devel/diet/diet of a cvs checkout of the DIET sources
hierarchy.
Additionally, assume we created a build tree directory and \verb+cd+
to it (in the example below we chose \verb+CVS_DIET_HOME/Bin+ as
build tree, but feel free to follow your conventions):
\begin{itemize}
\item
  \verb+cd CVS_DIET_HOME/Bin+
\item
  \verb+ccmake ..+ to enter the GUI
  \begin{itemize}
  \item press \verb+c+ (equivalent of bootstrap.sh of the autotools)
  \item toggle the desired options e.g. \verb+DIET_BUILD_EXAMPLES+ or
     \verb+DIET_USE_JXTA+. 
  \item specify the \verb+CMAKE_INSTALL_PREFIX+ parameter (if you wish
     to install in a directory different from \verb+/usr/local+,
  \item press \verb+c+ again, for checking required dependencies
  \item check all the parameters preceded with the * (star) character
     whose value was automatically retrieved by \verb+cmake+.
  \item provide the required information i.e. feel in the proper values
     for all parameters whose value is terminated by NOT-FOUND
  \item iterate the above process of parameter checking, toggle/specification
     and configuration until all configuration information is satisfied
  \item press \verb+g+ to generate the makefile
  \item press \verb+q+ to exit ccmake
  \end{itemize}
\item
  \verb+make+ in order to classically launch the compilation process
\item
  \verb+make install+ when installation is required
\end{itemize}

%==================
\subsubsection{DIET's main optional configuration flags}

Here are the main configuration flags:
\begin{itemize}
\item
  \verb+OMNIORB4_DIR+ is the path to the omniORB4 installation directory
  (only relevant when omniORB4 was not installed in /usr/local).
  Example: \verb+cmake .. -DOMNIORB4_DIR:PATH=$HOME/local/omniORB-4.0.7+

\item
  \verb+DIET_BUILD_DOCUMENTATION+ for building LaTeX-based user's guide,
  the developer's guide and the doxygenated documentation.
  This option is disabled by default, since it is very sensitive to the
  version of your \LaTeX\ compiler and it also relies on many subdependencies
  (e.g. \verb+doxygen+ or \verb+fig2dev+)

\item
  \verb+DIET_BUILD_EXAMPLES+ activates the compilation of a set of
  general client/server examples. Note that some specific examples
  (e.g. \verb+DIET_USE_BLAS+) require some additional flag to be activated
  too.

\item
  \verb+DIET_BUILD_LIBRARIES+ which is enabled by default, activates the
  compilation of the DIET libraries. This option is only useful if you 
  wish to restrict the compilation to the construction of the documentation.

\item
  \verb+DIET_WITH_STATISTICS+ enables the generation of statistics logs
\end{itemize}

\noindent
DIET has many extensions (some of them are still) experimental. These
extensions most often rely on external packages than need to be pre-installed
and some of those extensions offer concurrent functionalities. This explains
the usage of configuration flags in order to obtain the compilation of
the desired extensions.

\begin{itemize}
\item
  \verb+DIET_USE_BLAS+ option activates the compilation of
  the DIET BLAS examples, as a sub-module of examples.
  The BLAS~\footnote{\url{http://www.netlib.org/blas/}} (Basic Linear
  Algebra Subprograms) are high quality ``building block'' routines for
  performing basic vector and matrix operations.
  Level 1 BLAS do vector-vector operations, Level 2 BLAS do matrix-vector
  operations, and Level 3 BLAS do matrix-matrix operations.
  Because the BLAS are efficient, portable, and widely available,
  they're commonly used in the development of high quality linear algebra
  software.
  DIET uses BLAS to build demonstration examples of client/server.
  Note that the  \verb+DIET_USE_BLAS+ can only be effective when 
  \verb+DIET_BUILD_EXAMPLES+ is enabled.
  \verb+DIET_USE_BLAS+ is disabled by default.

\item
  \verb+DIET_USE_CORI+ CoRI, which stands for COllector of Resource
  Information, provides a framework for probing hardware and performance
  information about the SeD.
  CoRI also yields a very basic set of probing resources which are
  heavily dependent on the system calls available for the considered platform.
  When this option is activated (disabled by default), the user can either
  define new collectors or use existing collectors (like FAST, see the
  \verb+DIET_USE_FAST+ option) through CoRI's interface.
  CoRI thus provides a possible tactical approach for tuning the performance
  of your favorite plug-in scheduler.
  Chapter~\ref{chapter:performance} describes in more details CoRI and its
  possible usage within DIET.

\item
  \verb+DIET_USE_FAST+ activates DIET support of FAST (refer to
  \url{http://graal.ens-lyon.fr/FAST/} a grid aware dynamic forecasting
  library. FAST in turn relies on many sub-libraries (GSL, BDB, NWS, LDAP).
  Thus, the activation of this option can only be recommended for advanced
  users...

\item
  \verb+DIET_USE_BATCH+ activates DIET support of Appleseeds (refer to
  \url{http://grail.sdsc.edu/projects/appleseeds/}) based batch
  extensions.

\item
  \verb+DIET_USE_JXTA+ activates the so called MULTI-Master-Agent
  support. This support is based on the JXTA layer (refer to
  \url{http://www.jxta.org/}). This is to be opposed with
  \verb+DIET_WITH_MULTI_MA+ which offers similar functionalities
  but based on CORBA.

\item
  \verb+DIET_WITH_MULTI_MA+ activates the so called MULTI Master Agent
  support. This support is based on the CORBA layer (which is to be
  opposed with \verb+DIET_USE_JXTA+ which offers similar functionalities
  but based on JXTA.

\item
  \verb+DIET_USE_FD+ for activating Fault Detector.

\item
  \verb+DIET_USE_WORKFLOW+ enables the support of workflow.
\end{itemize}

\noindent
Eventually, some configuration flags control the general result of the
compilation or some developers extensions:
\begin{itemize}
\item
  \verb+BUILD_SHARED_LIBS+ is a cmake internal variable which specifies
  whether the libraries should be dynamics as opposed to static

\item
  \verb+DIET_USE_DART+ enables DART reporting system (refer to 
  \url{http://public.kitware.com/Dart}) which is used for constructing
  DIET's dashboard (see \url{http://graal.ens-lyon.fr/DietDashboard}).
\end{itemize}

%====[ Compilation with the autotools ]========================================
\subsection{Compiling DIET with the autotools (under deprecation)}
A configure script will prepare DIET for
compiling: its main options are described below, but please, run
\texttt{configure --help} to get an up-to-date and complete usage
description.\\

\noindent
{\footnotesize
\texttt{$\sim$> tar xzf DIET\_}\dietversion\texttt{.tgz} \\
\texttt{$\sim$> cd DIET} \\
\texttt{$\sim$/DIET> ./configure --help=short} \\
\texttt{Configuration of DIET} \dietversion \texttt{:} \\
\texttt{...}
}

\subsubsection{Optional features for configuration}

{\footnotesize
\begin{verbatim}
  --disable-FEATURE       do not include FEATURE (same as --enable-FEATURE=no)
  --enable-FEATURE[=ARG]  include FEATURE [ARG=yes]
  --enable-maintainer-mode enable make rules and dependencies not useful
                          (and sometimes confusing) to the casual installer
\end{verbatim}

\begin{verbatim}
  --enable-doc            enable the module doc, documentation about DIET
\end{verbatim}
}
\noindent This option activates the compilation and installation of
the DIET documents, which is disabled by default, because it is very
sensitive to the version of your \LaTeX\ compiler. The output
postscript files are provided in the archive.

{\footnotesize
\begin{verbatim}
  --disable-examples      disable the module examples, basic DIET examples
\end{verbatim}
}
\noindent This option deactivates the compilation of the DIET
examples, which is enabled by default.

{\footnotesize
\begin{verbatim}
  --enable-BLAS           enable the module BLAS, an example for calling BLAS
                          functions through DIET
\end{verbatim}
}
\noindent This option activates the compilation of the DIET BLAS
examples, as a sub-module of examples (which means that this option
has no effect if examples are disabled) - disabled by default.


{\footnotesize
\begin{verbatim}
 --enable-ScaLAPACK      enable the module ScaLAPACK, an example for calling
                          ScaLAPACK functions through DIET
\end{verbatim}
}
\noindent \textit{Experimental}.  This option activates the
compilation of the DIET \scalapack\ examples, as a sub-module of
examples (which means that this option has no effect if examples are
disabled) - disabled by default.

{\footnotesize
\begin{verbatim}
  --enable-JXTA-mode      enable DIET/JXTA architecture
\end{verbatim}
}
\noindent \textit{Experimental}.  This option allows the user to
deploy DIET$_{JXTA}$ architectures.  When this option is activated
(disabled by default), DIET allows only this type of architecture to
be deployed.

\label{sec:multimainstall}
{\footnotesize
\begin{verbatim}
  --enable-multi-MA       enable multi-MA architecture
\end{verbatim}
}
\noindent \textit{Experimental}.  This option allows the user to
connect several MA to be linked together. When this option is
activated, it allow a MA to looking for a SeD into the other MA
hierarchies.

{\footnotesize
\begin{verbatim}
  --enable-JuxMem       enable the use of JuxMem for managing persistent data
\end{verbatim}
}
\noindent \textit{Experimental}.  This option allows the user to
interact with JuxMem for the management of persistent data. When
this option is activated (disabled by default), a SeD can store data
blocks within JuxMem. Chapter~\ref{ch:juxmem} describes in more
details JuxMem and its use inside DIET.

{\footnotesize
\begin{verbatim}
  --enable-cori       enable the use of CoRI for collecting information 
                      about the SeD
\end{verbatim}
}
\noindent \textit{Experimental}.  This option allows the user to
get hardware and performance information about the SeD. When
this option is activated (disabled by default), the user can call FAST and other 
collectors by the interface CoRI. It can be used to improve the plug-in scheduler.
Chapter~\ref{chapter:performance} describes in more details CoRI and its use inside DIET.

{\footnotesize
\begin{verbatim}
  --enable-stats          enable generation of statistics logs
  --disable-dependency-tracking Speeds up one-time builds
  --enable-dependency-tracking  Do not reject slow dependency extractors
  --enable-shared=PKGS    build shared libraries default=yes
  --enable-static=PKGS    build static libraries default=yes
  --enable-fast-install=PKGS optimize for fast installation default=yes
  --disable-libtool-lock  avoid locking (might break parallel builds)
\end{verbatim}
}

%%{\footnotesize
%%\begin{verbatim}
%%  --enable-logservice     enable monitoring through LogService
%%\end{verbatim}
%%}
%%\noindent \textit{Deprecated}.  Support for the LogService monitoring
%%software is now always compiled in.  Note that support for LogService
%%is built-in and you do not have to install LogService to compile
%%DIET.

\subsubsection{Optional packages for configuration}


{\footnotesize
\begin{verbatim}
  --with-PACKAGE[=ARG]    use PACKAGE [ARG=yes]
  --without-PACKAGE       do not use PACKAGE (same as --with-PACKAGE=no)
  --with-gnu-ld           assume the C compiler uses GNU ld default=no
  --with-pic              try to use only PIC/non-PIC objects default=use both
\end{verbatim}
}

%% \subsubsection{The \texttt{--with-PKG-extra} option}

%% For all packages that DIET depends on, the \texttt{configure} script provides an
%% option that let the user define the arguments needed to compile with the
%% libraries of the package PKG: \texttt{--with-PKG-extra}. This is useful when
%% the \texttt{configure} script does not succeed on its own to get necessary
%% compilation option.

%% Let us take the example of a user who wishes to compile the BLAS examples (see
%% the BLAS subsection below). On the platform he/she uses, it is necessary to
%% specify the following arguments to compile with the BLAS library:

%% {\footnotesize \texttt{-lm -L/home/theuser/lib -lblas -lg2c -lm} }

%% \noindent
%% Let us assume that specifying the option
%% \texttt{--with-BLAS-libraries=/home/theuser/lib} is not enough for the
%% \texttt{configure} script to compile DIET examples. Then, the user will have to
%% add the following arguments to his/her \texttt{configure} command line:

%% {\footnotesize \texttt{--with-BLAS-extra="-lm -L/home/theuser/lib -lblas -lg2c
%%   -lm"} }

\paragraph{omniORB}
{\footnotesize
\begin{verbatim}
  --with-omniORB=DIR      specify the root installation directory of omniORB
  --with-omniORB-includes=DIR
                          specify exact header directory for omniORB
  --with-omniORB-libraries=DIR
                          specify exact library directory for omniORB
  --with-omniORB-extra=ARG|"ARG1 ARG2 ..."
                          specify extra conftest.c -o conftest for the linker
                          to find the omniORB libraries (use "" in case of 
                          several conftest.c -o conftest)
\end{verbatim}
}
\noindent This group of options lets the user define all necessary
paths to compile with omniORB. Generally, \texttt{--with-omniORB=DIR}
should be enough, and the other options are provided for ugly
installations of omniORB.\\ \textbf{NB:} having the executable
\texttt{omniidl} in the PATH environment variable should be enough in
most cases.

\paragraph{FAST}\label{fast_compil}
{\footnotesize
\begin{verbatim}
  --with-FAST=DIR         installation root directory for FAST (optional)
  --with-FAST-bin=DIR     installation directory for fast-config (optional)
\end{verbatim}
}
\noindent This group of options lets the user define all necessary
paths to compile with FAST; Generally, \texttt{--with-FAST=DIR} should
be enough, and the other options are provided for difficult
installations of FAST. There is no need to specify includes, libraries
and extra arguments, since FAST provides a tool \texttt{fast-config}
that does this job automatically.\\ \textbf{NB1:} having the
executable \texttt{fast-config} in the PATH environment variable
should be enough in most cases.\\ \textbf{NB2:} it is possible to
specify \texttt{--without-FAST}, which overrides \texttt{fast-config}
detection.

\paragraph{BLAS}

The BLAS~\footnote{\url{http://www.netlib.org/blas/}} (Basic Linear
Algebra Subprograms) are high quality ``building block'' routines for
performing basic vector and matrix operations.  Level 1 BLAS do
vector-vector operations, Level 2 BLAS do matrix-vector operations,
and Level 3 BLAS do matrix-matrix operations. Because the BLAS are
efficient, portable, and widely available, they're commonly used in
the development of high quality linear algebra software.

{\footnotesize
\begin{verbatim}
  --with-BLAS=DIR  specify the root installation directory of BLAS (Basic
                   Linear Algebric Subroutines)

  --with-BLAS-includes=DIR
                   specify exact header directory for BLAS
  --with-BLAS-libraries=DIR
                   specify exact library directory for BLAS
  --with-BLAS-extra=ARG|"ARG1 ARG2 ..."
                   specify extra conftest.c -o conftest for the linker to find
                   the BLAS libraries (use "" in case of several 
                   conftest.c -o conftest)
\end{verbatim}
}
\noindent This group of options lets the user define all necessary
paths to compile with the BLAS libraries. Generally,
\texttt{--with-BLAS=DIR} should be enough, and the other options are
provided for difficult installation of the BLAS.\\ \textbf{NB:} these
options have no effect if the module example and/or its sub-module
BLAS are disabled.

\paragraph{\scalapack}

The ScaLAPACK\footnote{\url{http://www.netlib.org/scalapack/}} (or
Scalable LAPACK) library includes a subset of LAPACK routines
redesigned for distributed memory MIMD parallel computers. It is
currently written in a Single-Program-Multiple-Data style using
explicit message passing for interprocessor communication. It assumes
matrices are laid out in a two-dimensional block cyclic decomposition.

Like LAPACK, the ScaLAPACK routines are based on block-partitioned
algorithms in order to minimize the frequency of data movements
between different levels of the memory hierarchy. For such machines,
the memory hierarchy includes the off-processor memory of other
processors, in addition to the hierarchy of registers, cache, and
local memory on each processor.  The fundamental building blocks of
the ScaLAPACK library are distributed memory versions (PBLAS) of the
Level 1, 2 and 3 BLAS, and a set of Basic Linear Algebra Communication
Subprograms (BLACS) for communication tasks that arise frequently in
parallel linear algebra computations. In the ScaLAPACK routines, all
interprocessor communication occurs within the PBLAS and the BLACS.
One of the design goals of ScaLAPACK was to have the ScaLAPACK
routines resemble their LAPACK equivalents as much as possible.

For detailed information on ScaLAPACK, please refer to the ScaLAPACK
Users' Guide~\cite{BCC+97}.


{\footnotesize
\begin{verbatim}
  --with-ScaLAPACK=DIR    specify the root installation directory of ScaLAPACK
                          (parallel version of the LAPACK library)

  --with-ScaLAPACK-includes=DIR
                          specify exact header directory for ScaLAPACK
  --with-ScaLAPACK-libraries=DIR
                          specify exact library directory for ScaLAPACK
  --with-ScaLAPACK-extra=ARG|"ARG1 ARG2 ..."
                          specify extra conftest.c -o conftest for the linker to find 
                          the ScaLAPACK libraries (use "" in case of several 
                          conftest.c -o conftest)
\end{verbatim}
}

\noindent This group of options lets the user define all necessary
paths for compilation with the \scalapack\ libraries. Normally,
\texttt{--with-ScaLAPACK=DIR} could be enough, but the other options
are provided because the installation of the \scalapack\ libraries is
often difficult, and thus difficult to detect automatically. For
instance, the \texttt{--with-ScaLAPACK-extra} option is useful to
integrate BLACS and MPI libraries, which are useful for ScaLAPACK.  \\
\textbf{NB:} these options have no effect if the module example and/or
its sub-module \scalapack\ are disabled.

\subsubsection{From configuration to compilation}

An important option for configure scripts is \texttt{--prefix=}, which
specifies where binary files, documents and configuration files will
be installed. It is important to set this option; otherwise it 
defaults to \texttt{DIET/install}.

The configuration will return with an error if no ORB was found. So
please help the configure script to find the ORB with the
\texttt{--with-omniORB*} options.\\


If everything went OK, the configuration ends with a summary of the
options that were selected and what it was possible to get. The output
will look like: {\footnotesize
\begin{verbatim}
~/DIET > ./configure --enable-examples --enable-doc
DIET successfully configured as follows:
 - Batch:           no
 - documents:       yes
 - examples:        dmat_manips, file_transfer, scalars
 - FAST:            disabled
 - CORI:            disabled
 - ORB:             omniORB 4 in /usr
 - prefix:          /home/username/DIET/install
 - multi-MA:        no
 - DIET J:          no
 - Juxmem:          no
 - target platform: i686-pc-linux-gnu
 
Please run make help to get compilation instructions.
~/DIET > make help
====================================================================
 Usage : make help     : shows this help screen
         make agent    : builds DIET agent executable
         make SeD      : builds DIET SeD library
         make client   : builds DIET client library
         make examples : builds basic examples
         make          : builds everything configured
         make install  : copy files into /home/username/DIET/install
====================================================================
\end{verbatim}
}

This helps the user to choose which DIET parts to compile and
install. It is recommended for beginners to compile the client,
the SeD and the agent in one single step with \texttt{make all}. But
please pay attention to the fact that \texttt{make all} does not
install DIET in the prefix provided at configuration time. To do this,
run \texttt{make install}.

\texttt{make install} will run the compilation for all DIET entities
before the installation itself. Thus, if the user wants to compile
only the agent, for instance, and install it, he must run:
{\footnotesize
\begin{verbatim}
~/DIET > cd src/agent
~DIET/src/agent > make install
\end{verbatim}
}


\section{Compiling the examples}

Four series of examples are provided in the DIET archive:
\begin{itemize}
\item{\texttt{file\_transfer}}: the server computes the sizes of two
  input files and returns them. A third output parameter may be
  returned; the server decides randomly whether to send back the
  first file. This is to show how to manage a variable number of
  arguments: the profile declares all arguments that may be filled,
  even if they might not be all filled at each request/computation.

\item{\texttt{dmat\_manips}}: the server offers matrix manipulation
  routines: transposition (\texttt{T}), product (\texttt{MatPROD}) and
  sum (\texttt{MatSUM}, \texttt{SqMatSUM} for square matrices, and
  \texttt{SqMatSUM\_opt} for square matrices but re-using the memory
  space of the second operand for the result). Any subset of these
  operations can be specified on the command line. The last two of
  them are given for compatibility with a BLAS server as explained below.
  
\item{\texttt{BLAS}}: the server offers the \texttt{dgemm} BLAS
  functionality.  We plan to offer all BLAS (Basic Linear Algebric
  Subroutines) in the future. Since this function computes
  $C = \alpha AB + \beta C$, it can also compute a matrix-matrix
  product, a sum of square matrices, etc. All these services are
  offered by the BLAS server. Two clients are designed to use these
  services: one (\texttt{dgemm\_client.c}) is designed to use the
  \texttt{dgemm\_} function only, and the other one
  (\texttt{client.c}) to use all BLAS functions (but currently only
  \texttt{dgemm\_}) and sub-services, such as \texttt{MatPROD}.
  
\item{\texttt{\scalapack}}: the server is designed to offer all
  \scalapack\  (parallel version of the LAPACK library) functions but
  only manages the \texttt{pdgemm\_} function so far. The
  \texttt{pdgemm\_} routine is the parallel version of the
  \texttt{dgemm\_} function, so that the server also offers all the
  same sub-services. Two clients are designed to use these services:
  one (\texttt{pdgemm\_client.c}) is designed to use the
  \texttt{pdgemm\_} function only, and the other one
  (\texttt{client.c}) to use all \scalapack\ functions and
  sub-services, such as \texttt{MatPROD}.
\end{itemize}

\subsection{Compiling the examples with cmake}
\verb+Cmake+ will compile the examples when setting the 
\verb+DIET_BUILD_EXAMPLES+ to \verb+ON+ which can be achieved by
toggling the corresponding entry of \verb+ccmake+ GUI's or by adding
\verb+-DDIET_BUILD_EXAMPLES:BOOL=ON+ to the command line
arguments of \verb+[c]cmake+ invocation.

\subsection{Compiling the examples with the autotools}

Running \texttt{make install} in the \texttt{examples} directory (or
in the root directory, when DIET is configured with
\texttt{--enable-examples}) does not copy binary files into
\texttt{$<$install\_dir$>$/bin}, but in \texttt{examples/bin}: this
odd behaviour is due to some limitations of the \textsf{automake} tool.

Likewise, the samples of configuration files located in
\texttt{examples/cfgs} are processed by \texttt{make install} to
create ready-to-use configuration files in \texttt{examples/etc} and
then copied into \texttt{$<$install\_dir$>$/etc}. Successive calls to
make install will not erase the configuration files created and copied
the first time, except if \texttt{make uninstall} is called inbetween
; thus the user will not lose changes to these files.



%
% DIET data
%
%\newpage
%****************************************************************************%
%* DIET User's Manual data chapter file                                     *%
%*                                                                          *%
%*  Author(s):                                                              *%
%*    - Philippe COMBES (Philippe.Combes@ens-lyon.fr)                       *%
%*                                                                          *%
%* $LICENSE$                                                                *%
%****************************************************************************%
%* $Id$
%* $Log$
%* Revision 1.12  2005/06/14 08:16:06  ecaron
%* Typo
%*
%* Revision 1.11  2005/04/22 10:35:53  ycaniou
%* The fourth arg of diet_string_set(), size_t length, does not exist
%*
%* Revision 1.10  2004/08/27 17:04:36  ctedesch
%* - update DIET J chapter :
%* use of the PIF DIET
%* JXTA -> DIET J
%*
%* - syntax corrections in data chapter (_ -> \_)
%*
%* Revision 1.9  2004/07/20 08:34:20  bdelfabr
%* corrections. Thanks Matthias.
%*
%* Revision 1.8  2004/07/01 08:53:10  bdelfabr
%* add data management part in section 3.
%*
%* Revision 1.7  2004/02/10 01:58:08  ecaron
%* Add suggesstions from Matthias Colin. Thanks!
%*
%* Revision 1.6  2004/02/10 00:38:25  ecaron
%* Add suggestions from Jean-Yves L'Excellent. Thanks !
%*
%* Revision 1.5  2004/01/29 17:08:47  ecaron
%* Add suggestions from Frederic Desprez. Thanks !
%*
%* Revision 1.3  2004/01/21 00:25:13  ecaron
%* Add suggestions from Holly Dail. Thanks !
%*
%* Revision 1.2  2003/12/12 10:15:29  avernois
%*
%* precision :les types DIET_SCOMPLEX et DIET_DCOMPLEX 
%*            sont encore en status de TODO
%*
%* Revision 1.1  2003/09/09 12:38:20  pcombes
%* Reorganization of doc: UM becomes UsersManual.
%*
%* Revision 1.7  2003/06/02 13:47:05  pcombes
%* Fix footnotesize.
%*
%* Revision 1.6  2003/05/23 09:23:35  pcombes
%* Add suggestions from Jean-Yves. Thanks !
%*
%* Revision 1.5  2003/05/15 14:17:58  pcombes
%* UM 0.7
%*
%* Revision 1.2  2003/01/24 16:58:54  pcombes
%* UM 0.6.4
%*
%* Revision 1.1  2003/01/22 17:34:53  pcombes
%* User Manual, v. 0.6.4
%****************************************************************************%

\chapter{DIET data}
\label{ch:data}

It is important that DIET can manipulate data to optimize copies and memory
allocation, to estimate data transfer and computation time, etc.  Therefore the
data must be fully described: their types first, then all the attributes
associated to these types.



\section{Data types}
\label{sec:types}

DIET defines a precise set of data types to be used to describe the arguments of
the services (on the server side) and of the problems (on the client side).

The DIET data types are defined in the file
\texttt{$<$install\_dir$>$/include/DIET\_data.h}. The user will also find in
this file various function prototypes to manipulate all DIET data types. Please
refer to this file for a complete and up-to-date API description.

To keep DIET type descriptions generic, two main sets are used: base and
composite types.

\subsection{Base types}
\label{ssec:base}

Base types are defined in an enum type \texttt{diet\_base\_type\_t} and have the
following semantics:
\begin{center}
\footnotesize
\begin{tabular}{|l|l|c|}
\hline
\textbf{Type}&\textbf{Description}&\textbf{Size in octets}\\
\hline
\textsf{DIET\_CHAR}     & Character                &  1\\
\textsf{DIET\_BYTE}     & Octet                    &  1\\
\textsf{DIET\_INT}      & Signed integer           &  4\\
\textsf{DIET\_LONGINT}  & Long signed integer      &  8\\
\textsf{DIET\_FLOAT}    & Simple precision real    &  4\\
\textsf{DIET\_DOUBLE}   & Double precision real    &  8\\
\hline\hline
\textsf{DIET\_SCOMPLEX} & Simple precision complex &  8\\
\textsf{DIET\_DCOMPLEX} & Double precision complex & 16\\
\hline
\end{tabular}
\end{center}

\begin{itemize}
\item[NB:] \textsf{DIET\_SCOMPLEX} and \textsf{DIET\_DCOMPLEX} are not implemented yet.
\end{itemize}


\subsection{Composite types}
\label{ssec:complex}

Composite types are defined in an enum type \texttt{diet\_type\_t}:
\begin{center}
\footnotesize
\begin{tabular}{|l|l|}
\hline
\textbf{Type}&\textbf{Possible base types}\\
\hline
\textsf{DIET\_SCALAR} & all base types\\
\textsf{DIET\_VECTOR} & all base types\\
\textsf{DIET\_MATRIX} & all base types\\
\textsf{DIET\_STRING} & \textsf{DIET\_CHAR}\\
\textsf{DIET\_FILE}   & \textsf{DIET\_CHAR}\\
\hline
\end{tabular}
\end{center}

Each of these types requires specific parameters to completely describe the
data (see Figure \ref{fig:data}).


\subsection{Persistence mode}
\label{ssec:persismode}
Persistence mode is defined in an enum type \texttt{diet\_persistence\_mode\_t}

\begin{center}
\footnotesize
\begin{tabular}{|l|l|}
\hline
\textbf{mode}&\textbf{Description}\\
\hline
\textsf{DIET\_VOLATILE} & not stored\\
\textsf{DIET\_PERSISTENT\_RETURN} & stored on server, movable and copy back to client\\
\textsf{DIET\_PERSISTENT} & stored on server and movable\\
\textsf{DIET\_STICKY} & stored and non movable\\
\textsf{DIET\_STICKY\_RETURN} & stored, non movable and copy back to client\\
\hline
\end{tabular}
\end{center}

\begin{itemize}
\item[NB:] \textsf{DIET\_STICKY\_RETURN} is not implemented yet.
\end{itemize}

\section{Data description}
\label{sec:datadesc}

Each parameter of a client problem is manipulated by DIET using the following
structure:
{\footnotesize
\begin{verbatim}
typedef struct diet_arg_s diet_arg_t;
struct diet_arg_s{
  diet_data_desc_t desc;
  void            *value;
};
typedef diet_arg_t diet_data_t;
\end{verbatim}
}

The second field is a pointer to the memory zone where the parameter data are
stored. The first one consists of a complete DIET data description, which is
better described by a figure than with C code, since it can be set and accessed
through API functions. Figure \ref{fig:data} shows the data classification used
in DIET. Every ``class'' inherits from the root ``class'' \texttt{data}, and
could also be a parent of more detailed classes of data in future versions of
DIET.

\begin{figure}[hpt]
 \begin{center}
  \includegraphics[scale=.5]{fig/data.eps}
  \caption{Argument/Data structure description.}
  \label{fig:data}
 \end{center}
\end{figure}


\section{Data management}
\label{sec:datamgt}

\subsection{Data identifier}
\label{ssec:dataid}
The data identifier is generated by the MA. The data identifier is a
string field that contains the MA name, the number of the session plus
the number of the data in the problem (incremental) plus the string
``id''.  This is the \texttt{id} field of the
\texttt{diet\_data\_desc\_t} structure.

{\footnotesize
\begin{verbatim}
typedef struct {
  char* id;  
  diet_persistence_mode_t  mode;
  ....
} diet_data_desc_t;
\end{verbatim}
}

For example, \textbf{id.MA1.1.1} will identify the first data
in the first session submitted on the Master Agent \textbf{MA1}.


\begin{itemize}
\item[NB:] the field ``id'' of the identifier will be next replaced by a
client identifier. This is not implemented yet.
\end{itemize}

\subsection{Data file}
\label{ssec:datafile}

The name of the file is generated by a Master Agent. It is created
during the \texttt{diet\_initialize()} call. The name of the file is
the aggregation of the string ID\_FILE plus the name of the MA plus
the number of the session.  

A file is created only when there are some persistent data in the
session.  

For example, \textbf{ID\_FILE.MA1.1} means the identifiers
of the persistent data stored are in the file corresponding to the
first session in the Master Agent \textbf{MA1}.

The file is stored in the \texttt{/tmp} directory.

\begin{itemize}
\item[NB:] for the moment, when a data is erased from the platform, the
file isn't updated.
\end{itemize}


\section{Manipulating DIET structures}
\label{sec:manip}

The user will notice that the API to the DIET data structures consists of
modifier and accessor functions only: no allocation function is required, since
\texttt{diet\_profile\_alloc} (see Section \ref{sec:pbdesc}) allocates all
necessary memory for all argument \textbf{descriptions}. This avoids the
temptation for the user to allocate the memory for these data structures twice
(which would lead to DIET errors while reading profile arguments). Please see
the example in Section \ref{sec:pbex} for a typical use.
\\

Moreover, the user should know that arguments of the \texttt{\_set} functions
that are passed by pointers are \textbf{not} copied, in order to save memory.
This is true for the \emph{value} arguments, but also for the \emph{path} in
\texttt{diet\_file\_set}. Thus, the user keeps ownership of the memory zones
pointed at by these pointers, and he/she must be very careful not to alter it
during a call to DIET.

\subsection{Set functions}

%The data persistence is not available in this current
%version~\footnote{the data persistence will be available in DIET
%  v1.1}, thus fix the mode of the diet persistence parameter to 0.

\label{sec:setfun}
{\footnotesize
\begin{verbatim}
/**
 * On the server side, these functions should not be used on arguments, but only
 * on convertors (see section 5.5).
 * If mode                                is DIET_PERSISTENCE_MODE_COUNT, 
 * or if base_type                        is DIET_BASE_TYPE_COUNT,
 * or if order                            is DIET_MATRIX_ORDER_COUNT,
 * or if size, nb_rows, nb_cols or length is 0,
 * or if path                             is NULL,
 * then the corresponding field is not modified.
 */

int
diet_scalar_set(diet_arg_t* arg, void* value, diet_persistence_mode_t mode,
                diet_base_type_t base_type);
int
diet_vector_set(diet_arg_t* arg, void* value, diet_persistence_mode_t mode,
                diet_base_type_t base_type, size_t size);

/* Matrices can be stored by rows or by columns */
typedef enum {
  DIET_COL_MAJOR = 0,
  DIET_ROW_MAJOR,
  DIET_MATRIX_ORDER_COUNT
} diet_matrix_order_t;

int
diet_matrix_set(diet_arg_t* arg, void* value, diet_persistence_mode_t mode,
                diet_base_type_t base_type,
                size_t nb_rows, size_t nb_cols, diet_matrix_order_t order);
int
diet_string_set(diet_arg_t* arg, char* value, diet_persistence_mode_t mode);

/* Computes the file size
   ! Warning ! The path is not duplicated !!! */
int
diet_file_set(diet_arg_t* arg, diet_persistence_mode_t mode, char* path);
\end{verbatim}
}


\subsection{Access functions}
\label{sec:accessfun}
{\footnotesize
\begin{verbatim}
/**
 * A NULL pointer is not an error (except for arg): it is simply IGNORED.
 * For instance,
 *   diet_scalar_get(arg, &value, NULL),
 * will only set the value to the value field of the (*arg) structure.
 * 
 * NB: these are macros that let the user not worry about casting (int **)
 * or (double **) etc. into (void **).
 */

/**
 * Type: int diet_scalar_get((diet_arg_t *), (void *),
 *                           (diet_persistence_mode_t *))
 */
#define diet_scalar_get(arg, value, mode) \
        _scalar_get(arg, (void *)value, mode)
/**
 * Type: int diet_vector_get((diet_arg_t *), (void **),
 *                           (diet_persistence_mode_t *), (size_t *))
 */
#define diet_vector_get(arg, value, mode, size) \
        _vector_get(arg, (void **)value, mode, size)
/**
 * Type: int diet_matrix_get((diet_arg_t *), (void **),
 *                           (diet_persistence_mode_t *),
 *                           (size_t *), (size_t *), (diet_matrix_order_t *))
 */
#define diet_matrix_get(arg, value, mode, nb_rows, nb_cols, order) \
        _matrix_get(arg, (void **)value, mode, nb_rows, nb_cols, order)
/**
 * Type: int diet_string_get((diet_arg_t *), (char **),
 *                           (diet_persistence_mode_t *), (size_t *))
 */
#define diet_string_get(arg, value, mode, length) \
        _string_get(arg, (char **)value, mode, length)
/**
 * Type: int diet_file_get((diet_arg_t *),
 *                         (diet_persistence_mode_t *), (size_t *), (char **))
 */
#define diet_file_get(arg, mode, size, path) \
        _file_get(arg, mode, size, (char **)path)
\end{verbatim}
}


\section{Data Management functions}

\begin{itemize}
\item {The \texttt{store\_id} method} is used to store the identifier of a persistent data.
 It also allows to give a description of the data stored.
 This method has to be called after the \texttt{diet\_call()}
\begin{verbatim}
  store_id(char* argID,char *msg);
\end{verbatim}

\item the \texttt{diet\_use\_data} method allows the client to use a data
that is already inside the platform.
\begin{verbatim}
  diet_use_data(diet_arg_t* arg,char* argID);
\end{verbatim}
This function replaces the set functions (see Section \ref{sec:setfun}).



\begin{itemize}
\item[NB:] a mecanisms of publication of data identifiers isnt
implemented yet. So, exchanges of identifiers between end-users that
want to share data must be explicitly done.
\end{itemize}



\item {The \texttt{diet\_free\_persistent\_data} method} allows the
client to remove a persistent data from the platform.
\begin{verbatim}
  diet_free_persistent_data(char *argID);
\end{verbatim}

\end{itemize}


{\footnotesize
\begin{verbatim}

/*******************************************************************
 *   Add handler argID and text message msg in the identifier file *
 ******************************************************************/

void 
store_id(char* argID, char* msg);


/** sets only identifier : data is present inside the platform */

void
diet_use_data(diet_arg_t* arg, char* argID);


/******************************************************************
 *  Free persistent data identified by argID                     *
 *****************************************************************/
int
diet_free_persistent_data(char* argID);

\end{verbatim}
}


\subsection{Free functions}
\label{sec:freefun}

The amount of data  pointed at by value fields should be freed through a DIET
API function:
{\footnotesize
\begin{verbatim}
/****************************************************************************/
/* Free the amount of data pointed at by the value field of an argument.    */
/* This should be used ONLY for VOLATILE data,                              */
/*    - on the server for IN arguments that will no longer be used          */
/*    - on the client for OUT arguments, after the problem has been solved, */
/*      when they will no longer be used.                                   */
/* NB: for files, this function removes the file and frees the path (since  */
/*     it has been dynamically allocated by DIET in both cases)             */
/****************************************************************************/

int
diet_free_data(diet_arg_t* arg);
\end{verbatim}
}


\section{Problem description}
\label{sec:pbdesc}

For DIET to match the client problem with a service, servers and clients must
``speak the same language'', \emph{ie} they must use the same problem
description. A unified way to describe problems is to use a name and define its
profile with the type \texttt{diet\_profile\_t}:
{\footnotesize
\begin{verbatim}
typedef struct {
  int         last_in, last_inout, last_out;
  diet_arg_t *parameters;
} diet_profile_t;
\end{verbatim}
}

%
% FIXME:
% as soon as persistency is integrated, this should become the table Eddy
% prepared to explain the transfer policy, depending on the modes.
%

The field \emph{parameters} consists of a \texttt{diet\_arg\_t} array of size
$last\_out + 1$. Arguments can be
\begin{description}
\item{IN:}    The data are sent to the server. The memory is allocated
  by the user.
\item{INOUT:} The data are allocated by the user as for the IN
  arguments, then sent to the server and brought back into the same memory zone
  after the computation has completed, without any copy. Thus freeing this
  memory while the computation is performed on the server would result in a
  segmentation fault when the data are brought back onto the client.
\item{OUT:} The data are created on the server and brought back into a
  newly allocated zone on the client. This allocation is performed by
  DIET. After the call has returned, the user can find its result in
  the zone pointed at by the \emph{value} field. Of course, DIET
  cannot guess how long the user needs these data for, so it lets
  him/her free the memory allocated with \texttt{diet\_free\_data}.
\end{description}

This behaviour will be modified soon with the introduction of the persistence
modes that will let the user leave some data on the servers for later
computations.

The fields \emph{last\_in}, \emph{last\_inout} and \emph{last\_out} of the
\texttt{diet\_profile\_t} structure respectively point at the indexes in the
\emph{parameters} array of the last IN, INOUT and OUT arguments.

Functions to create and destroy such profiles are defined with the prototypes
below:
{\footnotesize
\begin{verbatim}
diet_profile_t *diet_profile_alloc(int last_in, int last_inout, int last_out);
int diet_profile_free(diet_profile_t *profile);
\end{verbatim}
}



\section{Examples}
\label{sec:pbex}

\subsection{Example 1 : without persistency}
Let us consider the product of a scalar by a matrix: the matrix must be
multiplied in-place, and the computation time must be returned.  This
problem has one IN argument (the scalar factor), one INOUT argument (the matrix)
and one OUT argument (the computation time), so its profile will be built as
follows:
\begin{center}
\includegraphics[scale=.35]{fig/smprod.eps}
\end{center}

Here are the lines of C code to generate such a profile:
{\footnotesize
\begin{verbatim}
  double  factor;
  double *matrix;
  float  *time;
  // Init matrix at least, factor and time too would be better ...
  // ...
  diet_profile_t profile = diet_profile_alloc(0, 1, 2); // last_in, last_inout, last_out
  diet_scalar_set(diet_parameter(profile,0), &factor, 0, DIET_DOUBLE);
  diet_matrix_set(diet_parameter(profile,1), matrix,  0, DIET_DOUBLE, 5, 6, DIET_ROW_MAJOR);
  diet_scalar_set(diet_parameter(profile,2), NULL,    0, DIET_FLOAT);
\end{verbatim}
}

\begin{itemize}
\item[NB1:] If there is no IN argument, \emph{last\_in} must be set to
  -1, if there is no INOUT argument is needed, \emph{last\_inout} must
  be equal to \emph{last\_in}, and if there is no OUT argument,
  \emph{last\_out} must be equal to \emph{last\_inout}.
\item[NB2:] The \emph{value} argument for \texttt{\_set} functions
  (\ref{sec:setfun}) is ignored for OUT arguments, since DIET
  allocates the necessary memory space when the corresponding data are
  transfered from the server, so set value to NULL.
\end{itemize}

\subsection{Example 2 : using persistency}


Let us consider the following problem : $C=A*B$, with A,B and C persistent matrices.


{\footnotesize
\begin{verbatim}
  double *A, *B, *C; 
  // matrices initialization
  ...
  diet_initialize();
  strcpy(path,"MatPROD");
  profile = diet_profile_alloc(path, 1, 1, 2);
  diet_matrix_set(diet_parameter(profile,0),
                  A, DIET_PERSISTENT, DIET_DOUBLE, mA, nA, oA);
  print_matrix(A, mA, nA, (oA == DIET_ROW_MAJOR));
  diet_matrix_set(diet_parameter(profile,1),
                  B, DIET_PERSISTENT, DIET_DOUBLE, mB, nB, oB);
  print_matrix(B, mB, nB, (oB == DIET_ROW_MAJOR));
  diet_matrix_set(diet_parameter(profile,2),
                  NULL, DIET_PERSISTENT_RETURN, DIET_DOUBLE, mA, nB, oC);
  
  if (!diet_call(profile)) {
    diet_matrix_get(diet_parameter(profile,2),&C, NULL, &mA, &nB, &oC);
    store_id(profile->parameters[2].desc.id,"matrix C of doubles");
    store_id(profile->parameters[1].desc.id,"matrix B of doubles");
    store_id(profile->parameters[0].desc.id,"matrix A of doubles");
    print_matrix(C, mA, nB, (oC == DIET_ROW_MAJOR));
      
  }

  diet_profile_free(profile);
  diet_finalize();
\end{verbatim}
}

Then, a client submits the problem : $D=E+C$ with C already present in the platform.
We consider that the handle of C is ``id.MA1.1.3''.

{\footnotesize
\begin{verbatim}
  double *C, *D, *E; 
  // matrices initialization
  ...
  diet_initialize();

  strcpy(path,"MatSUM");
  profile2 = diet_profile_alloc(path, 1, 1, 2);
  
  printf("second pb\n\n");
  diet_use_data(diet_parameter(profile2,0), "id.MA1.1.3");
  diet_matrix_set(diet_parameter(profile2,1),
                  E, DIET_PERSISTENT, DIET_DOUBLE, mA, nB, oE);
  print_matrix(E, mA, nB, (oE == DIET_ROW_MAJOR));
  diet_matrix_set(diet_parameter(profile2,2),
                  NULL, DIET_PERSISTENT_RETURN, DIET_DOUBLE, mA, nB, oD);
  
  if (!diet_call(profile2)) {
    diet_matrix_get(diet_parameter(profile2,2), &D, NULL, &mA, &nB, &oD);
   print_matrix(D, mA, nB, (oD == DIET_ROW_MAJOR));
   store_id(profile2->parameters[2].desc.id,"matrice D de doubles");
   store_id(profile2->parameters[1].desc.id,"matrice E de doubles");
  
  }
  diet_profile_free(profile2);
  diet_free_persistent_data("id.MA1.1.3");
  diet_finalize();
\end{verbatim}
}  

Note that when in the same session, when a data is persistent and used
twice, we cannot know what will be its identifier. However, the
identifier is stored in the structure of the first profile use for
computations. For example, consider a matrix A built with
\texttt{diet\_matrix\_set()} method as following~: {\footnotesize
\begin{verbatim}
  ...
  diet_profile_t *profile;
  ...
  diet_matrix_set(diet_parameter(profile,0),
                  E, DIET_PERSISTENT, DIET_DOUBLE, mA, nA, oA);
  ...
\end{verbatim}
} After the first \texttt{diet\_call}, the identifier of A is stored in
the profile (in \texttt{profile->parameters[0].desc.id}). So, for the
second call, we will have the following instruction in order to use A~:
{\footnotesize
\begin{verbatim}
  ...
  diet_profile_t *profile2;
  ...
  diet_use_data(diet_parameter(profile2,0),profile->parameters[0].desc.id);
  ...
\end{verbatim}
}

\begin{itemize}
\item[NB:] when using this method, the first profile (here
\texttt{profile}) must not be freed before its use. 
\end{itemize}


%
% Building a client program
%
\newpage
%****************************************************************************%
%* DIET User's Manual client chapter file                                   *%
%*                                                                          *%
%*  Author(s):                                                              *%
%*    - Eddy CARON      (Eddy.Caron@ens-lyon.fr)                            *%
%*    - Philippe COMBES (Philippe.Combes@ens-lyon.fr)                       *%
%*    - Christophe Pera (Christophe.Pera@ens-lyon.fr)                       *%
%*                                                                          *%
%* $LICENSE$                                                                *%
%****************************************************************************%
%* $Id$
%* $Log$
%* Revision 1.17  2010/01/21 14:05:58  bdepardo
%* DIET -> \diet
%* SeD -> \sed
%* GoDIET -> \godiet
%*
%* Revision 1.16  2008/07/11 10:11:53  ecaron
%* - Ajout def API (un user a demand�)
%* - Pas mal de modifications sur le chapitre des workflows
%* 	* Corrections
%* 	* Prises en compte des remarques de Brice
%* - Fixme pour Benjamin corriger la vision 2 archi -> 1 archi
%*
%* Revision 1.15  2006/11/15 22:36:50  eboix
%*   Documentation concerning user compilation updated. --- Injay2461
%*
%* Revision 1.14  2005/07/12 21:44:28  hdail
%* - Correcting small problems throughout
%* - Modified deployment chapter to have a real section for deploying via GoDIET
%* - Adding short xml example without the comments to make a figure in GoDIET
%*   section.
%*
%* Revision 1.13  2005/06/27 08:59:41  hdail
%* Updating interfaces to agree with current versions and adding memory cleanup
%* to end of example programs.
%*
%* Revision 1.12  2005/06/14 08:15:40  ecaron
%* Typo
%*
%* Revision 1.11  2004/07/08 15:59:10  mcolin
%* correct the asynchronous example from the tutorial and user manual
%* FIXME :
%*  - still a bug with the INOUT parameter : the matrix is not modified
%*  - User Manuel [5.1] : the first example service/solve_service doesn't
%*  match with the actual version of DIET (pb with pointer and diet_string_get
%*  which doesn't exist)
%*
%* Revision 1.10  2004/02/10 00:44:24  ecaron
%* Add suggestions from Christophe Pera. Thanks !
%*
%* Revision 1.9  2004/02/02 15:57:15  ecaron
%* Add suggestion from Christophe Pera. Thanks !
%*
%* Revision 1.8  2004/01/29 17:08:47  ecaron
%* Add suggestions from Frederic Desprez. Thanks !
%*
%* Revision 1.7  2004/01/27 00:21:17  ecaron
%* Add suggestions from Jean-Yves (Thanks!)
%*
%* Revision 1.6  2004/01/21 00:25:13  ecaron
%* Add suggestions from Holly Dail. Thanks !
%*
%* Revision 1.5  2004/01/07 10:27:33  cpera
%* Add asynchronous call example and useAsyncAPI config parameter.
%*
%* Revision 1.4  2004/01/05 13:06:53  ecaron
%* Update example to DIET 1.0
%****************************************************************************%

\chapter{Building a client program}
\label{ch:client}

The most difficult part of building a client program is to understand how
to describe the problem interface. Once this step is done, it is
fairly easy to build calls to \diet.

\section{Structure of a client program}
\label{sec:cl_struct}

Since the client side of \diet is a library, a client program has to define a
\texttt{main} function that uses \diet through function calls. The complete
client-side interface is described in the files
\texttt{DIET\_data.h} (see Chapter \ref{ch:data}) and
\texttt{DIET\_client.h} found in \texttt{$<$install\_dir$>$/include}.
Please refer to these two files for a complete and
up-to-date API~\footnote{Application programming interface}
description, and include at least the latter at the beginning of
your source code (\texttt{DIET\_client.h} includes \texttt{DIET\_data.h}):
{\footnotesize
\begin{verbatim}
#include <stdio.h>
#include <stdlib.h>

#include "DIET_client.h"

int main(int argc, char *argv[])
{
  diet_initialize(configuration_file, argc, argv);
  // Successive DIET calls ...
  diet_finalize();
}
\end{verbatim}
}

The client program must open its \diet session with a call to
\texttt{diet\_initialize}, which parses the configuration file to set
all options and get a reference to the \diet Master Agent. The session
is closed with a call to \texttt{diet\_finalize}, which frees all
resources associated with this session on the client. Note that
memory allocated for all INOUT and OUT arguments brought back onto
the client during the session is not freed during
\texttt{diet\_finalize}; this allows the user to continue to use the
data, but also requires that the user explicitly free the memory.
The user must also free the memory he or she allocated for IN
arguments.

\section{Client API}
\label{sec:clAPI}

The client API follows the GridRPC definition \cite{gridRPC:02}: all
\texttt{diet\_} functions are ``duplicated'' with \texttt{grpc\_}
functions.  Both \texttt{diet\_initialize}/\texttt{grpc\_initialize}
and \texttt{diet\_finalize}/\texttt{grpc\_finalize} belong to the
GridRPC API. 
 
    A problem is managed through a \emph{function\_handle}, that
associates a server to a problem name. For compliance with GridRPC
\diet accepts \texttt{diet\_function\_handle\_init}, but the server 
specified in the call will be ignored; \diet is designed to
automatically select the best server. The structure allocation is
performed through the function
\texttt{diet\_function\_handle\_default}.

The \emph{function\_handle} returned is associated to the problem description,
its profile, in the call to \texttt{diet\_call}.

\section{Examples}
\label{sec:cl_ex}

Let us consider the same example as in Section \ref{sec:pbex}, but
for synchronous and asynchronous calls.  Here, the client
configuration file is given as the first argument on the command
line, and we decide to hardcode the matrix, its factor, and the name
of the problem.

\subsection{Synchronous call}
\texttt{smprod}
%~\footnote{Source code available in \texttt{doc/tutorial/solutions/exercise2/client\_smprod.c}} 
for scalar by matrix product.

{\footnotesize
\begin{verbatim}
#include <stdio.h>
#include <stdlib.h>
#include <math.h>
#include "DIET_client.h"

int main(int argc, char **argv)
{
  int i;
  double  factor = M_PI; /* Pi, why not ? */
  double *matrix;        /* The matrix to multiply */
  float  *time   = NULL; /* To check that time is set by the server */

  diet_profile_t         *profile;

  /* Allocate the matrix: 60 lines, 100 columns */
  matrix = malloc(60 * 100 * sizeof(double));
  /* Fill in the matrix with dummy values (who cares ?) */
  for (i = 0; i < (60 * 100); i++) {
    matrix[i] = 1.2 * i;
  }
  
  /* Initialize a DIET session */
  diet_initialize("./client.cfg", argc, argv);

  /* Create the profile as explained in Chapter 3 */
  profile = diet_profile_alloc("smprod",0, 1, 2); // last_in, last_inout, last_out
  
  /* Set profile arguments */
  diet_scalar_set(diet_parameter(profile,0), &factor, 0, DIET_DOUBLE);
  diet_matrix_set(diet_parameter(profile,1), matrix,  0, DIET_DOUBLE, 60, 100, DIET_COL_MAJOR);
  diet_scalar_set(diet_parameter(profile,2), NULL,    0, DIET_FLOAT);
  
  if (!diet_call(profile)) { /* If the call has succeeded ... */
     
    /* Get and print time */
    diet_scalar_get(diet_parameter(profile,2), &time, NULL);
    if (time == NULL) {
      printf("Error: time not set !\n");
    } else {
      printf("time = %f\n", *time);
    }

    /* Check the first non-zero element of the matrix */
    if (fabs(matrix[1] - ((1.2 * 1) * factor)) > 1e-15) {
      printf("Error: matrix not correctly set !\n");
    }
  }

  /* Free profile */
  diet_profile_free(profile);
  diet_finalize();
  free(matrix);
  free(time);
}
\end{verbatim}
}

\subsection{Asynchronous call}
\texttt{smprod}
%~\footnote{Source code available in \texttt{doc/tutorial/solutions/exercise2/client\_smprodAsync.c}} 
for scalar by matrix product.
{\footnotesize
\begin{verbatim}
#include <stdio.h>
#include <stdlib.h>
#include <math.h>
#include "DIET_client.h"

int main(int argc, char **argv)
{
  int i, j;
  double  factor = M_PI; /* Pi, why not ? */
  size_t m, n; /* Matrix size */
  double *matrix[5];        /* The matrix to multiply */
  float  *time   = NULL; /* To check that time is set by the server */

  diet_profile_t         *profile[5];
  diet_reqID_t rst[5] = {0,0,0,0,0};

  m = 60;
  n = 100;
 
  /* Initialize a DIET session */
  diet_initialize("./client.cfg", argc, argv);

  /* Create the profile as explained in Chapter 3 */
  for (i = 0; i < 5; i++){
    /* Allocate the matrix: m lines, n columns */
    matrix[i] = malloc(m * n * sizeof(double));
    /* Fill in the matrix with dummy values (who cares ?) */
    for (j = 0; j < (m * n); j++) {
      matrix[i][j] = 1.2 * j;
    }
    profile[i] = diet_profile_alloc("smprod",0, 1, 2); // last_in, last_inout, last_out
  
    /* Set profile arguments */
    diet_scalar_set(diet_parameter(profile[i],0), &factor, 0, DIET_DOUBLE);
    diet_matrix_set(diet_parameter(profile[i],1), matrix[i],  0, DIET_DOUBLE,
                    m, n, DIET_COL_MAJOR);
    diet_scalar_set(diet_parameter(profile[i],2), NULL,    0, DIET_FLOAT);
  }
  
  /* Call Diet */
  int rst_call = 0;
  
  for (i = 0; i < 5; i++){
     if ((rst_call = diet_call_async(profile[i], &rst[i])) != 0)  
        printf("Error in diet_call_async return -%d-\n", rst_call);
     else {
       printf("request ID value = -%d- \n", rst[i]);
       if (rst[i] < 0) {
         printf("error in request value ID\n");
         return 1;
       }
     }
     rst_call = 0;
  }   

  /* Wait for Diet answers */
  if ((rst_call = diet_wait_and((diet_reqID_t*)&rst, (unsigned int)5)) != 0)
     printf("Error in diet_wait_and\n");
  else {
    printf("Result data for requestID");
    for (i = 0; i < 5; i++) printf(" %d ", rst[i]);
    for (i = 0; i < 5; i++){
      /* Get and print time */
      diet_scalar_get(diet_parameter(profile[i],2), &time, NULL);
      if (time == NULL) {
        printf("Error: time not set !\n");
      } else {
        printf("time = %f\n", *time);
      }

      /* Check the first non-zero element of the matrix */
      if (fabs(matrix[i][1] - ((1.2 * 1) * factor)) > 1e-15) {
        printf("Error: matrix not correctly set !\n");
      }
    }
  }
  /* Free profiles */
  for (i = 0; i < 5; i++){
    diet_cancel(rst[i]);
    diet_profile_free(profile[i]);
    free(matrix[i]);
  }
  free(time);
  diet_finalize();
  return 0;
}
\end{verbatim}
}

\section{Compilation}
\label{sec:cl_comp}

After compiling the client program, the user must link it with the
\diet libraries and the CORBA libraries.

\subsection{Compilation when using Makefiles}

When using Makefiles, the easiest way to compile a program using \diet
with all necessary flags and link it with the proper libraries is to
trust the \texttt{Makefile.inc} available in 
\texttt{$<$include\_dir$>$/include} by including it at the beginning
of the program makefile.

The \texttt{Makefile.inc} defines the variables:
\begin{itemize}
\item \texttt{CCFLAGS\_DIET} which contains pre-processing instructions 
to be used when you compile C code,
\item \texttt{CXXFLAGS\_DIET} which contains pre-processing instructions
to be used when you compile C++ code,
\item \texttt{DIET\_CLIENT\_LIBS} which contains the linker instructions
in otder to link your program against the \diet client library.
\item \texttt{CC} and \texttt{CXX} which contains the name of the C compiler,
respectively C++ compiler, used to compiled \diet itself and which you
might choose to compile your own programs (in order to guarantee the 
compatibility of the compilers).
\end{itemize}
\noindent
The \texttt{doc/ExternalExample} directory contains an example of
such a user Makefile, which goes:
{\footnotesize
\begin{verbatim}
# The following inclusion provides convenient make variables for compiling
# and linking against the DIET library (DIET_HOME is an environment variable
# containing the path to a DIET installation directory):
include ${DIET_HOME}/include/Makefile.inc

all: simple_client simple_server

simple_client: simple_client.c $(DIET_CLIENT_PREREQ)
        $(CXX) -g $(CXXFLAGS_DIET) $< $(DIET_CLIENT_LIBS) -o $@
simple_server: simple_server.c  $(DIET_SERVER_PREREQ)
        $(CC) -g $(CCFLAGS_DIET) $< $(DIET_SERVER_LIBS) -o $@
clean:
        rm -f simple_client simple_server

\end{verbatim}
}

Since the \texttt{doc/ExternalExample} directory also contains the
cited \texttt{simple\_server.c} and \texttt{simple\_client.c}, one
can easily test this Makefile.

\subsection{Compilation when using cmake}

The \texttt{doc/ExternalExample} directory also contains a
\texttt{CMakeFile.txt} file which illustrates the cmake way of compiling
this simple client/server example:
{\footnotesize
\begin{verbatim}
PROJECT( DIETSIMPLEEXAMPLE )

SET( CMAKE_MODULE_PATH ${DIETSIMPLEEXAMPLE_SOURCE_DIR}/Cmake )
FIND_PACKAGE( Diet )

# On success use the information we just recovered:
INCLUDE_DIRECTORIES( ${DIET_INCLUDE_DIR} )
LINK_DIRECTORIES( ${DIET_LIBRARY_DIR} )

### Define a simple server...
ADD_EXECUTABLE( simple_server simple_server.c )
TARGET_LINK_LIBRARIES( simple_server ${DIET_SERVER_LIBRARIES} )
INSTALL_TARGETS( /bin simple_server )

### ... and it's associated simple client.
ADD_EXECUTABLE( simple_client simple_client.c )
TARGET_LINK_LIBRARIES( simple_client ${DIET_CLIENT_LIBRARIES} )
INSTALL_TARGETS( /bin simple_client )
\end{verbatim}
}

In order to test drive the cmake configuration of this example, and
assuming the \texttt{DIET\_HOME} points to a directory containing
an installation of \diet, simply try:

{\footnotesize
\begin{verbatim}
export DIET_HOME=<path_to_a_DIET_instal_directory>
cd doc/ExternalExample
mkdir Bin
cd Bin
cmake -DDIET_DIR:PATH=$DIET_HOME -DCMAKE_INSTALL_PREFIX:PATH=/tmp/DIETSimple ..
make
make install
\end{verbatim}
}


% Building a server application
%
\newpage
%**
%*  @file  server.tex
%*  @brief   DIET User's Manual server chapter file 
%*  @author   - Eddy CARON (Eddy.Caron@ens-lyon.fr)
%*            - Philippe COMBES (Philippe.Combes@ens-lyon.fr)
%*            - Georg Hoesh (Georg.Hoesh@ens-lyon.fr)
%*  @section Licence 
%*    |LICENSE|


\chapter{Building a server application}
\label{ch:server}

A \diet server program is the link between the \diet Server Deamon
(SeD) and the libraries that implement the service to offer.

\section{Structure of the program}
\label{sec:sv_struct}

Much like the client side, the \diet SeD is a library. So the server developer
needs to define the \texttt{main} function. Within the \texttt{main}, the \diet
server will be launched with a call to \texttt{diet\_SeD} which will never
return (except if some errors occur, or if a SIGINT or SIGTERM signal is sent
to the \sed). The complete server side interface is described in the files
\texttt{DIET\_data.h} (see Chapter~\ref{ch:data}) and \texttt{DIET\_server.h}
found in \texttt{$<$install\_dir$>$/include}. Do not forget to include the
\texttt{DIET\_server.h} (\texttt{DIET\_server.h} includes
\texttt{DIET\_data.h}) at the beginning of your server source code.

{\footnotesize
\begin{verbatim}
#include <stdio.h>
#include <stdlib.h>

#include "DIET_server.h"
\end{verbatim}
}

The second step is to define a function whose prototype is ``\diet-normalized''
and which will be able to convert the function into the library function prototype.
Let us consider a library function with the following prototype:
{\footnotesize
\begin{verbatim}
int service(int arg1, char *arg2, double *arg3);
\end{verbatim}
}

This function cannot be called directly by \diet, since such a prototype is hard
to manipulate dynamically. The user must define a ``solve'' function whose
prototype only consists of a \texttt{diet\_profile\_t}.
This function will be called by the \diet \sed through a pointer.
{\footnotesize
\begin{verbatim}
int solve_service(diet_profile_t *pb)
{
   int    *arg1;
   char   *arg2;
   double *arg3;

   diet_scalar_get(diet_parameter(pb,0), &arg1, NULL);
   diet_string_get(diet_parameter(pb,1), &arg2, NULL);
   diet_scalar_get(diet_parameter(pb,2), &arg3, NULL);
   return service(*arg1, arg2, arg3);
}
\end{verbatim}
}

Several API functions help the user to write this ``solve''
function, particularly for getting IN arguments as well as setting
OUT arguments.

\subsubsection*{Getting IN, INOUT and OUT arguments}

The \texttt{diet\_*\_get} functions defined in \texttt{DIET\_data.h} are still
usable here. Do not forget that the necessary memory space for OUT arguments is
allocated by \diet. So the user should call the \texttt{diet\_*\_get} functions
to retrieve the pointer to the zone his/her program should write to.

\subsubsection*{Setting INOUT and OUT arguments}

To set INOUT and OUT arguments, use the \texttt{diet\_*\_desc\_set} defined
in \texttt{DIET\_server.h}, these are helpful for writing ``solve''
functions only. Using these functions, the server developer must keep in
mind the fact that he cannot alter the memory space pointed to by
value fields on the server. Indeed, this would make \diet confused
about how to manage the data{\footnote{And the server developer
should not be confused by the fact that
\texttt{diet\_scalar\_desc\_set} uses a value, since scalar values
are copied into the data descriptor.}}.

{\footnotesize
\begin{verbatim}
/**
 * If value                 is NULL,
 * or if order              is DIET_MATRIX_ORDER_COUNT,
 * or if nb_rows or nb_cols is 0,
 * or if path               is NULL,
 * then the corresponding field is not modified.
 */

int
diet_scalar_desc_set(diet_data_t* data, void* value);

// No use of diet_vector_desc_set: size should not be altered by server

// You can alter nb_r and nb_c, but the total size must remain the same
int
diet_matrix_desc_set(diet_data_t* data,
                     size_t nb_r, size_t nb_c, diet_matrix_order_t order);

// No use of diet_string_desc_set: length should not be altered by server

int
diet_file_desc_set(diet_data_t* data, char* path);
\end{verbatim}
}


\section{Server API}
\label{sec:svAPI}


\subsubsection*{Defining services}

First, declare the service(s) that will be offered{\footnote{It is
possible to declare several services in a single SeD.}}.
Each service is described by a profile description called
\texttt{diet\_profile\_desc\_t} since the service does not specify
the sizes of the data. The \texttt{diet\_profile\_desc\_t} type is
defined in \texttt{DIET\_server.h}, and is very similar to
\texttt{diet\_profile\_t}. The difference is that the prototype is
described with the generic parts of \emph{diet\_data\_desc} only,
whereas the client description uses full \emph{diet\_data}.
{\footnotesize
\begin{verbatim}
file DIET_data.h:
     struct diet_data_generic {
       diet_data_type_t type;
       diet_base_type_t base_type;
     };

file DIET_server.h:
     typedef struct diet_data_generic diet_arg_desc_t;

     typedef struct {
       char*            path;
       int              last_in, last_inout, last_out;
       diet_arg_desc_t* param_desc;
     } diet_profile_desc_t;

diet_profile_desc_t* diet_profile_desc_alloc(const char* path,
                        int last_in, int last_inout, int last_out);
int diet_profile_desc_free(diet_profile_desc_t* desc);

diet_profile_desc_t *diet_profile_desc_alloc(int last_in, int last_inout, int last_out);

int diet_profile_desc_free(diet_profile_desc_t *desc);
\end{verbatim}
}

Each profile can be allocated with \texttt{diet\_profile\_desc\_alloc} with the
same semantics as for \texttt{diet\_profile\_alloc}. Every argument of the
profile will then be set with \texttt{diet\_generic\_desc\_set} defined in
\texttt{DIET\_server.h}.

\subsubsection*{Declaring services}

Every service must be added in the service table before the server is
launched. The complete service table API is defined in \texttt{DIET\_server.h}:
{\footnotesize
\begin{verbatim}
typedef int (* diet_solve_t)(diet_profile_t *);
int diet_service_table_init(int max_size);
int diet_service_table_add(diet_profile_desc_t *profile,
                           diet_convertor_t    *cvt,
                           diet_solve_t         solve_func);
void diet_print_service_table();
\end{verbatim}
}

The parameter \texttt{diet\_solve\_t solve\_func} is the type of the
\texttt{solve\_service} function: a function pointer used by \diet to launch the
computation.

The parameter \texttt{diet\_convertor\_t *cvt} is to be used in combination
with scheduling facilities (if available). It is there to allow profile conversion (for
multiple services, or when differences occur between the \diet and the scheduling facility
profile). Profile conversion is complicated and will be treated
separately in Chapter~\ref{chapter:performance}.

\section{Example}
\label{sec:sv_ex}

Let us consider the same example as in Chapter \ref{ch:client}, where
a function \texttt{scal\_mat\_prod} performs the product of a matrix
and a scalar and returns the time required for the computation: {\footnotesize
\begin{verbatim}
int scal_mat_prod(double alpha, double *M, int nb_rows, int nb_cols, float *time);
\end{verbatim}
}
Our program will first define the solve function that consists of the link
between \diet and this function. Then, the \texttt{main} function defines one service and
adds it in the service table with its associated solve function.
{\footnotesize
\begin{verbatim}
#include <stdio.h>
#include "DIET_server.h"
#include "scal_mat_prod.h"

int solve_smprod(diet_profile_t *pb)
{
  double *alpha;
  double *M;
  float  time;
  size_t m, n;
  int res;

  /* Get arguments */
  diet_scalar_get(diet_parameter(pb,0), &alpha, NULL);
  diet_matrix_get(diet_parameter(pb,1), &M, NULL, &m, &n, NULL);
  /* Launch computation */
  res = scal_mat_prod(*alpha, M, m, n, &time);
  /* Set OUT arguments */
  diet_scalar_desc_set(diet_parameter(pb,2), &time);
  /* Free IN data */
  diet_free_data(diet_parameter(pb,0));

  return res;
}

int main(int argc, char* argv[])
{
  diet_profile_desc_t *profile;
  
  /* Initialize table with maximum 1 service */
  diet_service_table_init(1);
  /* Define smprod profile */
  profile = diet_profile_desc_alloc("smprod",0, 1, 2);
  diet_generic_desc_set(diet_param_desc(profile,0), DIET_SCALAR, DIET_DOUBLE);
  diet_generic_desc_set(diet_param_desc(profile,1), DIET_MATRIX, DIET_DOUBLE);
  diet_generic_desc_set(diet_param_desc(profile,2), DIET_SCALAR, DIET_FLOAT);
  /* Add the service (the profile descriptor is deep copied) */
  diet_service_table_add(profile, NULL, solve_smprod);
  /* Free the profile descriptor, since it was deep copied. */
  diet_profile_desc_free(profile);

  /* Launch the SeD: no return call */
  diet_SeD("./SeD.cfg", argc, argv);

  return 0;
}
\end{verbatim}
}

\section{Compilation}
\label{sec:sv_comp}

After compiling her/his server program, the user must link it with the \diet
and CORBA libraries. This process is very similar to the one described
for the client in section \ref{sec:cl_comp}. Please refer to this section
for details.

%%% Local Variables:
%%% mode: latex
%%% ispell-local-dictionary: "american"
%%% mode: flyspell
%%% fill-column: 79
%%% End:


%
% Parallel and Batch submission in DIET
%
\newpage
%****************************************************************************%
%* DIET Programmer' guide, batch/parallel submissions                       *%
%*                                                                          *%
%*  Author(s):                                                              *%
%*    - Yves Caniou (yves.caniou@ens-lyon.fr)                               *%
%*                                                                          *%
%* $LICENSE$                                                                *%
%****************************************************************************%
%* $Id$
%* $Log$
%* Revision 1.6  2008/06/10 16:59:56  ycaniou
%* Typos
%* Use the "${DIET_DOC_SOURCE_DIR}/../src/.." non portable to get path to .c to
%*   be included in the doc. Should be ok if the doc is not removed from the DIET
%*   project
%* Batch chapter � priori completed
%*
%* Revision 1.5  2008/06/09 08:14:33  ycaniou
%* Correction, typos and begin of // and batch chapter
%*
%* Revision 1.4  2008/04/08 10:08:12  ecaron
%* Remove dependancie to src/examples
%* Todo: Fix with Yves this part
%*
%* Revision 1.2  2008/04/07 23:55:37  ecaron
%* Fix overfull box
%*
%* Revision 1.10  2008/04/07 22:26:26  ecaron
%* Updated files to pdflatex compilation
%*
%* Revision 1.9  2006/11/30 14:14:16  ycaniou
%* diet -> textsc{Diet} (Tkx Abdelkader)
%*
%* Revision 1.8  2006/11/30 13:58:07  aamar
%* Remove an extra \
%*
%* Revision 1.7  2006/11/28 20:40:30  ycaniou
%* Only headers
%*
%* Revision 1.6  2006/11/27 08:13:58  ycaniou
%* Added missing fields Id and Log in headers
%****************************************************************************%

\chapter{Batch and parallel submissions}\label{chapter:parallelSubmission}
\section{Introduction}

Most of resources in a grid are parallel, either clusters of
workstations or parallel machines. Computational grids are even
considered as hierachical sets of parallel resources, as we can see in
ongoing project like the french research grid project,
Grid'5000\cite{grid5000} (for the moment, 9 sites are involved), or
like the \textsc{Egee}\footnote{\url{http://public.eu-egee.org/}}
project ({\it Enabling Grids for E-science in Europe}), composed of
more than a hundred centers in 31 countries. Then, in order to provide
transparent access to resources, grid middleware must supply efficient
mechanisms to provide parallel services.

Because parallel resources are managed differently on each site, it is
neither the purpose of \diet to deal with the deployment of parallel
tasks inside the site, nor manage copies of data which can possibly be
on NFS. \diet implements mechanisms for a \sed programmer to easily
provide a service that can be portable on different sites; for clients
to request services which can be explicitly sequential, parallel or
solved in the real transparent and efficient metacomputing way: only
the name of the service is given and \diet chooses the best resource
where to solve the problem.

%****************************************************************************%
\section{Terminology}
%****************************************************************************%

%Because a good understanding comes with correct terms, we provide here
%the definition of the terms that we will use thereafter.

Servers provide {\it services}, \eg instanciation of {problems} that a
server can solve: for example, two services can provide the resolution
of the same problem, one being sequential and the other parallel. A
\diet~{\it task}, also called a {\it job}, is created by the {\it
request} of a client: it refers to the resolution of a service on a
given server.

A service can be sequential or parallel, in which case its resolution
requires numerous processors of a parallel resource (a parallel
machine or a cluster of workstations). If parallel, the task can be
modeled with the MPI standard, or composed of multiple sequential
tasks (deployed for example with \verb!ssh!) resolving a single
service: it is often the case with data parallelism problems.

Note that when dealing with batch reservation systems, we will likely
speak about {\it jobs} rather than about {\it tasks}.

%****************************************************************************%
\section{Configuration for compilation}
%****************************************************************************%

You must enable the batch flag in cmake arguments. Typically, if you
build \diet from the command line, you can use the following:

\begin{lstlisting}[language=bash,label=dietConfig.sh,basewidth={.5em,.4em},fontadjust]

   ccmake $diet_src_path
          -DDIET_USE_ALT_BATCH:BOOL=ON

\end{lstlisting}

%% $$$

%****************************************************************************%
\section{Parallel systems}
%****************************************************************************%

Single parallel systems are surely the less deployed in actual
computing grids. They are usually composed of a frontale node where
clients log in, and from which they can log on numerous nodes and
execute their parallel jobs, {\it without any kind of
reservation}. Some problems occur with such a use of parallel
resources: multiple parallel tasks can share a single processor, hence
delaying the execution of all applications using it; during the
deployment, the application must at least check the connectivity of
the resources; if performance is wanted, some monitoring has to be
performed by the application.

For the moment, \diet has some internals mechanisms to launch external
scripts and monitor them, but they will only be part of the API in the
next release ($>2.3$).

%****************************************************************************%
\section{Batch system}
%****************************************************************************%

Generally, a parallel resource is managed by a batch system, and jobs
are submitted to a site queue. The batch system is responsible for
managing parallel jobs: it schedules each job and, it determines and
allocates the resources needed for its execution. 

There are many batch system, among which
Torque\footnote{\url{http://old.clusterresources.com/products/torque/}}
(a fork of
PSB\footnote{\url{http://www.clusterresources.com/pages/products/torque-resource-manager.php}}),
Loadleveler\footnote{\url{http://www-03.ibm.com/servers/eserver/clusters/software/loadleveler.html}}
(developped by IBM), SunGrid
Engine\footnote{\url{http://www.sun.com/software/gridware/}} (SGE,
developped by Sun), OAR\footnote{\url{http://oar.imag.fr}} (developped
at the IMAG lab). Each one implements its own language syntax (with
its own mnemonics), as well as its own scheduler. Jobs can generally
access the identity of the reserved nodes through a file during their
execution, and are assured to exclusively possess them.

%Diagram batch

%****************************************************************************%
\section{Client extended API}
%****************************************************************************%

Even if older client codes must be recompiled (because internal
structures have evolved), they don't necessarily need modifications.

\diet provides means to request exclusively sequential services,
parallel services, or let \diet choose the best implementation of a
problem for efficiency purposes (according to the scheduling metric
and the performance function).
% The API now incorporates the calling
%functions to perform the former, the latter being the default and
%accessed through the traditional \verb!diet_call()! function:

\begin{lstlisting}[language=c,basewidth={.5em,.4em},fontadjust]

/* To explicitly call a sequential service */
diet_error_t
diet_parallel_call(diet_profile_t * profile) ;

diet_error_t
diet_sequential_call_async(diet_profile_t* profile, diet_reqID_t* reqID);

/* To explicitly call a parallel service in sync or async way */
diet_error_t
diet_sequential_call(diet_profile_t * profile) ;

diet_error_t
diet_parallel_call_async(diet_profile_t* profile, diet_reqID_t* reqID);

/* To mark a profile as parallel or sequential. The default call to 
   diet_call() or diet_call_async() will perform a call to the correct
   previous call */
int
diet_profile_set_parallel(diet_profile_t * profile) ;
int
diet_profile_set_sequential(diet_profile_t * profile) ;

/* To let the user choose a given amount of resources */ 
int
diet_profile_set_nbprocs(diet_profile_t * profile, int nbprocs) ;

\end{lstlisting}

%****************************************************************************%
\section{Batch server extended API and configuration file}
%****************************************************************************%

There are too many diverse scenarii about the communication and
execution of parallel applications: the code can be a MPI code or
composed of different interacting programs possibly launched via
\verb!ssh!  on every nodes; input and output files can use NFS if this
file system is present, or they can be splitted and uploaded to each
node participating to the calculus.

Then, we will see: what supplementary information has to be provided
in the server configuration file; how to write a batch submission
meta-script in a \sed; and how to record the parallel/batch service.

\section{Server API}

\begin{lstlisting}[language=C,label=dietConfig.sh,basewidth={.5em,.4em},fontadjust]

/* Set the status of the SeD among SERIAL and BATCH */
void 
diet_set_server_status( diet_server_status_t st ) ;

/* Set the nature of the service to be registered to the SeD */
int
diet_profile_desc_set_sequential(diet_profile_desc_t * profile) ;

int
diet_profile_desc_set_parallel(diet_profile_desc_t * profile) ;

/* A service MUST call this command to perform the submission to the batch system */
int
diet_submit_parallel(diet_profile_t * profile, const char * command) ;

\end{lstlisting}

\subsection{Registering the service}

A server is mostly built like described in section~\ref{ch:server}. In
order to let the \sed know that the service defined within the profile
is a parallel one, the \sed programmer must use the function:

\begin{lstlisting}[language=c,basewidth={.5em,.4em},fontadjust]
void diet_profile_desc_set_parallel(diet_profile_desc_t* profile)
\end{lstlisting}

By default, a service is registered as sequential. Nevertheless, for
code readability reasons, we also give the pendant function to
explicitly register a sequantial service:

\begin{lstlisting}[language=c,basewidth={.5em,.4em},fontadjust]
void diet_profile_desc_set_sequential(diet_profile_desc_t* profile)
\end{lstlisting}

\subsection{Server configuration file}

The programmer of a batch service available in a \sed has not to worry
to which batch system to submit except for its name, because \diet
provides all the mechanisms to transparently submit the job to them.

%FIXME: � quel point parle t-on d'elagi ?
\diet is able to submit batch scripts to %numerous batch schedulers,
oar1.6 and loadleveler.
%among which: "condor", "dqs", "loadleveler", "lsf", "pbs", "sge" or
%"oar".
The name of the batch scheduler managing the parallel resource where
the \sed is running has to be incorporated with the keyword
\verb!batchName! in the server configuration file. Only this makes the
\sed know how to submit a job correctly.

Furthermore, if there is no default queue, the \diet deployer must
also provide the queue on which jobs have to be submitted, with the
keyword \verb!batchQueue!.

You also have to provide a directory where the \sed can read and write
data on the parallel resource. Pease note that this directory is used
by \diet to store the new built script that is submitted to the batch
scheduler. In consequence, because certain batch schedulers (like OAR)
need the script to be available on all resources, {\it this directory
might be on NFS} (remember that \diet cannot replicate the script on
all resources before submission because of access rights).  Note that
concerning OAR, in order to use the CoRI\_batch features, the Batch
\sed deployer must also provide the keyword \verb$internQueue$ (see
Section~\ref{section:cori_batch}).

For example, the server configuration file can contain the following
lines:

\begin{lstlisting}[language=bash,label=dietConfig.sh,basewidth={.5em,.4em},fontadjust]
batchName = oar
batchQueue = queue_9_13
pathToNFS = /home/ycaniou/tmp/nfs
pathToTmp = /tmp/YC/
internOARbatchQueueName = 913
\end{lstlisting}

%% \subsubsection{\bf Parallel server}

%% With the aim to not extend the \diet API too much, we consider that a
%% parallel server corresponds to an ordinary shell that submits the
%% generated script. In consequence, the server configuration file should
%% contain the following line

%% \begin{lstlisting}[language=bash,label=dietConfig.sh,basewidth={.5em,.4em},fontadjust]
%% batchName = shellscript
%% \end{lstlisting}

\subsection{Server API for writing services}

The writing of a service corresponding to a parallel or batch job is
very simple. The \sed programmer builds a shell script that he would
have normally used to execute the job, \ie a script that must take
care of data replication and executable invocation depending on the
site.

In order for the service to be system independent, the \sed API
provides some meta-variables which can be used in the script. 

\begin{itemize}
\item \verb!$DIET_NAME_FRONTALE!: frontale name
\item \verb!$DIET_USER_NBPROCS!: number of processors
\item \verb!$DIET_BATCH_NODESLIST!: list of reserved nodes
\item \verb!$DIET_BATCH_NBNODES!: number of reserved nodes
\item \verb!$DIET_BATCH_NODESFILE!: name of the file containing the
identity of the reserved nodes
\item \verb!$DIET_BATCH_JOBID!: batch job ID
\item \verb!$DIET_BATCHNAME!: name of the batch system
\end{itemize}

%%% $$$

Once the script written in a string, it is given as an argument to the
following function:
\begin{lstlisting}[language=C,label=dietConfig.sh,basewidth={.5em,.4em},fontadjust]
int 
diet_submit_parallel(diet_profile_t * pb, char * script)
\end{lstlisting}
\subsection{Example of the client/server 'concatenation' problem}

There are fully commented client/server examples in
\verb!<diet_src>/src/examples/Batch! directory. The root directory
contains a simple example, and \verb!TestAllBatch! and
\verb!SparseSolver! are more practical, the latter being a code to
explain the \verb!CoRI\_batch! API.

The root directory contains 2 servers, one sequential
%, one parallel 
and one batch, and a single client. The latter is configurable to
simply ask for only sequential, or explicitly parallel services, or to
let \diet choose the best (by default, 2 processors are used and the
scheduling algorithm is Round-Robin). We consequently give the MPI
code which is called from the batch \sed (not reproduced here). Note
that the user {\it must change} some paths in the \sed codes,
according to the site where he deploys \diet. We reproduce the codes
here.

\newpage

%\parbox[b]{.5\textwidth}
{
  \tiny
  \lstinputlisting[title={\bf Synchronous client code},language=c,label=client.c,basewidth={.5em,.4em},fontadjust]{Data/examples_Batch_client.c}
}

%\newpage

{%\twocolumn
  \tiny
  \lstinputlisting[title={\bf Batch server code},language=c,basewidth={.5em,.4em},fontadjust]{Data/examples_Batch_batch_server.c}
}

{%\twocolumn
  \tiny
%  \lstinputlisting[title={\bf Parallel server code},language=c,basewidth={.5em,.4em},fontadjust]{@CMAKE_SOURCE_DIR@/src/examples/Batch/parallel_server.c}
}

{%\twocolumn
  \tiny
  \lstinputlisting[title={\bf Sequential server code},language=c,basewidth={.5em,.4em},fontadjust]{Data/examples_Batch_sequential_server.c}
}

\onecolumn



%
% Cloud system
%
\newpage
%****************************************************************************%
%* diet Programmer's guide, Cloud submission                                *%
%*                                                                          *%
%*  Author(s):                                                              *%
%*    - Adrian Muresan (adrian.muresan@ens-lyon.fr)                         *%
%*                                                                          *%
%* $LICENSE$                                                                *%
%****************************************************************************%
%* $Id$
%* $Log$
%* Revision 1.3  2010/07/08 14:28:11  amuresan
%* Added cloud.tex in cmakelists for user and programmer guids; added cloud.tex to distro file list.
%*
%* Revision 1.2  2010/07/08 11:44:12  amuresan
%* Completed entry in the ProgrammersGuide for the cloud component
%*
%* Revision 1.1  2010/07/07 15:10:51  amuresan
%* Added Cloud entry for the UsersGuide and ProgrammersGuide.
%*
%****************************************************************************%

The current chapter details the conceptual and implementation details of \diet's Cloud
component. It contains details about the design of the component, the cloud interface
that was used and the API exposed to the \diet programmer.

\section{Objectives}

The goal of the Cloud component is to allow \diet services to use an Amazon EC2 compatible
Cloud platform for on-demand resource provisioning.

\section{Implementation}

Given the goals, the easiest way to use a Cloud platform in \diet is to consider it
a new type of batch system. \diet is easy to extend in this field and all that is needed
is an interface to the Cloud provider and a new implementation for the \textbf{BatchSystem}
abstract class.

\subsection{Eucalyptus SOAP interface}

\textsc{Eucalyptus} has been used as the cloud provider during the development process. It has
been chosen because of its open-source nature and its compatibility with the Amazon EC2 interface.
Managing Virtual Machines in \textsc{Eucalyptus}\footnote{\url{http://open.eucalyptus.com/}}
is done via its web service interface. During
the development, the implemented version of the EC2 interface is 2009-08-15. This corresponds to
version 1.6 of \textsc{Eucalyptus}.

In order to generate a C stub for the web service interface, the gSOAP\footnote{\url{http://www.cs.fsu.edu/~engelen/soap.html}}
package has been used. This automatically generates the interface. The resulting files
are placed in the \verb!<diet_src>/src/util/EucaLib! directory. Please note that the WSSE
plugin for gSOAP should also be installed. This enables Web service security.

Generating the SOAP stub is done in two steps:
\begin{enumerate}
\item{Generate the intermediary header file} - this is necessary for gSOAP:

\verb!wsdl2h -Nec2 -c -o euca.h -t WS-typemap.dat ec2.2008-12-01.wsdl!

In the above command, \textbf{ec2.2008-12-01.wsdl} is a WSDL file describing the web service
interface of the Cloud platform and S-typemap.dat contains type definitions that wsdl2h uses
to parse the wsdl and are required to enable ws-security. The \textbf{-Nec2} option creates
a friendly name (\textbf{ec2}) for the generated structures and functions.

\textbf{Note:} it is necessary to make sure that the generated .h file contains an '\#import "wsse.h"'
directive somewhere at the beginning of its content. The generated .h files from ec2 wsdl files do not
contain this directive by default and this causes errors later on. If the generated .h does not contain
the directive, then it should be manually added: \verb!#import "wsse.h"!. One must pay attention as this
statement is an \textbf{import} which is internally used by gSOAP in the second phase and not a C/C++
\textbf{include} statement.
\item{Generate the stub} with a pure C output and client-side only (the server side stub is not needed):

\verb!soapcpp2 -I import -c -C euca.h!

In the above statement, \textbf{-I import} must specify the directory that contains the Web Service Security plugin, \textbf{wsse.h}, which
is used internally by gSOAP. The resulting source file will contain structure definitions for the types
used by the SOAP interface and methods used for calling the desired web method.
\end{enumerate}

\textbf{Note:} when including the generated files in a compilation, linking should also be done agains
\textbf{libssl} and \textbf{libcrypto}.

Calling a method from the SOAP interface is done by going through the following steps:
\begin{enumerate}
\item Generating the SOAP message with the three security headers required by the EC2 interface
\footnote{\url{http://docs.amazonwebservices.com/AWSEC2/latest/DeveloperGuide/index.html?using-soap-api.html}}:
\begin{enumerate}
\item \textbf{Binary security token} contains the X.509 certificate encoded in base64
\item \textbf{Signature} contains an XML digital signature using a signature algorithm and digest method
\item \textbf{Timestamp} requests to Amazon EC2 are only valid for 5 minutes to prevent replay attacks
\end{enumerate}
\item Instantiating and filling in the structures corresponding to the request and reply of the methods that is to be invoked.
\item Performing the method invokation by calling its corresponding generated C method from the stub.
\item Using the information from the response structure passed to the invoked method.
\end{enumerate}

\subsection{Eucalyptus Batch System}

The Cloud component has been implemented as a \textbf{BatchSystem}. This has been done by subclassing
\textbf{BatchSystem} and implementing its virtual methods.

Running a service call is done in 3 steps:
\begin{enumerate}
\item \textbf{Obtaining the Virtual Machines} through a SOAP call to the corresponding method of the EC2
interface. Note that the VMs are not obtained instantly. Booting a VM takes time. The method returns
instantly and polling is performed until all the VMs have been booted and have an associated IP address.
To prevent infinite waiting, a maximum number of tries is performed.
\item \textbf{Running the service} script on the SeD machine. It has access to the instantiated VMs
inside the script via their IP addresses by using the \verb!DIET_CLOUD_VMS! meta-variable.
\item \textbf{Terminating the VMs} by running another SOAP request to the Cloud front-end coresponding
to the method responsible for termination.
\end{enumerate}

The configuration for a SeD Cloud is done normally through the configuration file. Details about
the configuration file and the valid options can be found in the user's manual.

\section{Installation}

Please refer to the user's manual.




%
% Scheduling in DIET
%
\newpage
%****************************************************************************%
%* DIET User's Manual plugin scheduler chapter file                         *%
%*                                                                          *%
%*  Author(s):                                                              *%
%*    - Alan SU (Alan.SU@ens-lyon.fr)                                       *%
%*                                                                          *%
%* $LICENSE$                                                                *%
%****************************************************************************%

%%%%%%%%%%%%%%%%%%%%%%%%%%%%%%%%%%%%%%%%
%% \documentclass{article}

\newenvironment{code}
{\begin{list}{}{\setlength{\leftmargin}{1em}}\item\bfseries\tt}
{\end{list}}

\newenvironment{tinycode}
{\begin{list}{}{\setlength{\leftmargin}{1em}}\item\tiny\bfseries\tt}
{\end{list}}

%% \begin{document}
%%%%%%%%%%%%%%%%%%%%%%%%%%%%%%%%%%%%%%%%

\chapter{Scheduling in DIET}
\label{ch:plugin}

\section{Introduction}

We introduce a
\emph{plugin scheduling} facility, designed to allow DIET service
developers to define application-specific performance measures and
to implement corresponding scheduling strategies.  This section
describes the default scheduling policy in DIET and the interface to
the plugin scheduling facility.

\section{Default Scheduling Strategy}\label{sect:default_sched}

The DIET scheduling subsystem is based on the notion that, for the
sake of system efficacy and scalability, the work of
determining the appropriate schedule for a parallel workload should be
distributed across the computational platform.  When a task in
such a parallel workload is submitted to the system for processing,
each Server Daemon (SeD) provides a
\emph{performance estimate}~-- a collection of data pertaining to
the capabilities of a particular server in the context of a particular
client request~-- for that task.  These estimates are
passed to the server's parent agent; agents then
sort these responses in a manner that optimizes certain performance
criteria.
Effectively, candidate SeDs are identified through a distributed
scheduling algorithm based on
pairwise comparisons between these
performance estimations; upon receiving server responses from its
children, each agent performs a local scheduling operation called
\emph{server response aggregation}.  The end result of the agent's
aggregation phase is a list of server responses (from servers in the
subtree rooted at said agent), sorted according to the
aggregation method in effect.
By default, the aggregation phase
implements the following ordered sequence of tests:

\begin{enumerate}
\item \textbf{FAST/NWS data}: SeDs compiled and properly configured with
  FAST~\cite{Qui02} and
  NWS~\cite{WSH99}
  are capable of making dynamic
  performance estimates.  If such data
  were generated by the SeDs, these are the metrics on which agents
  select servers.
\item \textbf{Round-robin}: In the absence of application- and
  platform-specific performance
  data, the DIET scheduler attempts to probabilistically achieve load
  balance by assigning client requests on a round-robin
  basis.  Essentially each server records a timestamp indicating the
  last time at which it was assigned a job for execution.  Each time a
  request is received, the SeD computes the time elapsed since its
  last execution, and among the responses it receives, DIET agents
  select SeDs with a longer elapsed time.
\item \textbf{Random}: If the SeD is unable to store
  timestamps, the DIET scheduler will chose randomly when
  comparing two otherwise equivalent SeD performance estimations.
\end{enumerate}

\textbf{Warning:} If DIET is compiled with option \texttt{DIET\_USE\_CORI},
FAST/NWS Scheduling is deactivated (See
Chapter~\ref{chapter:performance} for more information about CoRI).

In principle, this scheduling policy prioritizes servers that are able
to provide useful performance prediction information (as provided by
the FAST and NWS facilities).  In general, this approach works well
when all servers in a given DIET hierarchy are capable of making such
estimations.  However, in platforms composed of SeDs with varying
capabilities, load imbalances may occur: since DIET systematically
prioritizes server responses containing FAST and/or NWS data, servers
that do not respond with such performance data will never be
chosen.

We have designed a plugin scheduler facility to
enable the application developer to tailor the DIET scheduling to the
targeted application.
This functionality provides
the application developer the means to extend the notion of a
performance estimation to include metrics that are
application-specific, and to instruct DIET how to treat those data in
the aggregation phase.
We describe these interfaces in the following sections.


\section{Plugin Scheduler Interface}

Distributed applications are varied and often exhibit performance
behavior specific to the domain from which they arise.  Consequently,
application-specific scheduling approaches are often necessary to
achieve high-performance execution.  We propose an extensible
framework to build
\emph{plugin schedulers}, enabling application developers to specify
performance estimation metrics that are tailored to their individual
needs.

%% This section introduces the principal components of the basic plugin
%% scheduler framework.

\subsection{Estimation Metric Vector}\label{sect:estvector}

The new type \texttt{estVector\_t} represents an
\emph{estimation vector}, logically a structure that can manage a
dynamic collection of performance estimation values.  It contains
values that represent the performance profile provided by a
SeD in response to a DIET service request.  This collection of values
may include either standard performance measures that are available
through DIET, or developer-defined values that are meaningful solely in
the context of the application being developed.

\subsection{Standard Estimation Tags}\label{sect:estTags}

To access to the different fields of the \texttt{estVector\_t}, it
is necessary to specify the tag that correspond to a specific information type.
The Table~\ref{t:tags} describe this correspondance.
Some tags represent a list of values, use the \texttt{diet\_est\_array\_*} 
functions to acces to them. In the Table~\ref{t:tags}, 
the second column marks this multi-value tags.

The tag \textit{ALLINFOS} is a special: his field is 
always empty, but it allows to fill the vector with all known tags 
by the particular collector.
 
\begin{table}[h]
 \footnotesize
 \centering
 \begin{tabular}[c]{|c|c|c|}\hline
%FAST ligne

  \texttt{Information tag}  &multi-& \texttt{Explication} \\[5pt]
 \texttt{starts with EST\_} &value& \\[5pt]
  \hline

%TAGS lines
 \textit{TCOMP        }&& the predicted time 
                       to solve a problem \\[5pt]
%  \hline
  \textit{TIMESINCELASTSOLVE} &   & time since last solve has been made (sec) \\[5pt]
  \hline
  \textit{FREECPU      }&& amount of free cpu between 0 and 1 \\[5pt]
  \hline
  \textit{FREEMEM      }&& amount of free  memory (Mb) \\[5pt]
  \hline
  \textit{NBCPU        }&& number of available processors  \\[5pt]
  \hline
  \textit{CPUSPEED     }&x& frequence of CPUs (MHz) \\[5pt]
  \hline
  \textit{TOTALMEM     }&& total memory size (Mb)  \\[5pt]
  \hline
  \textit{AVGFREECPU   }&& average amount of free CPU [0..1] \\[5pt]
  \hline
  \textit{BOGOMIPS     }&x& the bogomips \\[5pt]
  \hline
  \textit{CACHECPU     }&x& cache size CPUs (Kb) \\[5pt]
  \hline
  \textit{TOTALSIZEDISK}&& size of the partition (Mb)\\[5pt]
  \hline
  \textit{FREESIZEDISK }&& amount of free place on partition (Mb)\\[5pt]
  \hline
  \textit{DISKACCESREAD}&& average time to read on disk (Mb/sec) \\[5pt]
  \hline
  \textit{DISKACCESWRITE}&& average time to write to disk (sec) \\[5pt]
  \hline
  \textit{ALLINFOS     }&x& [empty] fill all possible fields \\[5pt]
  \hline
 \end{tabular}
 \caption{Explication of the estimation tags}
 \label{t:tags}
\end{table}

\subsubsection{Standard Performance Metrics}

To acces to the existing default performance estimation
routines (as described in Chapter~\ref{chapter:performance}), the following
functions are available to facilitate the construction of custom
performance estimation functions:
\begin{itemize}
\item FAST- and NWS-based performance estimation metrics can be used in the plugin scheduler. 
See the Section~\ref{subsection:callFAST} how to use them.
\item The time elapsed since the last execution (to enable
  the round-robin scheduler) is stored in an estimation metric vector
  by calling
  \begin{tabbing}
    \texttt{int diet\_estimate\_lastexec(}\=\texttt{estVector\_t ev,} \\
    \> \texttt{const diet\_profile\_t* const profilePtr)};
  \end{tabbing}
  with an appropriate value for \texttt{ev} and the
  \texttt{profilePtr} corresponding to the current DIET request.
\item The number of waiting jobs when using the maximum concurrent jobs
  limit is stored in an estimation metric vector by calling
  \begin{tabbing}
    \texttt{int diet\_estimate\_waiting\_jobs(}\=\texttt{estVector\_t ev)};
  \end{tabbing}
\item CoRI allows to access in an easy way to basic performance 
prediction. See Chapter~\ref{sec:CORI} to know more about the use of it.

\end{itemize}

In the future, we plan to expand the suite of default estimation
metrics to include dynamic internal DIET system state information
(e.g.,~queue lengths).

\subsubsection{Developer-defined Performance Metrics}

Application developers may also define performance values to be
included in a SeD response to a client request.  For example, a DIET
SeD that provides a service to query particular databases may need
to include information about which databases are currently resident in
its disk cache, in order that an appropriate server be identified for
each client request.  To store such values, the SeD developer should
first choose a unique integer identifier, referred to as the
\emph{tag} to denote each logical datum to be stored.  Values are
associated with tags using the following interface:
\begin{code}
int diet\_est\_set(estVector\_t ev, int userTag, double value);
\end{code}
The \texttt{ev} parameter is the estimation vector into the
value will be stored, the \texttt{userTag} parameter denotes the
chosen tag, and \texttt{value} indicates the value to be associated
with the tag.  Tagged data are used to effect
scheduling policies by defining custom server response
aggregation methods, described in Section~\ref{sect:agg_methods}.

\subsection{Estimation Function}\label{sect:est_fn}

The default behavior of a SeD when a service request arrives from
its parent agent is to store the following information in the
request profile:
\begin{enumerate}
\item \textbf{FAST-based execution time predictions}: DIET SeDs
  attempt to call FAST
  routines to obtain execution time predictions based on the type of
  service requested, if FAST was available at compilation time.  If
  available, such predictions are stored in the
  performance estimate.
\item \textbf{NWS-based dynamic resource information}: If NWS library
  functions are available, performance estimates may include dynamic
  resource performance information about CPU availability, free
  memory, and network bandwidth.
\item \textbf{Elapsed time since last execution}: To implement the
  default round-robin behavior in absence of FAST and NWS facilities,
  each SeD stores a timestamp of its last execution.  When a service
  request arrives, the difference between that timestamp and the
  current time is added to the performance estimate.
\end{enumerate}
This is accomplished by using the \texttt{diet\_estimate\_fast} and
\texttt{diet\_estimate\_lastexec} functions described in
Section~\ref{sect:estvector}.

To implement a plugin scheduler, we define an
interface that admits customizable performance estimation routines:
\begin{tabbing}
  \texttt{typedef void (* diet\_perfmetric\_t)(}
    \=\texttt{diet\_profile\_t*,} \\
    \>\texttt{estVector\_t);} \\
\end{tabbing}
\begin{tabbing}
  \texttt{diet\_perfmetric\_t} \\
  \texttt{diet\_service\_use\_perfmetric(diet\_perfmetric\_t perfmetric\_fn);}\\
\end{tabbing}
%% \begin{code}
%%   diet\_perfmetric\_t\\
%%   diet\_service\_use\_perfmetric(diet\_perfmetric\_t perfmetric\_fn);\\
%% \end{code}
Thus, the type \texttt{diet\_perfmetric\_t} is a function pointer
takes as arguments a performance estimation (represented by the
\texttt{estVector\_t} object) and a DIET service request
profile.  The application
developer can associate such a function, or
\emph{performance estimation routine}, with DIET services via the
\texttt{diet\_service\_use\_perfmetric} interface.  This interface
returns the previously registered performance estimation routine, if
one was defined (and
\texttt{NULL} otherwise).  At this point, a service added using the
\texttt{diet\_service\_table\_add} function will be associated with
the declared performance estimation routine.
Additionally, a performance estimation routine so specified will be
associated with \emph{all} services added into the service table until
another call to the
\texttt{diet\_service\_use\_perfmetric} interface is made.
In the performance estimation routine, the SeD developer should store
in the provided estimation vector
any performance data to be used in the server response aggregation
methods (described in the next section).

\subsection{Aggregation Methods}\label{sect:agg_methods}

At the time a DIET service is defined, an \emph{aggregation method}~--
the logical mechanism by which SeD responses are sorted~-- is
associated with the service; the default behavior was described in
Section~\ref{sect:default_sched}.

If application-specific data \emph{are} supplied (i.e.,~the
estimation function has been redefined), an alternative method for
aggregation is needed.  Currently, a basic
\emph{priority scheduler} has been implemented, enabling an
application developer to specify a series of performance values that
are to be optimized in succession.  A developer may implement a
priority scheduler using the following interface:
\begin{code}
\begin{tabbing}
diet\_aggregator\_desc\_t* \\
diet\_profile\_desc\_aggregator(diet\_profile\_desc\_t* profile); \\
\\
int diet\_aggregator\_set\_type(\=diet\_aggregator\_desc\_t* agg, \\
\> diet\_aggregator\_type\_t atype); \\
\\
int diet\_aggregator\_priority\_max(\=diet\_aggregator\_desc\_t* agg, \\
\> diet\_est\_tag\_t tag); \\
\\
int diet\_aggregator\_priority\_min(\=diet\_aggregator\_desc\_t* agg, \\
\> diet\_est\_tag\_t tag); \\
\\
int diet\_aggregator\_priority\_maxuser(\=diet\_aggregator\_desc\_t* agg, \\
\> int val); \\
\\
int diet\_aggregator\_priority\_minuser(\=diet\_aggregator\_desc\_t* agg, \\
\> int val); \\
\end{tabbing}
\end{code}
The \texttt{diet\_profile\_desc\_aggregator} and
\texttt{diet\_aggregator\_set\_type} functions fetch and configure the
aggregator corresponding to a DIET service profile, respectively.
In particular, a priority scheduler is declared by invoking the latter
function with \texttt{DIET\_AGG\_PRIORITY} as the \texttt{agg}
parameter.
Recall that from the point of view of an agent, the aggregation phase
is essentially a sorting of the server responses from its children.
A priority scheduler logically uses a series of user-specified tags to
perform the pairwise server comparisons needed to construct the
sorted list of server responses.

To define the tags and the order in which they should be compared,
four functions are introduced.  These functions, of the form
\texttt{diet\_aggregator\_priority\_*}, serve to identify the
estimation values to be optimzed during the aggregation phase.  The
\texttt{\_min} and \texttt{\_max} forms indicate that a standard
performance metric (e.g.,~time elapsed since last execution, from the
\texttt{diet\_estimate\_lastexec} function) is to be either
minimized or maximized, respectively.  Similarly, the
\texttt{\_minuser} and \texttt{\_maxuser} forms indicate the analogous
operations on user-supplied estimation values.  Calls to these
functions indicate the order of \textbf{precedence} of the tags.

Each time two server responses need to be compared, the values
associated with the tags specified in the priority aggregator are
retrieved.  In the specified order, pairs of corresponding values are
successively compared, passing to the next tag only if the values for
the current tag are identical.  If one server response contains a
value for the metric currently being compared, and another does not,
the response with a valid value will be selected.  If at any point
during the treatment of tags \emph{both} responses lack the necessary
tag, the comparison is declared indeterminate.
This process continues until one response is
declared superior to the other, or all tags in the priority aggregator
are exhausted (and the responses are judged equivalent).


\section{Example}

A new example has been added to the DIET distribution to illustrate
the usage of the plugin scheduler functionality; this code is
available in the directory
\begin{code}
src/examples/plugin\_example/
\end{code}
Provided are a DIET server and client, corresponding to a simulation
of a database research application.  If the construction of examples
was enabled during DIET configuration, two binaries \texttt{server}
and \texttt{client} will be built in this directory.  Having deployed
a DIET agent hierarchy, the server may be instantiated:
\begin{code}
  \$ server <SeD\_config> <DB> [ <DB> ... ]
\end{code}
where \texttt{<DB>} are string(s) that represent the existence of
a particular database at the SeD's site.  A client would pose a query
against a set of databases:
\begin{code}
  \$ client <client\_config> <DB> [ <DB> ... ]
\end{code}
The application uses the plugin scheduling facility to prioritize the
existence of databases in selecting a server, and thus, the expected
result is that one of the SeDs with the fewest number of database
mismatches will be selected.

In the \texttt{main} function of the \texttt{server.c} file, the
following block of code (a)~specifies the use of the priority
aggregator for this service, (b)~declares a performance estimation
function to supply the necessary data at request-time, and
(c)~defines the order of precedence of the performance values
(i.e.,~minimizing the number of database mismatches, and then
maximizing the elapsed execution time).
\begin{verbatim}
  {
    /* new section of the profile: aggregator */
    diet_aggregator_desc_t *agg;
    agg = diet_profile_desc_aggregator(profile);

    /* install our custom performance function */
    diet_service_use_perfmetric(performanceFn);                /* (a) */

    /* for this service, use a priority scheduler */
    diet_aggregator_set_type(agg, DIET_AGG_PRIORITY);          /* (a) */
    diet_aggregator_priority_minuser(agg, 0);                  /* (c) */
    diet_aggregator_priority_max(agg, EST_TIMESINCELASTSOLVE); /* (c) */
  }
\end{verbatim}
The performance function \texttt{performanceFn} is defined as follows:
\begin{verbatim}
static void performanceFn(diet_profile_t* pb, estVector_t perfValues);

[...]

/*
** performanceFn: the performance function to use in the DIET
**   plugin scheduling facility
*/
static void
performanceFn(diet_profile_t* pb, estVector_t perfValues)
{
  const char *target;
  int numMismatch;

  /* string value must be fetched from description; value is NULL */
  target = (diet_paramstring_get_desc(diet_parameter(pb, 0)))->param;
  numMismatch = computeMismatches(target);

  /*
  ** store the mismatch value in the user estimate space,
  ** using tag value 0
  */
  diet_est_set(perfValues, 0, numMismatch);

  /* also store the timestamp since last execution */
  diet_estimate_lastexec(perfValues, pb);
}
\end{verbatim}
The function \texttt{computeMismatches} (defined earlier in
\texttt{server.c}) calculates the number of requested databases that
are not present on the SeD making the evaluation.
Together, these two code segments serve to customize the generation of
performance information and the treatment of these data in the context
of the simulated database search.
Finally, it should be noted that the existence of a plugin scheduler
is completely transparent to the client, and thus client code need not
be changed.

\section{Future Work}

We have two primary efforts planned for extensions to the plugin
scheduler.
\begin{itemize}
\item \textbf{Additional information services}: We plan to add
  functionality to enable the application developer to access and use
  data concerning the internal state of the DIET server (e.g.,~the
  current length of request queues).  As other performance measurement
  and evaluation tools are developed both within and external to the
  DIET project (see Chapter~\ref{chapter:performance}), some
  tools are already available to enable such 
  information to be incorporated
  in the context of the plugin scheduler.
\item \textbf{Enhanced aggregation methods}: The plugin scheduler
  implemented in the current release enables the DIET system to
  account for user-defined factors in the server selection process.
  However, the priority aggregation method is fairly rudimentary and
  lacks the power to express many imaginable comparison mechanisms.
  We plan to investigate methods to embed code into DIET agents
  (e.g.,~a simple expression interpreter) in a manner that is secure
  and that preserves performance.
\end{itemize}

%%%%%%%%%%%%%%%%%%%%%%%%%%%%%%%%%%%%%%%%
%% \end{document}
%%%%%%%%%%%%%%%%%%%%%%%%%%%%%%%%%%%%%%%%



%
% Preformances prediction
%
\newpage
%**
%*  @file  prediction.tex
%*  @brief   DIET User's Manual Performance prediction chapter file 
%*  @author  - Martin QUINSON (Martin.Quinson@loria.fr) (FAST)
%*           - Peter FRAUENKRON (Peter.Frauenkron@gmail.com) (CoRI) 
%*  @section Licence 
%*    |LICENSE|


\chapter{Performance prediction}
\label{chapter:performance}
\section{Introduction}

As we have seen in Chapter~\ref{ch:plugin} the agent needs some information
from the \sed to make an optimal scheduling. This information is a performance
prediction of the \sed. The agent will ask the \sed to fill the data structure
defined in Chapter~\ref{ch:plugin} with the information it needs. The \sed
returns the information and the agent can make the scheduling.\\ Performance
prediction can be based on hardware information, the load of the \sed (the CPU
load, the memory availability, \etc) or an advanced performance prediction can
combine a set of basic performance predictions. Completely custom data can even be used for performance prediction, when using plugin schedulers. A performance prediction
scheduler module named CoRI  is available.
\\CoRI is described in Section~\ref{sec:CORI}.\\
In table~\ref{t:depcompil} you can see which
information is available by default with CoRI.

\begin{table}[h]
 \tiny
 \centering
 \begin{tabular}[c]{|l|}\hline
%Cori ligne
  \textbf{always available:} \\[5pt]
  \hline
  \hline
%TAGS lines
 \textit{TCOMP        } \\[5pt]
 \hline 
%  \textit{TIMESINCELASTSOLVE}\\[5pt]
%  \hline
  \textit{FREECPU      } \\[5pt]
  \hline
  \textit{FREEMEM      } \\[5pt]
  \hline
  \textit{NBCPU        } \\[5pt]
  \hline
  \textit{CPUSPEED     } \\[5pt]
  \hline
  \textit{TOTALMEM     } \\[5pt]
  \hline
  \textit{AVGFREECPU   } \\[5pt]
  \hline
  \textit{BOGOMIPS     } \\[5pt]
  \hline
  \textit{CACHECPU     } \\[5pt]
  \hline
  \textit{TOTALSIZEDISK} \\[5pt]
  \hline
  \textit{FREESIZEDISK } \\[5pt]
  \hline
  \textit{DISKACCESREAD} \\[5pt]
  \hline
  \textit{DISKACCESWRITE} \\[5pt]
  \hline
  \textit{ALLINFOS     } \\[5pt]
  \hline
  \hline
  \textbf{when compiled with -DDIET\_USE\_BATCH=ON} \\[5pt]  
  \textit{PARAL\_NB\_FREE\_RESOURCES\_IN\_DEFAULT\_QUEUE} \\[5pt] 
  \hline
 \end{tabular}
 \caption{Dependencies of the available information on the
 compiling options}
 \label{t:depcompil}
\end{table}

\subsubsection{Using convertors}

The service profiles offered by \diet are sometimes not understandable by the
service implementations. To solve this problem, a convertor processes each
profile before it is passed to the implementation. This is mainly used to hide
the implementation specific profile of a service from the user. It allows
different servers to declare the same service with the same profile using
different implementations of the service. If no convertor
is passed when declaring a new service, a default convertor is assigned to it
that does not change its profile nor its path.

To translate a profile, the convertor defines a new destination profile with a
new path. It then chooses for each argument of the new profile a predefined
function to assign this argument from the source profile. This allows the
following operations:

\begin{description}
\item{\textbf{Permutation of arguments}}. This is done implicitly by specifying
  which argument in the source profile corresponds to which argument in the
  destination profile.
\item{\textbf{Copy of arguments}}. Arguments can be simply used by applying the
  \texttt{DIET\_CVT\_IDENTITY} function. If the same source argument
  corresponds to two destination arguments it is automatically copied.
\item{\textbf{Creation of new arguments}}. New arguments can either contain
  static values or the properties of existing arguments. To create a new static
  value, the index for the source argument must be invalid (\eg -1) and the arg
  parameter must be set to the static argument. To extract a property of an
  existing argument, other functions than \texttt{DIET\_CVT\_IDENTITY} must be
  applied. The result of this function will then be used as the value for the
  destination argument.  Corresponding to the \diet datatypes, the following
  functions exist: \\
\begin{itemize}
\item{\texttt{DIET\_CVT\_IDENTITY}} Copy the argument
\item{\texttt{DIET\_CVT\_VECT\_SIZE}} Get the size of a vector
\item{\texttt{DIET\_CVT\_MAT\_NB\_ROW}} Get the number of rows of a matrix
\item{\texttt{DIET\_CVT\_MAT\_NB\_COL}} Get the number of columns of a matrix
\item{\texttt{DIET\_CVT\_MAT\_ORDER}} Get the order of a matrix
\item{\texttt{DIET\_CVT\_STR\_LEN}} Get the length of the string
\item{\texttt{DIET\_CVT\_FILE\_SIZE}} Get the size of the file
\end{itemize}
Only the \texttt{DIET\_CVT\_IDENTITY} function can be applied to any argument;
all other functions only operate on one type of argument.

\end{description}

\subsection{Example with convertors}

\noindent A short example is available below:
\footnotesize
\begin{verbatim}

/**
 * Example 1
 * Assume we declared a profile (INOUT MATRIX) with the path 'solve_T'.
 * This profile will be called by the client. Our implementation expects
 * a profile (IN INT, IN INT, INOUT MATRIX). This profile is known to
 * FAST with the path 'T_solve'.
 * We will write a convertor that changes the name and extracts the
 * matrix's dimensions.
 */
    // declare a new convertor with 2 IN, 1 INOUT and 0 OUT arguments
    cvt = diet_convertor_alloc("T_solve", 0, 1, 1);

    // apply the function DIET_CVT_MAT_NB_ROW to determine the
    // 0th argument of the converted profile. The function's
    // argument is the 0th argument of the source profile. As it
    // is an IN argument, the last parameter is not important.
    diet_arg_cvt_set(&(cvt->arg_convs[0]), DIET_CVT_MAT_NB_ROW, 0, NULL, 0);

    // apply the function DIET_CVT_MAT_NB_COL to determine the
    // 1st argument of the converted profile. The function's
    // argument is the 0th argument of the source profile. As it
    // is a IN argument, the last parameter is not important.
    diet_arg_cvt_set(&(cvt->arg_convs[1]), DIET_CVT_MAT_NB_COL, 0, NULL, 0);

    // apply the function DIET_CVT_IDENTITY to determine the
    // 2nd argument of the converted profile. The function's
    // argument is the 0th argument of the source profile and
    // it will be written back to the 0th argument of the source
    // profile when the call has finished.
    diet_arg-cvt_set(&(cvt->arg_convs[2]), DIET_CVT_IDENTITY, 0, NULL, 0);

    // NOTE: The last line could also be written as:
    //diet_arg_cvt_short_set(&(cvt->arg_convs[2]), 0, NULL);

    // add the service using our convertor
    diet_service_table_add(profile, cvt, solve_T);

    // free our convertor
    diet_convertor_free(cvt);
\end{verbatim}
\normalsize

\noindent More examples on how to create and use convertors are given in the
files\\ \texttt{examples/dmat\_manips/server.c} and
\texttt{examples/BLAS/server.c}.

\section{CoRI: Collectors of Ressource Information}
\label{sec:CORI}

CoRI manages access to different tools for collecting information about the
\sed. Currently, various tools, called collectors, are implemented: CoRI
Easy and CoRI batch. The user can choose which collector will provide the
information.

The idea is that there are various levels of expertise among users developing
\sed code and \diet client code:
\begin{enumerate}
\item Basic usage will rely on default scheduling strategies.
\item More advanced users will perhaps want to tweak the basic scheduling
  strategies, by providing different priorities between performance metrics
  (\verb+DIET_AGG_PRIORITY+), but still using the default performance metrics
  provided by \diet.
\item Even more advanced users can:
\begin{itemize}
\item ask CoRI to fill the estimation vector with particular sets of metrics,
\item or explicitely add to the default metrics, or replace the default
  metrics, with their own custom metrics in the estimation vector;
\item and optionnaly implement their own custom plugin scheduler
  (\verb+DIET_AGG_USER+), especially if opaque (not floating point, so
  unsortable with \verb+DIET_AGG_PRIORITY+ style scheduling) metrics are set in
  the estimation vector.
\end{itemize}
\end{enumerate}

To allow all these use cases, CoRI, by default, fills the estimation vector
with some default performance metrics. CoRI is also designed to allow managing
high level CoRI performance metrics collectors (see
figure~\ref{fig:cori-overview}), which will on demand fill the estimation
vector with a set of predefined performance metrics. Finally, as seen in
chapter~\ref{ch:plugin}, it is possible to manually fill the estimation vector
with any custom metrics.

\begin{figure}[h]
  \begin{center}
    \includegraphics[scale=0.5]{fig/overviewCori}
    \caption{CoRI overview}
    \label{fig:cori-overview}
  \end{center}
\end{figure}

\subsection{Functions and tags}
The tags for information are of type \texttt{integer} and defined in the
table~\ref{t:tags}. The second type of tag \texttt{diet\_est\_collect\_tag\_t}
is used to specify which collector will provide the information:
\texttt{EST\_COLL\_EASY} or \texttt{EST\_COLL\_BATCH}.
Three different functions are provided with CoRI.

The first function initializes a specific collector.

\footnotesize
\begin{verbatim}
  int
  diet_estimate_cori_add_collector(diet_est_collect_tag_t collector_type,
                                   void * data);
\end{verbatim}
\normalsize The second parameter is reserved for initializing collectors which
need additional information on initialization. For example, the BATCH collector
needs for its initialization the profile of the service to be solved.

After the initialization, accessing to the information is done by specifying
the collector and the information type.
\footnotesize
\begin{verbatim}
  int
  diet_estimate_cori(estVector_t ev,
                     int info_type,
                     diet_est_collect_tag_t collector_type,
                     void* data);
\end{verbatim}
\normalsize

Cori-Easy doesn't need more information, but BATCH need a profile of
type ``diet\_profile\_t''. The last parameter is reserved for it. \\ The last
function is used to test Cori-Easy. It prints all information Cori-Easy finds
to the standard output.

\footnotesize
\begin{verbatim}
  void
  diet_estimate_coriEasy_print();
\end{verbatim}
\normalsize
A result could be the following output:
\footnotesize
\begin{verbatim}
start printing CoRI values..
cpu average load : 0.56
CPU 0 cache : 1024 Kb
number of processors : 1
CPU 0 Bogomips : 5554.17
diskspeed in reading : 9.66665 Mbyte/s
diskspeed in writing : 3.38776 Mbyte/s
total disk size : 7875.51 Mb
available disk size  :373.727 Mb
total memory : 1011.86 Mb
available memory : 22.5195 Mb
end printing CoRI values
\end{verbatim}
\normalsize

\subsection{CoRI-Easy}
The CoRI-Easy collector makes some basic system calls to gather the
information. CoRI-Easy is available by default. The last column of the
table~\ref{t:depcompil} corresponds to the CoRI-Easy's functionality.

There is an example on how to use CoRI-Easy in the
\verb!<diet_src>/src/examples/cori/! directory.

\subsection{CoRI batch}\label{section:cori_batch}

With the help of the CoRI batch collector, a \sed programmer can use some
information obtained from the batch system. It is only available if \diet is
compiled with the option \texttt{-DDIET\_USE\_BATCH} set to
\texttt{ON}. Currently, only simple information can be accessed but
functionalities will be improved along with the number of batch systems \diet
is able to address.

There is an example on how to use CoRI batch in
the\\ \verb!<diet_src>/src/examples/Batch/Cori_cycle_stealing/! directory.

\section{Future Work}

There are two primary efforts for the CoRI manager:
\begin{itemize}
\item \textbf{Improving CoRI-Easy}: Some evaluation functions are very basic
  and should be revised to increase their response time speed and the accuracy
  of the information. There is a need for other information (\ie information
  about the network). Every operating systems provide other basic functions to
  get the information. CoRI-Easy doesn't know all functions. Use the
  \texttt{diet\_estimate\_cori\_print()} function to test what CoRI-Easy can
  find on your \sed. Send us a mail if not  all functions are working properly.

\item \textbf{Improving CoRI batch}: add new functionalities to access dynamic
  information as well as some kind of performance predictions for more batch
  systems.

\item \textbf{New collectors}: Integrating other external tools like
  Ganglia~\cite{Ganglia} or Nagios~\cite{Nagios} to the CoRI Manager can
  provide more useful and exact information.
\end{itemize}

%%% Local Variables:
%%% mode: latex
%%% ispell-local-dictionary: "american"
%%% mode: flyspell
%%% fill-column: 79
%%% End:


%
% Deploying a DIET platform
%
\newpage
%****************************************************************************%
%* DIET User's Manual deploying chapter file                                *%
%*                                                                          *%
%*  Author(s):                                                              *%
%*    - Philippe COMBES (Philippe.Combes@ens-lyon.fr)                       *%
%*                                                                          *%
%* $LICENSE$                                                                *%
%****************************************************************************%
%* $Id$
%* $Log$
%* Revision 1.2  2004/01/07 10:27:33  cpera
%* Add asynchronous call example and useAsyncAPI config parameter.
%*
%* Revision 1.1  2003/09/09 12:38:20  pcombes
%* Reorganization of doc: UM becomes UsersManual.
%*
%* Revision 1.10  2003/06/02 13:47:05  pcombes
%* Fix footnotesize.
%*
%* Revision 1.9  2003/05/23 09:23:35  pcombes
%* Add suggestions from Jean-Yves. Thanks !
%*
%* Revision 1.8  2003/05/15 14:17:58  pcombes
%* UM 0.7
%*
%* Revision 1.5  2003/01/22 17:34:53  pcombes
%* User Manual, v. 0.6.4
%*
%* Revision 1.4  2003/01/21 12:17:02  pcombes
%* Update UM to API 0.6.3, and "hide" data structures.
%*
%* Revision 1.3  2003/01/13 12:09:00  pcombes
%* UM: client part complete for users's day ...
%****************************************************************************%

\chapter{Deploying a DIET platform}
\label{ch:deploying}


%====[ Deploying CORBA services ]==============================================
\section{Deploying CORBA services}
\label{sec:CORBA_services}

So far, only one CORBA service is needed: the Naming Service. It is provided by
a name server that must be launched before all DIET entities. Then, these DIET
entities must be passed a reference to the $<$host:port$>$ of this name server.

In this section, we try to give some basic information to start using DIET with
a basic CORBA configuration. Please refer to the documentation of your ORB if
you need more details.

\subsection{omniNames}

\subsubsection{The server}

To launch the omniORB name server, first check that the path of the omniORB
libraries is in your environment variable \texttt{LD\_LIBRARY\_PATH}, then
specify the log directory, through the environment variable
\texttt{OMNINAMES\_LOGDIR} (or, with \textbf{omniORB 4}, at compile time,
through the \texttt{--with-omniNames-logdir} option of the omniORB configure
script). If there is no log file in this directory, \texttt{omniNames} needs to
be given the port through which its client will connect. It can be launched
as follows:
{\footnotesize
\begin{verbatim}
~ > omniNames -start

Fri Dec 13 14:46:02 2002:

Starting omniNames for the first time.
Wrote initial log file.
Read log file successfully.
Root context is IOR:010000002b00000049444c3a6f6d672e6f72672f436f734e616d696e672f4e61
6d696e67436f6e746578744578743a312e300000010000000000000060000000010102000d0000003134
302e37372e31332e34360000fa0a0b0000004e616d655365727669636500020000000000000008000000
0100000000545441010000001c0000000100000001000100010000000100010509010100010000000901
0100
Checkpointing Phase 1: Prepare.
Checkpointing Phase 2: Commit.
Checkpointing completed.
\end{verbatim}
}

This sets an omniORB name server which listens to clients
connections on default port 2809. If omniNames has already been launched once,
\emph{ie} there is already some log files in the log directory, using the
\texttt{-start} option causes an error. The port is actually read from old
log files:
{\footnotesize
\begin{verbatim}
~ > omniNames -start 2810

Fri Dec 13 15:02:36 2002:

Error: log file '/tmp/omninames-toto.log' exists.  Can't use -start option.
~ > omniNames  

Fri Dec 13 15:02:40 2002:

Read log file successfully.
Root context is IOR:010000002b00000049444c3a6f6d672e6f72672f436f734e616d696e672f4e61
6d696e67436f6e746578744578743a312e300000010000000000000060000000010102000d0000003134
302e37372e31332e34360000fa0a0b0000004e616d655365727669636500020000000000000008000000
0100000000545441010000001c0000000100000001000100010000000100010509010100010000000901
0100
Checkpointing Phase 1: Prepare.
Checkpointing Phase 2: Commit.
Checkpointing completed.
\end{verbatim}
}


\subsubsection{The client}

Every DIET entity needs to connect to the CORBA name server: it is the way to
discover each other. The reference to the omniORB name server is written in the
configuration file, whose path is given to omniORB through the environment
variable \texttt{OMNIORB\_CONFIG} (or, with \textbf{omniORB 4}, at compile time,
through the \texttt{--with-omniORB-config} option of the configure script). Some
examples of such a configuration file are given in the directory
\texttt{src/examples/cfgs} of the DIET source files and installed in
\texttt{$<$install\_dir$>$/etc}. The lines concerning the name server, in the
omniORB configuration file, are built as follows:
\begin{description}
 \item{omniORB 3:}
{\footnotesize
\begin{verbatim}
ORBInitialHost <name server hostname>
ORBInitialPort <name server port>
\end{verbatim}
}
 \item{omniORB 4:}
{\footnotesize
\begin{verbatim}
InitRef = NameService=corbaname::<name server hostname>:<name server port>
\end{verbatim}
}
The name server port is the one given in argument to the \texttt{-start} option
of \texttt{omniNames}.
\end{description}

%\subsection*{The TAO name server}
%\subsubsection{The server}
%\subsubsection{The client}


%====[ DIET configuration file ]===============================================
\section{DIET configuration file} 

Launching a DIET entity needs a configuration file. Some fully commented
examples of such configuration files are given in the directory
\texttt{src/examples/cfgs} of the DIET source files and installed in
\texttt{$<$install\_dir$>$/etc}. Please note that:
\begin{itemize}
\item comments start with '\#' and finish at the end of the current
  line,
\item meaningful lines have the format: \texttt{keyword = value}, following the
  format of configuration files for omniORB 4,
\item keywords are case sensitive.
\end{itemize}

\subsection{Tracing API}

\noindent
\texttt{traceLevel} \ \ \emph{default}\texttt{ = 1}\\
This option controls debugging trace output. The following levels are defined:

\begin{center}
 \footnotesize
 \begin{tabular}{p{.1\linewidth}p{.8\linewidth}}
  level $=$ 0  & Print only errors\\
  level $<$ 5  & Print errors and messages for the main steps (such as ``Got a
  request'') - default\\
  level $<$ 10 & Print errors and messages for all steps\\
  level $=$ 10 & Print errors, all steps, and some important structures (such
  as the list of offered services)\\
  level $>$ 10 & Print all DIET messages AND omniORB messages corresponding to
  an omniORB traceLevel of (level~-~10)
 \end{tabular}
\end{center}


\subsection{Client side parameters}

\noindent
\texttt{MAName} \ \ \emph{default}\texttt{ = }\emph{none}\\
This is a \textbf{compulsory} parameter that specifies the name of the Master
Agent to connect to. The MA must have registered with this same name to the
CORBA name server.


\subsection{Agent side parameters}

\noindent
\texttt{agentType} \ \ \emph{default}\texttt{ = }\emph{none}\\
As DIET offers only one executable for both types of agent, it is
\textbf{compulsory} to specify which kind of agent must be launched. Two values
are available: \texttt{DIET\_MASTER\_AGENT} and \texttt{DIET\_LOCAL\_AGENT}.
They have aliases, respectively \texttt{MA} and \texttt{LA}.
\\

\noindent
\texttt{name} \ \ \emph{default}\texttt{ = }\emph{none}\\
This is a \textbf{compulsory} parameter that specifies the name which the agent
will register with to the CORBA name server.


\subsection{LAs and SeD side parameters}

\noindent
\texttt{parentName} \ \ \emph{default}\texttt{ = }\emph{none}\\
This is a \textbf{compulsory} parameter for Local Agents only ; it indicates the
name of the parent (an LA or the MA) to register to.


\subsection{FAST options}

\noindent
\texttt{fastUse} \ \ \emph{default}\texttt{ = 1}\\
This option activates the requests to FAST. It is ignored if DIET was
compiled without FAST, and defaults to 1 otherwise. In the latter case, it is
useful to set it to 0 if one wishes to perform some tests
without deploying the whole NWS/LDAP platform.

The following options are ignored if DIET was compiled without FAST or if
\texttt{fastUse} is set to 0.

\subsubsection{LDAP options}

\noindent
\texttt{ldapUse} \ \ \emph{default}\texttt{ = 0} for agents, \texttt{ = 1} for
SeDs\\
This option activates the use of LDAP in FAST requests. It defaults to 0 for
agents (which do not need to connect to the LDAP base) and to 1 for SeDs.
\\

The following two options are ignored if \texttt{ldapUse} is set to 0.
\\

\noindent
\texttt{ldapBase} \ \ \emph{default}\texttt{ = }\emph{none}\\
Specify the \texttt{host:port} address of the LDAP base where FAST gets the
results of its benchmarks.
\\

\noindent
\texttt{ldapMask} \ \ \emph{default}\texttt{ = }\emph{none}\\
Specify the mask used for requests to the LDAP base. It must match the one given
in the \texttt{.ldif} file of the server that was added to the base.


\subsubsection{NWS options}

\noindent
\texttt{nwsUse} \ \ \emph{default}\texttt{ = 1}\\
This option activates the use of NWS in FAST requests. If 0, FAST will use an
internal sensor for the performance of the machine, but will not be able to
evaluate communication times.
\\

The following two options are ignored if \texttt{nwsUse} is set to 0.
\\

\noindent
\texttt{nwsNameserver} \ \ \emph{default}\texttt{ = }\emph{none}\\
Specify the \texttt{host:port} address of the NWS name server.
\\

\noindent
\texttt{nwsForecaster} \ \ \emph{default}\texttt{ = }\emph{none}\\
Specify the \texttt{host:port} address of the NWS forecaster.
\\


\subsection{Miscellanous options}

\texttt{endPoint} \ \ \emph{default}\texttt{ = }\emph{2809}\\
This option specifies the listening port of an agent or an SeD. It is very
useful when they are behind a firewall. It defaults to 2809, but if this port is
already busy, the OS chooses a port number.

\texttt{useAsyncAPI} \ \ \emph{default}\texttt{ = }\emph{1}\\
Specify use of asynchronous call and so create objects managing it
on client side.

%====[ Example ]=========================================
\section{Example}
\label{sec:deploy_ex}

As shown in Section \ref{init}, the hierarchy is built from top to down: the
children register to their direct parent.

Here is an example of a complete platform deployment. Let us assume that:

\begin{itemize}
\item DIET was compiled with FAST on all machines used,
\item the LDAP server is launched on the machine \texttt{ldaphost} and listens
  on the port 9000,
\item the NWS name server is launched on the machine \texttt{nwshost} and
  listens on the port 9001,
\item the NWS forecaster is launched on the machine \texttt{nwshost} and
  listens on the port 9002,
\item the NWS sensors are launched on every machine we use.
\end{itemize}


\subsubsection{Launching the MA}

For such a platform, the MA configuration file could be:
\tt
\begin{center}
 \footnotesize
 \begin{tabular}{lcll}
  \multicolumn{4}{l}{\# file MA\_example.cfg, configuration file for an MA}\\
  agentType    &=&DIET\_MASTER\_AGENT&\\
  name         &=&MA\_example        &\\
  \#traceLevel &=&5                  &\# default\\
  \#endPoint   &=&2809               &\# default\\
  \#fastUse    &=&1                  &\# default\\
  \#ldapUse    &=&0                  &\# default\\
  \#nwsUse     &=&1                  &\# default\\
  nwsNameserver&=&nwshost:9001       &\\
  nwsForecaster&=&nwshost:9002       &\\
 \end{tabular}
\end{center}
\rm

This configuration file is the only argument to the executable \texttt{dietAgent},
that is installed in \texttt{$<$install\_dir$>$/bin}. Provided
\texttt{$<$install\_dir$>$/bin} is in your PATH environment variable, run
{\footnotesize
\begin{verbatim}
~ > dietAgent MA_example.cfg

Master Agent MA_example started.
\end{verbatim}
}


\subsubsection{Launching an LA}

For such a platform, an LA configuration file could be:
\tt
\begin{center}
 \footnotesize
 \begin{tabular}{lcll}
  \multicolumn{4}{l}{\# file LA\_example.cfg, configuration file for an LA}\\
  agentType    &=&DIET\_LOCAL\_AGENT&\\
  name         &=&LA\_example       &\\
  parentName   &=&MA\_example       &\\
  \#traceLevel &=&5                 &\# default\\
  \#endPoint   &=&2809              &\# default\\
  \#fastUse    &=&1                 &\# default\\
  \#ldapUse    &=&0                 &\# default\\
  \#nwsUse     &=&1                 &\# default\\
  nwsNameserver&=&nwshost:9001      &\\
  nwsForecaster&=&nwshost:9002      &\\
 \end{tabular}
\end{center}
\rm

This configuration file is the only argument to the executable \texttt{dietAgent},
that is installed in \texttt{$<$install\_dir$>$/bin}. This LA will register as a
son of MA\_example. Run
{\footnotesize
\begin{verbatim}
~ > dietAgent LA_example.cfg

Local Agent LA_example started.
\end{verbatim}
}

\subsubsection{Launching a server}

For such a platform, an \sed\ configuration file could be:
\tt
\begin{center}
 \footnotesize
 \begin{tabular}{lcll}
  \multicolumn{4}{l}{\# file SeD\_example.cfg, configuration file for an SeD}\\
  parentName   &=&LA\_example        &\\
  \#traceLevel &=&5                 &\# default\\
  \#endPoint   &=&2809              &\# default\\
  \#fastUse    &=&1                 &\# default\\
  \#ldapUse    &=&1                 &\# default\\
  ldapBase     &=&ldaphost:9000     &\\
  ldapMask     &=&dc=LIP,dc=ens-lyon,dc=fr&\\
  \#nwsUse     &=&1                 &\# default\\
  nwsNameserver&=&nwshost:9001      &\\
  nwsForecaster&=&nwshost:9002      &\\
 \end{tabular}
\end{center}
\rm

This \sed\ will register as a son of LA\_example. Run the executable that you
linked with the DIET SeD library, and do not forget that the first argument of
\texttt{diet\_SeD} must be the path of the configuration file above.


\subsubsection{Launching a client}

Our client must connect to the MA\_example:
\tt
\begin{center}
 \footnotesize
 \begin{tabular}{lcll}
  \multicolumn{4}{l}{\# file SeD\_example.cfg, configuration file for an SeD}\\
  MAName       &=&LA\_example        &\\
  \#traceLevel &=&5                 &\# default\\
 \end{tabular}
\end{center}
\rm

Run the executable that you linked with the DIET client library, and do not
forget that the first argument of \texttt{diet\_initialize} must be the path of
the configuration file above.



%
% DIET Vizualisation
%
\newpage
%****************************************************************************%
%* DIET User's Manual installing chapter file                               *%
%*                                                                          *%
%*  Author(s):                                                              *%
%*    - Eddy CARON (Eddy.Caron@ens-lyon.fr)                                 *%
%*    - Pushpinder Kaur Chouhan (Pushpinder.Kaur.Chouhan@ens-lyon.fr)       *%
%*    - Philippe COMBES (Philippe.Combes@ens-lyon.fr)                       *%
%*                                                                          *%
%* $LICENSE$                                                                *%
%****************************************************************************%
%* $Id$
%* $Log$
%* Revision 1.1  2005/06/14 08:04:59  ecaron
%* Dashboard section (Todo: Rapha�l)
%*
%* Revision 1.18  2005/05/29 13:51:22  ycaniou
%* Moved the section concerning FAST from description to a new chapter about FAST
%* and performances prediction.
%* Moved the section about convertors in the FAST chapter.
%* Modified the small introduction in chapter 1.
%* The rest of the changes are purely in the format of .tex files.
%*
%* Revision 1.17  2005/05/20 19:06:01  mjan
%* Short description of how to configure DIET for JuxMem
%*
%* Revision 1.16  2004/10/25 08:59:56  sdahan
%* add the multi-MA documentation
%*
%* Revision 1.15  2004/07/12 08:33:58  rbolze
%* explain how to copy cfgs file in install_dir/etc directory and correct my english
%*
%* Revision 1.14  2004/07/09 14:34:42  rbolze
%* make changes relative to DIET 1.1 version
%*
%* Revision 1.13  2004/07/06 13:40:42  ctedesch
%* add corrections from Raphael Bolze.
%*
%* Revision 1.12  2004/04/09 11:19:32  rbolze
%* Add the testing platform : Linux/amd64
%*
%* Revision 1.11  2004/04/05 11:04:29  rbolze
%* add instruction to compile with logService
%*
%* Revision 1.10  2004/02/10 00:13:55  ecaron
%* Add bugzilla reference.
%*
%* Revision 1.9  2004/01/29 17:08:47  ecaron
%* Add suggestions from Frederic Desprez. Thanks !
%*
%* Revision 1.8  2004/01/21 23:23:03  ecaron
%* Add suggestions from Jean-Yves. Thanks !
%*
%* Revision 1.7  2004/01/21 00:25:13  ecaron
%* Add suggestions from Holly Dail. Thanks !
%*
%* Revision 1.6  2004/01/07 20:25:04  ecaron
%* Add ScaLAPACK and BLAS introduction
%*
%* Revision 1.5  2004/01/06 15:07:46  ecaron
%* Correct latex bug
%*
%* Revision 1.4  2003/12/12 14:42:44  pkchouha
%*  define \diet_version in UserManual.tex to be 1.0
%*
%* Revision 1.3  2003/12/03 11:06:57  pkchouha
%* 1. change the version of DIET to 1.0
%* 2. DIET.tgz to DIET_1.0.tgz
%* 3. added the unlisted options in  section 2.2.1
%* 4. commented the  all part of section 2.2.2 before the subcetion oniORB
%* 5. added the unlisted options in section 2.2.2
%* 6. changed the args to conftest.c -o conftest
%* 7. changed some words and sentances for simplification
%*
%* Revision 1.2  2003/11/28 11:51:36  pcombes
%* Correction about gcc-2.96 management of exception handling.
%*
%* Revision 1.1  2003/09/09 12:38:20  pcombes
%* Reorganization of doc: UM becomes UsersManual.
%*
%* Revision 1.12  2003/06/23 13:14:09  pcombes
%* Update example to new configuration summary.
%*
%* Revision 1.11  2003/06/16 17:39:55  pcombes
%* One word about gcc-2.96.
%*
%* Revision 1.10  2003/06/02 13:47:05  pcombes
%* Fix footnotesize.
%*
%* Revision 1.9  2003/05/23 09:23:35  pcombes
%* Add suggestions from Jean-Yves. Thanks !
%*
%* Revision 1.8  2003/05/15 14:17:58  pcombes
%* UM 0.7
%*
%* Revision 1.6  2003/01/24 16:58:54  pcombes
%* UM 0.6.4
%*
%* Revision 1.5  2003/01/22 17:34:53  pcombes
%* User Manual, v. 0.6.4
%****************************************************************************%


\chapter{DIET dashboard}
\label{ch:dashboard}

\fixme{Rapha�l: Quelques mots sur DIET Dashboard + Screenshots}

%====[ Dependencies ]==========================================================
\section{LogService}

\section{VizDIET}
\label{sec:VizDIET}

\section{Statistic}



%
% Multi-MA extension
%
\newpage
%****************************************************************************%
%* DIET User's Manual JXTA chapter file                                     *%
%*                                                                          *%
%*  Author(s):                                                              *%
%*    - Sylvain DAHAN (dahan@lifc.univ-fcomte.fr)                           *%
%*                                                                          *%
%*  This file is part of DIET 2.1.                                          *%
%*                                                                          *%
%*  Copyright (C) 2000-2003 ENS Lyon, LIFC, INSA and INRIA,                 *%
%*                          all rights reserved.                            *%
%*                                                                          *%
%*  Since DIET is open source, free software, you are free to use, modify,  *%
%*  and distribute the DIET source code and object code produced from the   *%
%*  source, as long as you include this copyright statement along with      *%
%*  code built using DIET.                                                  *%
%*                                                                          *%
%*  Redistribution and use in source and binary forms, with or without      *%
%*  modification, are permitted provided that the following conditions      *%
%*  are met.                                                                *%
%*                                                                          *%
%*  Redistributions of source code must retain the copyright notice below   *%
%*  this list of conditions and the following disclaimer. Redistributions   *%
%*  in binary form must reproduce the copyright notice below, this list     *%
%*  of conditions and the following disclaimer in the documentation         *%
%*  and/or other materials provided with the distribution. Neither the      *%
%*  name of ENS Lyon nor the names of its contributors (LIFC, INSA Lyon,    *%
%*  INRIA) may be used to endorse or promote products derived from this     *%
%*  software without specific prior written permission.                     *%
%*                                                                          *%
%*  THIS SOFTWARE IS PROVIDED BY THE COPYRIGHT HOLDERS AND CONTRIBUTORS     *%
%*  "AS IS" AND ANY EXPRESS OR IMPLIED WARRANTIES, INCLUDING, BUT NOT       *%
%*  LIMITED TO, THE IMPLIED WARRANTIES OF MERCHANTABILITY AND FITNESS       *%
%*  FOR A PARTICULAR PURPOSE ARE DISCLAIMED. IN NO EVENT SHALL THE          *%
%*  REGENTS OR CONTRIBUTORS BE LIABLE FOR ANY DIRECT, INDIRECT,             *%
%*  INCIDENTAL, SPECIAL, EXEMPLARY, OR CONSEQUENTIAL DAMAGES (INCLUDING,    *%
%*  BUT NOT LIMITED TO, PROCUREMENT OF SUBSTITUTE GOODS OR SERVICES ;       *%
%*  LOSS OF USE, DATA, OR PROFITS ; OR BUSINESS INTERRUPTION) HOWEVER       *%
%*  CAUSED AND ON ANY THEORY OF LIABILITY, WHETHER IN CONTRACT, STRICT      *%
%*  LIABILITY, OR TORT (INCLUDING NEGLIGENCE OR OTHERWISE) ARISING IN ANY   *%
%*  WAY OUT OF THE USE OF THIS SOFTWARE, EVEN IF ADVISED OF THE             *%
%*  POSSIBILITY OF SUCH DAMAGE.                                             *%
%*                                                                          *%
%****************************************************************************%

\chapter{Multi-MA extension}
\label{ch:multiMAextension}

The hierarchical organization of DIET is efficient when the set of resources is
shared by few individuals. However, the aim of grid computing is to share
resources between several individuals. In that case, the DIET hierarchy become
inefficient. The Multi-MA extension has been implemented to resolve this
issue. This chapter explains the different scalability issues of grid computing
and how to use the multi-MA extension to deal with them.

\section{Function of the Multi-MA extension}

The use of a monolithic architecture become more and more difficult when the
number of users and the number of resources grow simultaneously. When a user
try to resolve a problem, without the multi-MA extension, DIET looks for the
better SeD that can resolve it. This search involves the fact that each SeD has
to be queried to run a performance prediction as described in
section~\ref{sec:solvepb}.

The need to query every SeD that can resolve a problem is a serious
scalability issue. To avoid it, the multi-MA extension proposes to interconnect
several MA together. So, instead of having the whole set of SeD available under
a hierarchy of a unique MA, there are several MA and each MA manages a
subset of SeD. Those MA are interconnected in a way that they can share the
access to their SeD.

Each MA works like the usual: when they received a query from a user, they
looks for the best SeD which can resolve their problem inside their
hierarchy. If there is no SeD available in its hierarchy, the queried MA forward
the query to other MA to find a SeD that can be used by its client.  This
way, DIET is able to support more clients and more servers because each client
request are forwarded to a number of SeD that is independent of the total
number of available SeD.

\section{Deployment example}

The instructions about how to compile DIET with the multi-MA extension are
available in section~\ref{sec:multimainstall} and the configuration
instructions are available in section~\ref{sec:multimaconfig}.

This example is about four organizations which wish to share there
resources. The first organization, named alpha, have ten SeD which give access
to the service \textbf{a}. The second organization, named beta, have eight SeD
with the service \textbf{a} and three with the service \textbf{b}. The third
one, named gamma, have two SeD with the service \textbf{c}.  The last one,
named delta, have one SeD with the service \textbf{a}, but the server crash and
the SeD is unavailable.

Each organization has it's own DIET hierarchy. The MA of each organization are
connected with the multi-MA extension as shown in Figure~\ref{fig:multima}


\begin{figure}[h]
 \begin{center}
   \resizebox{.7\linewidth}{!}{\includegraphics{fig/multima.eps}}
   \label{fig:multima}
  \caption{Example of a multi-MA deployment}
 \end{center}
\end{figure}

The following lines appear in the MA configuration file of alpha. They tell
that the multi-MA extension should listen for incoming connection at port
2001. They also tell that the MA should create a link toward the MA of the
organization gamma and toward the MA of the organization beta. (The description
of each configuration parameter are available in
section~\ref{sec:multimaconfig}).

\begin{verbatim}
agentType = DIET_MASTER_AGENT
dietHostname = diet.alpha.com
bindServicePort = 2001
neighbours = diet.beta.com:2001,ma.gamma.com:6000
\end{verbatim}

The following lines appear in the MA configuration file of beta:

\begin{verbatim}
agentType = DIET_MASTER_AGENT
dietHostname = diet.beta.com
bindServicePort = 2001
neighbours = diet.alpha.com:2001,ma.gamma.com:6000
\end{verbatim}

The following lines appear in the MA configuration file of gamma. The
\texttt{neighbours} value is empty. This means that the gamma's MA will not try
to connect itself to other MA. However, the three others are configured to be
connected to gamma. So, after all, the gamma MA is connected to the other
three.

\begin{verbatim}
agentType = DIET_MASTER_AGENT
dietHostname = ma.gamma.com
bindServicePort = 6000
neighbours = 
\end{verbatim}

Finally the following lines appear in the MA configuration file of delta:

\begin{verbatim}
agentType = DIET_MASTER_AGENT
dietHostname = ma.delta.com
bindServicePort = 2001
neighbours = ma.gamma.com:6000
\end{verbatim}

\section{Search examples}

The following section explained how a \texttt{diet\_call} is managed when used
on the previous architecture.

If a client sends a \texttt{diet\_call} for the problem \textbf{a} to the
alpha's MA, the alpha's MA will return a reference of one of it's SeD. However,
if its scheduler (see section~\ref{ch:plugin}) says that no SeD is available,
it will forward the request to beta and gamma. If beta has an available SeD, it
will be used to resolve the problem. If not, the request is forwarded to delta.

Now, if a client sends a \texttt{diet\_call} for the problem \textbf{c} to the
delta's MA. The delta MA does not have a SeD that can resolve this problem. So,
it forwards the request to gamma. If gamma has not available SeD, the request
is forwarded to alpha and beta.


%
% Workflow support
%
\newpage
%****************************************************************************%
%* DIET User's Manual workflow chapter file                                 *%
%*                                                                          *%
%*  Author(s):                                                              *%
%*    - Abdelkader Amar (Abdelkader.Amar@ens-lyon.fr)                       *%
%*    - Raphael Bolze (Raphael.Bolze@ens-lyon.fr)                           *%
%*                                                                          *%
%* $LICENSE$                                                                *%
%****************************************************************************%

\chapter{Workflow management in \textsc{Diet}}

\section{Quick start}


\paragraph{Requirements and compilation}

The workflow supports in \textsc{DIET} needs the following:

\begin{itemize}
\item The Xerces library: the XML handling code is written with
  Xerces-C++ using the provided DOM API.
\item Enable the workflow support when compiling DIET.
\end{itemize}

To build DIET with workflow support using \textit{cmake}, two
configuration parameters need to be set:

\begin{itemize}
\item DIET\_USE\_WORKFLOW as follow: -DDIET\_USE\_WORKFLOW:BOOL=ON
\item XERCES\_DIR: defines the path to Xerces installation directory.
  (for example -DXERCES\_DIR:PATH=/usr/local/xerces)
\end{itemize}

This is an example of generating command line:

\verb|cmake .. -DMAINTAINER_MODE:BOOL=ON -DOMNIORB4_DIR=/usr/local/omniORB \|

\verb|      -DDIET_USE_WORKFLOW:BOOL=ON \|

\verb|      -DXERCES_DIR=/usr/local/xerces|


\paragraph{}
Workflow support was tested in the following configurations:

\begin{itemize}
\item gcc version 4.0.2 and higher.
\item \textit{Xerces} 2.7.
\end{itemize}

%\subsubsection{Previous version \todo{TO REMOVE}}
%
%\paragraph{Compilation using the autotools}
%
%The workflow supports in \textsc{DIET} needs the following:
%
%\begin{itemize}
%\item The QT4 library: the XML handling code is written with
%  QT4 using the provided DOM API.
%\item enable the workflow support by passing the
%  \verb|--enable-workflow| to the configure script.
%\end{itemize}
%
%To specify the QT4 installation directory, you can use the options
%\verb|--with-qt-inc| and \verb|--with-qt-lib| which indicate
%respectively where find headers and libraries. When QT4 is installed
%by binaries (RPM or APT package), the include directory is
%\verb|/usr/include/qt4|, when the library directory is
%\verb|/usr/lib|.
%
%Workflow support was tested in the following configurations:
%
%\begin{itemize}
%\item with FAST
%\item with or without cori support
%\end{itemize}
%
%\paragraph{Compilation using CMake}
%
%To build DIET with workflow support using \textit{cmake}, three
%configuration parameters need to be set:
%
%\begin{itemize}
%\item DIET\_USE\_WORKFLOW as follow: -DDIET\_USE\_WORKFLOW:BOOL=ON
%\item QT\_INC: defines the path to QT4 header, by default set to
%  /usr/include. (for example -DQT\_INC:PATH=/usr/include/qt4)
%\item QT\_LIB: defines the path to QT4 library, by default set to
%  /usr/lib. (for example -DQT\_LIB:PATH=/usr/local/lib)
%\end{itemize}
%
%This is an example of generating command line:
%
%\verb|cmake .. -DOMNIORB4_DIR=/usr/local/omniORB -DDIET_USE_WORKFLOW:BOOL=ON \|\verb 
%\verb|-DQT_LIB:PATH=/usr/lib -DQT_INC:PATH=/usr/include/qt4/| %|

\paragraph{Executing the examples}
%\label{sec:wf_examples}

The directory \texttt{examples/workflow} includes some examples of
workflows.  You can find a simple workflow (see
figure~\ref{fig:example1}) in the file \texttt{xml/scalar.xml} and you
can test it with the following command:

\verb|./scalar_client local_client.cfg scalar.xml |

You need to have a running DIET platform with the needed services. You
can launch separate SeDs (\textit{succ}, \textit{double}, \textit{sum}
and \textit{square}) or a single SeD that includes all the needed
services.

\begin{figure}[htbp]
  \centering
  \includegraphics[keepaspectratio,width=0.4\linewidth]{fig/wf_example1}
  \caption{Workflow example}
  \label{fig:example1}
\end{figure}

\section{Software architecture}

The workflow support in \textsc{DIET} provides two architectures and
can be used in three modes.
\begin{itemize}
\item The first one provides a workflow manager in the client side.
\item The second one use a special agent called the \textit{MA\_DAG}
  which is responsible to communicate with the \textit{Master Agent} of
  the platform and provide an ordering and mapping for workflow
  execution.
\item The third mode is similar to the second one, but in this mode
  the \textit{MA\_DAG} provides only an ordering for the workflow
  execution.
\end{itemize}

\subsection{Architecture 1 : Workflow manager inside the client}
\label{sec:archi1}

In this scheme (figure~\ref{fig:archi1}), the handling of workflow is
done by the client, and the execution don't require an MA\_DAG. The
execution steps are~:

\begin{enumerate}
\item The client read and process the workflow description; a dag
  structure is created.
\item The client send a set of the problem descriptions to the
  Master Agent to check if all services are available.
\item If all services are available, the client start the workflow
  execution. By default a random scheduler is used to execute the Dag,
  so only data dependencies are used to define the execution order.
  Between the ready nodes, the invocation order is random. In this
  mode, since all scheduling operation are done in the client side, we
  can use a customized scheduler. The client programmer can write a
  personal scheduler which must be a \textit{AbstractScheduler}
  subclass and implements the \textit{execute} method (for more
  details see the section~\ref{sec:wf_sched}).
\item If the execution was successful, the client can retrieve the
  results.
\end{enumerate}

To use this execution mode, nothing needs to be configured except that
the client configuration file must not include an MADAGNAME parameter.

\begin{figure}[htbp]
  \centering
  \includegraphics[keepaspectratio,width=0.7\linewidth]{fig/wf_archi1}
  \caption{Architecture 1 : workflow manager inside the client}
  \label{fig:archi1}
\end{figure}



\subsection{Architecture 2: Workflow manager = MA DAG}
\label{sec:archi2}

The second architecture (figure~\ref{fig:archi2}) of workflow
management in DIET use an additional entity called the \textit{MA DAG}
to handle the workflow. This agent can work in two modes, in the first
it defines a complete scheduling of the workflow (ordering and
mapping), while in the second it defines only an ordering for the
workflow execution, the mapping is done in the next step by the client
which pass by the Master Agent to find the server where execute the
workflow services.


To use the \textit{MA DAG}, the client configuration file must include
the parameter \textit{MADAGNAME} with the appropriate name.

\paragraph{First mode : the MA\_DAG provides a complete scheduling}

\begin{enumerate}
\item The client sends the workflow description (in a textual format)
  to the MA\_DAG.
\item The MA\_dAG request the platform Master Agent to check if all
  services are available.
\item If the workflow can be executed, the MA\_DAG defines a mapping and
  ordering and send it back to the client. Currently, the ordering is
  random like in the first architecture, while the mapping use a round
  robin policy.
\item The client start the workflow execution and retrieve the results
  if the execution was successful.
\end{enumerate}

When the \textit{MADAGNAME} is set in the configuration file, this is
the default mode. The client can ensure this mode by the method
\texttt{set\_ma\_dag\_sched}. 

\paragraph{Second mode : the MaDag provides only an ordering}

This mode is similar to the previous one, except that the MaDag
provides only an execution order (or a priority tag for each node),
the mapping is done when the client tries to execute each service, so
the server is the one chosen by the Master Agent when requested by the
client.

To use this mode, the client needs to use to the method
\texttt{set\_ma\dag\_sched(0)}. 

\begin{figure}[htbp]
  \centering
  \includegraphics[keepaspectratio,width=0.7\linewidth]{fig/wf_archi2}
  \caption{Architecture 2 : The MA\_DAG as an external workflow manager}
  \label{fig:archi2}
\end{figure}




\section{Client API}


\subsection{Structure of client program}
\label{sec:client_prg}

The structure of a client program is very close to the structure of
usual DIET client. The general structure is as follow:

\begin{verbatim}
diet_initialize

create the workflow profile

call the method diet_wf_call

if success retrieve the results

free the workflow profile

diet_finalize
\end{verbatim}

The table~\ref{tab::wf_api} shows a description of the different
methods provided by diet workflow API. 

\begin{table}[htbp]
  \centering
  \begin{tabular}[htbp]{|p{8cm}|p{7.5cm}|}\hline
    Workflow function & Description \\\hline
    %
    \texttt{diet\_wf\_desc\_t*  \newline
      diet\_wf\_profile\_alloc(const char* wf\_file\_name);} 
    &
    allocate a workflow profile to be used for a workflow submission.\newline
    \textit{wf\_file\_name} : the file name containing the workflow XML description.
    \\\hline
    % 
    \texttt{void  \newline
      diet\_wf\_profile\_free(diet\_wf\_desc\_t * profile);} 
    &
    free the workflow profile.
    \\\hline
    %
    \texttt{diet\_error\_t \newline
      diet\_wf\_call(diet\_wf\_desc\_t* wf\_profile);} 
    &
    execute the workflow associated to profile \textit{wf\_profile}.
    \\\hline
    %
    \texttt{int   \newline
      diet\_wf\_scalar\_get(const char * id, void** value);} 
    &
    Retrieve a workflow scalar result. \newline
    \textit{id} : the output port identifier.
    \\\hline
    % 
    \texttt{int   \newline
      diet\_wf\_string\_get(const char * id, char** value);} 
    &
    Retrieve a workflow string result. \newline
    \textit{id} : the output port identifier.
    \\\hline
    %
    \texttt{int    \newline
      diet\_wf\_file\_get(const char * id, size\_t* size, char** path);
    }
    &
    Retrieve a workflow file result. \newline
    \textit{id} : the output port identifier.
    \\\hline
    % 
    \texttt{int \newline
      diet\_wf\_matrix\_get(id, (void**)value, nb\_rows, nb\_cols, order)
    }
    &
    Retrieve a workflow matrix result. \newline
    \textit{id} : the output port identifier.
    \\\hline
    % 
    \texttt{void \newline
      set\_madag\_sched(int b);}&
    Define if the client use the Ma\_Dag ordering and mapping when
    executing the workflow ($b\neq1$) or just the ordering ($b=0$)
    \\\hline
    %
    \texttt{void  \newline
      set\_sched (struct AbstractWfSched * sched);}&
    Define a customized scheduler defined by the user. The Scheduler
    must be an AbstractScheduler subclass and implements the
    \textit{execute} method.
    \\\hline
    %
    \texttt{void \newline
      enable\_reordering(const char * name, int b);
    }
    & 
    enable/disable the reordering \newline
    reordering enabled (b = true) \newline
    reordering disabled (b = false)
    \\\hline
    %
    \texttt{void \newline
      void get\_all\_results();}
    &
    print all the results of the current executed workflow.
    \\\hline
    %
  \end{tabular}
  \caption{Diet workflow API}
  \label{tab::wf_api}
\end{table}


\subsection{Workflow description}
\label{sec:workflow_desc}

The workflow is described using an XML representation which is close
to diet profile representation. In addition to profile description
(problem path and arguments), this description represents also the
data dependencies between ports (source/sink), the node identifier
(unique) and the precedences between nodes. This last information can
be removed since it can be retrieved from the dependencies between
ports, however it can be useful to define a temporal dependency
without port linking.

The general structure of this description is:

\begin{verbatim}
<dag>
  <node id="..." path="...">
    <arg name="..." type="........" value=".."/>
    <in name="..." type="........" source="......."/>
    <out name="...." type="........" sink="......"/>
    <out name="...." type="........" sink="......"/>
  </node>
  ....
\end{verbatim}

The name argument represents the identifier of the port. To use it to
define a \textit{source} or a \textit{sink} value, it must be prefixed
with the node id. For example if the source of the input port
\textit{in3} is the port \textit{out2} of the node \textit{n1}, than
the element must be described as follow:

\begin{verbatim}
    <in name="in3" type="DIET_INT" source="n1#out2"/>
\end{verbatim}

The example shown in figure~\ref{fig:example1} can be represented by
this XML description:

\begin{verbatim}
<dag>
  <node id="n1" path="succ">
    <arg name="in1" type="DIET_INT" value="56"/>
    <out name="out1" type="DIET_INT" sink="n2#in2"/>
    <out name="out2" type="DIET_INT" sink="n3#in3"/>
  </node>
  <node id="n2" path="double">
    <prec id="n1"/>
    <in name="in2" type="DIET_INT" source="n1#out1"/>
    <out name="out3" type="DIET_INT" sink="n4#in4"/>
  </node>
  <node id="n3" path="double">
    <prec id="n1"/>
    <in name="in3" type="DIET_INT" source="n1#out2"/>
    <out name="out4" type="DIET_INT" sink="n4#in5"/>
  </node>
  <node id="n4" path="sum">
    <prec id="n2"/>
    <prec id="n3"/>
    <in name="in4" type="DIET_INT" source="n2#out3"/>
    <in name="in5" type="DIET_INT" source="n3#out4"/>
    <out name="out4" type="DIET_INT"/>
  </node>
</dag>
\end{verbatim}

\subsection{Examples}
\label{sec:examples}


\subsubsection{Example 1 : the simplest example}
\label{sec:ex1}

This examples represents the basic client code to execute a workflow.
The line 26 indicates that the workflow output is a double value named
\verb|n4#out4|. The example shown in figure~\ref{fig:example1} can be
fully (execution and result retrieving) executed with this client.

\begin{lstlisting}{1}
#include <string.h>
#include <unistd.h>
#include <stdlib.h>
#include <stdio.h>
#include <sys/stat.h>

#include "DIET_client.h"

int main(int argc, char* argv[])
{
  diet_wf_desc_t * profile;
  char * fileName;
  long * l; 
  if (argc != 3) {
    fprintf(stderr, "Usage: %s <file.cfg> <wf_file> \n", argv[0]);
    return 1;
  }
  
  if (diet_initialize(argv[1], argc, argv)) {
    fprintf(stderr, "DIET initialization failed !\n");
    return 1;
  } 
  fileName = argv[2];
  profile = diet_wf_profile_alloc(fileName);
  if (!diet_wf_call(profile)) {
    printf("get result = %d ", diet_wf_scalar_get("n4#out4", &l));
    printf("%ld\n", (long)(*l)); 
  }
  diet_wf_free(profile);
  return 0;
}
\end{lstlisting}

\paragraph{Example 2 :  Use the MA DAG modes}

This example is similar to the previous one but the user can specify
which mode he wants to use by command line options \verb|-madag_sched|
(the default) or \verb|-notmadag_sched|.

\begin{lstlisting}{2}
#include <string.h>
#include <unistd.h>
#include <stdlib.h>
#include <stdio.h>
#include <sys/stat.h>
#include "DIET_client.h"

void usage(char * s) {
  fprintf(stderr, "Usage: %s <file.cfg> <wf_file> [option]\n", s);
  fprintf(stderr, "option = -madag_sched | -notmadag_sched\n");
  exit(1);
}
int checkUsage(int argc, char ** argv) {
  if ((argc != 3) && (argc != 4)) {
    usage(argv[0]);
  }
  if (argc == 4) {
    if (strcmp(argv[3], "-madag_sched") && 
	strcmp(argv[3], "-notmadag_sched")) {
      usage(argv[0]);
    }
  }
  return 0;
}

int main(int argc, char* argv[])
{
  diet_wf_desc_t * profile;
  char * fileName;
  long * l; 
  checkUsage(argc, argv);
  if (diet_initialize(argv[1], argc, argv)) {
    fprintf(stderr, "DIET initialization failed !\n");
    return 1;
  } 
  fileName = argv[2];
  if (argc == 4)
    set_madag_sched(!strcmp(argv[3], "-madag_sched"));
  profile = diet_wf_profile_alloc(fileName);
  if (!diet_wf_call(profile)) {
    printf("get result = %d ", diet_wf_scalar_get("n4#out4", &l));
    printf("%ld\n", (long)(*l)); 
  }
  diet_wf_free(profile);
  return 0;
}
\end{lstlisting}


\section{Scheduling}


\subsection{Available schedulers}

The available schedulers are:

\begin{itemize}
\item In the client: a round robbin and HEFT scheduler.
\item In the MA\_DAG: a round robbin, HEFT, and FOFT (Fairness On
  Finish Time: in multiworkflow support -experimental-).
\end{itemize}

\subsection{Writing a new scheduler}

The workflow scheduler of the client is different little bit from the
MA\_DAG one. To write a new scheduler you need diet sources. The
following two section show how to develop your own scheduler and how
to plug it in your client or in the MA\_DAG.

\paragraph{Client scheduler}

To write a new workflow scheduler for your diet client you need to
create a derived class of the base abstract class
\textit{AbstractWfSched}. The main function of the scheduler (the SeDs
mapping) must be placed in the pure virtual function \textit{execute}.
These are the steps to follow:

\begin{enumerate}
\item Write a subclass of \textit{AbsWfSched} abstract class. The
  virtual method \textit{execute} needs to be implemented.
\item Create a c++ file where you can put the following code:
\begin{verbatim}
extern "C" {
  void set_my_personal_sched() {
    Personal_WfSched * mySched = new Personal_WfSched();
    set_sched(mySched);
  }
}
\end{verbatim}
\item In your client source code, call the method
  \textit{set\_my\_personal\_sched} before executing the
  \textit{diet\_wf\_call}.
\item Link the previous c++ object code to you DIET client.
\end{enumerate}



%
% DAGDA
%
\newpage
\chapter{DAGDA extension}
\textsc{DAGDA} (\textbf{D}ata \textbf{A}rrangement for \textbf{G}rid and
\textbf{D}istributed \textbf{A}pplications) is a new data manager for DIET.
DAGDA offers  to the DIET application developers a simple and efficient way 
to manage the data. It was not designed to replace the JuxMem extension
but to be possibly coupled with it. In a future work, DAGDA will be
divided in two parts: The DAGDA data manager and the DAGDA data interface.
The data interface will make interactions between DAGDA, JuxMem, FTP
etc. and other data transfer/management protocols. In this chapter, we will
present the current version of DAGDA which is an alternative data manager
for DIET with several advanced data management features.
\section{Overview}
DAGDA allows data explicit or
implicit replications and advanced data management on the grid. It was
designed to be backward compatible with previously developed
applications for DIET which benefit transparently of the data
replications. Moreover, DAGDA limits the data size loaded in memory to
a user-fixed value and avoids CORBA errors when transmitting too large
data regarding to the ORB configuration.

DAGDA offers a new way to manage the data on DIET. The API allows the
application developer to replicate, move, add or delete a data to be
reused later or by another application. Each component of DIET can
interact with DAGDA and the data manipulation can be done from a
client application, a server or an agent through a plug-in
scheduler.

A DAGDA component is associated to each element in a DIET platform
(client, Master Agent, Local Agent, SeD). These components are connected
following the DIET deployment topology. Figure \ref{fig:DAGDAarch} shows how
the DAGDA and DIET classical components are connected. In contrary of a DIET
architecture, each DAGDA component has the same role. It can store, transfer
or move a data. The client's DAGDA component is isolated of the architecture and
communicates only with the chosen SeDs DAGDA components when necessary. When
searching for a data, DAGDA uses its hierarchical topology to contact the
data managers. Among the data managers having one replicate of the data,
DAGDA chooses the \textit{"best"} source to transfer it. To make this choice
DAGDA uses some statistics collected from previous data transfers between
the nodes. By not using dynamic information, it is unsure that DAGDA really
chose the "best" nodes for the transfers. In a future version, we will
introduce some facilities to estimate the time needed to transfer a data and
to improve the choice of a data stored on the grid. To do the data transfers,
DAGDA uses the pull model: It is the destination node that ask for the data
transfer.
\begin{figure}[h]
\centerline{\includegraphics[width=0.7\linewidth]{fig/dagdaArch}}
\caption{DAGDA architecture in DIET.\label{fig:DAGDAarch}}
\end{figure}

Figure \ref{fig:DAGDAarch} presents how DAGDA manages the data when a client
submit a job. In this example, the client wants to use some data stored on the
grid and some personal data. He wants to obtain some results and to store
some others on the grid. Some of these output data are already stored on the
platform and they should be updated after the job execution.
\begin{enumerate}
  \item The client sends a request to the Master Agent.
  \item The Master agent returns one or more SeD references.
  \item The client sends its request to the chosen node. The parameters data
    are identified by a unique ID and the problem profile contains a reference
    to the client's data manager.
  \item Receiving the request the SeD asks the client to transfer the data of
    the user and it asks to the DAGDA architecture to obtain the persistent
    data already stored on the platform.
  \item The SeD executes the job. After the execution, the SeD stores the
    output data and it informs the client that the data are ready to be
    downloaded. It also asks to the architecture to update the modified output
    data.
  \item The client upload its results and the data are updated on the nodes.
\end{enumerate}

%Using DAGDA, each data stored on the platform can be replicated as many
%time as there is free space on the platform. To avoid DIET to take up all
%the storage ressources (disk or memory) of the nodes, DAGDA allows to fix
%a limit to the data size that can be stored for each DAGDA component. When
%a job needs a data, it asks DAGDA for it using its unique ID and gets it
%from the "best source". DAGDA uses some statistic informations to determine
%which node should be used for the transfers, but could be easily extended to
%use some dynamic metrology parameters. DAGDA also offers a way to manage
%automatically the data by choosing a \textit{cache replacement algorithm}
%for each node. The user can choose a persistence mode for each replica:
%\begin{itemize}
%  \item[-] Persistent: The data is kept as long as possible on the node
%    but can be deleted when the node needs some storage resources.
%  \item[-] Sticky: The data will stay on the node until a user chooses
%    to delete it explicitly.
%\end{itemize}
%Then, the three cache replacement algorithms implemented in DAGDA work as
%follows:
%\begin{itemize}
%  \item \emph{Least Recently Used} (LRU): The least recently used persistent
%    data of sufficient size is deleted.
%  \item \emph{Least Frequently Used} (LFU): The least frequently used
%    persistent data of sufficient size is deleted.
%  \item \emph{First In First Out} (FIFO): Among the persistent data of
%    sufficient size, the \textit{oldest} is deleted.
%\end{itemize}
%None of the above algorithms controls if another replica of the data
%exists on the platform before to delete it. The only way to ensure
%at least one replica of a data leaves on the platform is to declare
%one of them as \emph{sticky} data.

%Extending the DIET API, DAGDA allows to explicitly manage the data in the
%platform by communicating directly with the DAGDA hierarchy. However, DAGDA
%does not deal with the replicas consistency and the data availability. The
%user is responsible of what he does with the data. Even if DAGDA offers
%a way to update a persistent data, if a job or a user modifies the data
%while the updating process is executing, some side effects can appear.

\newcommand{\tabCell}[2]{%
  \begin{minipage}{#1}
    \vspace*{1mm}
    \scriptsize #2
    \vspace*{1mm}
  \end{minipage}
}
\section{The DAGDA configuration options}
DAGDA introduces new configuration options that can be defined for all the
DAGDA components. None of these options are mandatory to use DAGDA. Figure
\ref{fig:DAGDAoptions} presents all the DAGDA available options, their meaning
and default values.
\begin{figure}[h]
\begin{tabular}{|l|l|l|c|c|c|}
\hline
%& & & & & \\
\tabCell{2.9cm}{\vspace*{0.5cm}\centering\textbf{Option}} &
\tabCell{5cm}{\vspace*{0.5cm}\centering\textbf{Description}} &
\tabCell{4cm}{\vspace*{0.5cm}\centering\textbf{Default value}} &
\rotatebox{270}{\centering\bf Client} & \rotatebox{270}{\centering\bf Agent } &
\rotatebox{270}{\centering\bf SeD} \\
%& & & & &\\
\hline
%& & & & &\\
storageDirectory &
\tabCell{5cm}{The directory on which DAGDA will store the data files} &
\tabCell{4cm}{The \textit{/tmp} directory.} &
\ding{52} & \ding{52} & \ding{52} \\
%& & & & &\\
\hline
maxMsgSize &
\tabCell{5cm}{The maximum size of a CORBA message sent by DAGDA.} &
\tabCell{4cm}{The omniORB \textit{giopMaxMsgSize} size.} &
\ding{52} & \ding{52} & \ding{52} \\
%& & & & &\\
\hline
maxDiskSpace &
\tabCell{5cm}{The maximum disk space used by DAGDA to store the data. If set
to 0, DAGDA will not take care of the disk usage.} &
\tabCell{4cm}{The available disk space on the disk partition chosen by the
  \textit{storageDirectory} option.} &
\ding{52} & \ding{52} & \ding{52} \\
\hline
maxMemSpace &
\tabCell{5cm}{The maximum memory space used by DAGDA to store the data. If set
to 0, DAGDA will not take care of the memory usage.} &
\tabCell{4cm}{No maximum memory usage is set. Same effect than to choose 0.} &
\ding{52} & \ding{52} & \ding{52} \\
\hline
cacheAlgorithm &
\tabCell{5cm}{The cache replacement algorithm used when DAGDA needs more space
to store a data. Possible values are: \textit{LRU, LFU, FIFO}} &
\tabCell{4cm}{No cache replacement algorithm. DAGDA never replace a data by
another one.} &
\ding{52} & \ding{52} & \ding{52} \\
\hline
shareFiles &
\tabCell{5cm}{The DAGDA component shares its file data with all its children
(when the path is accessible by them, for example, if the storage directory is
on a NFS partition). Value can be 0 or 1.} &
\tabCell{4cm}{No file sharing - 0} &
\ding{56} & \ding{52} & \ding{56} \\
\hline
dataBackupFile &
\tabCell{5cm}{The path to the file that will be used when DAGDA save all its
stored data/data path when asked by the user (Checkpointing).} &
\tabCell{4cm}{No checkpointing is possible.} &
\ding{56} & \ding{52} & \ding{52} \\
\hline
restoreOnStart &
\tabCell{5cm}{DAGDA will load the \textit{dataBackupFile} file at start and
restore all the data recorded at the last checkpointing event. Possible values
are 0 or 1.} &
\tabCell{4cm}{No file loading on start - 0} &
\ding{56} & \ding{52} & \ding{52} \\
\hline
\end{tabular}
\caption{DAGDA configuration options}
\label{fig:DAGDAoptions}
\end{figure}

\section{Cache replacement algorithm}
When a data is replicated on a site, it is possible that not enough
disk/memory space is available. In that case, DAGDA allows to choose
a strategy to delete a persistent data. Only a simple persistent data
can be deleted, the sticky ones are never deleted by the chosen
algorithm. DAGDA offers three algorithm to manage the cache replacement:
\begin{itemize}
  \item LRU: The least recently used persistent data of sufficient size is
    deleted.
  \item LFU: The least frequently used persistent data of sufficient size
    is deleted.
  \item FIFO:  Among the persistent data of sufficient size, the
    \textit{oldest} is deleted.
\end{itemize}

\section{The DAGDA API}
By compiling DIET with the DAGDA extension activated, the
\textit{DIET\_Dagda.h} file is installed on the DIET include directory.
This file contains some data management functions and macros.
\subsection{Note on the memory management}
On the SeD side, DAGDA and the SeD share the
same data pointers, that means that if the pointer is a local variable
reference, when DAGDA will use the data, it will read an unallocated variable.
The users should allways allocate the data with a \textit{"malloc"/"calloc"} or
\textit{"new"} call on the SeD and agent sides. Because DAGDA takes the control
of the data pointer, there is no risk of memory leak even if the service
allocate a new pointer at each call. The data lifetime is managed by DAGDA
and the data will be freed according to its persistence mode.\\[4mm]
\begin{minipage}{2cm}
  \centering
  \textbf{{\Huge \Biohazard}}
\end{minipage}
\begin{minipage}{\textwidth - 2cm}
\textbf{On the SeD and agent sides, DAGDA takes the control of the data
pointers. To free a data may cause major bugs which could be very hard to
find. The users could only free a DIET data on the client side after the end of
a transfer.}
\end{minipage}
\subsection{Synchronous data transfers}
All of the following functions returns at the end of the transfer or
if an error occured. They all returns an integer value: 0 if the operation
succeed, another value if it failed.
\subsubsection{DAGDA \textit{put} data macros/functions.}
\label{sec:syncPutFunctions}
The following functions put a data on the DAGDA hierarchy to be used later.
The last parameter is allways a pointer to a C-string which will be
initialized with a pointer to the ID string of the data. This string is
allocated by DAGDA and can be freed when the user does not need it anymore.
The first parameter is allways a pointer to the data: For a scalar value
a pointer on the data, for a vector, matrix or string, a pointer on the
first element of the data. The \textit{"value"} argument for a file is
a C-string containing the path of this file. The persistence mode for
a data managed by DAGDA should allways be DIET\_PERSISTENT or DIET\_STICKY.
The VOLATILE and *\_RETURN modes do not make sense in this data management
context.

% \begin{itemize}
%   \item Synchronous: The called function returns when the transfer is ended.
%   \item Asynchronous with control: The called function returns immediately.
%     The user can wait the end of the transfer by calling to a DAGDA wait
%     function.
%   \item Asynchronous without control: The called function returns immediately.
%     The user cannot wait the end of the transfer. These functions should only
%     be used on the SeD or Agent sides.
% \end{itemize}
% \subsubsection{Synchronous data transfers.}
\begin{itemize}
  \item[-] \verb#dagda_put_scalar(void* value, diet_base_type_t base_type,#\\
           \verb#                 diet_persistence_mode_t mode, char** ID)#:\\
           This macro adds to the platform, the scalar data of type
           \textit{"base\_type"} pointed by \textit{"value"} with the
           persistence mode \textit{"mode"} (DIET\_PERSISTENT or DIET\_STICKY)
           and initializes \textit{"*ID"} with the ID of the data.
  \item[-] \verb#dagda_put_vector(void* value, diet_base_type_t base_type,#\\
           \verb#                 diet_persistent_mode_t mode, size_t size, char** ID)#:\\
           This macro adds to the platform, the vector of \textit{"size"}
           \textit{"base\_type"} elements pointed by \textit{"value"} with the
           persistence mode \textit{"mode"} and stores the data ID in
           \textit{"ID"}.
  \item[-] \verb#dagda_put_matrix(void* value, diet_base_type_t base_type,#\\
         \verb#                 diet_persistence_mode_t mode, size_t nb_rows,#\\
         \verb#                 size_t nb_cols, diet_matrix_order_t order, char** ID)#:\\
           This macro adds to the platform the \textit{"base\_type"} matrix of
           dimension \textit{"nb\_rows"} $\times$ \textit{"nb\_cols"} stored in
           \textit{"order"} order. The data ID is stored on \textit{"ID"}.
  \item[-] \verb#dagda_put_string(char* value, diet_persistence_mode_t mode, char** ID)#:\\
           This macro adds to the platform the string pointed by
           \textit{"value"} with the persistence mode \textit{"mode"} and
           stores the data ID into \textit{"ID"}.
  \item[-] \verb#dagda_put_file(char* path, diet_persistence_mode_t mode, char**ID)#:\\
           This macro adds the file of path \textit{"path"} with the persistence
           mode \textit{"mode"} to the platform and stores the data ID into
           \textit{"ID"}
 \end{itemize}

\subsubsection{DAGDA \textit{get} data macros/functions}
\label{sec:syncGetFunctions}
The following API functions are defined to obtain a data from DAGDA using
its ID:
\begin{itemize}
  \item[-] \verb#dagda_get_scalar(char* ID, void** value,#\\
           \verb#                 diet_base_type_t* base_type)#:\\
    The scalar value using the ID \textit{"ID"} is obtained from DAGDA and the
    \textit{"value"} argument is initialized with a pointer to the data.
    The \textit{"base\_type"} pointer content is set to the data base type.
    This last parameter is optionnal and can be set to NULL if the user does not
    want to get the \textit{"base\_type"} value.
  \item[-] \verb#dagda_get_vector(char* ID, void** value,#\\
           \verb#                 diet_base_type_t* base_type, size_t* size)#:\\
    The vector using the ID \textit{"ID"} is obtained from DAGDA. The
    \textit{"value"} argument is initialized with a pointer to the first
    vector element. The \textit{"base\_type"} content are initialized with
    the base type and size of the vector. These two parameters can be set to
    NULL if the user does not take care about it.
  \item[-] \verb#dagda_get_matrix(char* ID, void** value,#\\
           \verb#                 diet_base_type_t* base_type, size_t* nb_r,#\\
           \verb#                 size_t* nb_c, diet_matrix_order_t* order)#:\\
    The matrix using the ID \textit{"ID"} is obtained from DAGDA. The
    \textit{"value"} argument is initialized with a pointer to the first
    matrix element. The \textit{"base\_type"}, \textit{"nb\_r"},
    \textit{"nb\_c"} and \textit{"order"} arguments contents are repectively set
    to the base type of the matrix, the number of rows, the number of columns
    and the matrix order. All of these parameters can be set to NULL if the
    user does not take care about it.
  \item[-] \verb#dagda_get_string(char* ID, char** value)#:\\
    The string of ID \textit{"ID"} is obtained from DAGDA and the
    \textit{value} content is set to a pointer on the first string character.
  \item[-] \verb#dagda_get_file(char* ID, char** path)#:\\
    The file of ID \textit{"ID"} is obtained from DAGDA and the
    \textit{"path"} content is set to a pointer on the first path string
    character.
\end{itemize}

\subsection{Asynchronous data transfers.}
With DAGDA, there is two way to manage the asynchronous data transfers,
depending of the data usage:
\begin{itemize}
  \item With end-of-transfer control: DAGDA maintains a reference to the
    transfer thread. It only release this reference after a call to the
    corresponding waiting function. The client developer should allways use
    these functions, that's why a data ID is only returned by the
    \textit{"dagda\_wait\_*"} and \textit{"dagda\_wait\_data\_ID"} functions.
  \item Without end-of-transfer control: The data is loaded from/to the
    DAGDA hierarchy without the possibility to wait for the end of the transfer.
    These functions should only be called from an agent plugin scheduler, a
    SeD plugin scheduler or a SeD if the data transfer without usage of the
    data is one of the objectives of the called service. The data adding
    functions without control should be used very carefully because there is
    no way to be sure the data transfer is achieved or even started.    
\end{itemize}
With asynchronous transfers, the user should take care of the data lifetime
because DAGDA does not duplicate the data pointed by the passed pointer.
For example, if the program uses a local variable reference to add a data
to the DAGDA hierarchy and go out of the variable scope, a crash could
occured because the data pointer could be freed by the system before DAGDA has
finished to read it.
\subsubsection{DAGDA asynchronous \textit{put} macros/functions}
The arguments to these functions are the same than for the synchronous ones.
See Section \ref{sec:syncPutFunctions} for more details. All of these functions
return a reference to the data transfer which is an unsigned int. This value
will be passed to the \textit{"dagda\_wait\_data\_ID"} function.
\begin{itemize}
\item[-] \verb#dagda_put_scalar_async(void* value, diet_base_type_t base_type,#\\
         \verb#                       diet_persistence_mode_t mode)#
\item[-] \verb#dagda_put_vector_async(void* value, diet_base_type_t base_type,#\\
         \verb#                       diet_persistence_mode_t mode, size_t size)#
\item[-] \verb#dagda_put_matrix_async(void* value, diet_base_type_t base_type,#\\
         \verb#                       diet_persistence_mode_t mode, size_t nb_rows,#\\
         \verb#                       size_t nb_cols, diet_matrix_order_t order)#
\item[-] \verb#dagda_put_string_async(char* value, diet_persistence_mode_t mode)#
\item[-] \verb#dagda_put_file_async(char* path, diet_persistence_mode_t mode)#
\end{itemize}
After calling to one of these functions, the user can obtain the data ID by
calling to the \textit{"dagda\_wait\_data\_ID"} function by using a transfer
reference.
\begin{itemize}
  \item[-] \verb#dagda_wait_data_ID(unsigned int transferRef, char** ID)#:\\
    The \textit{"transferRef"} argument is the value returned by a
    \textit{"dagda\_put\_*\_async"} function. The \textit{"ID"} content will
    be initialized to a pointer on the data ID.
\end{itemize}

\subsubsection{DAGDA asynchronous \textit{get} macros/functions}
The only argument needed for one of these functions is the data ID.
All of these functions return a reference to the data transfer which is an
unsigned int. This value will be passed to the corresponding
\textit{"dagda\_wait\_*"} functions described later.
\begin{itemize}
\item[-] \verb#dagda_get_scalar_async(char* ID)#
\item[-] \verb#dagda_get_vector_async(char* ID)#
\item[-] \verb#dagda_get_matrix_async(char* ID)#
\item[-] \verb#dagda_get_string_async(char* ID)#
\item[-] \verb#dagda_get_file_async(char* ID)#
\end{itemize}

After asking for an asynchronous transfer, the user has to wait for the end
of it by calling the corresponding \textit{"dagda\_wait\_*"} function.
The arguments to these functions are the same than for the synchronous
\textit{"dagda\_get\_*"} functions. See Section \ref{sec:syncGetFunctions}
for more details.

\begin{itemize}
\item[-] \verb#dagda_wait_scalar(unsigned int transferRef, void** value,#\\
         \verb#                  diet_base_type_t* base_type)#
\item[-] \verb#dagda_wait_vector(unsigned int transferRef, void** value,#\\
         \verb#                  diet_base_type_t* base_type, size_t* size)#
\item[-] \verb#dagda_wait_matrix(unsigned int transferRef, void** value,#\\
         \verb#                  diet_base_type_t* base_type, size_t* nb_r,#\\
         \verb#                  size_t* nb_c, diet_matrix_order_t* order)#
\item[-] \verb#dagda_wait_string(unsigned int transferRef, char** value)#
\item[-] \verb#dagda_wait_file(unsigned int transferRef, char** path)#
\end{itemize}

It is frequent that a plugin scheduler developer wants to make an asynchronous
data transfer to the local DIET node. In that case, to wait the end of the
transfers before to return can be a problem. But with the previously defined
functions, DAGDA maintains a reference to the transfer thread which will be
released after a call to the waiting function. To avoid DAGDA to keep
infinitely these references, the user should call the \textit{"dagda\_load\_*"}
functions instead of the \textit{"dagda\_get\_*\_async"} ones.

\begin{itemize}
\item[-] \verb#dagda_load_scalar(char* ID)#
\item[-] \verb#dagda_load_vector(char* ID)#
\item[-] \verb#dagda_load_matrix(char* ID)#
\item[-] \verb#dagda_load_string(char* ID)#
\item[-] \verb#dagda_load_file(char* ID)#
\end{itemize}

\subsection{Data checkpointing with DAGDA}
DAGDA allows the SeD administrator to choose a file where DAGDA will store
all the data that it manages. When a SeD has a configured valid path name to a
backup file (\textit{"dataBackupFile"} option in the configuration file),
a client can ask to the agents or SeDs DAGDA components to save the data.\\

The \verb#dagda_save_platform()# function, which can only be
called from a client, records all the data managed by the agents or SeDs DAGDA
components that allow it.\\
Then, the \textit{"restoreOnStart"} configuration file option asks to the
DAGDA component to restore the data stored on the \textit{"dataBackupFile"}
file when the component starts. This mechanism allows to stop the DIET
platform for a while and to restart it conserving the same data distribution.

\subsection{Create data ID aliases.}
For many applications using large sets of data shared by several users, to
use an automatically generated ID to retrieve a data is impossible or difficult.
DAGDA allows the user to define data aliases, using human readable and
expressive strings to retrieve a data ID. Two functions are defined to do it:
\begin{itemize}
\item[-] \verb#dagda_data_alias(const char* id, const char* alias)#:\\
  Tries to associate \textit{"alias"} to \textit{"id"}. If the alias is
  already defined, returns a non zero value. A data can have several aliases
  but an alias is allways associated to only one data.
\item[-] \verb#dagda_id_from_alias(const char* alias, char** id)#:\\
  This function tries to retrieve the data id associated to the alias.
\end{itemize}

\subsection{Data replication}
After a data has been added to the DAGDA hierarchy, the users can choose to
replicate it explicitely on one or several DIET nodes. With the current
DAGDA version, we allow to choose the nodes where the data will be replicated
by hostname or DAGDA component ID. In future developments, it will be
possible to select the nodes differently. To maintain backward compatibility,
the replication function uses a C-string to define the replication rule.
\begin{itemize}
\item[-] \verb#dagda_replicate_data(const char* id, const char* rule)#
\end{itemize}
The replication rule is defined as follows:\\
"$<$Pattern target$>$:$<$identification pattern$>$:$<$Capacity overflow
behavior$>$"\\
\begin{itemize}
\item The \textit{pattern target} can be "ID" or "host".
\item The \textit{identification pattern} can contain some \textit{wildcards}
  characters. (for example \textit{"*.lyon.grid5000.fr"} is a valid pattern.
\item The \textit{capacity overflow behavior} can be "replace" or "noreplace".
  "replace" means the cache replacement algorithm will be used if available on
  the target node (a data could be deleted from the node to leave space to
  store the new one). "noreplace" means that the data will be replicated on
  the node if and only if there is enough storage capacity on it.
\end{itemize}

For example, \textit{"host:capricorne-*.lyon.*:replace"} is a valid replication
rule.

\section{Future works}
The next version of DAGDA will allow the users to develop their own cache
replacement algorithms and network capacity measurements methods.
DAGDA will be separated in two parts: A data management interface and a the
DAGDA data manager itself. DAGDA will implement the GridRPC data management API
extension.

%
% Dynamic hierarchy
%
\newpage
%****************************************************************************%
%* DIET User's Manual: Dynamic Management                                   *%
%*                                                                          *%
%*  Author(s):                                                              *%
%*    - Benjamin Depardon                                                   *%
%*                                                                          *%
%* $LICENSE$                                                                *%
%****************************************************************************%
%* $Id$
%* $Log$
%* Revision 1.4  2010/03/29 21:53:25  ecaron
%* Corrections
%*
%* Revision 1.3  2010/02/25 07:21:15  ycaniou
%* Add log infos at beginning of document
%* Ispelled
%*
%****************************************************************************%

\chapter{Dynamic management}
\label{ch:dynamic}

\section{Dynamically modifying the hierarchy}

\subsection{Motivations}

So far we saw that \diet's hierarchy was mainly static: once the shape
of the hierarchy chosen, and the hierarchy deployed, the only thing
you can do is kill part of the hierarchy, or add new subtrees to the
existing hierarchy. But whenever an agent is killed, the whole
underlying hierarchy is lost. This has several drawbacks: some \sed
will become unavailable, and if you want to reuse the machines on
which those \sed (or agents) are, you need to kill the existing \diet
element, and redeploy a new subtree. Another problem due to this
static asignement of the parent/children links is that if you have an
agent that is overloaded, you cannot move part of its children to an
underloaded agent somewhere else in the hierarchy without once again
killing part of the hierarchy, and deploying once again.


\subsection{``And thus it began to evolve''}

Hence, \diet also has a mode in which you can dynamically modify its
shape using CORBA calls. For this, you need to compile \diet with the
option \texttt{DIET\_USE\_DYNAMICS}. In this mode, if a \diet element
cannot reach its parent, when initializing, it won't exit, but will
wait for an order to connect itself to a new parent. Hence, you do not
need to deploy \diet starting from the MA down to the \sed, you can
launch all the elements at once, and then, send the orders for each
element to connect to its correct parent (you do not even need to
follow the shape of the tree, you can start from the bottom to the
tree up to the root, or use a random order, the service tables will be
correctly initialized.)

You now have access to the following CORBA methods:
\begin{itemize}
\item \verb|long bindParent(in string parentName)|: sends an order to
  a \sed or agent to bind to a new parent having the name
  ``\verb|parentName|'' if this parent can be contacted, otherwise the
  element keeps its old parent. If the element already has a parent,
  it unsubscribes itself from the parent, so that this latter is able
  to update its service table and list of children. A \texttt{null} value is
  returned if the change occurred, otherwise a value different from 0
  is returned if a problem occurred.

\item \verb|long disconnect()|: sends an order to disconnect an
  element from its parent. This does not kill the element, but merely
  removes the link between the element and its parent. Thus, the
  underlying hierarchy will be unreachable until the element is
  connected to a new parent.

\item \verb|long removeElement()|: sends an order to a \sed to kill
  itself. The \sed first unsubscribe from its parent before ending
  itself properly.

\item \verb|long removeElement(in boolean recursive)|: same as above
  but for agents. The parameter ``\verb|recursive|'' if true also
  destroys the underlying hierarchy, otherwise only the agent is
  killed.
\end{itemize}

Now, what happens if during a request submission an element receives
an order to change its parent? Actually, nothing will change, as
whenever a request is received a reference to the parent from which
the request originates is locally kept. So if the parent changes
before the request is sent back to the parent, as we keep a local
reference on the parent, the request will be sent back to the correct
``parent''. Hence, for a short period of time, an element can have
multiple parents.

\textbf{WARNING: currently no control is done on whether or not you
  are creating loops in the hierarchy when changing a parent.}

\subsection{Example}

Two examples on how to call those CORBA methods are present in\newline
\texttt{src/examples/dynamic\_hierarchy}:
\begin{itemize}
\item \texttt{connect.cc} sends orders to change the parent of an element.\newline
%
    \texttt{Usage: ./connect <SED|LA> <element name> <parent name>}.

\item \texttt{disconnect.cc} sends orders to disconnect an element
  from its parent. It does not kill the element, but only disconnects
  it from the \diet hierarchy (useful when your platform is not
  heavily loaded and you want to use only part of the
  hierarchy)\newline
%
  \texttt{Usage: ./disconnect <SED|LA> <element name>}.

\item \texttt{remove.cc} sends orders to remove an element.\newline
%
    \texttt{Usage: ./remove <SED|AGENT> <element name> [recursive: 0|1]}
\end{itemize}


\section{Changing offered services}

\subsection{Presentation}
A \sed does not necessarily need to declare all its services
initially, \ie as presented in Chapter~\ref{ch:server} before
launching the \sed via \verb|diet_SeD(...)|. One could want to
initially declare a given set of services, and then, depending on
parameters, or external events, one could want to modify this set of
services. An example of such usage is to spawn a service that is in
charge of cleaning temporary files when they won't be needed nor by
this \sed, nor by any other \sed or clients, and when this service is
called, it cleans whatever needs to be cleaned, and then this service
is removed from the service table.

Adding a service has already been introduced in
Chapter~\ref{ch:server}: using \verb|diet_service_table_add(...)| you
can easily add a new service (be it before running the \sed or within
a service). Well, removing a service is as easy, you only need to call
one of these methods: {\footnotesize
\begin{verbatim}
int diet_service_table_remove(const diet_profile_t* const profile);
int diet_service_table_remove_desc(const diet_profile_desc_t* const profile);
\end{verbatim}
}

So basically, when you want to remove the service that is called, you
only need to pass the \verb|diet_profile_t| you receive in the solve
function to \verb|diet_service_table_remove|. If you want to remove
another service, you need to build its profile description (just as if
you wanted to create a new service), and pass it to
\verb|diet_service_table_remove_desc|.


\subsection{Example}

The following example (present in \texttt{src/examples/dyn\_add\_rem})
initially declares one service. This service receives an integer $n$
as parameter. It creates $n$ services, and removes the service that
has just been called. Hence a service can only be called once, but it
spawns $n$ new services.

{\footnotesize
\begin{verbatim}
#include <iostream>
#include <sstream>
#include <cstring>

#include "DIET_server.h"
#include "DIET_Dagda.h"

/* begin function prototypes*/
int service(diet_profile_t *pb);
int add_service(const char* service_name);
/* end function prototypes*/

static unsigned int NB = 1;

template <typename T>
std::string toString( T t ) {
    std::ostringstream oss;
    oss << t;
    return oss.str();
}

/* Solve Function */
int
service(diet_profile_t* pb) {
  int *nb;

  if (pb->pb_name)
    std::cout << "## Executing " << pb->pb_name << std::endl;
  else {
    std::cout << "## ERROR: No name for the service" << std::endl;
    return -1;
  }

  diet_scalar_get(diet_parameter(pb,0), &nb, NULL);
  std::cout << "## Will create " << *nb << " services." << std::endl;

  for (int i = 0; i < *nb; i++) {
    add_service(std::string("dyn_add_rem_" + toString(NB++)).c_str());
  }

  std::cout << "## Services added" << std::endl;
  diet_print_service_table();

  /* Removing */
  std::cout << "## Removing service " << pb->pb_name << std::endl;
#ifdef HAVE_ALT_BATCH
  pb->parallel_flag = 1;
#endif
  diet_service_table_remove(pb);
  std::cout << "## Service removed" << std::endl;

  /* Print service table */
  diet_print_service_table();

  return 0;
}

/* usage function */
int
usage(char* cmd) {
  std::cerr << "Usage: " << cmd << " <SeD.cfg>" << std::endl;
  return -1;
}

/* add_service function: declares SeD's service */
int
add_service(const char* service_name) {
  diet_profile_desc_t* profile = NULL;
  unsigned int pos = 0;

  /* Set profile parameters: */
  profile = diet_profile_desc_alloc(strdup(service_name),0,0,0);

  diet_generic_desc_set(diet_param_desc(profile,pos++),DIET_SCALAR, DIET_INT);

  /* Add service to the service table */
  if (diet_service_table_add(profile, NULL, service )) return 1;

  /* Free the profile, since it was deep copied */
  diet_profile_desc_free(profile);

  std::cout << "Service '" << service_name << "' added!" << std::endl;

  return 0;
}

int checkUsage(int argc, char ** argv) {
  if (argc != 2) {
    usage(argv[0]);
    exit(1);
  }
  return 0;
}

/* MAIN */
int
main( int argc, char* argv[]) {
  int res;
  std::string service_name = "dyn_add_rem_0";

  checkUsage(argc, argv);

  /* Add service */
  diet_service_table_init(1);
  add_service(service_name.c_str());

  /* Print service table and launch daemon */
  diet_print_service_table();
  res = diet_SeD(argv[1],argc,argv);
  return res;
}
\end{verbatim}
}


\subsection{Going further}

Finally, another example is provided in
\texttt{src/examples/dynamicServiceMgr} showing how to dynamically
load and unload libraries containing services. Hence, a client can
send a library to as server, and for as long as the library is
compiled for the right architecture, the server will be able to load
it, and instanciate the service present in the library. The service
can further be called by other clients, and whenever it is not
required anymore, it can be easily removed.

%%% Local Variables:
%%% mode: latex
%%% ispell-local-dictionary: "american"
%%% mode: flyspell
%%% fill-column: 79
%%% End:


%
% DIET forwarders
%
\newpage
%****************************************************************************%
%* DIET User's Manual: DIET forwarders                                                *%
%*                                                                          *%
%*  Author(s):                                                              *%
%*    - Gaël Le Mahec                                                       *%
%*                                                                          *%
%* $LICENSE$                                                                *%
%****************************************************************************%
%* $Id$
%* $Log$
%* Revision 1.3  2010/10/25 08:30:05  bdepardo
%* Explanation on -C option which is mandatory to create the connection.
%*
%* Revision 1.2  2010/09/10 13:05:29  bdepardo
%* Typos
%*
%* Revision 1.1  2010/09/10 11:47:21  glemahec
%* The missing file...
%*
%****************************************************************************%
\chapter{\diet forwarders}
\label{ch:forwarders}
The \diet middleware uses CORBA as its communication layer. It is an easy
and flexible way for the different platform components to communicate.
However, deploying \diet on heterogeneous networks that are not
reachable from each other except through ssh connection is a complicated
task needing complex configuration. Moreover, to ensure that all
objects can contact each others, we need to set-up and launch ssh
tunnels between each of them, reducing significantly the \diet
scalability in that network configuration context.

The \textit{\diet forwarders} are the solution for such a situation, by
reducing the number of ssh tunnels to the minimum and making their
launch totally transparent for the final users. The \diet forwarders
configuration is very simple even for very complex network topologies.

The next section presents the global operation of \diet
forwarders. Section \ref{sec:ForwarderConfig} presents the \dietforwarder
executable, its command-line options and configuration file. Then,
section \ref{sec:ForwarderExamples} gives two examples of forwarder
configurations.

\section{Easy CORBA objects connections through ssh}
Each CORBA object in \diet is reachable through omniORB using a TCP
port and the hostname of the machine on which it is executed. By
default these parameters are fixed automatically by omniORB but it is
also possible to choose them statically using the \diet configuration
file.
When two objects are located on different networks that are reachable
only through ssh, it is easy to open an ssh tunnel to redirect the
communications between them on the good port and host. Then, correcting
the objects bindings into the omniNames servers is sufficient to ensure
the communications between the different objects. To allow two
CORBA objects to communicate through an ssh tunnel, users must:
\begin{itemize}
\item Start the process which declares the objects and register them
  into the omniNames server(s).
\item Open ssh tunnels that redirect the local objects ports to
  remote ports.
\item Modify the objects bindings in the omniNames server(s) to make
  them point to the forwarded ports.
\end{itemize}

When using few objects that do not need much interaction, these steps
can easily be done ``manually''. But \diet uses several different
objects for each element and is designed to manage thousands of nodes.
Moreover, creating an ssh tunnel between all the nodes of a Grid
cannot even be considered.

The \diet forwarders deal with this problem by creating \textit{proxy
  objects} that forward the communications to a peer forwarder through
a unique ssh tunnel. Then, only one ssh tunnel is created to connect
two different networks. This system also allows users to define complex
communications routing. Figure \ref{fig:forwarder} shows how the \diet
forwarders work to route communications through ssh tunnels. Moreover
most of the configuration of the forwarders automatically can be set
by \diet itself.

\begin{figure}[htp]
\begin{center}
  \includegraphics[width=12cm]{fig/Forwarder}
\end{center}
\caption{Forwarders routing \diet communication through ssh tunnels
  \label{fig:forwarder}}
\end{figure}

\section{The \dietforwarder executable}
\label{sec:ForwarderConfig}
Since \diet 2.5, when installing \diet, an executable called \textit{DIETForwarder}
is installed on the \textit{bin} directory. It allows to launch a
forwarder object following the configuration passed on the command
line.
\subsection{Command line options}
The \diet forwarder executable takes several options to be launched:
\begin{itemize}
\item \verb#--name#: to define the name of the forwarder.
\item \verb#--peer-name#: the name of its peer on the other network.
\item \verb#--ssh-host#: the ssh host for the tunnel creation.
\item \verb#--ssh-port#: the port to use to establish the ssh
  connection (by default: 22).
\item \verb#--ssh-login#: the login to use to establish the ssh
  connection (by default: current user login).
\item \verb#--ssh-key#: the private key to use to establish the ssh
  connection (by default: \verb#$HOME/.ssh/id_rsa#).
\item \verb#--remote-port#: the port to listen on the ssh host.
\item \verb#--net-config#: the path to the configuration file.
\item \verb#-C#: create the tunnel from this forwarder.
\end{itemize}
The remote port can be chosen randomly among the available TCP ports
on the remote host.\\

\noindent In order, to activate a \diet forwarder, users must:
\begin{itemize}
\item Launch omniNames on the remote and local hosts.
\item Launch the first peer on the remote host, only defining its name
  and its network configuration.
\item Launch the second peer, passing it it the first peer name, the
  ssh connection informations, the remote port to use and its network
  configuration, and the \verb#-C# option to force it to create the
  ssh tunnel.
\end{itemize}

\noindent\textit{Rem: The forwarders must be launched before the \diet
  hierarchy.}


\subsection{Configuration file}
To describe to which networks a forwarder gives an access to, users pass a
configuration file to the executable through the \verb#--net-config#
option. This file contains several rules describing which network is
reachable using this forwarder:
\begin{itemize}
\item The \verb#accept# rules: these rules describe which network can
  be accessed using the forwarder.
\item The \verb#reject# rules: these rules describe which network
  cannot be accessed usin the forwarder.\\
\end{itemize}

A rule is a regular expression that describes a set of hostnames or IP
addresses. \textit{localhost} is a special rule that represent, the
``localhost'' string, but also the 127.0.0.1 IP address and the main
network interface IP addresses.\\

A \diet forwarder configuration file line always starts with
\textit{accept:} or \textit{reject:} immediatly followed by a regular
expression.\\

The rules are evaluated starting by the \textit{accept} rules. Then
the rule \textit{accept:.*} (accept every hostname) followed by the
rule \textit{reject:localhost} means that the forwarder manage the
connections to every hosts except the local host.

Examples of configuration files are given in section
\ref{sec:ForwarderExamples}.


\section{Configuration examples}
\label{sec:ForwarderExamples}
The first example presents the configurations of \diet forwarders to
connect a SeD located on a network only reachable through ssh to a
\diet hierarchy located on another network.\\

The second example presents the configurations of \diet forwarders to
connect three networks: the first one can only reach the second one
through ssh and the third one can also only reach the second one
through ssh. We want to connect \diet elements distributed over the
three networks.
\subsection{Simple configuration}
\begin{itemize}
\item The two different domains are \textit{net1} and \textit{net2}. The forwarders will
  be launched on the hosts \textit{fwd.net1} and \textit{fwd.net2}.
\item There is no possibility to access \textit{fwd.net1} from
  \textit{fwd.net2} but users can access \textit{fwd.net2} from
  \textit{fwd.net1} using an ssh connection.
\item The forwarder on \textit{fwd.net1} is named \textit{Fwd1}, the
  forwarder on \textit{fwd.net2} is named \textit{Fwd2}.
\item One SeD is launched on \textit{fwd.net2}, the rest of the \diet
  hierarchy is launched on the \textit{net1} domain.\\
\end{itemize}

\noindent\textbf{Command line to launch \textit{Fwd1}: }\\
{\small \it fwd.net1\$ dietForwarder {\tiny$--$}name Fwd1
  {\tiny$--$}peer-name Fwd2 $\backslash$\\
  \hspace*{4.2cm}{\tiny$--$}ssh-host fwd.net2 {\tiny$--$}ssh-login
  dietUser $\backslash$\\
  \hspace*{4.2cm}{\tiny$--$}ssh-key id\_rsa\_net2
  {\tiny$--$}remote-port 50000 $\backslash$\\
  \hspace*{4.2cm}{\tiny$--$}net-config net1.cfg {\tiny$-$}C}\\[2mm]
\noindent\textbf{Command line to launch \textit{Fwd2}: }\\
{\small \it fwd.net2\$ dietForwarder {\tiny$--$}name Fwd2
  {\tiny$--$}net-config net2.cfg}\\[3mm]
\noindent\textbf{Configuration file for \textit{Fwd1}:}\\
In this example, the forwarders \textit{Fwd1} accepts only the
connections to \textit{fwd.net2}.
\begin{verbatim}
accept:fwd.net2
\end{verbatim}

\noindent\textbf{Configuration file for \textit{Fwd2}:}\\
In this example, the forwarders \textit{Fwd2} accepts all the
connections except those which are for the localhost.
\begin{verbatim}
accept:.*
reject:localhost
\end{verbatim}

Note that \textit{Fwd2} has to be launched before \textit{Fwd1}.
When the two forwarders are launched, the user can deploy his \diet
hierarchy. All the communications through  \diet forwarders are
transparent.

\subsection{Complex network topology}
To connect the three domains, we need 4 forwarders (2 pairs): one on
\textit{net1}, two on \textit{net2} and one on \textit{net3}.
\begin{itemize}
\item The three domains are: \textit{net1}, \textit{net2} and
  \textit{net3}.
\item The machines located on \textit{net1} and \textit{net3} are only
  reachable from \textit{fwd.net2} through ssh.
\item The four forwarders are named \textit{Fwd1}, \textit{Fwd2-1},
  \textit{Fwd2-3} and \textit{Fwd3}.
\item The \diet hierarchy is distributed on the three networks.\\
\end{itemize}

\noindent\textbf{Command line to launch \textit{Fwd1}: }\\
{\small \it fwd.net1\$ dietForwarder {\tiny$--$}name Fwd1
  {\tiny$--$}net-config net1.cfg}\\[2mm]

\noindent\textbf{Command line to launch \textit{Fwd2-1}: }\\
{\small \it fwd.net2\$ dietForwarder {\tiny$--$}name Fwd2-1
  {\tiny$--$}peer-name Fwd1 $\backslash$\\
  \hspace*{4.2cm}{\tiny$--$}ssh-host fwd.net1 {\tiny$--$}ssh-login
  dietUser $\backslash$\\
  \hspace*{4.2cm}{\tiny$--$}ssh-key id\_rsa\_net1
  {\tiny$--$}remote-port 50000 $\backslash$\\
  \hspace*{4.2cm}{\tiny$--$}net-config net2-1.cfg {\tiny$-$}C}\\[2mm]

\noindent\textbf{Command line to launch \textit{Fwd2-3}: }\\
{\small \it fwd.net2\$ dietForwarder {\tiny$--$}name Fwd2-3
  {\tiny$--$}peer-name Fwd3 $\backslash$\\
  \hspace*{4.2cm}{\tiny$--$}ssh-host fwd.net3 {\tiny$--$}ssh-login
  dietUser $\backslash$\\
  \hspace*{4.2cm}{\tiny$--$}ssh-key id\_rsa\_net3
  {\tiny$--$}remote-port 50000 $\backslash$\\
  \hspace*{4.2cm}{\tiny$--$}net-config net2-3.cfg {\tiny$-$}C}\\[2mm]

\noindent\textbf{Command line to launch \textit{Fwd3}: }\\
{\small \it fwd.net3\$ dietForwarder {\tiny$--$}name Fwd3
  {\tiny$--$}net-config net3.cfg}\\[3mm]

\noindent\textbf{Configuration file for \textit{Fwd1}:}\\
\textit{Fwd1} manages the communications for all the host outside
\textit{net1}.
\begin{verbatim}
accept:.*
reject:.*\.net1
\end{verbatim}

\noindent\textbf{Configuration file for \textit{Fwd2-1}:}\\
\textit{Fwd2-1} manages the communication for all the hosts located on
\textit{net1}.
\begin{verbatim}
accept:.*\.net1
\end{verbatim}

\noindent\textbf{Configuration file for \textit{Fwd2-3}:}\\
\textit{Fwd2-3} manages the communication for all the hosts located on
\textit{net3}.
\begin{verbatim}
accept:.*\.net3
\end{verbatim}

\noindent\textbf{Configuration file for \textit{Fwd3}:}\\
\textit{Fwd1} manages the communications for all the host outside
\textit{net3}.
\begin{verbatim}
accept:.*
reject:.*\.net3
\end{verbatim}

Using this configuration, a communication from a host on \textit{net1}
to a host on \textit{net3} is first routed from \textit{Fwd1} to
\textit{Fwd2-1} and then from \textit{Fwd2-3} to \textit{Fwd3}.
Note that \textit{Fwd1} has to be launched before \textit{Fwd2-1}, and
\textit{Fwd3} has to be launched before \textit{Fwd2-3}.

%
% Appendix
%
\newpage
\appendix
%****************************************************************************%
%* DIET User's Manual appendix file                                         *%
%*                                                                          *%
%*  Author(s):                                                              *%
%*    - Philippe COMBES (Benjamin.Depardon@ens-lyon.fr)                     *%
%*                                                                          *%
%* $LICENSE$                                                                *%
%****************************************************************************%
%* $Id$
%* $Log$
%* Revision 1.11  2010/11/09 03:58:28  bdepardo
%* Info on name for SeDs.
%* Reorganise parameters by alphabetical order
%*
%* Revision 1.10  2010/09/10 13:05:29  bdepardo
%* Typos
%*
%* Revision 1.9  2010/05/25 08:04:49  bdepardo
%* Fixmes removal
%*
%* Revision 1.8  2010/03/29 23:52:47  ecaron
%* Update configuration information for DIET v2.4
%*
%* Revision 1.7  2010/02/25 07:15:57  ycaniou
%* DAGDA -> macro + sc
%* Add info logs � dagda.tex
%* Formatage document + ispell
%*
%* Revision 1.6  2010/02/25 06:45:38  ycaniou
%* Add local variables
%*
%* Revision 1.5  2010/02/02 19:41:43  bdepardo
%* A few corrections
%*
%* Revision 1.4  2010/02/01 06:55:44  ycaniou
%* Typo
%*
%* Revision 1.3  2010/01/21 14:05:24  bdepardo
%* Available options present for the Diet elements.
%* Please fill in the blanks.
%*
%* Revision 1.2  2009/10/26 07:23:07  bdepardo
%* Added chapter on dynamic hierarchy management.
%*
%* Revision 1.1  2009/09/08 13:46:34  bdepardo
%* Added appendix.
%* Currently the page layout is broken in the appendix.
%*
%****************************************************************************%

\chapter{Appendix}
\label{ch:appendix}
\section{Configuration files}

\begin{description}


%%%% A
\item{\bf{ackFile}}
  \begin{itemize}
  \item Component: Agent and \sed
  \item Mode: Acknowledge file
  \item Type: String
  \item Description: Path to a file that will be created when the
    element is ready to execute.
  \end{itemize}

\item{\bf{agentType}}
  \begin{itemize}
  \item Component: Agent (MA and LA)
  \item Mode: All
  \item Type: Agent type
  \item Description: Master Agent or Local Agent? As there is only one
    executable for both agent types, it is COMPULSORY to specify the type
    of this agent: DIET\_MASTER\_AGENT (or MA) or DIET\_LOCAL\_AGENT (or
    LA).
  \end{itemize}

%%%% B
\item{\bf{batchName}}
  \begin{itemize}
  \item Component: \sed
  \item Mode: Batch
  \item Type: String
  \item Description: The reservation batch system's name.
  \end{itemize}

\item{\bf{batchQueue}}
  \begin{itemize}
  \item Component: \sed
  \item Mode: Batch
  \item Type: String
  \item Description: The name of the queue where the job will be submitted.
  \end{itemize}

\item{\bf{bindServicePort}}
  \begin{itemize}
  \item Component: MA
  \item Mode: All
  \item Type: Integer
  \item Description: port used by the Master Agent to share its IOR.
  \end{itemize}

%%%% C
\item{\bf{cacheAlgorithm}}
  \begin{itemize}
  \item Component: All
  \item Mode: \dagda
  \item Type: String
  \item Description: The cache replacement algorithm used when \dagda needs more space
to store a data. Possible values are: \textit{LRU, LFU, FIFO}. By default, 
no cache replacement algorithm. \dagda never replace a data by
another one.
  \end{itemize}

\item{\bf{clientMaxNbSeD}}
  \begin{itemize}
  \item Component: Client
  \item Mode: All
  \item Type: Integer
  \item Description: The maximum number of \sed the client should receive.
  \end{itemize}

%%%% D
\item{\bf{dataBackupFile}}
  \begin{itemize}
  \item Component: Agent and SeD
  \item Mode: \dagda
  \item Type: String
  \item Description: The path to the file that will be used when \dagda save all its
stored data/data path when asked by the user (Checkpointing). By default, no
checkpointing is possible.
  \end{itemize}

\item{\bf{dietHostName}}
  \begin{itemize}
  \item Component: All
  \item Mode: All
  \item Type: String
  \item Description: the listening interface of the agent. If not specified,
    let the ORB get the hostname from the system (the first one if several 
    one are available).
  \end{itemize}

\item{\bf{dietPort}}
  \begin{itemize}
  \item Component: All
  \item Mode: All
  \item Type: Integer
  \item Description: the listening port of the agent. If not
    specified, let the ORB get a port from the system (if the default
    2809 was busy).
  \end{itemize}

%%%% E

%%%% F
\item{\bf{fastUse}}
  \begin{itemize}
  \item Component: Agent and \sed
  \item Mode: FAST
  \item Type: Boolean
  \item Description: If set to 0, all LDAP and NWS parameters are
    ignored, and all requests to FAST are disabled (when \diet is compiled
    with FAST). This is useful for testing a \diet platform without
    deploying an LDAP base nor an NWS platform.
  \end{itemize}

%%%% G

%%%% H

%%%% I
\item{\bf{initRequestID}}
  \begin{itemize}
  \item Component: MA
  \item Mode: All
  \item Type: Integer
  \item Description: When a request is sent to the Master Agent, a request ID
  is associated and by default it begins at 1. If this parameter is provided, it
  will begins at initRequestID.
  \end{itemize}

\item{\bf{internOARbatchQueueName}}
  \begin{itemize}
  \item Component: \sed
  \item Mode: Batch
  \item Type: String
  \item Description: only useful when using CORI batch features with
    OAR 1.6
  \end{itemize}

%%%% J

%%%% K

%%%% L
\item{\bf{ldapBase}}
  \begin{itemize}
  \item Component: Agent and \sed
  \item Mode: FAST
  \item Type: Address 
  \item Description: <host:port> of the LDAP base that stores
    FAST-known services.
  \end{itemize}

\item{\bf{ldapMask}}
  \begin{itemize}
  \item Component: Agent and \sed
  \item Mode: FAST
  \item Type: String
  \item Description: the mask which is registered in the LDAP base.
  \end{itemize}

\item{\bf{ldapUse}}
  \begin{itemize}
  \item Component: Agent and \sed
  \item Mode: FAST
  \item Type: Boolean
  \item Description: 0 tells FAST not to look for the services in an
    LDAP base.
  \end{itemize}

\item{\bf{locationID}}
  \begin{itemize}
  \item Component: \sed
  \item Mode: \dagda
  \item Type: String
  \item Description: This parameter is used for alternative transfer cost
  prediction.
  \end{itemize}

\item{\bf{lsOutbuffersize}}
  \begin{itemize}
  \item Component: Agent and \sed
  \item Mode: All
  \item Type: Integer
  \item Description: the size of the buffer for outgoing messages.
  \end{itemize}

\item{\bf{lsFlushinterval}}
  \begin{itemize}
  \item Component: Agent and \sed
  \item Mode: All
  \item Type: Integer
  \item Description: the flush interval for the outgoing message buffer.
  \end{itemize}

%%%% M
\item{\bf{MADAGNAME}}
  \begin{itemize}
  \item Component: Client
  \item Mode: Workflow
  \item Type: String
  \item Description: the name of the \madag agent to wich the client
    will connect.
  \end{itemize}

\item{\bf{MAName}}
  \begin{itemize}
  \item Component: Client
  \item Mode: All
  \item Type: String
  \item Description: Master Agent name.  The ORB configuration files of the clients
    and the children of this MA (LAs and SeDs) must point at the same CORBA
    Naming Service as the one pointed at by the ORB configuration file of
    this agent.
  \end{itemize}

\item{\bf{maxConcJobs}}
  \begin{itemize}
  \item Component: \sed
  \item Mode: All
  \item Type: Integer
  \item Description: If useConcJobLimit == true, how many jobs can run at once?
  This shoudl be used in conjunction with \emph{maxConcJobs}.
  
  \end{itemize}

\item{\bf{maxDiskSpace}}
  \begin{itemize}
  \item Component: All
  \item Mode: \dagda
  \item Type: Integer
  \item Description: The maximum disk space used by \dagda to store the data. If set
to 0, \dagda will not take care of the disk usage. By default this value is
equal to the available disk space on the disk partition chosen by the
  \textit{storageDirectory} option. 
  \end{itemize}

\item{\bf{maximumNeighbours}}
  \begin{itemize}
  \item Component: MA
  \item Mode: Integer 
  \item Type: Multi MA
  \item Description: maximum number of connected neighbours. The agent
    does not accept a greater number of connection to build the federation
    than maximumNeighbours.
  \end{itemize}

\item{\bf{maxMemSpace}}
  \begin{itemize}
  \item Component: All
  \item Mode: \dagda
  \item Type: Integer
  \item Description: The maximum memory space used by \dagda to store the data. If set
to 0, \dagda will not take care of the memory usage. By default no maximum
memory usage is set. Same effect than to choose 0.
  \end{itemize}

\item{\bf{maxMsgSize}}
  \begin{itemize}
  \item Component: All
  \item Mode: \dagda
  \item Type: Integer
  \item Description: The maximum size of a CORBA message sent by \dagda. By
  default this value is equal to the omniORB \textit{giopMaxMsgSize} size.
  \end{itemize}

\item{\bf{minimumNeighbours}}
  \begin{itemize}
  \item Component: MA
  \item Mode: Multi MA
  \item Type: Integer
  \item Description: Minimum number of connected neighbours. If the
    agent has less that this number of connected neighbours, is going to
    find some new connections. 
  \end{itemize}

\item{\bf{moduleConfigFile}}
  \begin{itemize}
  \item Component: Agent
  \item Mode: User scheduling
  \item Type: String
  \item Description: Optional configuration file for the module.
  \end{itemize}

%%%% N
\item{\bf{name}}
  \begin{itemize}
  \item Component: Agent and \sed
  \item Mode: All
  \item Type: String
  \item Description: The name of the element. This parameter is not mandatory
    for a \sed as it can generate a name during launch time.
  \end{itemize}

\item{\bf{neighbours}}
  \begin{itemize}
  \item Component: MA
  \item Mode: Multi MA
  \item Type: String
  \item Description: A list of Master Agent that must be contacted to
    build a federation. The format is a list of host:port.
  \end{itemize}

\item{\bf{nwsForecaster}}
  \begin{itemize}
  \item Component: Agent and \sed
  \item Mode: FAST
  \item Type: Address
  \item Description: NWS forecast module used by FAST.
  \end{itemize}

\item{\bf{nwsNameserver}}
  \begin{itemize}
  \item Component: Agent and \sed
  \item Mode: FAST
  \item Type: Address
  \item Description: <host:port> of the NWS nameserver.
  \end{itemize}

\item{\bf{nwsUse}}
  \begin{itemize}
  \item Component: Agent and \sed
  \item Mode: FAST
  \item Type: Boolean
  \item Description: 0 tells FAST not to use NWS for its comm times
    forecasts.
  \end{itemize}

%%%% O

%%%% P
\item{\bf{parentName}}
  \begin{itemize}
  \item Component: LA and \sed
  \item Mode: All
  \item Type: String
  \item Description: the name of the agent to which the element will
    register. This agent must have registered at the same CORBA Naming
    Service that is  pointed to by your ORB configuration.
  \end{itemize}

\item{\bf{pathToNFS}}
  \begin{itemize}
  \item Component: \sed
  \item Mode: Batch
  \item Type: String
  \item Description: Path to an NFS directory where you have read/write rights.
  \end{itemize}

\item{\bf{pathToTmp}}
  \begin{itemize}
  \item Component: \sed
  \item Mode: Batch
  \item Type: String
  \item Description: Path to a temporary directory where you have
    read/write rights.
  \end{itemize}

%%%% Q

%%%% R
\item{\bf{restoreOnStart}}
  \begin{itemize}
  \item Component: Agent and SeD
  \item Mode: \dagda
  \item Type: Boolean
  \item Description: \dagda will load the \textit{dataBackupFile} file at start and
restore all the data recorded at the last checkpointing event. Possible values
are 0 or 1. By default, no file loading on start - 0.
  \end{itemize}


%%%% S
\item{\bf{schedulerModule}}
  \begin{itemize}
  \item Component: Agent
  \item Mode: User scheduling
  \item Type: String
  \item Description: The path to the scheduler library file containing the
  implementation of the plugin scheduler class.
  \end{itemize}

\item{\bf{shareFiles}}
  \begin{itemize}
  \item Component: Agent
  \item Mode: \dagda
  \item Type: Boolean
  \item Description: The \dagda component shares its file data with all its children
(when the path is accessible by them, for example, if the storage directory is
on a NFS partition). Value can be 0 or 1.  By default no file sharing - 0.
  \end{itemize}

\item{\bf{storageDirectory}}
  \begin{itemize}
  \item Component: All
  \item Mode: \dagda or Batch
  \item Type: String
  \item Description: The directory on which \dagda will store the data files.
  By default \texttt{/tmp} is used.
  \end{itemize}


%%%% T
\item{\bf{traceLevel}}
  \begin{itemize}
  \item Component: All
    \item Mode: All
  \item Type: Integer
  \item Description: traceLevel for the \diet agent:
    \begin{itemize}
    \item  0 \diet prints only warnings and errors on the standard error
      output,
    \item 1 [default] \diet prints information on the main steps
      of a call,
    \item 5 \diet prints information on all internal steps too,
    \item 10 \diet prints all the communication structures too,
    \item $>10$ (traceLevel - 10) is given to the ORB to print CORBA messages
      too.
    \end{itemize}
  \end{itemize}

%%%% U
\item{\bf{updateLinkPeriod}}
  \begin{itemize}
  \item Component: MA
  \item Mode: Multi MA
  \item Type: Integer
  \item Description: The agent check at a regular time basis that all
    it's neighbours are still alive and try to connect to a new one if the
    number of connections is less than
    \emph{minimumNeighbours}. \emph{updateLinkPeriod} indicate the period
    in second between two checks.
  \end{itemize}

\item{\bf{useConcJobLimit}}
  \begin{itemize}
  \item Component: \sed
  \item Mode: All
  \item Type: Boolean
  \item Description: should SeD restrict the number of concurrent solves?
  This should be used in conjunction with \emph{maxConcJobs}. 
  \end{itemize}

\item{\bf{useLogService}}
  \begin{itemize}
  \item Component: Agent and \sed
  \item Mode: All
  \item Type: Boolean
  \item Description: 1 to use the LogService for monitoring.
  \end{itemize}

\item{\bf{USE\_SPECIFIC\_SCHEDULING}}
  \begin{itemize}
  \item Component: Client
  \item Mode: Custom Client Scheduling (CCS)
  \item Type: String
  \item Description: 
    This option specifies the scheduler the client will use whenever it submits
    a request:
    \begin{itemize}
    \item BURST\_REQUEST: round robin on the available \sed
    \item BURST\_LIMIT: only allow a certain number of request per \sed in
      parallel the limit can be set with "void
      setAllowedReqPerSeD(unsigned ix)"
    \end{itemize}
  \end{itemize}

%%%% V

%%%% W

%%%% X

%%%% Y

%%%% Z





% OUTDATED
%
% \item{\bf{USEWFLOGSERVICE}}
%   \begin{itemize}
%   \item Component: 
%   \item Mode: Workflow
%   \item Type: String
%   \item Description: ?
%   \end{itemize}
% \fixme{USEWFLOGSERVICE a �t� supprim�, mais est ce que c'est bien plus utilis�,
% j'ai fait un grep sur le code et je ne trouve d'occurence plus que dans le
% parser, donc je dirais que c'est outdated. A valider}











\end{description}






%%% Local Variables:
%%% mode: latex
%%% ispell-local-dictionary: "american"
%%% mode: flyspell
%%% fill-column: 79
%%% End:


%%%%
% BIBLIO
%%%%

\bibliographystyle{plain}
\bibliography{UsersManual}

\end{document}

%%% Local Variables:
%%% mode: latex
%%% ispell-local-dictionary: "american"
%%% mode: flyspell
%%% fill-column: 79
%%% End:
