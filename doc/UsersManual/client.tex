%****************************************************************************%
%* DIET User's Manual client chapter file                                   *%
%*                                                                          *%
%*  Author(s):                                                              *%
%*    - Eddy CARON      (Eddy.Caron@ens-lyon.fr)                            *%
%*    - Philippe COMBES (Philippe.Combes@ens-lyon.fr)                       *%
%*    - Christophe Pera (Christophe.Pera@ens-lyon.fr)                       *%
%*                                                                          *%
%* $LICENSE$                                                                *%
%****************************************************************************%
%* $Id$
%* $Log$
%* Revision 1.9  2004/02/02 15:57:15  ecaron
%* Add suggestion from Christophe Pera (Thanks Christophe)
%*
%* Revision 1.8  2004/01/29 17:08:47  ecaron
%* Add suggestions from Frederic Desprez. Thanks !
%*
%* Revision 1.7  2004/01/27 00:21:17  ecaron
%* Add suggestions from Jean-Yves (Thanks!)
%*
%* Revision 1.6  2004/01/21 00:25:13  ecaron
%* Add suggestions from Holly Dail. Thanks !
%*
%* Revision 1.5  2004/01/07 10:27:33  cpera
%* Add asynchronous call example and useAsyncAPI config parameter.
%*
%* Revision 1.4  2004/01/05 13:06:53  ecaron
%* Update example to DIET 1.0
%*
%* Revision 1.3  2003/12/12 14:40:02  pkchouha
%* Bug correction
%*
%* Revision 1.2  2003/12/12 12:36:17  ecaron
%* Change call to diet_initialize()
%* Correct some bug
%*
%* Revision 1.1  2003/09/09 12:38:20  pcombes
%* Reorganization of doc: UM becomes UsersManual.
%*
%* Revision 1.11  2003/06/02 13:47:05  pcombes
%* Fix footnotesize.
%*
%* Revision 1.10  2003/05/23 09:23:35  pcombes
%* Add suggestions from Jean-Yves. Thanks !
%*
%* Revision 1.9  2003/05/15 14:17:58  pcombes
%* UM 0.7
%*
%* Revision 1.6  2003/01/24 16:58:54  pcombes
%* UM 0.6.4
%*
%* Revision 1.4  2003/01/21 12:17:02  pcombes
%* Update UM to API 0.6.3, and "hide" data structures.
%*
%* Revision 1.3  2003/01/14 08:04:36  pcombes
%* MAJ
%*
%* Revision 1.2  2003/01/13 12:09:00  pcombes
%* UM: client part complete for users's day ...
%****************************************************************************%

\chapter{Building a client program}
\label{ch:client}

The most difficult part of building a client program is to understand the way a
problem has to be described. Once this step is done, it is fairly easy to
build calls to DIET.


\section{Structure of a client program}
\label{sec:cl_struct}

Since the client side of DIET is a library, a client program has to define the
\texttt{main} function: it uses DIET through function calls. The complete
client-side interface is described in the \texttt{$<$install\_dir$>$/include}
files \texttt{DIET\_data.h} (see Chapter \ref{ch:data}) and
\texttt{DIET\_client.h}. Please refer to these two files for a complete and
up-to-date API description, and include at least the latter at the beginning of
your source code (\texttt{DIET\_client.h} includes \texttt{DIET\_data.h}):
{\footnotesize
\begin{verbatim}
#include <stdio.h>
#include <stdlib.h>

#include "DIET_client.h"

int main(int argc, char *argv[])
{
  diet_initialize(configuration_file, argc, argv);
  // Successive DIET calls ...
  diet_finalize();
}
\end{verbatim}
}

The client program must open its DIET session with a call to
\texttt{diet\_initialize}, which parses the configuration file to set
all options and get a reference to the DIET Master Agent. The session
is closed with a call to \texttt{diet\_finalize}, which frees all
resources, if any, associated with this session on the client, servers,
and agents, but not the memory allocated for all INOUT and OUT
arguments brought back onto the client during the session, so that the
user can still access them (and still has to free them !)


\section{Client API}
\label{sec:clAPI}

The client API follows the GridRPC definition \cite{gridRPC:02}: all
\texttt{diet\_} functions are ``duplicated'' with \texttt{grpc\_}
functions.  Both \texttt{diet\_initialize}/\texttt{grpc\_initialize}
and \texttt{diet\_finalize}/\texttt{grpc\_finalize} belong to the
GridRPC API. 
 
A problem is managed through a \emph{function\_handle}, that
associates a server to a problem name. Please do not use
\texttt{diet\_function\_handle\_init}, since it assumes that the
client already knows the server, and DIET is not conceived for such a
use. This function is only provided for GridRPC compliance. The
structure allocation is performed through the function
\texttt{diet\_function\_handle\_default}.

The \emph{function\_handle} returned is associated to the problem description,
its profile, in the call to \texttt{diet\_call}.

\section{Examples}
\label{sec:cl_ex}

Let us consider the same example as in Section \ref{sec:pbex} for synchronous 
and asynchronous call.  Here, the client configuration file is given as the 
first argument on the command line, and we decide to hardcode the matrix,
its factor, and the name of the problem:

\subsection{synchronous call}
\texttt{smprod}~\footnote{Source code available in \texttt{doc/tutorial/solutions/exercise2/client\_smprod.c}} for scalar by matrix product.

{\footnotesize
\begin{verbatim}
#include <stdio.h>
#include <stdlib.h>
#include <math.h>
#include "DIET_client.h"

int main(int argc, char **argv)
{
  int i;
  double  factor = M_PI; /* Pi, why not ? */
  double *matrix;        /* The matrix to multiply */
  float  *time   = NULL; /* To check that time is set by the server */

  diet_profile_t         *profile;

  /* Allocate the matrix: 60 lines, 100 columns */
  matrix = malloc(60 * 100 * sizeof(double));
  /* Fill in the matrix with dummy values (who cares ?) */
  for (i = 0; i < (60 * 100); i++) {
    matrix[i] = 1.2 * i;
  }
  
  /* Initialize a DIET session */
  diet_initialize("./client.cfg", argc, argv);

  /* Create the profile as explained in Chapter 3 */
  profile = diet_profile_alloc("smprod",0, 1, 2); // last_in, last_inout, last_out
  
  /* Set profile arguments */
  diet_scalar_set(diet_parameter(profile,0), &factor, 0, DIET_DOUBLE);
  diet_matrix_set(diet_parameter(profile,1), matrix,  0, DIET_DOUBLE, 60, 100, DIET_COL_MAJOR);
  diet_scalar_set(diet_parameter(profile,2), NULL,    0, DIET_FLOAT);
  
  if (!diet_call(profile)) { /* If the call has succeeded ... */
     
    /* Get and print time */
    diet_scalar_get(diet_parameter(profile,2), &time, NULL);
    if (time == NULL) {
      printf("Error: time not set !\n");
    } else {
      printf("time = %f\n", *time);
    }

    /* Check the first non-zero element of the matrix */
    if (fabs(matrix[1] - ((1.2 * 1) * factor)) > 1e-15) {
      printf("Error: matrix not correctly set !\n");
    }
  }

  /* Free profile */
  diet_profile_free(profile);
  diet_finalize();
}
\end{verbatim}
}


\subsection{asynchronous call}
\texttt{smprod}~\footnote{Source code available in \texttt{doc/tutorial/solutions/exercise2/client\_smprodAsync.c}} for scalar by matrix product.
{\footnotesize
\begin{verbatim}
#include <stdio.h>
#include <stdlib.h>
#include <math.h>
#include "DIET_client.h"

int main(int argc, char **argv)
{
  int i;
  double  factor = M_PI; /* Pi, why not ? */
  double *matrix[5];        /* The matrix to multiply */
  float  *time   = NULL; /* To check that time is set by the server */

  diet_profile_t         *profile[5];
  diet_reqID_t rst[5] = {0,0,0,0,0};

  
  /* Initialize a DIET session */
  diet_initialize("./client.cfg", argc, argv);

  /* Create the profile as explained in Chapter 3 */
  for (i = 0; i < 5; i++){
    /* Allocate the matrix: 60 lines, 100 columns */
    matrix[i] = malloc(60 * 100 * sizeof(double));
    /* Fill in the matrix with dummy values (who cares ?) */
    for (j = 0; j < (60 * 100); j++) {
      matrix[i][j] = 1.2 * j;
    }
    profile = diet_profile_alloc("smprod",0, 1, 2); // last_in, last_inout, last_out
  
    /* Set profile arguments */
    diet_scalar_set(diet_parameter(profile[i],0), &factor, 0, DIET_DOUBLE);
    diet_matrix_set(diet_parameter(profile[i],1), matrix[i],  0, DIET_DOUBLE,
                    60, 100, DIET_COL_MAJOR);
    diet_scalar_set(diet_parameter(profile[i],2), NULL,    0, DIET_FLOAT);
  }
  
  int rst_call = 0;
  if ((rst_call = diet_call_async(profile[i], &rst[i])) != 0)  
     printf("Error in diet_call_async return -%d-\n", rst_call);
  else {
    printf("request ID value = -%d- \n", rst[i]);
    if (rst[i] < 0) {
      printf("error in request value ID\n");
      return;
    }
  }
  rst_call = 0;
  if ((rst_call = diet_wait_and((diet_reqID_t*)&rst, (unsigned int)5)) != 0)
     printf("Error in diet_wait_and\n");
  else {
    printf("Result data for requestID");
    for (i = 0; i < 5; i++) printf(" %d ", rst[i]);
    printf(" and omnithreadID %d \n", omni_thread::self()->id());
    for (i = 0; i < 5; i++){
      sprintf(requestID, "%d", rst[i]);
      /* Get and print time */
      diet_scalar_get(diet_parameter(profile[i],2), &time, NULL);
      if (time == NULL) {
        printf("Error: time not set !\n");
      } else {
        printf("time = %f\n", *time);
      }

      /* Check the first non-zero element of the matrix */
      if (fabs(matrix[i][1] - ((1.2 * 1) * factor)) > 1e-15) {
        printf("Error: matrix not correctly set !\n");
      }
    }
  }
  /* Free profiles */
  for (i = 0; i < 5; i++){
    diet_cancel(rst[i]);
    diet_profile_free(profile[i]);
  }
  diet_finalize();
}
\end{verbatim}
}



\section{Compilation}
\label{sec:cl_comp}

After compiling the client program, the user must link it with the DIET
libraries and the CORBA libraries. The easiest way to compile a program using
DIET with all necessary flags and link it with the right libraries is to trust the
\texttt{Makefile.inc} available in \texttt{$<$include\_dir$>$/include}, and
include it at the beginning of the program makefile.

The \texttt{Makefile.inc} defines the variables:
\begin{itemize}
\item \texttt{CC} and \texttt{CCFLAGS} that are to be used if you compile C
 code,
\item \texttt{CXX} and \texttt{CXXFLAGS} that are to be used if you compile C++
  code,
\item \texttt{DIET\_CLIENT\_LIBS} that link the program to the CORBA and DIET
  client libraries.
\end{itemize}

For our C example, the Makefile should be something like:
{\footnotesize
\begin{verbatim}
include <install_dir>/Makefile.inc

client.o:  client.c
           $(CC) -c $< $(CCFLAGS) -o $@

client:    client.o
           $(CC) $< $(CCFLAGS) $(DIET_CLIENT_LIBS) -o $@
\end{verbatim}
}

